\chapter{مفاهيم اساسية}

\section{مقدمة}
في الآونة الاخيرة تعتبر الاختبارات اللا معلمية من الاختبارات شائعة الاستخدام وخصوصاً عندما يكون شرط الاختبار تطبيق الاختبارات المعلمية غير متوفرة. عندما تكون شروط الاختبار المعلمي غير متوفرة فإن الحل الوحيد هو اجراء اختبار لا معلمي بالاضافة الى ذلك لو ان ظروف الاختبار تدور حول اشياء وصيفة مثل "هل ان العينة عشوائية". هل هناك ارتباط بين متغيرين "هل المتغيرات مستقلة" فإن لابد من استخدام الاختبارات اللامعلمية ولو اننا تغاضينا عن استيفاء شروط الاختبار المعلمي واجريناه فإن النتائج التي سوف نحصل عليها ستكون غير دقيقة وتوقعنا فيه اخطاء كثيرة.

\section{مميزات الاختبارات اللامعلمية}

\begin{enumerate}
	\item الاختبارات اللا معلمية سهلة عند التطبيق.
	\item لا تحتاج الاختبارات اللا معلمية الى شروط كثيرة عند تطبيقها.
	\item يكون من السهل على الباحث المستخدم للاختبارات اللا معلمية وغير المتخصص في الاحصاء التعرف على الشروط البسيطة اللازمة لتطبيق الاختبار اللا معلمي وبالتالي يسهل عليه تحقيقها قبل البدء في استخدام الاختبار مما يجعل استنتاجاته ونتائجه منطقية وقريبة جداّ من الصحة.
\end{enumerate}

\section{شروط الاختبار اللامعلمي}

\begin{enumerate}
	\item العينة المختارة يجب ان تكون عشوائية.
	\item لابد من التشابه في الشكل وفي الاختلاف (التباين) للتوزيعات المستخدمة في التحليل.
	\item احياناً يتطلب الاختبار ان يكون هناك استقلال بين العينات.
\end{enumerate}

\subsection*{فيما يلي بعض الاختبارات اللامعلمية التي سوف نتطرق لها في البحث}

\begin{enumerate}
	\item اختبار مربع كاي.
	\item اختبار جودة التوفيق.
	\item اختبار الاستقلال.
	\item اختبار كولومجروف سيمنروف لعينة واحدة.
	\item اختبار عينتين مستقلتين.
	\item اختبار عينتين غير مستقلتين.
	\item اختبار اكثر من عينتين مستقلتين.
	\item اختبار اكثر من عينتين غير مستقلتين.
\end{enumerate}

\section{بعض المفاهيم الاحصائية}

\subsection{النموذج الاحصائي}
هو عبارة عن تعبير رياضي عن العوامل التي تؤثر في المشاهدة طبقاً لافتراضات التجربة ولابد ان يعكس النموذج العلاقة بين متغير الاستجابة (المعتمد) ومتغير الرئيسي (المستقل) المسؤول عن احداث تغيير في معامل الاستجابة.

\subsection{المجتمع والعينة}
يختلف معنى كلمة المجمتع في علم الاحصاء عن المعنى الشائع لدى عامة الناس حيث تستخدم كلمة المجتمع لدى العامة للاشارة الى مجموعة من الاشخاص الذين يقيمون في منطقة معينة. في حين يعبر المجتمع في علم الاحصاء بأنه جميع الوحدات التي تكون الظاهرة محل للدراسة.\\
\noindent
اما العينة فهي جزء من المجتمع التي يتم اختبارها في الغالب عشوائياً ومن المفترض ان تمثل المجتمع محل الدراسة تمثيلاً صادقاً.

\subsection{المعلمة واحصاء العينة}
المعلمة هي خاصية من خصائص المجتمع التي يتم قياسها.\\
اما احصاء العينة فهو قيمة رقمية تصف خاصية معينة يتم قياسها كمياًعن طريق عينة تمثل مجتمع الدراسة. اي ان احصاء العينة مقدر لعينة المجتمع.

\subsection{الفرضية الاحصائية}
تصريح او ادعاء قد يكون صائباً او يكون خاطئاًحول معلمة او اكثر لمجتمع او لمجموعة من المجتمعات.

\subsection{مستوى المعنوية}
هو احتمال الوقوع في الخطأ من النوع الاول.

\subsection{الاحصاء الاستدلالي}
هو سحب عينة للاستدلال بها من مجتمع او عينات للاستدلال بها عن مجتمعات عدة.

\subsection{البيانات}
عبارة عن ممجوعة من الحقائق والارقام غير المنظمة من مصادر مختلفة يمكن ان تختلف مصادر البينات اعتماداً على ما يحتاجه البحث.

\subsection{الخطأ العشوائي }
هو اي خطأ لا يظهر بانتظام في كل البيانات وليس له علاقة بأي من المتغيرات والخطأ الشعوائي عادة ما يكون له صفة التوزيع الطبيعي بمتوسط قدره صفر $e_{i,j} \sim N(0, \sigma^2)$.

\subsection{درجات الحرية}
هي عدد القيم القابلة للتغير في حساب خاصية احصائية معينة.

\subsection{التوزيع الطبيعي}
من اهم التوزيعات ذات التوزيع المستمر وهو الاكثر شيوعاً والتوزيع الطبيعي يتحدد بمعلمتين هما المتوسط $\mu$ والانحراف المعياري $\sigma$ وتتحدد قيم  المتغير من $-\infty$ الى $-\infty$ معادلة المنحني الطبيعي $f(x)$ 
\[
f(x) = 
\begin{cases}
	\dfrac{1}{\sigma\sqrt{2\pi}} e^{-\frac{1}{2}\left(\frac{x-\mu}{\sigma}\right)^2} & ,-\infty < x < \infty \\[5pt]
	0 & , \text{otherwise}
\end{cases}
\]

\subsection{انواع الفرضيات}

\begin{enumerate}
	\item $H_0$ هي فرضية العدم.
	\item $H_1$ هي الفرضية البديلة
\end{enumerate}

\subsection{انواع الاخطاء}
\begin{enumerate}
	\item \textbf{الخطأ من المنوع الاول} هو فرض فرضية العدم عندما تكون صحيحة.
	\item \textbf{الخطأ من المنوع الثاني} هو قبول الفرضية البديلة عندما تكون خاطئة.
\end{enumerate}

\section{الاختبارات المعلمية  (\en{Parametric Tests})}
الاختبارات المعلمية هي اختبارات إحصائية تُستخدم عندما تكون البيانات تتبع توزيعًا معينًا، مثل التوزيع الطبيعي. هذه الاختبارات تفترض أن البيانات تأتي من مجتمع له خصائص معينة، مثل التوزيع الطبيعي والتجانس في التباين. فيما يلي بعض من أهم الاختبارات المعلمية

\subsection{اختبار (Z-Test)}
يُستخدم عندما يكون حجم العينة كبيرًا (عادةً أكبر من 30).
يُستخدم لاختبار فرضية حول متوسط المجتمع إذا كان الانحراف المعياري للمجتمع معروفًا.

\subsection{اختبار (T-Test)}
يُستخدم عندما يكون حجم العينة صغيرًا (أقل من 30) أو عندما يكون الانحراف المعياري غير معروف.
له أنواع مختلفة: 
\begin{enumerate}
\item  	اختبار T لعينة واحدة: لمقارنة متوسط العينة بمتوسط المجتمع.
\item  	اختبار T لعينتين مستقلتين: لمقارنة متوسطين لمجموعتين مختلفتين.
\item  	اختبار T لعينتين مرتبطتين (\en{Matched-Pair T-Test}): لمقارنة بيانات مترابطة مثل قبل وبعد التجربة.
\end{enumerate}

\subsection{اختبار تحليل التباين (ANOVA)}
يُستخدم لمقارنة متوسطات أكثر من مجموعتين.
له أنواع مختلفة: 
\begin{enumerate}
\item  	ANOVA أحادي الاتجاه (\en{One-Way ANOVA}): لمقارنة مجموعة واحدة من البيانات عبر عدة فئات.
\item  	ANOVA ثنائي الاتجاه (\en{Two-Way ANOVA}): عندما يكون هناك أكثر من متغير مستقل.
\end{enumerate}

\subsection{اختبار F-Test}
\begin{enumerate}[label=$\bullet$]
\item  	يُستخدم لمقارنة التباين بين عينتين أو أكثر.
\item  	يُستخدم في تحليل التباين واختبارات المقارنة بين النماذج الإحصائية.
\item 	متى نستخدم الاختبارات المعلمية؟
\item  	عندما تكون البيانات كمية (عددية).
\item  	عندما يكون التوزيع طبيعيًا أو قريبًا من الطبيعي.
\item 	عندما تكون التباينات متساوية تقريبًا بين المجموعات.
\item  	إذا لم تكن هذه الشروط متوفرة، يتم اللجوء إلى الاختبارات اللامعلمية مثل اختبار ويلكوكسون، كروسكال-واليس، واختبار ميديان.
\end{enumerate}
