\chapter*{مقدمة}
\addcontentsline{toc}{chapter*}{مقدمة}

تُعدُّ الاختبارات الإحصائية أداةً أساسيةً في تحليل البيانات واتخاذ القرارات بناءً على الأدلة التجريبية. وتنقسم هذه الاختبارات إلى نوعين رئيسيين: الاختبارات المعلمية
 \LR{(Parametric Tests)} التي تفترض وجود توزيع معين للبيانات، والاختبارات اللا معلمية \LR{(Nonparametric Tests)} التي لا تتطلب مثل هذه الافتراضات الصارمة. تلعب الاختبارات اللا معلمية دورًا حيويًا في الإحصاء، لا سيما عند التعامل مع بيانات لا تحقق شروط الاختبارات المعلمية، مثل التوزيع الطبيعي أو تساوي التباينات بين المجموعات المختلفة.
أهمية الاختبارات اللا معلمية

تنبع أهمية الاختبارات اللا معلمية من مرونتها الكبيرة في تحليل البيانات، إذ إنها لا تفترض أن العينة مأخوذة من مجتمع ذي توزيع محدد، مما يجعلها مناسبة للتعامل مع البيانات التي قد تكون منحرفة، أو غير متجانسة، أو حتى مرتبة بدلًا من أن تكون كمية مباشرة. على سبيل المثال، في حالات البيانات التي تأخذ شكل رُتب 
\LR{(Ordinal Data)} أو البيانات الفئوية \LR{(Categorical Data)}، تكون هذه الاختبارات أكثر كفاءة من نظيراتها المعلمية.

بالإضافة إلى ذلك، فإن هذه الاختبارات تُستخدم عند التعامل مع عينات صغيرة الحجم، حيث قد يكون من الصعب التحقق من تحقيق البيانات للفرضيات التي تتطلبها الاختبارات المعلمية مثل اختبار t أو ANOVA. كما أنها مفيدة في تحليل البيانات التي تتضمن قيمًا متطرفة (Outliers)، حيث يمكن لهذه القيم أن تؤثر بشكل كبير على نتائج الاختبارات المعلمية، بينما تكون الاختبارات اللا معلمية أكثر مقاومةً لمثل هذه التأثيرات.