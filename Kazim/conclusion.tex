\chapter*{استنتاجات}
\addcontentsline{toc}{chapter*}{استنتاجات}

\begin{itemize}
	\item \textbf{فعالية الاختبارات اللامعلمية:} من خلال تحليل نتائج البحث، تبين أن الاختبارات اللامعلمية توفر نتائج دقيقة وموضوعية في قياس القدرات العقلية للطلاب، مقارنة بالاختبارات التقليدية التي قد تتأثر بتوقعات المعلمين أو تحيزاتهم.
	
	\item \textbf{مزايا الاختبارات اللامعلمية:} توفر الاختبارات اللامعلمية بيئة أكثر حيادية لقياس المهارات والقدرات، مما يساهم في توفير فرص متساوية لجميع الطلاب بغض النظر عن خلفياتهم الاجتماعية أو الثقافية.
	
	\item \textbf{التحديات التي تواجه تطبيق الاختبارات اللامعلمية:} بالرغم من مزاياها، فإن تطبيق الاختبارات اللامعلمية يتطلب تقنيات متقدمة وموارد ضخمة، مثل الأدوات التكنولوجية المتطورة والقدرة على تحليل البيانات الضخمة.
	
	\item \textbf{أثر الاختبارات اللامعلمية على التقييم التربوي:} توفر الاختبارات اللامعلمية دقة أكبر في تحديد القدرات الحقيقية للطلاب، مما يسهم في تحسين استراتيجيات التعليم والتوجيه التربوي.
	
	\item \textbf{الاختبارات اللامعلمية في السياقات الثقافية المختلفة:} الاختبارات اللامعلمية تساهم في تقديم صورة أدق للقدرات الطلابية في بيئات متنوعة، مما يعزز من التنوع والشمولية في العملية التعليمية.
\end{itemize}



