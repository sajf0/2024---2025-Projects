\chapter*{توصيات}
\addcontentsline{toc}{chapter*}{توصيات}
\begin{itemize}
	\item \textbf{تحسين التدريب للمعلمين:} يُنصح بتطوير برامج تدريبية مستمرة للمعلمين حول كيفية دمج واستخدام الاختبارات اللامعلمية في التعليم، مع توفير الدعم الفني للتغلب على التحديات التقنية.
	
	\item \textbf{توسيع استخدام التكنولوجيا:} ينبغي تعزيز استخدام أدوات تكنولوجية متطورة لدعم الاختبارات اللامعلمية وتحليل النتائج بشكل أكثر فعالية، مما يعزز من دقة التقييم وسرعة الوصول إلى البيانات.
	
	\item \textbf{إجراء دراسات مستقبلية:} يُوصى بإجراء دراسات أوسع وأعمق لفهم الأبعاد النفسية والاجتماعية لتأثير الاختبارات اللامعلمية على الطلاب في سياقات تعليمية متنوعة.
	
	\item \textbf{توفير الدعم المؤسسي:} من المهم أن تقوم المؤسسات التعليمية بتوفير الدعم المادي والفني لضمان استدامة تطبيق الاختبارات اللامعلمية في مختلف المدارس والجامعات.
	
	\item \textbf{تعزيز التنوع والشمولية:} يجب العمل على تحسين تصميم الاختبارات اللامعلمية لتكون شاملة لجميع الفئات الطلابية، بما في ذلك الطلاب من خلفيات ثقافية أو اجتماعية متنوعة.
\end{itemize}