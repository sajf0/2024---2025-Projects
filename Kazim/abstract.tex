\chapter*{الملخص}
\addcontentsline{toc}{chapter*}{الملخص}

تناول هذا البحث أهم الاختبارات اللّا معلمية 
\LR{(Non-parametric Tests)}، التي تُعد من الأدوات الإحصائية الأساسية عند عدم تحقق الافتراضات المطلوبة في الاختبارات المعلمية، كعدم توفر التوزيع الطبيعي أو تجانس التباين. وقد تطرق البحث إلى تعريف الاختبارات اللا معلمية، وبيان الفرق بينها وبين الاختبارات المعلمية من حيث الشروط والمرونة في التطبيق، مع توضيح مميزاتها وعيوبها.

كما شمل البحث عرضًا لأبرز الاختبارات اللا معلمية المستخدمة في تحليل البيانات، مثل اختبار مان-ويتني 
\LR{(Mann-Whitney U Test)} للمجموعات المستقلة، واختبار ويلكوكسون \LR{(Wilcoxon Signed-Rank Test)} للبيانات المرتبطة وغيرها من الاختبارات الشائعة. وتم توضيح استخدامات كل اختبار، وخطوات تطبيقه، وكيفية تفسير نتائجه.

واختُتم البحث بالتأكيد على أهمية هذه الاختبارات في مجالات متعددة، خصوصًا في البحوث التي تتعامل مع بيانات رتبية أو ذات توزيع غير طبيعي، مما يجعلها أدوات فعالة في دعم القرارات الإحصائية عندما تكون الشروط المعلمية غير متحققة.