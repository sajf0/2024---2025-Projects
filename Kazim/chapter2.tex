\chapter{اهم الاختبارات اللا معلمية}

\section{اختبار مربع كاي (\en{Chi Square Test})}
هو اختبار احصائي لا معلمي يستخدم لتحليل البيانات الفئوية. يقوم الاختبار بمقارنة التووزيعات الملحوظة مع التوزيعات المتوقعة تحت فرضية العدم (اي عدم وجود فرق او ارتباط بين الفئات). يتم حساب القيمة باستخدام معادلة تجمع مربعات الفروق بين القيم الملحوظة والقيم المتوقعة. كل منها مقسوم على القيمة المتوقعة. يستخدم الاختبار مثلاً لاختبار ما اذا كانت النسب في جدول التكرارات تختلف عن النسب المتوقعة نظرياً.

\subsection*{خطوات الحل}
	
\begin{enumerate}
	\item \textbf{وضع الفرضيات:}
	\begin{itemize}
		\item فرضية العدم (\en{$H_0$}): لا يوجد فرق بين التوزيع الفعلي والتوزيع المتوقع.
		\item الفرضية البديلة (\en{$H_1$}): يوجد فرق كبير بين التوزيع الفعلي والتوزيع المتوقع.
	\end{itemize}
	
	\item \textbf{جمع البيانات:}
	\begin{itemize}
		\item قم بجمع البيانات المناسبة التي ستستخدمها لاختبار مربع كاي، مثل التكرار في الفئات.
	\end{itemize}
	
	\item \textbf{حساب القيم المتوقعة:}
	\begin{itemize}
		\item القيم المتوقعة هي القيم التي كنت تتوقعها بناءً على التوزيع أو فرضية العدم.
		\item لحساب القيمة المتوقعة لأي فئة:
		\[
		E = \frac{\text{عدد الحالات في الفئة} \times \text{الإجمالي}}{\text{الإجمالي الكلي}}
		\]
	\end{itemize}
	
	\item \textbf{حساب قيمة مربع كاي (\en{Chi-square statistic}):}
	\begin{itemize}
		\item احسب قيمة مربع كاي باستخدام المعادلة التالية:
		\[
		\chi^2 = \sum \frac{(O_i - E_i)^2}{E_i}
		\]
		حيث:
		\begin{itemize}
			\item \(O_i\): الترددات المُلاحَظة (\en{Observed frequencies}).
			\item \(E_i\): الترددات المتوقعة (\en{Expected frequencies}).
		\end{itemize}
	\end{itemize}
	
	\item \textbf{تحديد \en{Degrees of Freedom}:}
	\begin{itemize}
		\item يتم حساب درجات الحرية باستخدام المعادلة التالية:
		\[
		\text{الدرجات الحرة} = (\text{عدد الفئات} - 1)
		\]
	\end{itemize}
	
	\item \textbf{مقارنة القيمة المحسوبة مع القيمة الحرجة:}
	\begin{itemize}
		\item قارن القيمة المحسوبة لمربع كاي مع القيمة الحرجة من جدول مربع كاي بناءً على \en{Degrees of freedom} ومستوى الثقة.
		\item إذا كانت القيمة المحسوبة أكبر من القيمة الحرجة، يتم رفض فرضية العدم.
	\end{itemize}
	
	\item \textbf{القرار النهائي:}
	\begin{itemize}
		\item إذا كانت القيمة المحسوبة أكبر من القيمة الحرجة، نرفض فرضية العدم ونقبل الفرضية البديلة.
		\item إذا كانت القيمة المحسوبة أصغر من القيمة الحرجة، لا نرفض فرضية العدم.
	\end{itemize}
\end{enumerate}


\noindent
\textbf{مثال}\\
شركة ما تريد ان تعرف هل ان رضا العميل مرتبط بنوعية الخدمة التي يتلقونها (الكترونياً ام في الواقع). لدينا جدول البيانات التالي

\begin{table}[H]
    \centering
	\begin{tabular}{| c| c| c| c|}
		\hline
		نوع الخدمة & راضٍ & غير راضٍ & المجموع \\
		\hline 
		الكتروني & 30 & 20 & 50 \\
		\hline
		في الواقع & 40 & 10 & 50 \\
		\hline
		المجموع & 70 & 30 &100\\
		\hline
	\end{tabular}
\end{table}
نستخدم قانون مربع كاي
\[
\chi^2 = \sum_i \frac{(O_i - E_i)^2}{E_i} 
\]
نستخرج القيم المتوقعة $E_i$:
\begin{align*}
	E_1 = \frac{50\times 70}{100} = 35\\
	E_2 = \frac{50\times30}{100} = 15\\
	E_3 = \frac{50\times70}{100} = 35\\
	E_4 = \frac{50\times30}{100} = 15\\
\end{align*}
الان نحسب قيمة مربع كاي
\begin{align*}
	\chi^2 &= \sum_i \frac{(O_i - E_i)^2}{E_i} \\
	&= \frac{25}{35} + \frac{25}{15} + \frac{25}{35} + \frac{25}{15}\\
	&= 4.762
\end{align*}

الان نحدد درجة الحرية والتي تحسب من خلال القانون التالي
\[
df = (R-1)(C-1)
\]
حيث $R$ عدد الصفوف و $C$ عدد الاعمدالتالي
\[
df = (2-1)(2-1) = 1
\]
اذن $\chi^2 = 4.76$مع درجة حرية 1 ومن جدول مربع كاي فإن قيمة $p$ تكون تقريباً $0.03$ اذن $p < 0.05$ اذن نرفض فرضية العدم وبالتالي وجود ارتباط بين البيانات.
\section{اختبار جودة التوفيق (\en{Goodness of Fit Test})}
يهدف هذا الاختبار إلى التحقق مما إذا كانت التوزيعة الملحوظة لعينة من البيانات تتوافق مع توزيع نظري محدد (مثل توزيع متساوي أو توزيع طبيعي). يقوم بمقارنة الترددات الملحوظة في كل فئة مع الترددات المتوقعة بناءً على التوزيع النظري المفترض. إذا كان الفارق بين القيم الملحوظة والمتوقعة كبيراً، فإنه يشير إلى أن العينة لا تتبع التوزيع المفترض.

\begin{enumerate}
	\item \textbf{وضع الفرضيات:}
	\begin{itemize}
		\item فرضية العدم (\en{$H_0$}): التوزيع الفعلي يتوافق مع التوزيع المتوقع.
		\item الفرضية البديلة (\en{$H_1$}): التوزيع الفعلي لا يتوافق مع التوزيع المتوقع.
	\end{itemize}
	
	\item \textbf{جمع البيانات:}
	\begin{itemize}
		\item قم بجمع البيانات المناسبة التي ستستخدمها لاختبار جودة التوفيق، مثل التكرار في الفئات.
	\end{itemize}
	
	\item \textbf{حساب القيم المتوقعة:}
	\begin{itemize}
		\item القيم المتوقعة هي القيم التي كنت تتوقعها بناءً على التوزيع أو فرضية العدم.
		\item لحساب القيمة المتوقعة لأي فئة:
		\[
		E = \frac{\text{عدد الحالات في الفئة} \times \text{الإجمالي}}{\text{الإجمالي الكلي}}
		\]
	\end{itemize}
	
	\item \textbf{حساب قيمة مربع كاي (\en{Chi-square statistic}):}
	\begin{itemize}
		\item احسب قيمة مربع كاي باستخدام المعادلة التالية:
		\[
		\chi^2 = \sum \frac{(O_i - E_i)^2}{E_i}
		\]
		حيث:
		\begin{itemize}
			\item \(O_i\): الترددات المُلاحَظة (\en{Observed frequencies}).
			\item \(E_i\): الترددات المتوقعة (\en{Expected frequencies}).
		\end{itemize}
	\end{itemize}
	
	\item \textbf{تحديد \en{Degrees of Freedom}:}
	\begin{itemize}
		\item يتم حساب درجات الحرية باستخدام المعادلة التالية:
		\[
		\text{الدرجات الحرة} = (\text{عدد الفئات} - 1)
		\]
	\end{itemize}
	
	\item \textbf{مقارنة القيمة المحسوبة مع القيمة الحرجة:}
	\begin{itemize}
		\item قارن القيمة المحسوبة لمربع كاي مع القيمة الحرجة من جدول مربع كاي بناءً على \en{Degrees of freedom} ومستوى الثقة.
		\item إذا كانت القيمة المحسوبة أكبر من القيمة الحرجة، يتم رفض فرضية العدم.
	\end{itemize}
	
	\item \textbf{القرار النهائي:}
	\begin{itemize}
		\item إذا كانت القيمة المحسوبة أكبر من القيمة الحرجة، نرفض فرضية العدم ونقبل الفرضية البديلة.
		\item إذا كانت القيمة المحسوبة أصغر من القيمة الحرجة، لا نرفض فرضية العدم.
	\end{itemize}
\end{enumerate}
\noindent
\textbf{مثال}\\
تم رمي حجر نرد 60 مرة وتمت ملاحظة عدد تكرارات كل وجه و حصلنا على البيانات التالية
\begin{table}[H]
	\centering
	\begin{tabular}{|c|c|c|}
		\hline
		الوجه & عدد التكرارات ($O$) & عدد التكرارت المتوقعة ($E=60/6=10$)\\
		\hline
		1 & 8&10 \\
		2 &12&10\\
		3&14&10\\
		4&10&10\\
		5&9&10\\
		6&7&10\\
		\hline
	\end{tabular}
\end{table}
نحسب مربع كاي
\begin{align*}
	\chi^2 &= \sum_i \frac{(O_i - E_i)^2}{E_i} \\
	&= \frac{4}{10} + \frac{4}{10} + \frac{16}{10} + 0 + \frac{1}{10} + \frac{9}{10}\\
	&= 3.4
\end{align*}
حساب درجة الحرية 
\[
df = C -1 = 6-1 = 5
\]
حيث $C$ عدد الوجوه. مع هذه البيانات نحصل على قيمة $p = 0.639$ والتي تكون اكبر من $\alpha = 0.05$ وبالتالي فشلنا في رفض فرضية العدم و يظهر ان النرد يكون عادل.  

\section{اختبار الاستقلالية (\en{Independence Test})}
يستخدم هذا الاختبار لتحديد ما إذا كان هناك ارتباط بين متغيرين فئويين. يتم ذلك من خلال إعداد جدول تكراري (جدول تقاطع) يحوي التوزيعات الملحوظة لكل تركيبة من مستويات المتغيرين، ومن ثم حساب التوزيعات المتوقعة لو افترضنا استقلال المتغيرين. يُحسب الفرق بين القيم الملحوظة والمتوقعة باستخدام صيغة كاي-تربيع، ويتم استنتاج وجود علاقة أو ارتباط إذا كان الفرق كبيراً بما فيه الكفاية (أي إذا كانت قيمة p أقل من مستوى الدلالة). 

\subsection*{خطوات الحل}

\begin{enumerate}
	\item \textbf{وضع الفرضيات:}
	\begin{itemize}
		\item فرضية العدم (\en{$H_0$}): لا توجد علاقة بين المتغيرين (المتغيران مستقلان).
		\item الفرضية البديلة (\en{$H_1$}): يوجد علاقة بين المتغيرين (المتغيران غير مستقلين).
	\end{itemize}
	
	\item \textbf{جمع البيانات:}
	\begin{itemize}
		\item قم بجمع البيانات المناسبة التي ستستخدمها لاختبار الاستقلالية، وهي عادة تكون في شكل جدول تكراري.
	\end{itemize}
	
	\item \textbf{حساب القيم المتوقعة:}
	\begin{itemize}
		\item القيم المتوقعة هي القيم التي كنت تتوقعها بناءً على فرضية العدم.
		\item لحساب القيمة المتوقعة لأي خلية في الجدول:
		\[
		E = \frac{(\text{إجمالي الصف} \times \text{إجمالي العمود})}{\text{الإجمالي الكلي}}
		\]
	\end{itemize}
	
	\item \textbf{حساب قيمة مربع كاي (\en{Chi-square statistic}):}
	\begin{itemize}
		\item احسب قيمة مربع كاي باستخدام المعادلة التالية:
		\[
		\chi^2 = \sum \frac{(O_i - E_i)^2}{E_i}
		\]
		حيث:
		\begin{itemize}
			\item \(O_i\): الترددات المُلاحَظة (\en{Observed frequencies}).
			\item \(E_i\): الترددات المتوقعة (\en{Expected frequencies}).
		\end{itemize}
	\end{itemize}
	
	\item \textbf{تحديد \en{Degrees of Freedom}:}
	\begin{itemize}
		\item يتم حساب درجات الحرية باستخدام المعادلة التالية:
		\[
		\text{الدرجات الحرة} = (\text{عدد الصفوف} - 1) \times (\text{عدد الأعمدة} - 1)
		\]
	\end{itemize}
	
	\item \textbf{مقارنة القيمة المحسوبة مع القيمة الحرجة:}
	\begin{itemize}
		\item قارن القيمة المحسوبة لمربع كاي مع القيمة الحرجة من جدول مربع كاي بناءً على \en{Degrees of freedom} ومستوى الثقة.
		\item إذا كانت القيمة المحسوبة أكبر من القيمة الحرجة، يتم رفض فرضية العدم.
	\end{itemize}
	
	\item \textbf{القرار النهائي:}
	\begin{itemize}
		\item إذا كانت القيمة المحسوبة أكبر من القيمة الحرجة، نرفض فرضية العدم ونقبل الفرضية البديلة.
		\item إذا كانت القيمة المحسوبة أصغر من القيمة الحرجة، لا نرفض فرضية العدم.
	\end{itemize}
\end{enumerate}


\noindent
\textbf{مثال}\\
باحث اراد معرفة ما اذا كان التدخين مرتبط بالجنس و حصل على البيانات التالية من عينة ما\\
\begin{table}[H]
	\centering
	\begin{tabular}{|c|c|c|c|}
		\hline
		الحالة & الذكور & الاناث & المجموع\\
		\hline
		مدخن & 30 & 20 & 50\\
		\hline
		غير مدخن & 40 & 60 & 100\\
		\hline
		المجموع & 70 & 80 & 150\\
		\hline
	\end{tabular}
\end{table}
نحسب القيم المتوقعة
\begin{align*}
	E_1 = 70\times 50 / 150 = 23.33\\
	E_2 = 50\times80/ 150 = 26.67\\
	E_3 = 100\times 70/ 150 = 46.67\\
	E_4 = 100\times 80 /150 = 53.33
\end{align*}
نحسب قيمة مربع كاي 
\begin{align*}
	\chi^2 & = \sum_i \frac{(O_i - E_i)^2}{E_i} \\
	&= 1.91 + 1.67 + 0.95 + 0.83\\
	&= 6.36
\end{align*}
نحسب درجة الحرية
\[
df = (R-1)(C-1) = (2-1)(2-1) = 1
\]
حيث $C$ عدد الاعمدة و $R$ عدد الصفوف و بالتالي من جدول مربع كاي فإن $p = 0.012$ وهي اقل من $0.05$ اذن نرفض فرضية العدم وبالتالي فإن هنالك ارتباط بين نوع الجنس والتدخين. 

\section{اختبار كولموغوروف-سمرنوف لأحادية العينة \\(\en{One-Sample Kolmogorov-Smirnov Test})}
هو اختبار غير معلمي يُستخدم لمقارنة توزيع عينة مع توزيع نظري معين (مثل التوزيع الطبيعي). يقوم الاختبار بحساب دالة التوزيع التجريبية (ECDF) للعينة ومقارنتها بدالة التوزيع النظرية. يُقاس الفرق الأقصى بين هاتين الدالتين؛ وإذا كان الفرق أكبر من قيمة حرجة معينة (أو ما يقابله قيمة p أقل من مستوى الدلالة)، فإنه يتم رفض فرضية أن العينة تتبع التوزيع النظري.

\subsection*{خطوات الحل}

\begin{enumerate}
	\item \textbf{وضع الفرضيات:}
	\begin{itemize}
		\item فرضية العدم (\en{$H_0$}): البيانات تتبع التوزيع النظري المحدد.
		\item الفرضية البديلة (\en{$H_1$}): البيانات لا تتبع التوزيع النظري المحدد.
	\end{itemize}
	
	\item \textbf{جمع البيانات:}
	\begin{itemize}
		\item قم بجمع البيانات التي ستختبرها، ويجب أن تكون البيانات متوافقة مع التوزيع النظري المُفترض (مثل التوزيع الطبيعي أو أي توزيع آخر).
	\end{itemize}
	
	\item \textbf{حساب التوزيع النظري:}
	\begin{itemize}
		\item حدد التوزيع النظري الذي ترغب في مقارنته بالبيانات.
		\item احسب التوزيع التراكمي النظري (\en{Cumulative Distribution Function}) باستخدام المعادلات الرياضية أو القيم من الجدول المخصص.
	\end{itemize}
	
	\item \textbf{حساب قيمة اختبار كولموغوروف-سمرنوف:}
	\begin{itemize}
		\item احسب الفرق بين التوزيع التراكمي الفعلي والبيانات التي تم جمعها والتوزيع التراكمي النظري.
		\item قيمة اختبار كولموغوروف-سمرنوف هي:
		\[
		D = \max \left( |F_n(x) - F(x)| \right)
		\]
		حيث:
		\begin{itemize}
			\item \( F_n(x) \): التوزيع التراكمي الفعلي للبيانات.
			\item \( F(x) \): التوزيع التراكمي النظري.
		\end{itemize}
	\end{itemize}
	
	\item \textbf{تحديد القيمة الحرجة:}
	\begin{itemize}
		\item حدد القيمة الحرجة بناءً على مستوى الثقة وعدد العينة. القيمة الحرجة يتم تحديدها من الجداول الخاصة لاختبار كولموغوروف-سمرنوف.
	\end{itemize}
	
	\item \textbf{مقارنة القيمة المحسوبة مع القيمة الحرجة:}
	\begin{itemize}
		\item إذا كانت القيمة المحسوبة \( D \) أكبر من القيمة الحرجة، يتم رفض فرضية العدم.
		\item إذا كانت القيمة المحسوبة أصغر أو تساوي القيمة الحرجة، لا نرفض فرضية العدم.
	\end{itemize}
	
	\item \textbf{القرار النهائي:}
	\begin{itemize}
		\item إذا كانت القيمة المحسوبة أكبر من القيمة الحرجة، نرفض فرضية العدم ونقبل الفرضية البديلة.
		\item إذا كانت القيمة المحسوبة أصغر أو تساوي القيمة الحرجة، لا نرفض فرضية العدم.
	\end{itemize}
\end{enumerate}

\noindent
\textbf{مثال}\\
مدرس جمع نتائج 10 طلاب في مادة ما وحصل على البيانات التالية

\begin{table}[H]
	\centering
	\begin{tabular}{|c|c|c|c|c|c|c|c|c|c|c|}
		\hline
		45&50&52&55 &60&62&65 &70&72&75\\
		\hline
	\end{tabular}
\end{table}
نحسب الدالة التجميعية التجريبية
\[
F_n(x_i) = \frac{i}{n} , \quad i = 1,2,\dots,10
\]
الان نقدر المعالم الخاصة بالتوزيع الطبيعي\\
نحسب متوسط العينة ($\bar{x}$) و الانحدار المعياري ($S$)
\[
\bar{x} = \sum_{i=1}^{10} \frac{x_i}{10} = 60.6 
\]
\[
S = \SQRT{\frac{1}{n-1} \sum_{i=1}^{10} (x_i - \bar{x})^2} = 10.05
\]
الان لكل $x_i$ نحسب $z_i$ من خلال القانون
\[
z_i = \frac{x_i - \bar{x}}{S}
\]
\section{اختبار مان-ويتني (\en{Mann-Whitney U Test})}
هو اختبار غير معلمي يُستخدم لمقارنة عينتين مستقلتين عندما لا يكون بالإمكان افتراض التوزيع الطبيعي للبيانات. بدلاً من مقارنة المتوسطات كما في اختبار t، يقوم الاختبار بترتيب جميع القيم معاً ومن ثم يقارن مجموع الرتب لكل مجموعة. إذا كانت الرتب موزعة بشكل غير متساوٍ، فهذا يشير إلى وجود اختلاف في التوزيعات بين العينتين. يُستخدم هذا الاختبار كبديل لاختبار t في الحالات التي لا تحقق فيها البيانات افتراضات الاختبارات المعلمية.

\subsection*{خطوات الحل}

\begin{enumerate}
	\item \textbf{وضع الفرضيات:}
	\begin{itemize}
		\item فرضية العدم (\en{$H_0$}): لا يوجد فرق بين المجموعتين (أي أن التوزيع في المجموعتين متساوي).
		\item الفرضية البديلة (\en{$H_1$}): يوجد فرق بين المجموعتين (أي أن التوزيع في المجموعتين غير متساوي).
	\end{itemize}
	
	\item \textbf{جمع البيانات:}
	\begin{itemize}
		\item قم بجمع البيانات الخاصة بالمجموعتين اللتين ترغب في مقارنة توزيعهما.
	\end{itemize}
	
	\item \textbf{ترتيب القيم:}
	\begin{itemize}
		\item قم بترتيب القيم في المجموعتين بشكل مشترك في ترتيب تصاعدي.
		\item يتم إعطاء ترتيب لكل قيمة من القيم في المجموعتين.
		\item إذا كانت هناك قيم مكررة، يتم إعطاء ترتيب مشترك لها.
	\end{itemize}
	
	\item \textbf{حساب إحصائية مان ويتني (\en{U statistic}):}
	\begin{itemize}
		\item احسب إحصائية مان ويتني باستخدام المعادلة التالية:
		\[
		U_1 = n_1 n_2 + \frac{n_1(n_1 + 1)}{2} - R_1
		\]
		حيث:
		\begin{itemize}
			\item \( n_1 \): عدد العينات في المجموعة الأولى.
			\item \( n_2 \): عدد العينات في المجموعة الثانية.
			\item \( R_1 \): مجموع التراكيب المرتبطة بالمجموعة الأولى.
		\end{itemize}
		\item كذلك يمكن حساب \( U_2 \) باستخدام:
		\[
		U_2 = n_1 n_2 - U_1
		\]
	\end{itemize}
	
	\item \textbf{مقارنة مع القيمة الحرجة:}
	\begin{itemize}
		\item حدد القيمة الحرجة لاختبار مان ويتني بناءً على \en{n1} و \en{n2} ومستوى الثقة.
		\item قارن القيمة المحسوبة لإحصائية \( U \) مع القيمة الحرجة.
		\item إذا كانت القيمة المحسوبة \( U \) أصغر من القيمة الحرجة، يتم رفض فرضية العدم.
	\end{itemize}
	
	\item \textbf{القرار النهائي:}
	\begin{itemize}
		\item إذا كانت القيمة المحسوبة لإحصائية مان ويتني \( U \) أصغر من القيمة الحرجة، نرفض فرضية العدم.
		\item إذا كانت القيمة المحسوبة أكبر من أو تساوي القيمة الحرجة، لا نرفض فرضية العدم.
	\end{itemize}
\end{enumerate}

\noindent
\textbf{مثال}\\
باحث اراد ان يقارن بين نتائج 6 طلاب مستخدمين طريقتين للحل. فحصل على البيانات التالية

\begin{table}[H]
	\centering
	\begin{tabular}{|c|c|}
		\hline
		الطريقة A $n=6$ & الطريقة B $n=6$\\
		\hline
		85 & 75 \\
		90 & 80\\
		78 & 70\\
		88 & 82\\
		84 & 79\\
		91 & 77\\
		\hline
	\end{tabular}
\end{table}
نصنف كل النتائج الـ12 تصاعدياً مع اعطاء تصنيف كالآتي
\begin{table}[H]
	\centering
	\begin{tabular}{|c|c|c|}
		\hline
		النتيجة & الطريقة & التصنيف \\
		\hline
		70 & B&1\\
		75&B &2\\
		77&B&3\\
		78&A&4\\
		79&B&5\\
		80&B&6\\
		82&B&7\\
		84&A&8\\
		85&A&9\\
		88&A&10\\
		90&A&11\\
		91&A&12\\
		\hline
	\end{tabular}
\end{table}
الان نحسب مجموع التصنيفات لكل طريقة
\[
R_A = 4+8+9+10+11+12 = 54
\]
\[
R_B = 1+2+3+5+6+7 = 24
\]
نحسب قيمة الاحصاءة $U$ من خلال القانون
\[
U_A = n_1 n_2 + \frac{n_1(n_2+1)}{2} - R_A = 3
\]
\[
U_B = n_1 n_2 - U_A = 36 - 3 = 33
\]
الان نحسب المعدل للاحصاءة $U$
\[
\mu_U = \frac{n_1n_2}{2} = \frac{36}{2} = 18
\]
الانحدار المعياري
\[
\sigma_U = \sqrt{\frac{n_1n_2(n_1+n_2+1)}{12}} = 6.245
\]
نحسب الاحصاءة $z$
\[
z = \frac{U-\mu_U + 0.5}{\sigma_U} = -2.32
\]
من جدول $z$ نحصل على قيمة $p = 0.02$ وبالتالي اقل من $0.05$. اي نرفض فرضية العدم وبالتالي يوجد فرق بين الطريقتين.

\section{اختبار ولكوكسون \en{Wilcoxon Test}}

اختبار ولكوكسون هو اختبار غير معلمي يُستخدم لمقارنة المجموعات المترابطة أو العينة الواحدة عندما لا يمكن افتراض أن البيانات تتبع التوزيع الطبيعي. يمكن استخدام اختبار ولكوكسون في حالتين:

\begin{itemize}
	\item \textbf{اختبار ولكوكسون للعينات المترابطة} (\en{Wilcoxon Signed-Rank Test}): يستخدم للمقارنة بين قياسات قبل وبعد العلاج على نفس الأفراد.
	\item \textbf{اختبار ولكوكسون للعينات المستقلة} (\en{Wilcoxon Rank-Sum Test}): يُستخدم للمقارنة بين مجموعتين مستقلتين.
\end{itemize}

\subsection*{خطوات الحل}

\begin{enumerate}
	\item \textbf{وضع الفرضيات:}
	\begin{itemize}
		\item فرضية العدم (\en{$H_0$}): لا يوجد فرق بين المجموعتين (أو لا يوجد فرق بين القياسات قبل وبعد).
		\item الفرضية البديلة (\en{$H_1$}): يوجد فرق بين المجموعتين (أو يوجد فرق بين القياسات قبل وبعد).
	\end{itemize}
	
	\item \textbf{جمع البيانات:}
	\begin{itemize}
		\item قم بجمع البيانات المناسبة للمقارنة بين المجموعات أو القياسات قبل وبعد.
		\item يجب أن تكون البيانات مكونة من أزواج مترابطة من القياسات.
	\end{itemize}
	
	\item \textbf{حساب الفروق:}
	\begin{itemize}
		\item احسب الفرق بين القيم المتزاوجة (القياسات قبل وبعد، أو المجموعات المترابطة).
		\item إذا كان الفرق سلبيًا، تجاهل الإشارة واعتبر القيمة المطلقة.
	\end{itemize}
	
	\item \textbf{ترتيب القيم:}
	\begin{itemize}
		\item رتب الفروق حسب الحجم، وأعطِ كل فرق ترتيبًا.
		\item إذا كانت هناك فروق مكررة، قم بإعطاء ترتيب مشترك لها.
	\end{itemize}
	
	\item \textbf{حساب إحصائية ولكوكسون:}
	\begin{itemize}
		\item قم بحساب مجموع التراكيب المرتبطة بالإشارات الموجبة والسالبة.
		\item قم بحساب إحصائية ولكوكسون (\en{T}) كما يلي:
		\[
		T = \min(T_+ , T_-)
		\]
		حيث:
		\begin{itemize}
			\item \( T_+ \) هو مجموع التراكيب المرتبطة بالقيم الموجبة.
			\item \( T_- \) هو مجموع التراكيب المرتبطة بالقيم السالبة.
		\end{itemize}
	\end{itemize}
	
	\item \textbf{تحديد القيمة الحرجة:}
	\begin{itemize}
		\item حدد القيمة الحرجة لاختبار ولكوكسون بناءً على مستوى الثقة وعدد الأفراد (العينات) في الدراسة.
		\item يمكن العثور على القيمة الحرجة من الجداول الخاصة باختبار ولكوكسون.
	\end{itemize}
	
	\item \textbf{مقارنة مع القيمة الحرجة:}
	\begin{itemize}
		\item قارن القيمة المحسوبة \( T \) مع القيمة الحرجة من جدول ولكوكسون.
		\item إذا كانت القيمة المحسوبة \( T \) أصغر من القيمة الحرجة، يتم رفض فرضية العدم.
	\end{itemize}
	
	\item \textbf{القرار النهائي:}
	\begin{itemize}
		\item إذا كانت القيمة المحسوبة \( T \) أصغر من القيمة الحرجة، نرفض فرضية العدم.
		\item إذا كانت القيمة المحسوبة أكبر من أو تساوي القيمة الحرجة، لا نرفض فرضية العدم.
	\end{itemize}
\end{enumerate}

\noindent
\textbf{مثال}\\
\noindent
افترض أن هناك 6 طلاب قاموا بأداء اختبارين، أحدهما قبل التدريب والآخر بعد التدريب. نقارن النتائج لمعرفة ما إذا كان هناك فرق ذو دلالة بين الاختبارين. البيانات هي كما يلي:

\[
\begin{array}{|c|c|c|c|}
	\hline
	\text{الطالب} & \text{قبل التدريب} & \text{بعد التدريب} & \text{الفرق (قبل - بعد)} \\
	\hline
	1 & 75 & 85 & -10 \\
	2 & 82 & 88 & -6 \\
	3 & 68 & 70 & -2 \\
	4 & 91 & 87 & 4 \\
	5 & 78 & 80 & -2 \\
	6 & 84 & 90 & -6 \\
	\hline
\end{array}
\]

\subsection*{الخطوات:}

\begin{enumerate}
	\item \textbf{وضع الفرضيات:}
	\begin{itemize}
		\item فرضية العدم (\en{$H_0$}): لا يوجد فرق بين الاختبارات (أي لا يوجد تأثير للتدريب).
		\item الفرضية البديلة (\en{$H_1$}): يوجد فرق بين الاختبارات (أي أن التدريب أثر على الدرجات).
	\end{itemize}
	
	\item \textbf{حساب الفروق:}
	نقوم بحساب الفرق بين الدرجات قبل وبعد التدريب:
	\begin{itemize}
		\item الفروق هي: \(-10, -6, -2, 4, -2, -6\).
	\end{itemize}
	
	\item \textbf{ترتيب القيم:}
	نرتب القيم وفقًا للمطلقات:
	\begin{itemize}
		\item الفروق المطلقة: \(10, 6, 2, 4, 2, 6\).
		\item الترتيب: \[ 2, 2, 4, 6, 6, 10 \] (ترتيب القيم من الأصغر إلى الأكبر).
		\item يجب إعطاء الترتيب للقيم. إذا كانت هناك قيم مكررة، نأخذ ترتيبًا مشتركًا (ولكن هنا لا توجد قيم مكررة).
	\end{itemize}
	
	\item \textbf{تحديد الإشارات (الموجبة والسالبة):}
	\begin{itemize}
		\item الإشارات هي: سالب (-) للفرق \(-10, -6, -2, -2, -6\) وموجب (+) للفرق \( 4 \).
		\item الآن، نضيف الترتيب لكل قيمة مع الإشارة:
		\begin{itemize}
			\item الفرق \(-10\): الترتيب 6 مع الإشارة السالبة.
			\item الفرق \(-6\): الترتيب 5 مع الإشارة السالبة.
			\item الفرق \(-2\): الترتيب 2 مع الإشارة السالبة.
			\item الفرق \(4\): الترتيب 3 مع الإشارة الموجبة.
			\item الفرق \(-2\): الترتيب 2 مع الإشارة السالبة.
			\item الفرق \(-6\): الترتيب 5 مع الإشارة السالبة.
		\end{itemize}
	\end{itemize}
	
	\item \textbf{حساب إحصائية ولكوكسون (\(T\)):}
	\begin{itemize}
		\item الإحصائية الموجبة (\(T_+\)): مجموع التراكيب الموجبة = \( 3 \).
		\item الإحصائية السالبة (\(T_-\)): مجموع التراكيب السالبة = \( 6 + 5 + 2 + 2 + 5 = 20 \).
		\item نحسب إحصائية ولكوكسون:
		\[
		T = \min(T_+, T_-) = \min(3, 20) = 3.
		\]
	\end{itemize}
	
	\item \textbf{تحديد القيمة الحرجة:}
	\begin{itemize}
		\item بناءً على عدد العينة (6 طلاب) ومستوى الثقة (مثل $0.05$)، نستخدم جدول ولكوكسون للبحث عن القيمة الحرجة لـ \( n = 6 \).
		\item القيمة الحرجة لـ \( T = 3 \) هي 2.
	\end{itemize}
	
	\item \textbf{مقارنة مع القيمة الحرجة:}
	\begin{itemize}
		\item إذا كانت \( T \) أصغر من أو تساوي القيمة الحرجة (التي هي 2)، نرفض فرضية العدم.
		\item في هذا المثال، \( T = 3 \)، وهي أكبر من القيمة الحرجة 2.
	\end{itemize}
	
	\item \textbf{القرار النهائي:}
	\begin{itemize}
		\item نظرًا لأن \( T = 3 \) أكبر من القيمة الحرجة (2)، لا نرفض فرضية العدم. وهذا يعني أنه لا يوجد دليل كافٍ على أن التدريب أثر على الدرجات بشكل كبير.
	\end{itemize}
\end{enumerate}

\textbf{النتيجة:} بناءً على الاختبار، لا يوجد فرق ذو دلالة بين الاختبارات قبل وبعد التدريب (أي أن فرضية العدم لا تُرفض).
	
	\section{اختبار عينتين مستقلتين \LR{(Independent Two-Sample Test)}}
	\begin{itemize}
		\item \textbf{الوصف}: يُستخدم هذا الاختبار عندما يتم جمع عينتين من مجموعات مستقلة عن بعضها البعض، ويهدف إلى المقارنة بين المتوسطات أو الفروق بين العينتين.
		\item \textbf{مثال}: 
		\begin{itemize}
			\item مقارنة متوسط درجات طلاب قسمين مختلفين في نفس المادة.
			\item مقارنة نتائج مبيعات منتجين مختلفين في فترتين مختلفتين.
		\end{itemize}
		\item \textbf{الفكرة الأساسية}: هنا الفرضية الأساسية هي أن المجموعتين مستقلتين، أي أنه لا يوجد ارتباط بين العناصر في المجموعة الأولى والعناصر في المجموعة الثانية.
		\item \textbf{الافتراضات}:
		\begin{itemize}
			\item البيانات موزعة بشكل طبيعي.
			\item التباين في المجموعتين متساوي (أو يمكن تعديل ذلك باستخدام اختبار لتفاوت التباين).
			\item عينات مستقلة.
		\end{itemize}
		\item \textbf{الاختبار الإحصائي}: يُستخدم اختبار \textit{t} لعينتين مستقلتين، وفي حالة عدم التوزيع الطبيعي يمكن استخدام اختبارات غير معلمية مثل اختبار \textit{Mann-Whitney U}.
	\end{itemize}
	
	\section{اختبار عينتين غير مستقلتين \LR{(Paired Two-Sample Test)}}
	\begin{itemize}
		\item \textbf{الوصف}: يُستخدم عندما تكون العينات مترابطة أو مرتبطة ببعضها البعض. يتم جمع البيانات في شكل أزواج، حيث يتم قياس نفس الأفراد في نقطتين مختلفتين أو تحت حالتين مختلفتين.
		\item \textbf{مثال}:
		\begin{itemize}
			\item مقارنة نتائج نفس الطلاب قبل وبعد اختبار أو تدريب.
			\item قياس ضغط الدم قبل وبعد تناول دواء معين لدى مجموعة من المرضى.
		\end{itemize}
		\item \textbf{الفكرة الأساسية}: الفرضية الأساسية هي أن البيانات ترتبط ببعضها البعض، أي أن كل عنصر في المجموعة الأولى له علاقة مباشرة مع عنصر في المجموعة الثانية.
		\item \textbf{الافتراضات}:
		\begin{itemize}
			\item البيانات تمثل قياسات تكرارية لنفس المجموعة.
			\item البيانات تتبع التوزيع الطبيعي.
			\item الفروق بين الأزواج متوزعة بشكل طبيعي.
		\end{itemize}
		\item \textbf{الاختبار الإحصائي}: يُستخدم اختبار \textit{t} لعينتين غير مستقلتين أو اختبار \textit{Wilcoxon Signed Rank} إذا كانت البيانات غير متوزعة بشكل طبيعي.
	\end{itemize}
	
	\section{اختبار أكثر من عينتين مستقلتين \LR{(One-Way ANOVA)}}
	\begin{itemize}
		\item \textbf{الوصف}: يُستخدم لاختبار الفرق في المتوسطات بين أكثر من عينتين (ثلاثة عينات أو أكثر) تكون مستقلة عن بعضها البعض. هذا الاختبار يهدف إلى معرفة ما إذا كانت هناك فروق ذات دلالة إحصائية بين مجموعات متعددة.
		\item \textbf{مثال}:
		\begin{itemize}
			\item مقارنة درجات اختبار بين ثلاث مجموعات من طلاب المدارس (مدرسة أ، مدرسة ب، مدرسة ج).
			\item مقارنة فعالية ثلاثة أنواع من الأدوية في علاج نفس المرض.
		\end{itemize}
		\item \textbf{الفكرة الأساسية}: الهدف هو اختبار ما إذا كانت الفروق بين المجموعات أكبر من الفروق داخل المجموعات. إذا كانت الفروق بين المجموعات أكبر بشكل كبير من الفروق داخل المجموعات، فهذا يشير إلى أن هناك فرقًا ذا دلالة إحصائية بين المتوسطات.
		\item \textbf{الافتراضات}:
		\begin{itemize}
			\item توزيع البيانات في كل مجموعة يجب أن يكون طبيعيًا.
			\item التباين بين المجموعات يجب أن يكون متساويًا (تجانس التباين).
			\item المجموعات مستقلة عن بعضها.
		\end{itemize}
		\item \textbf{الاختبار الإحصائي}: يتم استخدام \textit{ANOVA} أحادي الاتجاه \LR{(One-Way ANOVA)}، وإذا تبين وجود فروق معنوية، يمكن استخدام اختبار \textit{Post-hoc} (مثل اختبار \textit{Tukey}) لتحديد أي المجموعات المختلفة بشكل واضح.
	\end{itemize}
	
	\section{اختبار أكثر من عينتين غير مستقلين \LR{(Repeated Measures ANOVA)}}
	\begin{itemize}
		\item \textbf{الوصف}: يُستخدم هذا الاختبار عندما يكون لديك أكثر من عينتين مترابطتين. يُستخدم عادة عندما يتم قياس نفس العينة في أكثر من وقت أو تحت أكثر من حالة.
		\item \textbf{مثال}:
		\begin{itemize}
			\item قياس مستويات الضغط الدموي لنفس المجموعة من المرضى عدة مرات على مدار أشهر.
			\item متابعة تقدم الطلاب في اختبار دوري متكرر خلال فصل دراسي.
		\end{itemize}
		\item \textbf{الفكرة الأساسية}: في هذا النوع من الاختبارات، لا تكون العينات مستقلة، بل يتم قياس نفس العينة في حالات أو أوقات مختلفة. وهذا يسمح بالمقارنة بين الحالات أو الأوقات المختلفة لمعرفة ما إذا كان هناك تأثير أو تغير.
		\item \textbf{الافتراضات}:
		\begin{itemize}
			\item البيانات يجب أن تكون متوزعة بشكل طبيعي.
			\item التباين داخل الأفراد يجب أن يكون متساويًا.
			\item التغييرات أو التفاعلات بين القياسات المتكررة يجب أن تكون موجودة.
		\end{itemize}
		\item \textbf{الاختبار الإحصائي}: يُستخدم \textit{\LR{Repeated Measures ANOVA}}، وفي حالة عدم التوزيع الطبيعي يمكن اللجوء إلى اختبار \textit{Friedman} غير المعلمي.
	\end{itemize}

	




