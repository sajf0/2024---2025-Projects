\chapter{انواع فضاء المتجهات}

\section[متجه الفضاء الاقليدي $\R^n$]{متجه الفضاء الاقليدي $\R^n$ \cite{key3}}
مجموعة العناصر على الشكل
\[
\R^n = \{(x_1, x_2, \dots, x_n) : x_i \in \R\}
\]
حيث تعرف عمليتي الجمع والضرب كالآتي
\[
(x_1, x_2, \dots, x_n) + (y_1, y_2, \dots, y_n)  = (x_1+x_1, x_2+y_2, \dots, x_n+y_n) 
\]  
\[
\alpha (x_1, x_2, \dots, x_n)  = (\alpha x_1,\alpha x_2, \dots, \alpha x_n) 
\]



\subsection*{اثبات ان $\R^n$ فضاء متجهات}
حسب التعريف 2 - 1 لكي يكون فضاء متجهات يجب ان يحقق ما يلي\\

\setLR
\noindent
(1 $
\begin{aligned}[t]
	x + y &= (x_1, x_2, \dots, x_n) + (y_1, y_2, \dots, x_n)\\
	&= (x_1+y_1, x_2+y_2, \dots, x_n + y_n)\\
	&= (y_1, y_2, \dots, y_n) + (x_1, x_2, \dots, x_n) = y + x
\end{aligned}
$\\ \\
(2 $
\begin{aligned}[t]
	x + (y + z) &= (x_1, x_2, \dots, x_n) + [(y_1, y_2, \dots, y_n) + (z_1, z_2, \dots, z_n)]\\
	&= (x_1, x_2, \dots, x_n) + (y_1+z_1, y_2+z_2, \dots, y_n+z_n)\\
	&=  (x_1+y_1+z_1, x_2+y_2+z_2, \dots, x_n+y_n+z_n)\\
	&=  (x_1+y_1, x_2+y_2, \dots, x_n+y_n) + (z_1+z_2,\dots,z_n)\\
	&= (x+y) + z
\end{aligned}
$\\

\selectlanguage{arabic}
\noindent
يوجد عنصر محايد وهو الصفر $0 = (0, 0, \dots, 0)$ بحيث

\selectlanguage{english}
\noindent
3) $
\begin{aligned}[t]
	x + 0 &= (x_1,x_2,\dots,x_n) + (0, 0, \dots, 0)\\
	&= x
\end{aligned}
$\\

\selectlanguage{arabic}
\noindent
لكل متجه $x = (x_1, x_2, \dots, x_n) \in \R^n$ نظيره $-x = (-x_1, -x_2, \dots, -x_n) \in \R^n$

\selectlanguage{english}
\noindent
4) $x + (-x) = (x_1, x_2, \dots, x_n)= (0, 0, \dots, 0) = 0$ \\ 
\\
5) $
\begin{aligned}[t]
	\forall x = (x_1, x_2, \dots, x_n) \in \R^n: 1\cdot x
	&= 1 \cdot (x_1, x_2, \dots, x_n)\\
	&= (x_1, x_2, \dots, x_n) \\
	&= x
\end{aligned}
$\\
\noindent
6) $
\begin{aligned}[t]
	\forall \alpha\in\R , \forall x,y \in \R^n: \alpha (x+y)
	&= \alpha(x_1+y_1, x_2+y_2, \dots, x_n+y_n)\\
	&= (\alpha(x_1+y_1), \alpha(x_2+y_2), \dots, \alpha(x_n+y_n))\\
	&= (\alpha x_1+\alpha y_1, \alpha x_2+\alpha y_2, \dots, \alpha x_n+\alpha y_n)\\
	&= \alpha x + \alpha y
\end{aligned}
$\\\\
\noindent
7) $
\begin{aligned}[t]
	\forall \alpha, \beta \in \R, \forall x\in \R^n: (\alpha + \beta)\cdot x
	&= (\alpha + \beta)\cdot (x_1, x_2, \dots, x_n)\\
	&= ((\alpha+\beta)x_1, (\alpha+\beta)x_2, \dots, (\alpha+\beta)x_n)\\
	&= (\alpha x_1 + \beta x_1, \alpha x_2 + \beta x_2, \dots , \alpha x_n + \beta x_n)\\
	&= \alpha x + \beta x
\end{aligned}
$\\
\\
8) $
\begin{aligned}[t]
	\forall \alpha, \beta \in \R, \forall x\in \R^n: (\alpha \beta)\cdot x
	&= (\alpha\beta)(x_1, x_2, \dots, x_n)\\
	&= (\alpha\beta x_1, \alpha\beta x_2, \dots, \alpha\beta x_n)\\
	&= (\alpha(\beta x_1), \alpha(\beta x_2), \dots, \alpha(\beta x_n))\\
	&= \alpha \cdot (\beta x)
\end{aligned}
$
\selectlanguage{arabic}
\begin{example}
	لتكن $W = \{ (x, y, z) \in \mathbb{R}^3 \mid x + y + z = 0 \}$. أثبت أن $W$ فضاء جزئي من $\mathbb{R}^3$.
\end{example}

\begin{solution}
	للتحقق من أن $W$ فضاء جزئي من $\mathbb{R}^3$، نتحقق من الشرطين:\\
	\noindent
	أولاً، إذا كانت $\vec{u} = (x_1, y_1, z_1) \in W$ و$\vec{v} = (x_2, y_2, z_2) \in W$، فإن:
	\[
	x_1 + y_1 + z_1 = 0 \quad \text{و} \quad x_2 + y_2 + z_2 = 0
	\]
	وبالتالي:
	\[
	\vec{u} + \vec{v} = (x_1 + x_2, y_1 + y_2, z_1 + z_2)
	\]
	ومجموع مركباته:
	\[
	(x_1 + x_2) + (y_1 + y_2) + (z_1 + z_2) = (x_1 + y_1 + z_1) + (x_2 + y_2 + z_2) = 0 + 0 = 0
	\]
	أي أن $\vec{u} + \vec{v} \in W$.\\
	\noindent
	ثانياً، إذا كانت $\vec{u} = (x, y, z) \in W$ و$c \in \mathbb{R}$، فإن:
	\[
	c\vec{u} = (cx, cy, cz) \quad \text{و} \quad cx + cy + cz = c(x + y + z) = c \cdot 0 = 0
	\]
	إذًا $c\vec{u} \in W$.
	
	بما أن الشروط الثلاثة محققة، فإن $W$ فضاء جزئي من $\mathbb{R}^3$.
\end{solution}


\section[فضاء المصفوفات $M_{m\times n}(\R)$]{فضاء المصفوفات $M_{m\times n}(\R)$ \cite{key2}}
تعرف $M_{m\times n}(\R)$ على انها مجموعة كل المصفوفات ذات البعد $m \times n$ على حقل الاعداد الحقيقية $\R$. 
\textbf{الجمع:} لتكن $A=[a_{ij}]$ و $B=[b_{ij}]$ مصفوفات من $M_{m\times n}(\R)$ فإن 
\[
A + B = [a_{ij} + b_{ij}] , 1\leq i\leq m, 1\leq j\leq n  
\]
\textbf{الضرب:} ليكن $c\in \R$ و $A = [a_{ij}]\in M_{m\times n}(\R)$ فإن 
\[
cA = [c\, a_{ij}] , 1\leq i\leq m, 1\leq j\leq n
\]


\subsection*{اثبات ان $M_{n\times m}(\R)$ فضاء متجهات}
حسب التعريف 2 - 1 لكي يكون فضاء متجهات يجب ان يحقق ما يلي\\

\selectlanguage{english}
\noindent
1. $
\begin{aligned}[t]
	A + B &= [a_{ij}] + [b_{ij}]\\
	&= [a_{ij} + b_{ij}]\\
	&= [b_{ij} + a_{ij}]\\
	&= B+A
\end{aligned}
$\\
2. $
\begin{aligned}[t]
	A + (B+C) &= [a_{ij}] + ([b_{ij}] + [c_{ij}])\\
	&= [a_{ij}] + [b_{ij} + c_{ij}]\\
	&= [a_{ij} + b_{ij} + c_{ij}]\\
	&= [a_{ij} + b_{ij}] + [c_{ij}]\\
	&= (A+B)+C
\end{aligned}
$

\selectlanguage{arabic}
\noindent
المصفوفة الصفرية $0=[0]$ تمثل العنصر المحايد

\selectlanguage{english}
\noindent
3. $A+0 = [a_{ij}] + [0] = A$

\selectlanguage{arabic}
\noindent
لكل $A=[a_{ij}]$ نظيرها المصفوفة $-A=[-a_{ij}]$ بحيث 

\selectlanguage{english}
\noindent
4. $A + (-A) = [a_{ij}] + [-a_{ij}] = [0] =0$  \\
5. $
\begin{aligned}[t]
	\forall A \in M_{n\times m}(\R) : 1\cdot A
	&= 1\cdot [a_{ij}]\\
	&= [a_{ij}] =A
\end{aligned}
$\\
6. $
\begin{aligned}[t]
	\forall\alpha\in\R, \forall A, B\in M_{n\times m}(\R) : \alpha(A+B)
	&= \alpha [a_{ij}+b_{ij}]\\
	&= [\alpha (a_{ij} + b_{ij})]\\
	&= [\alpha a_{ij} + \alpha b_{ij}]\\
	&= [\alpha a_{ij} ] + [\alpha b_{ij}]\\
	&= \alpha A + \alpha B
\end{aligned}
$\\
7. $
\begin{aligned}[t]
	\forall \alpha, \beta \in \R, \forall A\in M_{n\times m}(\R) : (\alpha+\beta)A
	&= (\alpha+\beta) [a_{ij}]\\
	&= [(\alpha+\beta)a_{ij}]\\
	&= [\alpha a_{ij} + \beta a_{ij}]\\
	&= \alpha A + \beta A
\end{aligned}
$\\
8. $
\begin{aligned}[t]
	\forall \alpha, \beta \in \R, \forall A\in M_{n\times m}(\R) : (\alpha\beta)A
	&= (\alpha\beta) [a_{ij}]\\
	&= [(\alpha\beta)a_{ij}]\\
	&= [\alpha(\beta a_{ij})]\\
	&= \alpha (\beta A)
\end{aligned}
$
\selectlanguage{arabic}
\begin{example}
	لتكن $W = \left\{ \begin{bmatrix} a & b \\ 0 & 0 \end{bmatrix} \mid a, b \in \mathbb{R} \right\}$. أثبت أن $W$ فضاء جزئي من $M_2(\mathbb{R})$.
\end{example}

\begin{solution}
	للتحقق من أن $W$ فضاء جزئي من $M_2(\mathbb{R})$، نتحقق من الشرطين:\\
	\noindent
	أولاً، إذا كانت 
	\[
	A = \begin{bmatrix} a_1 & b_1 \\ 0 & 0 \end{bmatrix}, \quad B = \begin{bmatrix} a_2 & b_2 \\ 0 & 0 \end{bmatrix} \in W
	\]
	فإن:
	\[
	A + B = \begin{bmatrix} a_1 + a_2 & b_1 + b_2 \\ 0 & 0 \end{bmatrix} \in W
	\]
	لأن مجموع الصف السفلي يبقى صفراً.
	
	\noindent
	ثانياً، إذا كانت 
	\[
	A = \begin{bmatrix} a & b \\ 0 & 0 \end{bmatrix} \in W \quad \text{و} \quad c \in \mathbb{R}
	\]
	فإن:
	\[
	cA = \begin{bmatrix} ca & cb \\ 0 & 0 \end{bmatrix} \in W
	\]
	
	بما أن الشرطين محققان، فإن $W$ فضاء جزئي من $M_2(\mathbb{R})$.
\end{solution}

\section[فضاء كثيرات الحدود $P_n$]{فضاء كثيرات الحدود $P_n$ \cite{key4}}
مجموعة كثيرات الحدود من الدرجة $n$ او اقل تعرف كالآتي
\[
P_n = \{p(x) = a_0 + a_1 x + \cdots + a_n x^n\}
\]
\textbf{الجمع:} لكل $p(x), q(x) \in P_n$ لدينا 
$
(p+q)(x) = p(x) + q(x)
$\\
\textbf{الضرب:} لكل $\alpha \in \R$ ولكل $p(x)\in P_n$ لدينا
$
(\alpha p)(x) = \alpha p(x)
$

\subsection*{اثبات ان $P_n$ فضاء متجهات}
حسب التعريف 2 - 1 لكي يكون فضاء متجهات يجب ان يحقق ما يلي\\

\setLR
\noindent
.1 $(p+q)(x) = p(x) + q(x) = q(x) + p(x) = (q+p)(x)$\\
.2 $
\begin{aligned}[t]
	[p+(q+r)](x) &= p(x) + (q+r)(x)\\
	&= p(x) + q(x) + r(x) \\
	&= (p+q) + r(x)\\
	&= [(p+q)+r](x)
\end{aligned}
$\\


\setRL
\noindent
كثيرة الحدود الصفرية $0(x)=0 $ تمثل العنصر المحايد 

\setLR

\noindent
.3 $(p+0)(x) = p(x) + 0(x) = p(x) + 0 = p(x)$

\setRL
\noindent
لكل $p(x)\in P_n$ نظيرها كثيرة الحدود $(-p)(x) = -p(x)$ 

\setLR
\noindent
.4 $(p+(-p))(x) = p(x) + (-p)(x) = p(x) - p(x) = 0$\\
.5 $\forall p(x) \in P_n : (1\cdot p)(x) = 1\cdot p(x) = p(x)$\\
.6 $
\begin{aligned}[t]
	\forall\alpha\in\R, \forall p,q\in P_n : (\alpha(p+q))(x)
	&= \alpha (p+q)(x)\\
	&= \alpha[p(x) + q(x)]\\
	&= \alpha p(x) + \alpha q(x)\\
	&= (\alpha p)(x) + (\alpha q)(x)\\
	&= [\alpha p + \alpha q](x)
\end{aligned}
$\\
\\
.7 $
\begin{aligned}[t]
	\forall\alpha,\beta\in\R, \forall p\in P_n : [(\alpha + \beta)p](x)
	&= (\alpha+\beta)p(x)\\
	&= \alpha p(x) + \beta p(x)\\
	&= (\alpha p)(x) + (\beta p)(x)\\
	&= (\alpha p + \beta p)(x) 
\end{aligned}
$\\
.8 $
\begin{aligned}[t]
	\forall\alpha,\beta\in\R, \forall p\in P_n : [(\alpha  \beta)p](x)
	&= (\alpha\beta)p(x)\\
	&= \alpha (\beta p(x))\\
	&= \alpha [\beta p](x)\\
	&= [\alpha(\beta p)](x)
\end{aligned}
$

\setRL
\begin{example}
	لتكن $W = \{ p(x) \in P_2 \mid p(1) = 0 \}$. أثبت أن $W$ فضاء جزئي من $P_2$.
\end{example}
\noindent
\textbf{الحل}\\
\noindent
	للتحقق من أن $W$ فضاء جزئي من 
	$P_2$، نتحقق من الشرطين:\\
	\noindent
	أولاً، إذا كانت $p(x), q(x) \in W$، أي أن $p(1) = 0$ و$q(1) = 0$، فإن:
	\[
	(p + q)(1) = p(1) + q(1) = 0 + 0 = 0
	\]
	إذًا $p(x) + q(x) \in W$.
	ثانياً، إذا كانت $p(x) \in W$ و$c \in \mathbb{R}$، فإن:
	\[
	(cp)(1) = c \cdot p(1) = c \cdot 0 = 0
	\]
	إذًا $cp(x) \in W$.
	بما أن الشرطين محققان، فإن $W$ فضاء جزئي من 
	$P_2$.


