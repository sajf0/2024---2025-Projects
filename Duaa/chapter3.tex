\chapter{انواع فضاء المتجهات}

\section{الفضاء الاقليدي $\R^n$}
\subsection{التعريف}

مجموعة العناصر على الشكل
\[
\R^n = \{(x_1, x_2, \dots, x_n) : x_i \in \R\}
\]
حيث تعرف عمليتي الجمع والضرب كالآتي
\[
(x_1, x_2, \dots, x_n) + (y_1, y_2, \dots, y_n)  = (x_1+x_1, x_2+y_2, \dots, x_n+y_n) 
\]  
\[
\alpha (x_1, x_2, \dots, x_n)  = (\alpha x_1,\alpha x_2, \dots, \alpha x_n) 
\]

\subsection{الاساس والبعد}
اساس الفضاء الاقليدي هو المجموعة $B = \{e_1, e_2, \dots, e_n \}$
حيث 
\[
e_1 = (1,0,\dots,0) , e_2 = (0,1,\dots,0) , \dots , e_n = (0,0,\dots,1)
\]
ومن تعريف البعد : $\dim\R^n = n$

\subsection{الضرب النقطي}
يعرف الضرب النقطي في الفضاء الاقليدي كالآتي : لكل $x, y\in \R^n$ فإن
\[
x \cdot y = \sum_{i=1}^{n} x_i y_i
\]

\subsection{اثبات ان $\R^n$ فضاء متجهات}

\selectlanguage{english}
\noindent
1) $
\begin{aligned}[t]
	x + y &= (x_1, x_2, \dots, x_n) + (y_1, y_2, \dots, x_n)\\
	&= (x_1+y_1, x_2+y_2, \dots, x_n + y_n)\\
	&= (y_1, y_2, \dots, y_n) + (x_1, x_2, \dots, x_n) = y + x
\end{aligned}
$\\ \\
2) $
\begin{aligned}[t]
	x + (y + z) &= (x_1, x_2, \dots, x_n) + [(y_1, y_2, \dots, y_n) + (z_1, z_2, \dots, z_n)]\\
	&= (x_1, x_2, \dots, x_n) + (y_1+z_1, y_2+z_2, \dots, y_n+z_n)\\
	&=  (x_1+y_1+z_1, x_2+y_2+z_2, \dots, x_n+y_n+z_n)\\
	&=  (x_1+y_1, x_2+y_2, \dots, x_n+y_n) + (z_1+z_2,\dots,z_n)\\
	&= (x+y) + z
\end{aligned}
$\\

\selectlanguage{arabic}
\noindent
يوجد عنصر محايد وهو الصفر $0 = (0, 0, \dots, 0)$ بحيث

\selectlanguage{english}
\noindent
3) $
\begin{aligned}[t]
	x + 0 &= (x_1,x_2,\dots,x_n) + (0, 0, \dots, 0)\\
	&= x
\end{aligned}
$\\

\selectlanguage{arabic}
\noindent
لكل متجه $x = (x_1, x_2, \dots, x_n) \in \R^n$ نظيره $-x = (-x_1, -x_2, \dots, -x_n) \in \R^n$

\selectlanguage{english}
\noindent
4) $x + (-x) = (x_1, x_2, \dots, x_n)= (0, 0, \dots, 0) = 0$ \\ 
\\
5) $
\begin{aligned}[t]
	\forall x &= (x_1, x_2, \dots, x_n) \in \R^n: 1\cdot x\\
	&= 1 \cdot (x_1, x_2, \dots, x_n)\\
	&= (x_1, x_2, \dots, x_n) \\
	&= x
\end{aligned}
$\\
\noindent
6) $
\begin{aligned}[t]
	\forall \alpha&\in\R , \forall x,y \in \R^n: \alpha (x+y)\\
	&= \alpha(x_1+y_1, x_2+y_2, \dots, x_n+y_n)\\
	&= (\alpha(x_1+y_1), \alpha(x_2+y_2), \dots, \alpha(x_n+y_n))\\
	&= (\alpha x_1+\alpha y_1, \alpha x_2+\alpha y_2, \dots, \alpha x_n+\alpha y_n)\\
	&= \alpha x + \alpha y
\end{aligned}
$\\\\
\noindent
7) $
\begin{aligned}[t]
	\forall \alpha, \beta &\in \R, \forall x\in \R^n: (\alpha + \beta)\cdot x\\
	&= (\alpha + \beta)\cdot (x_1, x_2, \dots, x_n)\\
	&= ((\alpha+\beta)x_1, (\alpha+\beta)x_2, \dots, (\alpha+\beta)x_n)\\
	&= (\alpha x_1 + \beta x_1, \alpha x_2 + \beta x_2, \dots , \alpha x_n + \beta x_n)\\
	&= \alpha x + \beta x
\end{aligned}
$\\
\\
8) $
\begin{aligned}[t]
	\forall \alpha, \beta &\in \R, \forall x\in \R^n: (\alpha \beta)\cdot x\\
	&= (\alpha\beta)(x_1, x_2, \dots, x_n)\\
	&= (\alpha\beta x_1, \alpha\beta x_2, \dots, \alpha\beta x_n)\\
	&= (\alpha(\beta x_1), \alpha(\beta x_2), \dots, \alpha(\beta x_n))\\
	&= \alpha \cdot (\beta x)
\end{aligned}
$
\selectlanguage{arabic}
\begin{example}
	المجموعة 
	$W = \{(x_1, x_2, x_3) \in \R^3: x_1+x_2+x_3=0\}$
	تمثل فضاء جزئي من $\R^3$ ، اوجد القاعدة والبعد لـــ $W$.
\end{example}
\begin{solution}
	بما ان $x_1+x_2+x_3=0 $ اذن $x_3=-x_1-x_2 $ وبالتالي
	\begin{align*}
		(x_1, x_2, x_3) &= (x_1, x_2, -x_1-x_2)\\
		&= x_1(1,0,-1) + x_2 (0,1,-1)
	\end{align*}
	اذن القاعدة تكون 
	$B=\{(1,0,-1), (0,1,-1)\}$ اذن ومنه نحصل على البعد $\dim W=2$
\end{solution}

\section{فضاء المصفوفات $M_{m\times n}(\R)$}
تعرف $M_{m\times n}(\R)$ على انها مجموعة كل المصفوفات ذات البعد $m \times n$ على حقل الاعداد الحقيقية $\R$. 
\subsection{عمليتي الجمع والضرب}
\textbf{الجمع:} لتكن $A=[a_{ij}]$ و $B=[b_{ij}]$ مصفوفات من $M_{m\times n}(\R)$ فإن 
\[
A + B = [a_{ij} + b_{ij}] , 1\leq i\leq m, 1\leq j\leq n  
\]
\textbf{الضرب:} ليكن $c\in \R$ و $A = [a_{ij}]\in M_{m\times n}(\R)$ فإن 
\[
cA = [c\, a_{ij}] , 1\leq i\leq m, 1\leq j\leq n
\]

\subsection{الاساس والبعد}
اساس الفضاء $M_{m\times n}(\R)$ هو المجموعة  $B = \{E_{ij} : 1\leq i\leq m, 1\leq j\leq m\}$ حيث ان $E_{ij}$ مصفوفة تكون في الصف $i$ والعمود $j$ تساوي 1 وباقــــي المواقع تساوي صفراً. اذن البعد $\dim(M_{m\times n}(\R))=m\times n$
\subsection{الضرب النقطي}
لكل $A, B\in M_{m\times n}(\R)$ يعرف الضرب النقطي كالآتي
\[
A\cdot B = \sum_{i=1}^{m}\sum_{j=1}^{n} a_{ij}b_{ij}
\]

\begin{example}
	مجموعة المصفوفات المتناظرة
	 $W=\{A\in M_{m\times n}(\R) : A^T = A\}$
	 تمثل فضاء جزئي من الفضاء $M_{m\times n}(\R)$. اثبت ذلك
\end{example}
\noindent
\textbf{البرهان}\\
\noindent
1. لكل $A, B\in W$ فــــإن $A^T=A, B^T=B$ لذا 
\[
(A+B)^T = A^T + B^T = A+B
\]
اذن $A+B\in W$ \\
2. لكل $c\in\R$ ولكل $A\in W$ فــــإن $A^T=A$ وبالتالي
\[
(cA)^T = cA^T = cA
\]
اذن $cA \in W$ وبالتالي $W$ يمثل فضاء جزئي من $M_{m\times n}(\R)$.

\section{فضاء كثيرات الحدود $P_n$}
مجموعة كثيرات الحدود من الدرجة $n$ او اقل تعرف كالآتي
\[
P_n = \{p(x) = a_0 + a_1 x + \cdots + a_n x^n\}
\]
\subsection{عمليتي الجمع والضرب}
\textbf{الجمع:} لكل $p(x), q(x) \in P_n$ لدينا 
$
(p+q)(x) = p(x) + q(x)
$\\
\textbf{الضرب:} لكل $\alpha \in \R$ ولكل $p(x)\in P_n$ لدينا
$
(\alpha p)(x) = \alpha p(x)
$

\subsection{الاساس والبعد}
تكون المجموعة
$B = \{1, x, x^2, \dots, x^n\}$
اساس الفضاء $P_n$ \\
ويكون الفضاء ذات بعد $\dim P_n = n+1 $.

\subsection{اثبات ان $P_n$ فضاء متجهات}

\begin{example}
	مجموعة كثيرات الحدود الزوجية 
	$W = \{p(x) \in P_n : p(-x) = p(x) : a_i \in \R\}$
	تمثل فضاء جزئي من $P_n$. اثبت ذلك
\end{example}
\noindent
\textbf{البرهان}\\
\noindent
1. لكل $p(x), q(x) \in P_n$ اذن $p(-x) = p(x) \wedge q(-x) = q(x)$ ، اذن
\[
(p+q)(-x) = p(-x) + q(-x)= p(x) + q(x) = (p+q)(x)
\]
اذن $(p+q)(x) \in P_n$.\\ \noindent
2. لكل $\alpha \in\R$ ولكل $p(x) \in W$ فــــإن $p(-x) = p(x)$ وبالتالي
\[
(\alpha p)(-x) = \alpha p(-x) = \alpha p(x) = (\alpha p)(x) 
\] 
اذن $(\alpha p)(x) \in W$.

\section{فضاء الدوال المستمرة $C[a, b]$}
مجموعة الدوال المستمرة على الفترة $[a, b]$ تعرف بالشكل
\[
C[a, b] = \{f:[a, b] \to \R \mid \text{$f$ دالة مستمرة}\}
\]

\subsection{عمليتي الجمع والضرب}
\textbf{الحمع:} لكل $f(x), g(x) $ دوال مستمرة على الفترة $[a, b]$ :$(f+g)(x) = f(x) + g(x)$
\textbf{}