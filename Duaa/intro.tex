\chapter*{مقدمة}
\addcontentsline{toc}{chapter*}{مقدمة}
\noindent
	المتجهات هي كميات رياضية تتميز بامتلاكها مقدارًا واتجاهًا، وتُستخدم على نطاق واسع في العديد من المجالات العلمية والهندسية. تتمثل أهميتها في الحياة العملية في وصف الظواهر الفيزيائية مثل القوة والسرعة والتسارع، حيث تعتمد العديد من التطبيقات الهندسية والتقنية على تحليل المتجهات لفهم حركة الأجسام والتفاعل بين القوى المختلفة. بالإضافة إلى ذلك، تلعب المتجهات دورًا أساسيًا في الرسومات الحاسوبية، والملاحة الجوية، والذكاء الاصطناعي، وحتى في الاقتصاد والتمويل عند تحليل البيانات واتجاهات السوق.
	
	\noindent
	أما \textbf{فضاء المتجهات}، فهو مفهوم رياضي يُعرّف على أنه مجموعة من المتجهات التي تخضع لعمليات الجمع والضرب العددي، ويمثل الأساس للعديد من النظريات الرياضية مثل الجبر الخطي والتحليل العددي. يُستخدم فضاء المتجهات في حل المعادلات التفاضلية، والنمذجة العلمية، والتشفير، مما يجعله عنصرًا جوهريًا في فهم وتطوير العديد من العلوم والتقنيات الحديثة.
	

