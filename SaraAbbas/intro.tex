\chapter*{مقدمة}
\addcontentsline{toc}{chapter*}{مقدمة}

نظرية التقريب تلعب دوراً مهماً في التحليل الرياضي والتحليل العددي. حيث تقدم طرائق لتقريب دوال معقدة بواسطة دوال ابسط منها. واحد من اكثر الادوات المعروفة هو مؤثر برينستين. الذي قدمه العالم برينستين في عام 1912 كجزء من برهانه لنظرية وايرستراس للتقريب.\\
\noindent
يكون التقريب للدالة الذي قدمه برينستين بواسطة مجموع لكثيرات حدود تسمى بكثيرات حدود القاعدة لبرينستين التي تمثل اداة قوية وفعّالة في التقريب لانها تحافظ على خصائص معينة للدالة. مثل عدم السالبية والرتابة. كما تستخدم في نطاق واسع في الرسومات الحاسبية والتحليل العددية ونظرية الاحتمالات وفي دراسة منحنيات بيزير \en{\textbf{Bezier curves}}.