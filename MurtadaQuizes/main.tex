\documentclass[14pt, a4paper]{extarticle}

\usepackage[
margin=30pt
]{geometry}

\pagestyle{empty}

\usepackage{amsmath, amssymb, mathtools, amsthm}

\usepackage{polyglossia}

\setmainlanguage[numerals=maghrib]{arabic}

\newfontfamily\arabicfont[Script=Arabic]{Times New Roman}

\usepackage[lite]{mtpro2}

\begin{document}
	\begin{minipage}{0.2\textwidth}
		\raggedleft
\textbf{الاسم : }\\[15pt]
\textbf{الشعبة : }
	\end{minipage}
	\hfill
		\begin{minipage}{0.25\textwidth}
		\raggedright\includegraphics[scale=0.11]{school.jpg}   
	\end{minipage}
	\hfill
	\begin{minipage}{0.35\textwidth}
		\raggedright
		\textbf{مدارس البصير الاهلية}\\
		\textbf{الصف : الاول المتوسط}\\
				\textbf{المادة: الرياضيات}\\
		\textbf{امتحان الفصل الثاني / الشهر الثاني}
	\end{minipage}
	\hrule
	\vspace{20pt}
	\textbf{س 1 /} جد مجموع قياس الزوايا الداخلية في سباعي منتظم. \hfill \textbf{(25 درجة)} 
	\vfill
	\hrule
	\vspace{20pt}
	\textbf{س 2/} ارسم المستوي الاحداثي و حدد النقاط التالية عليه. \hfill \textbf{(25 درجة)}
	\[
	A(1,-2), B(3,0), C(-4,4), D(2,3), E(-3,-1)
	\]
	\vfill
	\hrule
	\vspace{1pt}
	\begin{center}
\textbf{		(اقلب الصفحة)}
	\end{center}
	\newpage
	\textbf{س 3/} أجرٍ انسحاب النقاط التالية وحدتان الى الاسفل و 5 وحدات نحو اليمين. \hfill \textbf{(25 درجة)}
	\[
	A(1,1) ,\,\,\, B(4,2),\,\,\, C(-1, 2)
	\]
	\vfill
	\hrule
	\vspace{20pt}
	\textbf{س 4/} جد انعكاس النقاط التالية حول محور السينات. \hfill \textbf{(25 درجة)}
	\[
	A(2,2)\,\,\, B(-3,2),\,\,\, C(1,-3)
	\]
	\vfill
	\hrule
	\vspace{1pt}

	\begin{center}
		\textbf{مدرس المادة}\\
		\textbf{مرتضى حسين مهدي}
	\end{center}
\end{document}