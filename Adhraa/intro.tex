\chapter*{مقدمة}
\addcontentsline{toc}{chapter*}{مقدمة}


طريقة التغاير التكراري
(\en{Variational Iteration Method}) هي واحدة من الطرق الرياضية الحديثة التي تستخدم لحل المعادلات التفاضلية، سواء كانت خطية أو غير خطية. تعتبر هذه الطريقة من الأساليب الفعّالة التي تتيح إيجاد حلول تقريبية للمعادلات المعقدة التي يصعب حلها باستخدام الطرق التقليدية.\\
\noindent
تم تطوير طريقة التغاير التكراري في أوائل التسعينات بواسطة الباحثين\en{J. H. He} و \en{H. S. Zhang}، وهي تقوم على فكرة استخدام تكرارات لتحسين تقريب الحلول للمعادلات التفاضلية. تعتمد هذه الطريقة على تعديل المعادلة التفاضلية الأصلية بشكل يسمح بتطوير سلسلة تكرارية تؤدي إلى تقريب الحل بفعالية مع كل خطوة.\\
\noindent
تتميز طريقة التغاير التكراري بقدرتها على توفير حلول دقيقة، حتى عندما تكون المعادلات التفاضلية غير خطية أو تحتوي على معقدات متعددة. يمكن استخدامها لحل مجموعة واسعة من المعادلات التفاضلية الجزئية والعادية، بما في ذلك معادلات بواسون، معادلات لايبنز، معادلات فاراداي، وغيرها.