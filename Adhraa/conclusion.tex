\chapter*{الخلاصة}
\addcontentsline{toc}{chapter*}{الخلاصة}

في نهاية البحث قد تعرفنا على طريقة التغاير التكراري  (\textbf{VIM}) في حل المعادلات التفاضلية الجزئية و طبقناها على حل مثال. حيث هذه الطريقة تستخدم مفهوم مضروب لاكرانج في نظرية التغاير للحصول حل امثلي (\en{\textbf{optimal solution}})
 وعرفنا ان التكرارات في هذه الطريقة هي عبارة عن دوال تقترب الى الحل الحقيقي
 \en{(\textbf{exact solution})}. وفي كثير من الحالات يكون افضل من الطرق العددية التي تعطي حلول عددية بدل من الدوال.