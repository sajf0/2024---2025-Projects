\chapter{مفاهيم أساسية}

\section[المعادلة التفاضلية الاعتيادية]{المعادلة التفاضلية الاعتيادية \cite{ode}}
هي علاقة بين المتغير المعتمد ومتغير مستقل واحد وتدخل فيها المشتقات او التفاضلات. الصيغة العامة للمعادلة التفاضلية الاعتيادية
\begin{equation}
	F(x, y, y', y'', \dots) = 0
\end{equation}
حيث $y$ المتغير المعتمد و $x$ المتغير المستقل و $F$ اي دالة. الان سوف نعطي بعض الامثلة على المعادلات التفاضلية الاعتيادية
\begin{align}
	\frac{dy}{dx} + y &= 3x^2\\[5pt]
	x\frac{d^3 y}{dx^3} + (2\sin x)\frac{d^2 y}{dx^2}\frac{dy}{dx} &= (3-x^2)y\\[5pt]
	\frac{d^2 y}{dx^2} + \frac{1}{x}\, \frac{dy}{dx} + y &=0 \\[5pt]
	(x-y)dx + (x+y)dy &= 0
\end{align}

\section[تصنيف المعادلات التفاضلية الاعتيادية]{تصنيف المعادلات التفاضلية الاعتيادية \cite{ode}}

سوف نصنف المعادلات التفاضلية الاعتيادية حسب الاتي
\subsection*{1. رتبة المعادلات التفاضلية الاعتيادية}
اذا كانت المشتقة النونية $y^{(n)}$ هي اعلى مشتقة تظهر بالمعادلة التفاضلية الاعتيادية قيل أن هذه المعادلة من الرتبة $n$.\\
\textbf{مثال:} لدينا المعادلات (2) و (5) معادلات تفاضلية من الرتبة الاولى. اما المعادلة (2) هي من الرتبة الثانية و المعادلة (4) من الرتبة الثالثة.

\subsection*{2. درجة المعادلة التفاضلية الاعتيادية}
هي الاس المرفوع اليها اعلى مشتقة تظهر بالمعادلة التفاضلية. وقبل تحديد درجة المعادلة يجب وضعها على صورة قياسية وصحيحة من حيث المشتقات.\\
\textbf{مثال:} المعادلات من (2) الى (5) كلها من الدرجة الاولى اما المعادلة
\begin{equation}
	\left(\frac{d^2 y}{dx^2}\right)^3 + x\left(\frac{dy}{dx}\right) + x^2 y^2 = e^x \sin x
\end{equation}
فهي معادلية تفاضلية من الرتبة الثانية و الدرجة الثالثة.

\subsection*{3. المعادلة التفاضلية الخطية}
هي المعادلة التفاضلية التي تكون خطية في المتغير المعتمد ومشتقاته جميعاً\\
\textbf{مثال:} المعادلة
\begin{equation}
	x^2y'' + xy' + x^2 y = \sin x
\end{equation}
هي معادلة خطية من الرتبة الثانية حيث ان كلاً من المتغير المعتمد $y$ ومشتقاته $y', y''$ خطية

اذا لم تكن المعادلية التفاضلية خطية فانها معادلة تفاضلية لا خطية\\
\textbf{مثال:} المعادلات التفاضلية التالية غير خطية
\begin{align}
	yy'' + y' &=0\\[5pt]
	y' + x\sqrt{y} &= \sin x\\[5pt]
	y''' + x^2 y'' + \sin y &= 0
\end{align}

\subsection*{4. المعادلة التفاضلية الاعتيادية الخطية المتجانسة}
تكون على الصيغة
\begin{equation}
	P_n(x) y^{(n)} + P_{n-1}(x) y^{(n-1)}+ \cdots + P_1(x) y' + P_0(x) y = 0
\end{equation}
\textbf{مثال:} المعادلة
\begin{equation}
	x^2 y'' + xy' + (x^2 - 1) y =0
\end{equation}
معادلة تفاضلية اعتيادية خطية متجانسة من الرتبة الثانية.

\setcounter{equation}{0}
\section[المعادلة التفاضلية الجزئية]{المعادلة التفاضلية الجزئية \cite{pde1}}
المعادلة التفاضلية الجزئية هي معادلة تتضمن متغير معتمد ذات متغيرين او اكثر ، والمشتقات الجزئية لهذه بالنسبة الى بعض تلك المتغيرات او كلها\\ [5pt]
الصيغة العامة للمعادلة التفاضلية الجزئية هي 
\begin{equation}
	F(x,y,z,t,\dots,u_x, u_y, u_z, u_t, \dots) = 0
\end{equation}
حيث $u$ هو المتغير المعتمد و $x,y,z,t,\dots$ هي المتغيرات المستقلة.\\[5pt]
\textbf{مثال:} كل مما يأتي معادلة تفاضلية جزئية
\begin{align}
	&\frac{\partial u}{\partial x} = t\\[5pt]
	&x\frac{\partial u}{\partial x} - t\frac{\partial u}{\partial t} = 4\\[5pt]
	&(x^2 + t^2)\frac{\partial^2 u}{\partial x\partial t} - 3\frac{\partial u}{\partial x} + x\left(\frac{\partial u}{\partial t}\right)^4 = e^x\\[5pt]
	&x\left(\frac{\partial u}{\partial x}\right)^3 - y\left(\frac{\partial^3 u}{\partial x\partial t^2}\right)^2 = 6x\frac{\partial u}{\partial x} \frac{\partial u}{\partial t}
\end{align}

\section[تصنيف المعادلات التفاضلية الجزئية]{تصنيف المعادلات التفاضلية الجزئية\cite{pde1}}
كما الحال في المعادلات التفاضلية الاعتيادية سوف نصنف المعادلات التفاضلية الجزئية كالآتي

\subsection*{1. رتبة ودرجة المعادلة التفاضلية}
كما في حالة المعادلات التفاضلية الاعتيادية تعرف رتبة المعادلة التفاضلية الجزئية بانها اعلى مشتقة فيها كما ان درجة المعادلة التفاضلية الجزئية هي اس اعلى مشتقة فيها بشرط ان يكون عدداً صحيحاً غير سالب.\\ [5pt]
\textbf{مثال:} المعادلتان (2) و (3) في المثال اعلاه من الرتبة الاولى و الدرجة الاولى والمعادلة (4) من الرتبة الثانية والدرجة الاولى والمعادلة (5) من الرتبة الثالثة والدرجة الثانية

\subsection*{2. المعادلة التفاضلية الجزئية الخطية}
تسمى المعادلة التفاضلية الجزئية خطية اذا كان مجموع اسس المتغير المعتمد ومشتقاته في كل من حدوده لا يزيد على (1) بشرط ان تكون الاسس اعداداً صحيحة غير سالبة وعلى سبيل المثال فإن المعادلة:
\begin{equation*}
	(x-t) \frac{\partial^3 u}{\partial x^3} + 4x \frac{\partial^3 u}{\partial x\partial t^2} - \frac{\partial ^2 u}{\partial x^2} + t \frac{\partial u}{\partial t} + tu = e^x
\end{equation*}
خطية، ولكن المعادلات
\begin{align*}
	& u \frac{\partial^2  u}{\partial x^2} - 4x \frac{\partial^2 u}{\partial x\partial y} = 0\\[5pt]
	&\left(\frac{\partial u}{\partial x}\right)^2 - 5xu = 0
\end{align*}
غير خطية

\subsection*{3. المعادلة التفاضلية الجزئية الخطية المتجانسة}
تسمى المعادلة التفاضلية الجزئية الخطية متجانسة اذا كانت كل المشتقات الجزئية فيها متساوية في الرتبة.\\
وعلى سبيل المثال فإن المعادلتين الاتيتين
\begin{align*}
&u_{xx} - x u_{tx} = 4x^2\\
&x^2 u_{xxx} + 5xt u_{txx} - u_{xxx} = x^2 + t^2
\end{align*}
متجانستين، ولكن المعادلتين
\begin{align*}
	&x^2 u_{xxx} + 5xt u_{txx} + u_{xx} = x^2\\
	&u_x + xu = t
\end{align*}
غير متجانستين

\section[عدد المتغيرات و الانماط الثلاث الاساسية]{عدد المتغيرات و الانماط الثلاث الاساسية \cite{pde3}}
سوف نهتم بدراسة المعادلات التفاضلية الجزئية بمتغيرين مستقلين والتي تأخذ الشكل العام
\begin{equation}
	a u_{xx} + 2bu_{tx} + cu_{tt} + du_x + eu_t + fu + g = 0
\end{equation}
حيث $a,b,c,d,e,f$ دوال للمتغيرين $x,t$. سوف ندرس المعادلة (6) من خلال المقدار المميز
\begin{equation}
	\Delta = b^2 - 4ac
\end{equation}

\subsection*{1. المعادلة التفاضلية الجزئية المكافئية}
تكون المعادلة (6) مكافئية اذا كان $\Delta =0 $ \\
\textbf{مثال:}
 \begin{equation*}
 	u_{xx} - u_y = 0
 \end{equation*}
 
 \subsection*{2. المعادلة التفاضلية الجزئية الناقصية}
 تكون المعادلة (6) ناقصية اذا كان $\Delta < 0 $ \\
 \textbf{مثال:}
 \begin{equation*}
 	u_{xx} + u_{yy} = 0
 \end{equation*}
 
 \subsection*{3. المعادلة التفاضلية الجزئية الزائدية}
 تكون المعادلة (6) زائدية اذا كان $\Delta > 0$ \\
 \textbf{مثال:}
 \begin{equation*}
 	u_{xx} - u_{yy} = 0
 \end{equation*}
 
 \section[الشروط الابتدائية و الشروط الحدودية]{الشروط الابتدائية والشروط الحدودية \cite{pde2}}
 
 \subsection*{1. الشرط الابتدائي}
 يتم تفسير احد المتغيرات المستقلة على انه الزمن $t$. و نفرض الشروط عند لحظة معينة. على سبيل المثال
 $u(x, t_0) = u_0(x)$
 
 \subsection*{2. الشرط الحدودي}
 نفرض الشروط على حدود المجال $\Omega$. على سبيل المثال
 $u|_{\partial \Omega} = \phi$
 حيث $\partial\Omega$ هي حدود المجال $\Omega$.
 
 