% compile with XeLaTeX
% this template was created by salim bou 
\documentclass[dvipsnames,mathserif]{beamer}
\usepackage{setspace, float}
\setstretch{1.2}
\usepackage{tikz}
\usepackage{subcaption}

\usepackage{amsmath, amssymb, amsthm, nicematrix}

\usepackage{polyglossia}
\setdefaultlanguage[numerals=maghrib,locale=algeria]{arabic} % locale=mashriq, libya, algeria, tunisia, morocco, or mauritania  for names of months in \date 
\setotherlanguage{english}
\newfontfamily\arabicfont[Script=Arabic]{Amiri}
\newfontfamily\arabicfontsf[Script=Arabic]{Amiri}
\newfontfamily{\timesfont}{Times New Roman}

\newcommand{\ar}{\textarabic}
\newcommand{\en}{\textenglish}

\usepackage[T1]{fontenc}
\usepackage{times}

\usepackage[lite, zswash]{mtpro2}

\DeclareMathSymbol{0}{\mathalpha}{operators}{`0}
\DeclareMathSymbol{1}{\mathalpha}{operators}{`1}
\DeclareMathSymbol{2}{\mathalpha}{operators}{`2}
\DeclareMathSymbol{3}{\mathalpha}{operators}{`3}
\DeclareMathSymbol{4}{\mathalpha}{operators}{`4}
\DeclareMathSymbol{5}{\mathalpha}{operators}{`5}
\DeclareMathSymbol{6}{\mathalpha}{operators}{`6}
\DeclareMathSymbol{7}{\mathalpha}{operators}{`7}
\DeclareMathSymbol{8}{\mathalpha}{operators}{`8}
\DeclareMathSymbol{9}{\mathalpha}{operators}{`9}

\usetheme{Warsaw}
%\usecolortheme{crane}

% for RTL liste
\makeatletter
\newcommand{\RTListe}{\raggedleft\rightskip\leftm}
\newcommand{\leftm}{\@totalleftmargin}
\makeatother



% RTL frame title
\setbeamertemplate{frametitle}
{\vspace*{-1mm}
	\nointerlineskip
	\begin{beamercolorbox}[sep=0.3cm,ht=2.2em,wd=\paperwidth]{frametitle}
		\vbox{}\vskip-2ex%
		\strut\hskip1ex\insertframetitle\strut
		\vskip-0.8ex%
	\end{beamercolorbox}
}


% align subsection in toc
\makeatletter
\setbeamertemplate{subsection in toc}
{\leavevmode\rightskip=5ex%
	\llap{\raise0.1ex\beamer@usesphere{subsection number projected}{bigsphere}\kern1ex}%
	\inserttocsubsection\par%
}
\makeatother

% RTL triangle for itemize
\setbeamertemplate{itemize item}{\scriptsize\raise1.25pt\hbox{\donotcoloroutermaths$\blacktriangleleft$}} 

%\setbeamertemplate{itemize item}{\rule{4pt}{4pt}}

\defbeamertemplate{enumerate item}{square2}
{\LR{
		%
		\hbox{%
			\usebeamerfont*{item projected}%
			\usebeamercolor[bg]{item projected}%
			\vrule width2.25ex height1.85ex depth.4ex%
			\hskip-2.25ex%
			\hbox to2.25ex{%
				\hfil%
				{\color{fg}\insertenumlabel}%
				\hfil}%
		}%
}}

\setbeamertemplate{enumerate item}[square2]

\setbeamertemplate{navigation symbols}{}





\title{\textbf{اهم الاختبارات اللا معلمية}}
\author{\textbf{الطالب : كاظم لطيف كاظم}}
\date{\textbf{اشراف : د.جاسم محمد علي العيساوي}}

\begin{document}
	\abovedisplayskip=7pt
	\belowdisplayskip=7pt
	\maketitle	
	\begin{frame}
		\begin{huge}
			\begin{center}
				\bfseries
				\vspace*{\stretch{1}}
				﷽\\
				\vspace{15pt}
				\quranayah[10][10][8]
			\end{center}
			\begin{flushleft}
				\textbf{سورة يونس}
			\end{flushleft}
			\vspace*{\stretch{1}}
		\end{huge}
	\end{frame}
	
		\timesfont
		
		\begin{frame}{الملخص}
			تناول هذا البحث أهم الاختبارات اللّا معلمية 
			\LR{(Non-parametric Tests)}، التي تُعد من الأدوات الإحصائية الأساسية عند عدم تحقق الافتراضات المطلوبة في الاختبارات المعلمية، كعدم توفر التوزيع الطبيعي أو تجانس التباين. وقد تطرق البحث إلى تعريف الاختبارات اللا معلمية، وبيان الفرق بينها وبين الاختبارات المعلمية من حيث الشروط والمرونة في التطبيق، مع توضيح مميزاتها وعيوبها.
			
			كما شمل البحث عرضًا لأبرز الاختبارات اللا معلمية المستخدمة في تحليل البيانات، مثل اختبار مان-ويتني 
			\LR{(Mann-Whitney U Test)} للمجموعات المستقلة، واختبار ويلكوكسون \LR{(Wilcoxon Signed-Rank Test)} للبيانات المرتبطة وغيرها من الاختبارات الشائعة. وتم توضيح استخدامات كل اختبار، وخطوات تطبيقه، وكيفية تفسير نتائجه.
			
			واختُتم البحث بالتأكيد على أهمية هذه الاختبارات في مجالات متعددة، خصوصًا في البحوث التي تتعامل مع بيانات رتبية أو ذات توزيع غير طبيعي، مما يجعلها أدوات فعالة في دعم القرارات الإحصائية عندما تكون الشروط المعلمية غير متحققة.
		\end{frame}
		
		\begin{frame}
			\Huge
			\begin{center}
				\textbf{الفصل الاول}\\
				\textbf{مفاهيم اساسية}
			\end{center}
		\end{frame}
		
		\begin{frame}
			\begin{exampleblock}{النموذج الاحصائي}
				هو عبارة عن تعبير رياضي عن العوامل التي تؤثر في المشاهدة طبقاً لافتراضات التجربة ولابد ان يعكس النموذج العلاقة بين متغير الاستجابة (المعتمد) ومتغير الرئيسي (المستقل) المسؤول عن احداث تغيير في معامل الاستجابة.
			\end{exampleblock}
			
			\pause
			\begin{exampleblock}{المجتمع والعينة}
				يختلف معنى كلمة المجمتع في علم الاحصاء عن المعنى الشائع لدى عامة الناس حيث تستخدم كلمة المجتمع لدى العامة للاشارة الى مجموعة من الاشخاص الذين يقيمون في منطقة معينة. في حين يعبر المجتمع في علم الاحصاء بأنه جميع الوحدات التي تكون الظاهرة محل للدراسة.\\
				\noindent
				اما العينة فهي جزء من المجتمع التي يتم اختبارها في الغالب عشوائياً ومن المفترض ان تمثل المجتمع محل الدراسة تمثيلاً صادقاً.
			\end{exampleblock}
			
			\pause
			\begin{exampleblock}{المعلمة و احصاء العينة}
				المعلمة هي خاصية من خصائص المجتمع التي يتم قياسها.
				اما احصاء العينة فهو قيمة رقمية تصف خاصية معينة يتم قياسها كمياًعن طريق عينة تمثل مجتمع الدراسة. اي ان احصاء العينة مقدر لعينة المجتمع.
			\end{exampleblock}
		\end{frame}
		
		\begin{frame}
			\begin{exampleblock}{الفرضية الاحصائية}
				تصريح او ادعاء قد يكون صائباً او يكون خاطئاً حول معلمة او اكثر لمجتمع او لمجموعة من المجتمعات.
			\end{exampleblock}
			
			\pause
			\begin{exampleblock}{مستوى المعنوية}
				هو احتمال الوقوع في الخطأ من النوع الاول.
			\end{exampleblock}
			
			\pause
			\begin{exampleblock}{درجات الحرية}
				هي عدد القيم القابلة للتغير في حساب خاصية احصائية معينة.
			\end{exampleblock}
			
			\pause
			\begin{exampleblock}{التوزيع الطبيعي}
				من اهم التوزيعات ذات التوزيع المستمر وهو الاكثر شيوعاً والتوزيع الطبيعي يتحدد بمعلمتين هما المتوسط $\mu$ والانحراف المعياري $\sigma$ وتتحدد قيم  المتغير من $-\infty$ الى $-\infty$ معادلة المنحني الطبيعي $f(x)$ 
				\[
				f(x) = 
				\begin{cases}
					\dfrac{1}{\sigma\sqrt{2\pi}} e^{-\frac{1}{2}\left(\frac{x-\mu}{\sigma}\right)^2} & ,-\infty < x < \infty \\
					0 & , \text{otherwise}
				\end{cases}
				\]
			\end{exampleblock}
		\end{frame}
		
		\begin{frame}
			\begin{exampleblock}{انواع الفرضيات}
				\begin{enumerate}
					\item $H_0$ هي فرضية العدم.
					\item $H_1$ هي الفرضية البديلة
				\end{enumerate}
			\end{exampleblock}
			
			\pause
			\begin{exampleblock}{انواع الاخطاء}
				\begin{enumerate}
					\item \textbf{الخطأ من النوع الاول} هو فرض فرضية العدم عندما تكون صحيحة.
					\item \textbf{الخطأ من النوع الثاني} هو قبول الفرضية البديلة عندما تكون خاطئة.
				\end{enumerate}
			\end{exampleblock}
			
			\pause
			\begin{exampleblock}{الاختبارات المعلمية}
				الاختبارات المعلمية هي اختبارات إحصائية تُستخدم عندما تكون البيانات تتبع توزيعًا معينًا، مثل التوزيع الطبيعي. هذه الاختبارات تفترض أن البيانات تأتي من مجتمع له خصائص معينة، مثل التوزيع الطبيعي والتجانس في التباين.
			\end{exampleblock}
		\end{frame}
		
		\begin{frame}
			\Huge
			\begin{center}
				\textbf{الفصل الثاني}\\
				\textbf{الاختبارات اللامعلمية}
			\end{center}
		\end{frame}
		
		\begin{frame}
			\begin{exampleblock}{اختبار مربع كاي (\en{Chi Square Test})}
				هو اختبار احصائي لا معلمي يستخدم لتحليل البيانات الفئوية. يقوم الاختبار بمقارنة التووزيعات الملحوظة مع التوزيعات المتوقعة تحت فرضية العدم (اي عدم وجود فرق او ارتباط بين الفئات). يتم حساب القيمة باستخدام معادلة تجمع مربعات الفروق بين القيم الملحوظة والقيم المتوقعة. كل منها مقسوم على القيمة المتوقعة. يستخدم الاختبار مثلاً لاختبار ما اذا كانت النسب في جدول التكرارات تختلف عن النسب المتوقعة نظرياً.
			\end{exampleblock}
			
			\pause
			\begin{exampleblock}{مثال}
				شركة ما تريد ان تعرف هل ان رضا العميل مرتبط بنوعية الخدمة التي يتلقونها (الكترونياً ام في الواقع). لدينا جدول البيانات التالي
				
				\begin{table}[H]
					\centering
					\begin{tabular}{| c| c| c| c|}
						\hline
						نوع الخدمة & راضٍ & غير راضٍ & المجموع \\
						\hline 
						الكتروني & 30 & 20 & 50 \\
						\hline
						في الواقع & 40 & 10 & 50 \\
						\hline
						المجموع & 70 & 30 &100\\
						\hline
					\end{tabular}
				\end{table}
			\end{exampleblock}
		\end{frame}
		
		\begin{frame}
		\begin{exampleblock}{}
				نستخدم قانون مربع كاي
				\[
				\chi^2 = \sum_i \frac{(O_i - E_i)^2}{E_i} 
				\]
				نستخرج القيم المتوقعة $E_i$:
				\begin{align*}
					E_1 = \frac{50\times 70}{100} = 35, \\
					E_2 = \frac{50\times30}{100} = 15\\
					E_3 = \frac{50\times70}{100} = 35, \\
					E_4 = \frac{50\times30}{100} = 15\\
				\end{align*}
			\end{exampleblock}
		\end{frame}
		
		\begin{frame}
			\begin{exampleblock}{}
				
				الان نحسب قيمة مربع كاي
				\begin{align*}
					\chi^2 &= \sum_i \frac{(O_i - E_i)^2}{E_i} 
					= \frac{25}{35} + \frac{25}{15} + \frac{25}{35} + \frac{25}{15}
					= 4.762
				\end{align*}
				
				الان نحدد درجة الحرية والتي تحسب من خلال القانون التالي
				\[
				df = (R-1)(C-1)
				\]
				حيث $R$ عدد الصفوف و $C$ عدد الاعمدالتالي
				\[
				df = (2-1)(2-1) = 1
				\]
				اذن $\chi^2 = 4.76$مع درجة حرية 1 ومن جدول مربع كاي فإن قيمة $p$ تكون تقريباً $0.03$ اذن $p < 0.05$ اذن نرفض فرضية العدم وبالتالي وجود ارتباط بين البيانات.
			\end{exampleblock}
		\end{frame}
		
		
		\begin{frame}{الفرق بين الاختبارات المعلمية واللامعلمية}
			\begin{center}
				\begin{tabular}{|c|>{\raggedleft\arraybackslash}p{5cm}|>{\raggedleft\arraybackslash}p{5cm}|}
					\hline
					\textbf{ت} & \textbf{الاختبارات المعلمية} & \textbf{الاختبارات اللامعلمية} \\
					\hline
					1 & يجب ان يكون حجم العينة كبيرا نسبيا & لا يشترط ان يكون حجم العينة كبيرا \\
					\hline
					2 & يشترط توفر معلومات حول معلمات المجتمع & لا يشترط توفر معلومات حول معلمات المجتمع \\
					\hline
					3 & يشترط ان تكون البيانات خاضعة للتوزيع الطبيعي & لا يشترط ان تكون البيانات خاضعة للتوزيع الطبيعي \\
					\hline
					4 & تناسب البيانات الكمية او النسبية & تناسب البيانات النوعية ويشترط ان تكون البيانات اسمية او ترتيبية \\
					\hline
					5 & اختبارات قوية & اكثر قوة \\
					\hline
					6 & يستغرق وقتا و جهدا & اسرع واسهل استخداما \\
					\hline
				\end{tabular}
			\end{center}
		\end{frame}
		
		\begin{frame}{استنتاجات}
			\begin{itemize}
				\item \textbf{فعالية الاختبارات اللامعلمية:} من خلال تحليل نتائج البحث، تبين أن الاختبارات اللامعلمية توفر نتائج دقيقة وموضوعية في قياس القدرات العقلية للطلاب، مقارنة بالاختبارات التقليدية التي قد تتأثر بتوقعات المعلمين أو تحيزاتهم.
				
				\item \textbf{مزايا الاختبارات اللامعلمية:} توفر الاختبارات اللامعلمية بيئة أكثر حيادية لقياس المهارات والقدرات، مما يساهم في توفير فرص متساوية لجميع الطلاب بغض النظر عن خلفياتهم الاجتماعية أو الثقافية.
				
				\item \textbf{التحديات التي تواجه تطبيق الاختبارات اللامعلمية:} بالرغم من مزاياها، فإن تطبيق الاختبارات اللامعلمية يتطلب تقنيات متقدمة وموارد ضخمة، مثل الأدوات التكنولوجية المتطورة والقدرة على تحليل البيانات الضخمة.
				
				\item \textbf{أثر الاختبارات اللامعلمية على التقييم التربوي:} توفر الاختبارات اللامعلمية دقة أكبر في تحديد القدرات الحقيقية للطلاب، مما يسهم في تحسين استراتيجيات التعليم والتوجيه التربوي.
				
				\item \textbf{الاختبارات اللامعلمية في السياقات الثقافية المختلفة:} الاختبارات اللامعلمية تساهم في تقديم صورة أدق للقدرات الطلابية في بيئات متنوعة، مما يعزز من التنوع والشمولية في العملية التعليمية.
			\end{itemize}
		\end{frame}
		
		\begin{frame}{توصيات}
			\begin{itemize}
				\item \textbf{تحسين التدريب للمعلمين:} يُنصح بتطوير برامج تدريبية مستمرة للمعلمين حول كيفية دمج واستخدام الاختبارات اللامعلمية في التعليم، مع توفير الدعم الفني للتغلب على التحديات التقنية.
				
				\item \textbf{توسيع استخدام التكنولوجيا:} ينبغي تعزيز استخدام أدوات تكنولوجية متطورة لدعم الاختبارات اللامعلمية وتحليل النتائج بشكل أكثر فعالية، مما يعزز من دقة التقييم وسرعة الوصول إلى البيانات.
				
				\item \textbf{إجراء دراسات مستقبلية:} يُوصى بإجراء دراسات أوسع وأعمق لفهم الأبعاد النفسية والاجتماعية لتأثير الاختبارات اللامعلمية على الطلاب في سياقات تعليمية متنوعة.
				
				\item \textbf{توفير الدعم المؤسسي:} من المهم أن تقوم المؤسسات التعليمية بتوفير الدعم المادي والفني لضمان استدامة تطبيق الاختبارات اللامعلمية في مختلف المدارس والجامعات.
				
				\item \textbf{تعزيز التنوع والشمولية:} يجب العمل على تحسين تصميم الاختبارات اللامعلمية لتكون شاملة لجميع الفئات الطلابية، بما في ذلك الطلاب من خلفيات ثقافية أو اجتماعية متنوعة.
			\end{itemize}
		\end{frame}
		
		\begin{frame}{المصادر}
			\begin{thebibliography}{9}
				\addcontentsline{toc}{chapter*}{المصادر}
				\begin{english}
					\bibitem{wasserman2006}
					Wasserman, L. (2006).
					\textit{All of Nonparametric Statistics}.
					Springer Science \& Business Media.
					
					\bibitem{hollander2013}
					Hollander, M., Wolfe, D. A., \& Chicken, E. (2013).
					\textit{Nonparametric Statistical Methods} (3rd ed.).
					John Wiley \& Sons.
					
					\bibitem{corder2014}
					Corder, G. W., \& Foreman, D. I. (2014).
					\textit{Nonparametric Statistics: A Step-by-Step Approach} (2nd ed.).
					John Wiley \& Sons.
					
					\bibitem{kolassa2020}
					Kolassa, J. E. (2020).
					\textit{An Introduction to Nonparametric Statistics}.
					CRC Press.
					
					\bibitem{kloke2024}
					Kloke, J. D., \& McKean, J. W. (2024).
					\textit{Nonparametric Statistical Methods Using R} (2nd ed.).
					Chapman \& Hall/CRC.
				\end{english}
			\end{thebibliography}
		\end{frame}
		
\end{document}