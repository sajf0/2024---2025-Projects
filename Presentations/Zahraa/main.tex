% compile with XeLaTeX
% this template was created by salim bou 
\documentclass[dvipsnames,mathserif]{beamer}
\usepackage{setspace}
\setstretch{1.5}
\usepackage{tikz}
\usetikzlibrary{calc}
\usepackage{polyglossia}
\setdefaultlanguage[numerals=maghrib,locale=algeria]{arabic} % locale=mashriq, libya, algeria, tunisia, morocco, or mauritania  for names of months in \date 
\setotherlanguage{english}
\newfontfamily\arabicfont[Script=Arabic]{Amiri}
\newfontfamily\arabicfontsf[Script=Arabic]{Amiri}

\usepackage[T1]{fontenc}
\usepackage{times}

\usepackage[lite, zswash]{mtpro2}

\DeclareMathSymbol{0}{\mathalpha}{operators}{`0}
\DeclareMathSymbol{1}{\mathalpha}{operators}{`1}
\DeclareMathSymbol{2}{\mathalpha}{operators}{`2}
\DeclareMathSymbol{3}{\mathalpha}{operators}{`3}
\DeclareMathSymbol{4}{\mathalpha}{operators}{`4}
\DeclareMathSymbol{5}{\mathalpha}{operators}{`5}
\DeclareMathSymbol{6}{\mathalpha}{operators}{`6}
\DeclareMathSymbol{7}{\mathalpha}{operators}{`7}
\DeclareMathSymbol{8}{\mathalpha}{operators}{`8}
\DeclareMathSymbol{9}{\mathalpha}{operators}{`9}

\usetheme{Warsaw}
%\usecolortheme{crane}

% for RTL liste
\makeatletter
\newcommand{\RTListe}{\raggedleft\rightskip\leftm}
\newcommand{\leftm}{\@totalleftmargin}
\makeatother



% RTL frame title
\setbeamertemplate{frametitle}
{\vspace*{-1mm}
	\nointerlineskip
	\begin{beamercolorbox}[sep=0.3cm,ht=2.2em,wd=\paperwidth]{frametitle}
		\vbox{}\vskip-2ex%
		\strut\hskip1ex\insertframetitle\strut
		\vskip-0.8ex%
	\end{beamercolorbox}
}


% align subsection in toc
\makeatletter
\setbeamertemplate{subsection in toc}
{\leavevmode\rightskip=5ex%
	\llap{\raise0.1ex\beamer@usesphere{subsection number projected}{bigsphere}\kern1ex}%
	\inserttocsubsection\par%
}
\makeatother

% RTL triangle for itemize
\setbeamertemplate{itemize item}{\scriptsize\raise1.25pt\hbox{\donotcoloroutermaths$\blacktriangleleft$}} 

%\setbeamertemplate{itemize item}{\rule{4pt}{4pt}}

\defbeamertemplate{enumerate item}{square2}
{\LR{
		%
		\hbox{%
			\usebeamerfont*{item projected}%
			\usebeamercolor[bg]{item projected}%
			\vrule width2.25ex height1.85ex depth.4ex%
			\hskip-2.25ex%
			\hbox to2.25ex{%
				\hfil%
				{\color{fg}\insertenumlabel}%
				\hfil}%
		}%
}}

\setbeamertemplate{enumerate item}[square2]

\setbeamertemplate{navigation symbols}{}





\begin{document}
	\author{\textbf{الطالبة : زهراء مؤيد}}
	\title{\textbf{كثيرات حدود شيبشيف من النوع الثاني}}
	\date{\textbf{اشراف\\
		م.م. ايمان عزيز عبدالصمد}}
	
	\begin{frame}
		\maketitle
	\end{frame}
	
	\timesfont
	\begin{frame}{مقدمة}
		
		\pause
		كانت مسألة التقريب محط اهتمام الكثير من العلماء في الرياضيات والفيزياء. وتعد القدرة على استبدال دالة معطاة بأخرى ابسط منها مثل كثيرات الحدود من الامور المفيدة جداً في مسائل الرياضيات التي توصف ظواهر معينة في الفيزياء والكيمياء وغيرها، حيث توضع شروط وقيود تكون في معظم الاحيان شروطاً صعبة ليس من السهل معها الحصول على الحلول الدقيقة ولا تطابق قيك هذه الدالة المعطاة ولكن اما ان تكون قريبة منها بالقدر الكافي او انها تتحكم بدرجة التقريب. \\[5pt]
		\pause
		في الرياضيات حدوديات شيبشيف هي حدوديات يعود اسمها الى عالم الرياضيات الروسي بافنوتي شيبشيف في عام (1953) (مؤسس علم التقريب المنتظم) هي متتالية من حدوديات متعامدة ذات اهمية اساسية في العديد من العلوم وفروع الرياضيات ونظرية التقريب وتطبيقاتها. وساهم الكثير من الباحثين باستخدام كثيرات الحدود المتعامدة في حل مسائل قيم حدية ومسائل قيم ابتدائية المتمثلة بمعادلات تفاضلية اعتيادية غير خطية والتي لها تطبيقات عملية عديدة في الهندسة التفاضلية والفيزياء اللا خطية والعلوم التطبيقية وغيرها. سنهتم بشكل اساسي بدراسة النوع الثاني لكثيرات حدود شيبشيف.  
	\end{frame}
	
		\begin{frame}
		\begin{center}
			\Huge
			\textbf{النوع الاول}
		\end{center}
	\end{frame}
	
	\begin{frame}{كثيرات حدود شيبشيف من  النوع الاول}
		\pause
		\begin{exampleblock}{ تعريف}
			
			تعرف كثيرات حدود شيبشيف من النوع الاول بالشكل التالي
			\begin{equation}
				T_n(x) = \cos(n\cos^{-1}x);\quad n\geq 0 , x\in [-1,1]
			\end{equation}
			
			\pause\noindent
			نفرض $x = \cos\theta$ فتصبح المعادلة (1) بالشكل $T_n(x) = \cos n\theta$
		\end{exampleblock}
		
		\pause
		\begin{exampleblock}{ الصيغة التكرارية}
				$$T_{n+1(x)} = 2xT_n(x) - T_{n-1}(x)$$ 
		\end{exampleblock}
	\end{frame}
	
	
		\begin{frame}
		\begin{center}
			\Huge
			\textbf{النوع الثاني}
		\end{center}
	\end{frame}
	
	\begin{frame}{كثيرات حدود شيبشيف من النوع الثاني}
		\begin{exampleblock}{تعريف}
				كثيرات حدود شيبشيف من النوع الثاني $U_n(x)$ تعرف كالاتي
			\begin{equation}
							U_n(x) = \frac{\sin(n+1)\theta}{\sin\theta},\quad n\geq 0 \quad -1 \leq x \leq 1
			\end{equation}

			
			حيث $x=\cos\theta \iff \theta = \cos^{-1}(x)$
		\end{exampleblock}
		
		\pause
		\begin{exampleblock}{الصيغة التكرارية}
			$$
				U_{n+1}(x) = 2x U_n(x) - U_{n-1}(x), \quad n = 1,2,3,\dots
		$$
	يمكن من خلال الصيغة التكرارية ، ايجاد $U_2(x), U_3(x), \dots$
		\end{exampleblock}
	\end{frame}
	
	\begin{frame}{}
		\begin{exampleblock}{مثال}
			
			\pause
			نلاحظ
			
			\pause
			\[
			U_0(x) = 1
			\]
			
			\pause
			\[
			 U_1(x) = \frac{\sin2\theta}{\sin\theta} = \frac{2\sin \theta\cos\theta}{\sin\theta}= 2\cos\theta=2x
			\]
			
			\pause
			وبإستخدام الصيغة التكرارية
			\[
				U_2(x) = 2x U_1(x) - U_0(x) = 2x(2x) - 1 = 4x^2 - 1
			\]
		\end{exampleblock}
	\end{frame}
	

	
		\begin{frame}{التعبير عن الدوال \LR{\textit{x}\textsuperscript{\textit{n}}} بكثيرات حدود شيبشيف من النوع الثاني}  
			
			\pause
			يمكن التعبير عن أي دالة أسية $x^n$ لأي متعددة حدود بإستخدام كثيرات حدود شيبشيف بالشكل التالي
			\begin{align*}
				&1 = U_0(x)\\
				&x = \frac{1}{2}U_1(x)\\
				&x^2 = \frac{1}{4} [U_0(x) + U_2(x)]\\
				&x^3 = \frac{1}{8} [U_3(x) + 2U_1(x)]\\
				&x^4 = \frac{1}{16}[U_4(x) + 3U_2(x) + 2U_0(x)]\\
				&x^5 = \frac{1}{32} [U_5(x) + 4U_3(x) + 5U_1(x) ]
			\end{align*}
	\end{frame}
	
	\begin{frame}
		
		\pause
		\begin{exampleblock}{مثال}
			عبر عن الدالة $e^x$ للحد من الدرجة الثالثة بإستخدام متسلسلة تايلور و كثيرات حدود شيبشيف من النوع الثاني
		
		\pause
		\noindent
		\textbf{الحل}\\
		\noindent
		الحدود لغاية الدرجة الثالثة بإستخدام متسلسلة تايلور 
		\[
		e^x = 1 + x + \frac{x^2}{2!} + \frac{x^3}{3!}
		\]
		
		\pause
		بإستخدام كثيرات حدود شيبشيف من النوع الثاني نستطيع التعبير عنها
		\begin{align*}
			e^x &= U_0(x) + \frac{1}{2}U_1(x) + \frac{1}{2}\cdot\frac{1}{4}[U_0(x) + U_2(x)] + \frac{1}{6}\cdot\frac{1}{8}[2U_1(x) + U_3(x)]\\
			&= U_0(x) + \frac{1}{2}U_1(x) + \frac{1}{8}U_0(x) + \frac{1}{8}U_2(x) + \frac{1}{24}U_1(x) + \frac{1}{48}U_3(x)\\
			&= \frac{9}{8}U_0(x) + \frac{13}{24}U_1(x) + \frac{1}{8}U_2(x) + \frac{1}{48}U_3(x)
		\end{align*}
		\end{exampleblock}
	\end{frame}
	
	\begin{frame}{بعض خواص كثيرات حدود شيبشيف من النوع الثاني}
		\begin{enumerate}
			
			\pause
			\item $U_n(-x) = (-1)^{n+1} U_n(x)$
			
			\pause
			\item جذور كثيرة شيبشيف من النوع الثاني هي \quad $x_r = \cos\left(\dfrac{r}{n+1}\pi\right), r=1,2,\dots,n$
			
			\pause
			\item كثيرة حدود شيبشيف من النوع الثاني متعامدة في المجال $[-1,1]$ بالنسبة لدالة الوزن \\$w = \sqrt{1-x^2}$ حيث
			\[
			\int_{-1}^{1} \sqrt{1-x^2} U_n(x) U_m(x)\, dx = \begin{cases}
				0 & ;n\neq m \\
				\dfrac{\pi}{2} & ; n = m
			\end{cases}
			\]
		\end{enumerate}
	\end{frame}
	
	\begin{frame}{العلاقة بين النوع الاول و الثاني}

\pause
\noindent
هناك علاقة وثيقة بين متعددات حدود شيبشيف من النوع الاول  $T_n(x)$ والنوع الثاني $U_n(x)$ حيث ان كلاهما مشتقة من الدوال المثلثية ولهما خصائص متشابهة\\

\pause
 و بما أن
 \[
 T_n(x) = \cos(n \theta), \quad U_n(x) = \frac{\sin(n+1)\theta}{\sin\theta}
 \]
 
 \pause
 فإن
 \begin{align}
 	T_{n+1}(x) = \cos(n+1)\theta = \cos(n\theta)\cos\theta - \sin(n\theta)\sin\theta \\
 T_{n-1}(x) = \cos(n-1)\theta = \cos(n\theta)\cos\theta + \sin(n\theta)\sin\theta 
 \end{align}
 
 \pause
 و بالتالي
 \[
 T_{n-1}(x) - T_{n+1}(x) = 2 \sin(n\theta) \sin\theta = 2 \sin^2\theta \frac{\sin(n\theta)}{\sin\theta} = 2(1-\cos^2 \theta)\cdot\frac{\sin(n\theta)}{\sin \theta}
 \]
	\end{frame}
	
\begin{frame}
	
	\pause
		اذن
		\begin{equation}
		T_{n-1}(x) - T_{n+1}(x) = 2 (1-x^2) U_{n-1}(x) 
	\end{equation}
	
	\pause
	ويمكن الحصول على علاقة اخرى من خلال
	\begin{align}
		U_{n+1} (x) &= \frac{\sin(n+2)\theta}{\sin\theta}\notag\\[5pt]
		&= \frac{\sin[(n+1)\theta + \theta]}{\sin \theta}\notag\\[5pt]
		&= \frac{\sin(n+1)\theta\cos\theta + \sin\theta\cos(n+1)\theta}{\sin\theta}
	\end{align}
\end{frame}
\begin{frame} 
	و
	\begin{align}
		U_{n-1} (x) &= \frac{\sin n\theta}{\sin\theta}\notag\\[5pt]
		&= \frac{\sin[(n+1)\theta -\theta]}{\sin \theta}\notag\\[5pt]
		&= \frac{\sin(n+1)\theta\cos\theta - \sin\theta\cos(n+1)\theta}{\sin\theta}
	\end{align}
	
	\pause
	إذن بطرح المعادلتين (6) و (7)
	\begin{equation}
		U_{n+1}(x) - U_{n-1}(x)  = 2T_{n+1}(x)
	\end{equation}
\end{frame}

\begin{frame}{الاستنتاج}
	
	\pause
	تناول هذا البحث دراسة كثيرات حدود شيبشيف من النوع الاول والثاني، حيث تم تحليل خصائص كل منهما. تم استخدام كثيرات حدود شيبشيف من النوع الثاني في تقريب الدوال ، لما تتميز به من خصائص مناسبة في هذا السياق. وقد بينت الدراسة اهمية فهم النوعين معاً لتعزيز استخدامهما في التقريب والتحليل العددي.
\end{frame}

\begin{frame}
	\begin{center}
		\Huge
		\textbf{شكراً لحسن استماعكم}
	\end{center}
\end{frame}
\end{document}
