% compile with XeLaTeX
% this template was created by salim bou 
\documentclass[dvipsnames,mathserif]{beamer}
\usepackage{setspace, float}
\setstretch{1.2}
\usepackage{tikz}
\usepackage{subcaption}

\usepackage{amsmath, amssymb, amsthm, nicematrix}

\usepackage{polyglossia}
\setdefaultlanguage[numerals=maghrib,locale=algeria]{arabic} % locale=mashriq, libya, algeria, tunisia, morocco, or mauritania  for names of months in \date 
\setotherlanguage{english}
\newfontfamily\arabicfont[Script=Arabic]{Amiri}
\newfontfamily\arabicfontsf[Script=Arabic]{Amiri}
\newfontfamily{\timesfont}{Times New Roman}

\newcommand{\ar}{\textarabic}
\newcommand{\en}{\textenglish}

\usepackage[T1]{fontenc}
\usepackage{times}

\usepackage[lite, zswash]{mtpro2}

\DeclareMathSymbol{0}{\mathalpha}{operators}{`0}
\DeclareMathSymbol{1}{\mathalpha}{operators}{`1}
\DeclareMathSymbol{2}{\mathalpha}{operators}{`2}
\DeclareMathSymbol{3}{\mathalpha}{operators}{`3}
\DeclareMathSymbol{4}{\mathalpha}{operators}{`4}
\DeclareMathSymbol{5}{\mathalpha}{operators}{`5}
\DeclareMathSymbol{6}{\mathalpha}{operators}{`6}
\DeclareMathSymbol{7}{\mathalpha}{operators}{`7}
\DeclareMathSymbol{8}{\mathalpha}{operators}{`8}
\DeclareMathSymbol{9}{\mathalpha}{operators}{`9}

\usetheme{Warsaw}
%\usecolortheme{crane}

% for RTL liste
\makeatletter
\newcommand{\RTListe}{\raggedleft\rightskip\leftm}
\newcommand{\leftm}{\@totalleftmargin}
\makeatother



% RTL frame title
\setbeamertemplate{frametitle}
{\vspace*{-1mm}
	\nointerlineskip
	\begin{beamercolorbox}[sep=0.3cm,ht=2.2em,wd=\paperwidth]{frametitle}
		\vbox{}\vskip-2ex%
		\strut\hskip1ex\insertframetitle\strut
		\vskip-0.8ex%
	\end{beamercolorbox}
}


% align subsection in toc
\makeatletter
\setbeamertemplate{subsection in toc}
{\leavevmode\rightskip=5ex%
	\llap{\raise0.1ex\beamer@usesphere{subsection number projected}{bigsphere}\kern1ex}%
	\inserttocsubsection\par%
}
\makeatother

% RTL triangle for itemize
\setbeamertemplate{itemize item}{\scriptsize\raise1.25pt\hbox{\donotcoloroutermaths$\blacktriangleleft$}} 

%\setbeamertemplate{itemize item}{\rule{4pt}{4pt}}

\defbeamertemplate{enumerate item}{square2}
{\LR{
		%
		\hbox{%
			\usebeamerfont*{item projected}%
			\usebeamercolor[bg]{item projected}%
			\vrule width2.25ex height1.85ex depth.4ex%
			\hskip-2.25ex%
			\hbox to2.25ex{%
				\hfil%
				{\color{fg}\insertenumlabel}%
				\hfil}%
		}%
}}

\setbeamertemplate{enumerate item}[square2]

\setbeamertemplate{navigation symbols}{}





\title{\textbf{التشاكل في الزمر والحلقات}}
\author{\textbf{الطالبة : حنين عدنان اسماعيل}}
\date{\textbf{اشراف : م. جاسم محمد جواد}}

\begin{document}
	\abovedisplayskip=0pt
	\belowdisplayskip=0pt
	\maketitle
	
	\timesfont
	
	\begin{frame}{الملخص}
		قدمنا في هذا البحث نبذة عن نظرية الزمر ونظرية الحلقات حيث درسنا في الفصل الاول مفاهيم أولية في نظرية الزمر وكذلك في نظرية الحلقات ، اما في الفصل الثاني درسنا مفهوم التشاكل الزمري والتشاكل الحلقي وأهم المبرهنات النتائج التي تخصهما. 
	\end{frame}
	
	\begin{frame}
		\Huge
		\begin{center}
			\textbf{الفصل الاول}\\
			\textbf{مفاهيم اولية في الزمر والحلقات}
		\end{center}
	\end{frame}
	
	\begin{frame}
		\begin{exampleblock}{تعريف الزمرة}
				لتكن $(G, *)$ شبه زمرة بعنصر محايد فأن $G$ تسمى زمرة Group اذا كان كل عنصر فيها له معكوس بالنسبة للعملية الثنائية $*$. او نقول ان $(G, *)$ زمرة اذا تحققت الشروط التالية
			\begin{enumerate}
				\item مغلقة بالنسبة للعملية $*$ اي : $a*b \in G , \forall a, b\in G$.
				\item العملية $*$ تجميعية : $a*(b*c) = (a*b)*c , \forall a, b,c\in G$.
				\item $G$ تمتلك عنصر محايد مثل $e$ : $a*e =e*a=a, \forall a\in G$.
				\item كل عنصر $a\in G$ يمتلك معكوس : $\forall a\in G, \exists a^{-1}\in G : a*a^{-1}=a^{-1}*a=e$.
			\end{enumerate}
		\end{exampleblock}
		
		\pause
		\begin{exampleblock}{تعريف}
				الزمرة $(G, *)$ تسمى زمرة ابدالية اذا كانت العملية $*$ عملية ثنائية ابدالية.
		\end{exampleblock}
		
		\pause
		\begin{exampleblock}{تعريف}
				لتكن $(G, *)$ زمرة و $H$ مجموعة غير خالية جزئية من $G$ فإن $(H, *)$ تسمى زمرة جزئية من الزمرة $(G, *)$ اذا كانت $H$ هي زمرة كذلك ونكتب $(H, *)\leq (G, *)$
		\end{exampleblock}
	\end{frame}
	
	\begin{frame}
		\begin{exampleblock}{مبرهنة}
				لتكن $(G, *)$ زمرة و $\varnothing\neq H\subseteq G$ اذن $(H, *)$ تكون زمرة جزئية من $(G, *)$ اذا وفقط اذا حققت الشرط 
			\[
			a * b^{-1} \in H, \forall a, b\in H
			\]
		\end{exampleblock}
		
		\pause
		\begin{exampleblock}{تعريف}
			
			لتكن $(H, *)$ زمرة جزئية من الزمرة $(G, *)$ وأن $a\in G$ ،تسمى  المجموعة المعرفة بالشكل التالي 
			$a* H = \{a * h : h\in G\}$
			بالمجموعة المشاركة (المصاحبة)  اليسارية لــ $H$ في $G$  ، وتسمى $H*a = \{h*a:h\in G\}$ بالمجموعة المشاركة (المصاحبة) اليمينية. 
		\end{exampleblock}
		
		\pause
		\begin{exampleblock}{تعريف}
			 	لتكن $(H, *)$ زمرة جزئية من $(G, *)$ فإن $(H, *)$ تسمى زمرة جزئية ناظمية (سوية) اذا وفقط اذا كان $a*H = H*a$ لكل $a\in G$ ونكتب $ H \trianglelefteq G$.
		\end{exampleblock}
	\end{frame}
	
	\begin{frame}
		\begin{exampleblock}{تعريف}
			  	لتكن $(H, *)$ زمرة جزئية من $(G , *)$ ، نعرف العملية الثنائية $\circledast$ على $G/H$ بالشكل التالي\\
			$
			(a*H)\circledast (b*H) = (a*b) *H, \forall a, b\in G
			$
			ويسمى الزوج المرتب $(G/H, \circledast)$ بزمرة القسمة.
		\end{exampleblock}
		
		\pause
		\begin{exampleblock}{تعريف الحلقة}
			 	الحلقة هي ثلاثي مرتب $(R, +, \cdot)$ مكون من مجموعة غير خالية $R$ وعمليتي الجمع والضرب بحيث 
			\begin{enumerate}
				\item $(R, +)$ زمرة ابدالية.
				\item  $(R, \cdot) $ شبه زمرة.
				\item العملية $\cdot$ تتوزع على العملية + ، أي أن:
				\begin{gather*}
					a\cdot (b+c) = a\cdot b + a\cdot c \tag{التوزيع من اليسار}\\
					(b + c) \cdot a = b\cdot a + c\cdot a \tag{التوزيع من اليمين}
				\end{gather*}
				لكل $a ,b,c\in R$
			\end{enumerate}
		\end{exampleblock}
	\end{frame}
	
	\begin{frame}
		\begin{exampleblock}{تعريف}
			 	لتكن $(R, +, \cdot)$ حلقة مع عنصر محايد لعملية الضرب ، نقول ان $(R, +, \cdot)$ ساحة تامة اذا لم تحوي على قواسم الصفر وكانت ابدالية بالنسبة لعملية الضرب.
		\end{exampleblock}
		
		\pause
		\begin{exampleblock}{تعريف}
			لتكن $(R, +, \cdot)$ حلقة و أن $\varnothing\neq S\subseteq R$ ، اذا كانت $(S, +, \cdot)$ حلقة بحد ذاتها نقول بأنها حلقة جزئية من $(R, +, \cdot)$ و اختصاراً نقول $S$ حلقة جزئية من $R$.
		\end{exampleblock}
		
		\pause
		\begin{exampleblock}{تعريف}
			 	لتكن $R$ حلقة و $I$ مجموعة جزئية من $R$ ، نقول ان $I$ هي مثالية في $R$ اذا تحققت الشروط
			\begin{enumerate}
				\item $a-b\in I, \forall a, b\in I$.
				\item $r\cdot a \in I$ و $a\cdot r\in I$ لكل $r\in R, a\in I$.
			\end{enumerate}
		\end{exampleblock}
	\end{frame}
	
	\begin{frame}
		\Huge
		\begin{center}
			\textbf{الفصل الثاني}\\
			\textbf{التشاكل الزمري والتشاكل الحلقي}
		\end{center}
	\end{frame}
	
	\begin{frame}
		\begin{exampleblock}{تعريف}
				لتكن كل من $(G_1, *)$ و $(G_2, \circ)$ زمرة ، تسمى الدالة $f : G_1\to G_2 $ انها تشاكل زمري اذا حققت الشرط  $f(a * b) = f(a) \circ f(b)$ لكل $a, b \in G_1 $ 
		\end{exampleblock}
		
		\pause
		\begin{exampleblock}{مثال}
			لتكن 
			$f : (\Z, +) \to (\Z_n, +_n)$ دالة معرفة بالشكل التالي $f(a) = [a], \forall a\in \Z$. بين هل ان $f$ تمثل تشاكل
		\noindent
		\textbf{الحل}\\
		\noindent
		لكل $a, b\in \Z$
		\[
		f(a + b) = [a + b] = [a] +_n [b] = f(a) +_n f(b)
		\]
		بالتالي $f$ دالة تشاكل.
		\end{exampleblock}
	\end{frame}
			
		\begin{frame}
			\begin{exampleblock}{مبرهنة}
				ليكن 	$f : (G_1, *) \to (G_2, \circ)$ تشاكل زمري ، فإن
			\begin{enumerate}
				\item $f(e_1) = e_2$.
				\item  $f(a^{-1}) = [f(a)]^{-1}$.
			\end{enumerate}
		\end{exampleblock}
		
		\pause
		\begin{exampleblock}{تعريف}
					ليكن 	$f : (G_1, *) \to (G_2, \circ)$ تشاكل زمري ، فإن مجموعة كل عناصر المجموعة $G_1$ التي تكون صورتها عنصر المحايد للزمرة $G_2$ تسمى نواة التشاكل ويرمز لها بالرمز $\ker f$ اي
			\[
			\ker f = \{a\in G : f(a) = e_2\}
			\]
		\end{exampleblock}
		
		\pause
		\begin{exampleblock}{مبرهنة}
						ليكن 	$f : (G_1, *) \to (G_2, \circ)$ تشاكل زمري ، فإن $\ker f\leq G_1$.
		\end{exampleblock}
		
		\pause
		\begin{exampleblock}{مبرهنة}
								ليكن 	$f : (G_1, *) \to (G_2, \circ)$ تشاكل زمري ، فإن $\ker f=\{e_1\}$ اذا وفقط اذا كانت $f$ دالة متباينة.
		\end{exampleblock}
		\end{frame}
		
		\begin{frame}
		   \begin{exampleblock}{مبرهنة التشاكل الاساسية في الزمر}
		   	لتكن 
		   	$f : (G_1, *) \to (G_2, \circ)$ تشاكل شامل فإن 
		   	$(G/\ker f, \otimes) \cong (G_2, \circ)$
		   \end{exampleblock}
		   
		   \pause
		   \begin{exampleblock}{تعريف التشاكل الحلقي}
		   		لنفترض ان $R$ و $S$ حلقتان ، تسمى الدالة $f:R\to S$ تشاكل حلقي اذا وفقط اذا كان
		   	\setLR
		   	\begin{enumerate}
		   		\item $f(a + b) = f(a) + f(b)$.
		   		\item $f(a\cdot b) = f(a) \cdot f(b)$.
		   	\end{enumerate}
		   	\setRL
		   \end{exampleblock}
		   
		   \pause
		   \begin{exampleblock}{ملاحظة}
		   	\begin{enumerate}
		   		\item اذا كانت $f$ دالة شاملة فإن التشاكل يسمى تشاكل شامل.
		   		\item اذا كانت $f$ دالة متباينة فإن التشاكل يسمى تشاكل متباين.
		   		\item اذا كانت $f$ دالة تقابل فإن التشاكل يسمى تشاكل تقابلي.
		   	\end{enumerate}
		   \end{exampleblock}
		\end{frame}
		
		\begin{frame}
			\begin{exampleblock}{مثال}
				
				افترض ان $R$ حلقة ، نعرف الدالة $f : R\to R$ بالشكل 
				$f(a) = a, \forall a\in R$
				تكون تشاكل تقابلي.\\
				\noindent
				\textbf{الحل}\\
				\noindent
				(1) $f(a + b) = a + b = f(a) + f(b)$.\\
				(2) $f(a\cdot b) = a\cdot b = f(a) \cdot f(b)$. \\
				اذا كان 
				$f(x) = f(y)$ لبعض $x,y \in R$ فأن $x=y$. اذن $f$ تباين.\\
				الآن لكل $y\in R$ فأن $ f(y) = y$ اذن $f$ شاملة. بالتالي $f$ تشاكل تقابلي.
			\end{exampleblock}
			
			\pause
			\begin{exampleblock}{تعريف}
					لتكن $f$ تشاكل من الحلقة $R$ الى الحلقة $S$. فإن المجموعة 
				\[
				\ker f = \{a\in R : f(a) = 0\}
				\]
				تسمى بنواة التشاكل $f$.
			\end{exampleblock}
		\end{frame}

\begin{frame}
	\begin{exampleblock}{مبرهنة}
				لتكن $f$ تشاكل من الحلقة $R$ الى الحلقة $S$. فإن $\ker f$ هي مثالية في $R$.
	\end{exampleblock}
	
	\pause
	\begin{exampleblock}{تعريف}
		نفرض ان $R, S$ حلقتان بحيث $f:R\to S$ تشاكل تقابلي ، نقول ان $R$ تماثل $S$ ونكتب $R\simeq S$.
	\end{exampleblock}
	
	\pause
	\begin{exampleblock}{تعريف}
		ليكن $I$ مثاليًا (ideal) في الحلقة $(R, +, \cdot)$، نُعرِّف التشاكل الطبيعي 
		$\pi : R \to R/I$ بالشكل التالي:
		\[
		\pi(r) = r + I
		\]
		ويسمى هذا التشاكل {بالتشاكل الطبيعي}.
	\end{exampleblock}
	
	\pause
	\begin{exampleblock}{مبرهنة التشاكل الاساسية في الحلقات} 
		لتكن 
		$f : (R, +, \cdot) \to (S, +, \cdot)$ تشاكل شامل بين حلقتين، فإن 
		$(R/\ker f, +, \cdot) \cong (S, +, \cdot)$.
	\end{exampleblock}
\end{frame}

\begin{frame}
	\Huge
	\begin{center}
		\textbf{شكراً لحسن استماعكم}
	\end{center}
\end{frame}
	
\end{document}