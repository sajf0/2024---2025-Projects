% compile with XeLaTeX
% this template was created by salim bou 
\documentclass[dvipsnames,mathserif]{beamer}
\usepackage{setspace}
\setstretch{1.5}
\usepackage{tikz}
\usetikzlibrary{calc}
\usepackage{polyglossia}
\setdefaultlanguage[numerals=maghrib,locale=algeria]{arabic} % locale=mashriq, libya, algeria, tunisia, morocco, or mauritania  for names of months in \date 
\setotherlanguage{english}
\newfontfamily\arabicfont[Script=Arabic]{Amiri}
\newfontfamily\arabicfontsf[Script=Arabic]{Amiri}

\usepackage[T1]{fontenc}
\usepackage{times}

\usepackage[lite, zswash]{mtpro2}

\DeclareMathSymbol{0}{\mathalpha}{operators}{`0}
\DeclareMathSymbol{1}{\mathalpha}{operators}{`1}
\DeclareMathSymbol{2}{\mathalpha}{operators}{`2}
\DeclareMathSymbol{3}{\mathalpha}{operators}{`3}
\DeclareMathSymbol{4}{\mathalpha}{operators}{`4}
\DeclareMathSymbol{5}{\mathalpha}{operators}{`5}
\DeclareMathSymbol{6}{\mathalpha}{operators}{`6}
\DeclareMathSymbol{7}{\mathalpha}{operators}{`7}
\DeclareMathSymbol{8}{\mathalpha}{operators}{`8}
\DeclareMathSymbol{9}{\mathalpha}{operators}{`9}

\usetheme{Warsaw}
%\usecolortheme{crane}

% for RTL liste
\makeatletter
\newcommand{\RTListe}{\raggedleft\rightskip\leftm}
\newcommand{\leftm}{\@totalleftmargin}
\makeatother



% RTL frame title
\setbeamertemplate{frametitle}
{\vspace*{-1mm}
	\nointerlineskip
	\begin{beamercolorbox}[sep=0.3cm,ht=2.2em,wd=\paperwidth]{frametitle}
		\vbox{}\vskip-2ex%
		\strut\hskip1ex\insertframetitle\strut
		\vskip-0.8ex%
	\end{beamercolorbox}
}


% align subsection in toc
\makeatletter
\setbeamertemplate{subsection in toc}
{\leavevmode\rightskip=5ex%
	\llap{\raise0.1ex\beamer@usesphere{subsection number projected}{bigsphere}\kern1ex}%
	\inserttocsubsection\par%
}
\makeatother

% RTL triangle for itemize
\setbeamertemplate{itemize item}{\scriptsize\raise1.25pt\hbox{\donotcoloroutermaths$\blacktriangleleft$}} 

%\setbeamertemplate{itemize item}{\rule{4pt}{4pt}}

\defbeamertemplate{enumerate item}{square2}
{\LR{
		%
		\hbox{%
			\usebeamerfont*{item projected}%
			\usebeamercolor[bg]{item projected}%
			\vrule width2.25ex height1.85ex depth.4ex%
			\hskip-2.25ex%
			\hbox to2.25ex{%
				\hfil%
				{\color{fg}\insertenumlabel}%
				\hfil}%
		}%
}}

\setbeamertemplate{enumerate item}[square2]

\setbeamertemplate{navigation symbols}{}





\newcommand{\sen}[1]{\en{\textsuperscript{#1}}}
\title{\textbf{انظمة المعادلات الخطية وبعض طرق حلها}}
\author{\textbf{الطالبة : رسل حسين فاخر}}
\date{\textbf{إشراف : ا.م. مرتضى جاسم محمد}}

\begin{document}
\renewcommand{\theequation}{1-\arabic{equation}}
\abovedisplayskip=0pt
\belowdisplayskip=0pt
	\maketitle
	\timesfont
	
	\begin{frame}{المستخلص}
		تمت دراسة مفهوم المعادلة الرياضية وبعض أنواعها وتم إيجاد بعض الحلول العددية والجبرية لأنظمة من المعادلات الجبرية التي ليس لها حل بالطرق التحليلية من خلال بعض الطرق المباشرة وبعض طرق غير مباشرة أذا كانت خطية وغير خطية.
	\end{frame}
	
	%%%% CHAPTER 1 %%%%
	\begin{frame}
		\begin{center}
			\Huge
			\textbf{الفصل الاول}\\
			\textbf{مفاهيم اساسية}
		\end{center}
	\end{frame}
	
	\begin{frame}
		\begin{exampleblock}{	\textbf{المعادلة الرياضية (1 - 1)\sen{\cite{odesolapp}}:}}
			هي عبارات رياضية تربط بينها علامة المساواة ويكون لها حدان متساويان في القيمة احدهما على الجانب الايمن والآخر على الجانب الايسر حيث يتم استخدامها في ايجاد المتغير المجهول سواء كان متغير واحد او اكثر.
		\end{exampleblock}
		
		\begin{exampleblock}{\textbf{المعادلة الجبرية (1- 2)\sen{\cite{odesolapp}}:}}
			هي مساواة بين مقدارين جبريين يحوي احدهما او كلاهما متغيراً او اكثر حيث القيمة العددية للمقدار الاول لا تساوي القيمة العددية للمقدار الثاني الا مع قيم خاصة للمتغيرات.\\
			على سبيل المثال معادلة حدودية احادية المتغير تأخذ الشكل التالي
			\[
			a_n x^n + a_{n-1}x^{n-1} + \cdots + a_1 x + a_0 = 0
			\]
			حيث $a_0, \dots, a_n$ هي معاملات المعادلة، الهدف هو ايجاد جميع القيم المجهولة لــ $x$.
		\end{exampleblock}
	\end{frame}
	
	\begin{frame}
			\begin{exampleblock}{ملاحظة}
			يقال عن متعددة الحدود انها من الدرجة الاولى اذا كانت اعلى قوة لــ $x$ تظهر في المعادلة هي واحد. وانها من الدرجة الثانية اذا كانت اعلى قوة لــ $x$ هي 2 وهكذا ...
		\end{exampleblock}
		
		\begin{exampleblock}{\textbf{المعادلة التفاضلية (1 - 3)\sen{\cite{odesolapp}}:}}
			هي معادلة تحوي مشتقات و تفاضلات لبعض الدوال الرياضية وتظهر فيها بشكل متغيرات المعادلة ويكون الهدف من حل هذه المعادلات هو ايجاد هذه الدوال الرياضية التي تحقق مشتقاتها هذه المعادلات.\\
			\textbf{$\bullet$ درجة المعادلة الرياضية :}
			تتحدد درجة المعادلة التفاضلية حسب اس المشتقة ذات الرتبة الاعلى.\\
			\textbf{$\bullet$ رتبة المعادلة التفاضلية :}
			هي رتبة اعلى مشتقة تحتوي عليها هذه المعادلة.
		\end{exampleblock}
	\end{frame}
	
	\begin{frame}
		\begin{exampleblock}{مثال}
				\begin{table}[H]
				\renewcommand{\arraystretch}{1.4}
				\centering
				\begin{tabular}{|c|c|c|}
					\hline
					\textbf{المعادلة التفاضلية} & \textbf{الدرجة} & \textbf{الرتبة}\\
					\hline
					$(y')^2 + 5x^3 y = 2x + 5y$ & الثانية & الاولى\\
					\hline
				\end{tabular}
			\end{table}
		\end{exampleblock}
		
		\begin{exampleblock}{1 - 1 انواع المعادلات التفاضلية}
			\begin{enumerate}
				\item \textbf{معادلات تفاضلية اعتيادية (\en{ordinary differential equations})} تحتوي على توابع ذات متغير مستقل واحد ومشتقات هذا المتغير \\
				\textbf{مثال}:\qquad $y'' + 3y = x^2$
				\item \textbf{معادلات تفاضلية جزئية \en{partial differential equations}} هي معادلات تفاضلية تحتوي على دالة واحدة او اكثر من الدوال المجهولة ومشتقاتها الجزئية.\\
				\textbf{مثال:}\[
				\frac{\partial u}{\partial x} + 3 \frac{\partial u}{\partial y} = 0
				\]
			\end{enumerate}
		\end{exampleblock}
	\end{frame}
	
	\begin{frame}
		\begin{exampleblock}{1 - 2 بعض طرق حل المعادلات التفاضلية الجزئية}
			\begin{enumerate}
				\item \textbf{فصل المتغيرات:}\\ بهذه الطريقة يتم تحويل المعادلة التفاضلية الجزئية ذات $n$ من المتغيرات المستقلة الى معادلة تفاضلية اعتيادية.
				
				\item \textbf{التحويلات التكاملية:}\\ يتم تحويل المعادلة الجزئية ذات $n$ من المتغيرات المستقلة الى معادلة تفاضلية جزئية ذات $n-1$ من المتغيرات المستقلة ومن ثم بهذه الطريقة يمكن تحويل المعادلة الجزئية ذات المتغيرين الى معادلة اعتيادية.
				
				\item \textbf{طريقة الدوال الذاتية: }\\ يتم ايجاد حل المعادلة الجزئية كمجموع عدد غير منته من الدوال الذاتية وهذه الدوال توجد بحل يسمى المناظرة للمسائل الاصلية.
			\end{enumerate}
		\end{exampleblock}
	\end{frame}
	
	\begin{frame}
		\begin{exampleblock}{المعادلات التفاضلية الخطية وغير الخطية (1 - 4)}
			كل من المعادلات التفاضلية العادية و الجزئية يمكن ان تصنف الى خطية وغير خطية وتكون المعادلة التفاضلية خطية بشرطين:
			\begin{enumerate}
				\item اذا كانت معاملات المتغير التابع والمشتقات فيها دوال في المتغير المستقل فقط او ثوابت.
				
				\item اذا كان المتغير التابع والمشتقات غير مرفوعة لأسس اي كلها من الدرجة الاولى.
			\end{enumerate}  
			\noindent
			$\bullet$ وتكون غير خطية فيها عدا ذلك
			\\[10pt]
			كل معادلة تفاضلية خطية هي من الدرجة الاولى بينما ليست كل المعادلات التفاضلية من الدرجة الاولى هي خطية لان الدرجة تتحدد حسب اس التفاضل الاعلى ومن الممكن ان تكون التفاضلات الاقل مرفوعة لأسس غير الواحد دون ان يؤثر ذلك على الدرجة وهذا يخل بشرط المعادلة الخطية. وبهذا تكون غير خطية.
		\end{exampleblock}
		
		\begin{exampleblock}{\textbf{امثلة}}
			\begin{enumerate}
				\item $x^2 y'' + xy' + x^2 y = e^x \sin x$\qquad معادلة تفاضلية خطية.
				\item $yy'' + y' = x$\qquad\qquad\qquad\quad معادلة تفاضلية غير خطية.
			\end{enumerate}
		\end{exampleblock}
	\end{frame}
	
	\begin{frame}
		\begin{exampleblock}{النظام الخطي (1 - 5)\sen{\cite{odesolapp}}:}
			
			هو نظام مكون من $m$ من المعادلات و $n$ من المتغيرات. او هو مجموعة تحتوي $m$ من المعادلات الخطية لكل منها $n$ من المتغيرات ويعبر عن ذلك النظام عادة بالشكل

			\begin{equation}
				\begin{gathered}
					a_{11}x_1 + a_{12}x_2 + \cdots + a_{1n} x_n = b_1\\
					a_{21}x_1 + a_{22}x_2 + \cdots + a_{2n} x_n = b_2\\
					\vdots\\
					a_{m1}x_1 + a_{m2}x_2 + \cdots + a_{mn} x_n = b_m
				\end{gathered}
			\end{equation}
			بالتالي فإن المعادلة (1-1) هي $a_{i1}x_1+a_{i2}x_2 +\cdots+a_{in}x_n=b_i$. تسمى $a_{ij}$ بالثوابت. تسمى $S_i$ التي تحقق كل معادلة خطية في النظام اعلاه بحل النظام المعادلات الخطية
			\[
			a_1x_1 + a_2 + \cdots + a_n x_n = b
			\]
		\end{exampleblock}
	\end{frame}
	
	\begin{frame}
	    \begin{exampleblock}{حل المعادلة الخطية (1 - 6)\sen{\cite{odesolapp}}:}
	    	هو متتابعة من $n$ من الاعداد $S_1,S_2,\dots,S_n$ تحقق المعادلة عند اجراء التعويض وتسمى الفئة المكونة من كل  حلول المعادلة بفئة الحل لها.
	    \end{exampleblock}
	     \begin{exampleblock}{انظمة المعادلات الخطية في متغيرين (1 - 7)\sen{\cite{research}}:}
	    		    		    	هذا النظام يكون بالشكل
	    		    	\begin{gather*}
	    		    		A_1 X_1 + B_1X_2 = C_1\\
	    		    		A_2 X_1 + B_2 X_2 = C_2
	    		    	\end{gather*}
	    		    	حل هذا النظام هو مجموعة الازواج المرتبة من الاعداد الحقيقية والتي تحقق المعادلتين. سوف نستخدم طريقة الحذف والتعويض الخلفي لحل هذا النظام.
	   \end{exampleblock}
	\end{frame}
	
	\begin{frame}
		\begin{exampleblock}{مثال}
				اوجد حل النظام
			\begin{gather}
				3X - 4Y = 28\\
				X+2Y = 6
			\end{gather}
			\noindent
			\textbf{الحل}\\
			\noindent
			من المعادلة (1-3)
			\begin{equation}
				X = 6 -2Y
			\end{equation}
			نعوض (1-4) في (1-2)
			\begin{gather*}
				3(6-2Y) - 4Y = 28\\
				18 - 6Y - 4Y = 28\\
				-10 Y = 10
				\Rightarrow Y = -1
			\end{gather*}
			نعوض في (1-4)
			\[
			X = 6-2(-1) = 6+2 = 8
			\] 
		\end{exampleblock}
	\end{frame}
	
	\begin{frame}
		\begin{exampleblock}{انظمة المعادلات في ثلاث متغيرات (1 - 8)\sen{\cite{odesolapp}}:}
			تكون بالشكل
			\begin{gather*}
				a_1 x + b_1 y + c_1 z = d_1\\
				a_2 x + b_2 y + c_2 z = d_2\\
				a_3 x + b_3 y + c_3 z = d_3
			\end{gather*}
		\end{exampleblock}
		
		\begin{exampleblock}{مثال}
			حل النظام
			\begin{gather}
				2x + 2y - 3z = 1\\
				5x + 3y - 4z = 4 \\
				7x - 3y + 2z =6
			\end{gather}
		\end{exampleblock}
	\end{frame}
	
	\begin{frame}
		\begin{exampleblock}{الحل}
				يكون الحل بطريقة الحذف. نضرب المعادلة (1-5) بــ $-3$ و المعادلة (1-6) بــ 2
			نحصل على 
			\begin{gather}
				-6x -6y +9z =-3\\
				10x + 6y -8z =8
			\end{gather}
			بجمع (1-8) و (1-9) نحصل على 
			\begin{equation}
				4x + z =5
			\end{equation}
			الآن نجمع (1-6) مع (1-7) نحصل على
			\begin{equation}
				12x -2z = 10 \Rightarrow 6x - z = 5  
			\end{equation}
			بجمع (1-10) و (1-11) نحصل على
			\[
			10x = 10 \Rightarrow x=1
			\]
			نعوض في (1-10)
			\[
			4(1) + z = 5 \Rightarrow z=1
			\]
			\end{exampleblock}
	     	\end{frame}
	     	
	     	\begin{frame}
	     		\begin{exampleblock}{}
	     			نعوض الآن عن $x, z$ في (1-5)
	     		\[
	     		2(1) + 2y - 3(1) = 1 \Rightarrow y =1
	     		\]
	     			نتحقق من ان  $x=1, y=1,z=1$ يحقق حل النظام.
	     		\begin{gather*}
	     			2(1) + 2(1)- 3(1) =1\\
	     			5(1)+3(1)-4(1)=1\\
	     			7(1)-3(1)+2(1)=1
	     		\end{gather*}
	     		\end{exampleblock}
	     		
	     		\begin{exampleblock}{3 - 1 المعنى الهندسي للنظام الخطي\sen{\cite{odesolapp}}:}
	     			النظام الخطي العام المتكون من معادلتين خطيتين بالمتغيرين $x,y$ يمثل بالصيغة التالية
	     			\begin{gather*}
	     				a_1x + b_1 y = c_1\\
	     				a_2 + b_2 y = c_2
	     			\end{gather*}
	     			ان الشكل الهندسي لهذه المعادلات هو الخطوط المستقيمة $L_1, L_2$ كما في الشكل (1 - 1)
	     			ولما كانت النقطة $(x, y)$ تقع على المستقيم اذا وفقط اذا كانت $x, y$ تحقق معادلة المستقيم فإن حلول النظام الخطي تقابل المستقيمين كما موضح في الشكل (1-1)
	     		\end{exampleblock}	
	     	\end{frame}

	\begin{frame}
		\begin{figure}[H]
			\centering
			\begin{tikzpicture}
				\node at (1.3, -2.5) {(1)};
				\draw[->, thick] (-0.3,0) -- (3, 0);
				\draw[->, thick] (0, -0.3) -- (0, 3);
				
				\draw[thick] (-1, 2) -- (2, -1) node[below]{$L_1$};
				\draw[thick] (-1, 3) -- (3, -1) node[below]{$L_2$};
				
				\node at (5.3, -2.5) {(2)};
				\draw[->, thick] (3.7,0) -- (7, 0);
				\draw[->, thick] (4, -0.3) -- (4, 3);
				
				\draw[thick] (6, 2) -- (5, -1) node[below]{$L_1$};
				\draw[thick] (3, 3) -- (7, -1) node[below]{$L_2$};
				
				\node at (9.3, -2.5) {(3)};
				\draw[->, thick] (7.7,0) -- (11, 0);
				\draw[->, thick] (8, -0.3) -- (8, 3);
				
				\draw[thick] (7, 3) -- (11, -1) node[pos=0, below]{$L_1$};
				\draw[thick] (7, 3) -- (11, -1) node[below]{$L_2$};
			\end{tikzpicture}
		\end{figure}
		\begin{center}
			الشكل (1-1)
		\end{center}
	\end{frame}
	
	\begin{frame}
		\begin{exampleblock}{}
			من خلال الشكل (1-1) يتضح ان هناك ثلاث احتمالات للحلول هي
			\begin{enumerate}
				\item المستقيمان متوازيان اي لا يوجد نقطة تقاطع وعليه فليس للنظام الخطي حل (الشكل (1) من (1-1)).
				\item يتقاطعان بنقطة واحدة وهذا يعني ان النظام الخطي له حل واحد فقط. (الشكل (2) من (1-1)).
				\item المستقيمان متطابقان اي يوجد عدد غير محدد من الحلول (الشكل (3) من (1-1)).
			\end{enumerate}
			\noindent
			\\
			نستنتج من ذلك ان اي نظام خطي اما ليس له حل او له حل وحيد او له عدد غير منته من الحلول. تسمى المجموعة المنتهية من $m$ من المعادلات الخطية التي تحتوي على $n$ من المتغيرات 
			$X_1, X_2 , \dots, X_n$
			بنظام من المعادلات الخطية.\\ 
			وتسمى ايضاً بالنظام الخطي اما المتتابعة المتكونة من $n$ من الاعداد الحقيقية $S_1 , S_2, \dots, S_n=X_n$ حلاً لكل معادلة من النظام الخطي.
		\end{exampleblock}
	\end{frame}
	
	\begin{frame}
		\begin{exampleblock}{}
			ويمكن كتابة النظام الخطي من $m$ من المعادلات التي تحتوي على $n$ من المتغيرات بالصيغة
			\begin{equation*}
				\begin{gathered}
					a_{11}X_1 + a_{12}X_2 + \cdots + a_{1n} X_n = b_1\\
					a_{21}X_1 + a_{22}X_2 + \cdots + a_{2n} X_n = b_2\\
					\vdots\\
					a_{m1}X_1 + a_{m2}X_2 + \cdots + a_{mn} X_n = b_m
				\end{gathered}
			\end{equation*}
			حيث $X_1 , X_2, \dots,X_n$ متغيرات و $a_{ij}$ ثوابت حيث 
			$i=1,2,\dots,m, j=1,2,\dots,n$
			
			\noindent
			يعتبر وضع الدليلين لمعادلة المجاهيل وسيلة مقيدة نستخدمها لتحديد موضع المعامل في النظام. يشير الدليل الايسر للمعامل توجد $a_{ij}$ الى المعادلة التي تقع فيها المعامل ويشير الدليل الايمن الى المجهول المضروب فيه. ولهذا فإن $a_{n2}$ في المعادلة الاولى تضرب في المجهول $x_2$ يمكننا كتابة النظام الخطي على الشكل
			\[
			\left(
			\begin{array}{cccc|c}
				a_{11} & a_{12} & \cdots & a_{1n} & b_1\\
				a_{21} & a_{22} & \cdots & a_{2n} & b_2\\
				\vdots&\vdots&\ddots&\vdots &\vdots\\
				a_{m1} & a_{m2} & \cdots& a_{mn} & b_m
			\end{array}
			\right)
			\]
		\end{exampleblock}
	\end{frame}
	
	\begin{frame}
		\begin{exampleblock}{}
			
			ويسمى هذا الترتيب بالمصفوفة الممتدة للنظام.\\
			مستطيل من الاعداد وتظهر المصفوفات فيه. ويستخدم لفظ مصفوفة في الرياضيات ليدل على ترتيب مقامات عديدة.\\
			\noindent
			لتوضيح ان المصفوفة الممتدة لنظام المعادلات.
			\begin{gather*}
				X + 2Y + 2Z = 4\\
				X+Y +4Z = 6\\
				2X - 6Y - 2Z =-2
			\end{gather*}
			هي
			\[
			\left(
			\begin{array}{ccc|c}
				1&2&2&4\\
				1&1&4&6\\
				2 &-6& -2& -2
			\end{array}
			\right)
			\]
		\end{exampleblock}
	\end{frame}
	
	\begin{frame}
		\begin{exampleblock}{ملاحظة}
			 	عند بناء اي مصفوفة ممتدة يجب كتابة المجاهيل بنفس الترتيب في كل معادلة.
			الطريقة الاساسية لحل جديد له نفس الحل ولكن ابسط في الحل اي نظام لمعادلات خطية هي النظام المعطى بنظام يتم الحصول بشكل عام على النظام الجديد من سلسلة من الخطوات بواسطة تطبيق الانواع الثلاثة الآتية من عمليات حذف منتظم من المجاهيل.
			\begin{enumerate}
				\item ضرب المعادلة بأكملها بثابت غير صفري.
				\item التبديل بين  اي معادلتين.
				\item اضافة مضاعف لصف آخر.
			\end{enumerate}  
			تسمى هذه العمليات بعمليات اولية على المصفوفة. سنوضح في المثال التالي كيفية استخدام هذه العمليات.
		\end{exampleblock}
		
		\begin{exampleblock}{مثال}
			
			حل النظام الخطي
			\begin{gather*}
				X-2Y + 3Z=9\\
				-X +3Y =-4\\
				2X-5Y+5Z=17
			\end{gather*}
		\end{exampleblock}
	\end{frame}
	
	\begin{frame}
		\begin{exampleblock}{الحل}
			1. المصفوفة الممتدة للنظام
			\[
			\left(
			\begin{array}{ccc|c}
				1&-2&3&9\\
				-1&3&0&-4\\
				2 &-5& 5& 17
			\end{array}
			\right)
			\]
			2. نجمع الصف الاول مع الصف الثاني. ونضرب الصف الاول بــ $-2$ ونجمعها مع الصف الثالث نحصل على
			\[
			\left(
			\begin{array}{ccc|c}
				1&-2&3&9\\
				0&1&3&5\\
				0 &-1& -1& -1
			\end{array}
			\right)
			\]
			3. نجمع الصف الثاني والثالث نحصل على
			\[
			\left(
			\begin{array}{ccc|c}
				1&-2&3&9\\
				0&1&3&5\\
				0 &0& 2& 4
			\end{array}
			\right)
			\]
		\end{exampleblock}
	\end{frame}
	
	\begin{frame}
		\begin{exampleblock}{}
			
			اي نحصل على
			\begin{align}
				X-2Y+3Z&=9\\
				Y + 3Z=5\\
				2Z=4
			\end{align}
			من معادلة (1-14) نحصل على 
			\[
			Z=2
			\]
			نعوض في (1-13) نحصل على 
			\[
			Y + 3(2) = 5 \Rightarrow Y = -1
			\]
			الآن نعوض في (1-12) نحصل على
			\[
			X - 2(-1) + 3(2) = 9 \Rightarrow X = 1
			\]
		\end{exampleblock}
	\end{frame}
	
	
	%%%% CHAPTER 2 %%%%
		\begin{frame}
		\begin{center}
			\Huge
			\textbf{الفصل الثاني}\\
			\textbf{بعض طرق حل الانظمة الخطية}
		\end{center}
	\end{frame}
	%%%%%%%%%%%%%%%
	
	\begin{frame}
		\begin{exampleblock}{2 - 1 طريقة كرامر}
			لحل النظام $AX=b$ فإن 
			\[
			x_i = \frac{|A_i|}{|A|}, \quad i = 1,2,\dots,n
			\]
			حيث $A_i$ هي المصفوفة $A$ مع استبدال العمود $i$ مع المتجه $b$.
		\end{exampleblock}
		
		\begin{exampleblock}{مثال}
			اوجد حل النظام الخطي التالي بإستخدام طريقة كرامر
			\begin{align*}
				2X_1 - 3X_2 =8\\
				3X_1 + X_2 =1
			\end{align*}
		\end{exampleblock}
	\end{frame}
	
	\begin{frame}
		\begin{exampleblock}{الحل}
			
			نكتب النظام بالشكل
			\[
			\underbrace{
				\begin{pmatrix}
					2&-3\\
					3&1
			\end{pmatrix}}_{A}
			\underbrace{
				\begin{pmatrix}
					X_1\\X_2
			\end{pmatrix}}_{X}
			=
			\underbrace{
				\begin{pmatrix}
					8\\1
			\end{pmatrix}}_{b}
			\]
			\[
			\Rightarrow |A| = 2+9=11
			\]
			\[
			A_1 =
			\begin{pmatrix}
				8&-3\\
				1&1
			\end{pmatrix}
			\Rightarrow |A_1| = 8+3=11
			\]
			\[
			\Rightarrow X_1 = \frac{|A_1|}{|A|} = \frac{11}{11}=1
			\]
			\[
			A_1 =
			\begin{pmatrix}
				2&8\\
				3&1
			\end{pmatrix}
			\Rightarrow |A_2| = 2-24=-22
			\Rightarrow X_2 = \frac{|A_2|}{|A|} = \frac{-22}{11} = -2
			\]
		\end{exampleblock}
	\end{frame}
	
	\begin{frame}
		\begin{exampleblock}{2 - 2 طريقة معكوس المصفوفة}
			ليكن $AX=b$ منظومة المعادلات الخطية مكونة من $n$ من المعادلات والمتغيرات. في حال عدد المعادلات يساوي عدد المجاهيل نستخدم القانون التالي لحل النظام
			\[
			X = A^{-1}b
			\]
			حيث $X$ قيم المتغيرات و $A$ مصفوفة المعاملات وهي قابلة للانعكاس و $b$ متجه القيم المطلقة.
		\end{exampleblock}
		
		\begin{exampleblock}{مثال}
			بإستخدام معكوس المصفوفة جد حل النظام الخطي التالي
			\begin{align*}
				4X_1 - 2X_2 = 10\\
				3X_1 - 5X_2 = 11
			\end{align*}
		\end{exampleblock}
	\end{frame}
	
	\begin{frame}
		\begin{exampleblock}{الحل}
			\[
			A = 
			\begin{pmatrix}
				4&-2\\
				3&-5
			\end{pmatrix}
			\Rightarrow |A| = -20+6=-14
			\]
			\[
			A^{-1} = \frac{1}{|A|}\cdot \text{adj}(A)
			\]
			\begin{align*}
				C_{11} = (-1)^{1+1} |M_{11}| = -5\\
				C_{12} = (-1)^{1+2} |M_{12}| = -3\\
				C_{21} = (-1)^{2+1} |M_{21}| = 2\\
				C_{22} = (-1)^{2+2} |M_{22}| = 4
			\end{align*}
			\[
			\Rightarrow C =
			\begin{pmatrix}
				-5& -3\\
				2&4
			\end{pmatrix}
			\Rightarrow C^T
			=\begin{pmatrix}
				-5&2\\
				-3&4
			\end{pmatrix}
			\]
		\end{exampleblock}
	\end{frame}
	
	\begin{frame}
		\begin{exampleblock}{}
				\begin{align*}
				A^{-1} &= \frac{1}{|A|}C^T\\
				&=\frac{1}{-14} 
				\begin{pmatrix}
					-5&2\\
					-3&4
				\end{pmatrix}\\
				&=
				\begin{pmatrix}
					5/14 & -1/7\\
					3/14 & -2/7
				\end{pmatrix}
			\end{align*}
			نحسب قيم المتغيرات $X_1, X_2 $ باستخدام القانون $X=A^{-1}b$
			\[
			X = 
			\begin{pmatrix}
				5/14 & -1/7\\
				3/14 & -2/7
			\end{pmatrix} \cdot
			\begin{pmatrix}
				10\\11
			\end{pmatrix}=
			\begin{pmatrix}
				50/14 - 11/7\\
				30/14 - 22/7
			\end{pmatrix}
			\]
			\[
			X_1 = \frac{50}{14} - \frac{11}{7} = 2
			\]
			\[
			X_2 = \frac{30}{14} - \frac{22}{7} = -1
			\]
		\end{exampleblock}
	\end{frame}
	
	\begin{frame}
		\begin{exampleblock}{طريقة كاوس جوردان للحذف}
			لحل النظام $AX=B$ بطريقة كاوس - جوردان للحذف نتبع الخطوات التالية:
			\begin{enumerate}
				\item نحول النظام الخطي الى المصفوفة الممتدة.
				\item نحول المصفوفة الممتدة الى المصفوفة المحايدة.
				\item عند تحويل المصفوفة الى مصفوفة محايدة نستخدم العمليات الصفية الاولية.
				\item نصفر العناصر الواقعة تحت القطر الرئيسي.
				\item نصفر العناصر الواقعة فوق القطر الرئيسي.
				\item نجعل عناصر القطر الرئيسي تساوي 1.
			\end{enumerate}
		\end{exampleblock}
		
		\begin{exampleblock}{مثال}
			جد حل النظام الخطي التالي
			\begin{align*}
				X-6Y = -11\\
				5X - Y = 3
			\end{align*}
		\end{exampleblock}
	\end{frame}
	
	\begin{frame}
		\begin{exampleblock}{الحل}
			\[
			(A\mid b) =
			\left(
			\begin{array}{cc|c}
				1&-6&-11\\
				5&-1&3
			\end{array}
			\right)
			\]
			نصفر العناصر الواقعة تحت القطر الرئيسي. حيث نضرب الصف الاول بــ $-5$ ونضيفه الى الصف الثاني
			\[
			R_2\to-5R_1 + R_2  \quad 
			\left(
			\begin{array}{cc|c}
				1&-6&-11\\
				0&29&58
			\end{array}
			\right)
			\]
			الآن نضرب الصف الثاني بــ $6$ ونضرب الصف الاول بــ $29$ ونجمع
			\[
			R_1 \to 6R_2 + 29R_1 \quad
			\left(
			\begin{array}{cc|c}
				29&0&29\\
				0&29&58
			\end{array}
			\right) 
			\]
			نجعل عناصر القطر الرئيسي يسواي 1.
			\[
			\begin{array}{c}
				R_1 \to \dfrac{1}{29}R_1\\[10pt]
				R_2 \to \dfrac{1}{29}R_2
			\end{array}\quad
			\left(
			\begin{array}{cc|c}
				1&0&1\\
				0&1&2
			\end{array}
			\right) 
			\Rightarrow
			\begin{array}{c}
				X = 1\\
				Y= 2
			\end{array}
			\]
		\end{exampleblock}
	\end{frame}
	
	\begin{frame}
		\begin{exampleblock}{طريقة تجزئة $LU$}
			لدينا النظام الخطي $AX=b$ ، حيث $A$ مصفوفة المعاملات و $X$ قيم المتغيرات و $b$ متجه القيم المطلقة ، لحل هذا النظام بطريقة تجزئة $LU$  نقوم أولاً بكتابة المصفوفة $A$ بالشكل 
			\begin{equation*}
				A = LU
			\end{equation*}
			حيث $L$ مصفوفة مثلثية سفلى ، و $U$ مصفوفة مثلثية عليا. بعد ذلك نعوض هذا التحليل في النظام الاصلي ليصبح لدينا 
			\[
			(LU)X = b
			\]
			ومن ثم بأستخدام خواص فضاء المتجهات نحصل على $L(UX) = b$ ، لنفرض ان $UX = Y$ فنختزل النظام الى $L Y = b$ الذي يمكن حله بطريقة التعويض الامامي للحصول على المتجه $Y$ فيصبح متجه معلوم ، ونحل النظام $UX = Y$ بالتعويض الخلفي للحصول على المتجه $X$.\\
		\end{exampleblock}
		
		\begin{exampleblock}{مثال}
			
			حل النظام الخطي التالي بطريقة تجزئة $LU$
			\begin{align*}
				2X_1+ 3X_2 &= 8\\
				4X_1 + 7X_2 &= 18
			\end{align*}
		\end{exampleblock}
	\end{frame}
	
	\begin{frame}
		\begin{exampleblock}{الحل}
			مصفوفة النظام هي 
			\[
			A = 
			\begin{bmatrix}
				2&3\\
				4&7
			\end{bmatrix}
			\]
			نحلل المصفوفة بالشكل $A = LU$ حيث
			\[
			L = 
			\begin{bmatrix}
				1 & 0\\
				a & 1
			\end{bmatrix},\quad
			U = 
			\begin{bmatrix}
				b & c\\
				0 & d
			\end{bmatrix}
			\]
			بالتالي
			\begin{align*}
				\begin{bmatrix}
					2&3\\
					4&7
				\end{bmatrix} =
				\begin{bmatrix}
					1 & 0\\
					a & 1
				\end{bmatrix}\cdot
				\begin{bmatrix}
					b & c\\
					0 & d
				\end{bmatrix}
				= 
				\begin{bmatrix}
					b & c\\
					ab & ac + d
				\end{bmatrix}
			\end{align*}
			اذن 
			\begin{align*}
				b = 2,\quad
				c&= 3\\
				ab = 4,\quad
				ac + d &= 7 
			\end{align*}
		\end{exampleblock}
	\end{frame}
	
	\begin{frame}
		\begin{exampleblock}{}
			وبالتالي
			\[
			a=2, \quad b=2, \quad c=3, \quad d=1
			\]
			\[
			\Rightarrow 
			L = 
			\begin{bmatrix}
				1&0\\
				2&1
			\end{bmatrix},
			\quad
			U=
			\begin{bmatrix}
				2&3\\
				0&1
			\end{bmatrix}
			\]
			نفرض ان $ UX = Y$ حيث 
			\[
			Y = 
			\begin{bmatrix}
				Y_1\\
				Y_2
			\end{bmatrix}
			\]
			اذن يصبح لدينا النظام $LY= b$ اي ان 
			\begin{align*}
				Y_1 &= 8\\
				2Y_1 + Y_2 &= 18
			\end{align*}
			اذن $Y_1=8, Y_2=2 $ وبالتعويض في $UX = Y$ يصبح لدينا
			\begin{align*}
				2X_1 + 3X_2 &= 8\\
				X_2 &= 2
			\end{align*}
		\end{exampleblock}
	\end{frame}
	
	\begin{frame}
		\begin{exampleblock}{}
				اذن $2X_1+3(2) = 8 $ وبالتالي $X_1=1 $ ، اذن الحل النهائي
			\[
			X_1 = 1, \quad X_2 = 2
			\]
		\end{exampleblock}
	\end{frame}
	
	%%% The References %%%%
	\begin{frame}{المصادر}
		\begin{thebibliography}{9}
			\bibitem{odesolapp}
			اسماعيل بوقفه و عايش الهناودة ، المعادلات التفاضلية حلول وتطبيقات.
			
			\bibitem{research}
			آية عبدالعالي علي زيدان ، مشروع بحث ، ليبيا جامعة سبها ، 2015 - 2016.
			
			\bibitem{odepart2}
			حسن مصطفى العويضي ، المعادلات التفاضلية الجزء الثاني ، مكتبة رشيد.
			
			\bibitem{matrix}
			مجدي الطويل ، المصفوفات ، النظرية و التطبيق ، القاهرة ، 1417 هــ - 1996 م.
			
		\end{thebibliography}
	\end{frame}
\end{document}