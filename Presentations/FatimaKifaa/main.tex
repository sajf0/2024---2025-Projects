% compile with XeLaTeX
% this template was created by salim bou 
\documentclass[dvipsnames,mathserif]{beamer}
\usepackage{setspace, float}
\setstretch{1.2}
\usepackage{tikz}
\usepackage{subcaption}

\usepackage{amsmath, amssymb, amsthm, nicematrix}

\usepackage{polyglossia}
\setdefaultlanguage[numerals=maghrib,locale=algeria]{arabic} % locale=mashriq, libya, algeria, tunisia, morocco, or mauritania  for names of months in \date 
\setotherlanguage{english}
\newfontfamily\arabicfont[Script=Arabic]{Amiri}
\newfontfamily\arabicfontsf[Script=Arabic]{Amiri}
\newfontfamily{\timesfont}{Times New Roman}

\newcommand{\ar}{\textarabic}
\newcommand{\en}{\textenglish}

\usepackage[T1]{fontenc}
\usepackage{times}

\usepackage[lite, zswash]{mtpro2}

\DeclareMathSymbol{0}{\mathalpha}{operators}{`0}
\DeclareMathSymbol{1}{\mathalpha}{operators}{`1}
\DeclareMathSymbol{2}{\mathalpha}{operators}{`2}
\DeclareMathSymbol{3}{\mathalpha}{operators}{`3}
\DeclareMathSymbol{4}{\mathalpha}{operators}{`4}
\DeclareMathSymbol{5}{\mathalpha}{operators}{`5}
\DeclareMathSymbol{6}{\mathalpha}{operators}{`6}
\DeclareMathSymbol{7}{\mathalpha}{operators}{`7}
\DeclareMathSymbol{8}{\mathalpha}{operators}{`8}
\DeclareMathSymbol{9}{\mathalpha}{operators}{`9}

\usetheme{Warsaw}
%\usecolortheme{crane}

% for RTL liste
\makeatletter
\newcommand{\RTListe}{\raggedleft\rightskip\leftm}
\newcommand{\leftm}{\@totalleftmargin}
\makeatother



% RTL frame title
\setbeamertemplate{frametitle}
{\vspace*{-1mm}
	\nointerlineskip
	\begin{beamercolorbox}[sep=0.3cm,ht=2.2em,wd=\paperwidth]{frametitle}
		\vbox{}\vskip-2ex%
		\strut\hskip1ex\insertframetitle\strut
		\vskip-0.8ex%
	\end{beamercolorbox}
}


% align subsection in toc
\makeatletter
\setbeamertemplate{subsection in toc}
{\leavevmode\rightskip=5ex%
	\llap{\raise0.1ex\beamer@usesphere{subsection number projected}{bigsphere}\kern1ex}%
	\inserttocsubsection\par%
}
\makeatother

% RTL triangle for itemize
\setbeamertemplate{itemize item}{\scriptsize\raise1.25pt\hbox{\donotcoloroutermaths$\blacktriangleleft$}} 

%\setbeamertemplate{itemize item}{\rule{4pt}{4pt}}

\defbeamertemplate{enumerate item}{square2}
{\LR{
		%
		\hbox{%
			\usebeamerfont*{item projected}%
			\usebeamercolor[bg]{item projected}%
			\vrule width2.25ex height1.85ex depth.4ex%
			\hskip-2.25ex%
			\hbox to2.25ex{%
				\hfil%
				{\color{fg}\insertenumlabel}%
				\hfil}%
		}%
}}

\setbeamertemplate{enumerate item}[square2]

\setbeamertemplate{navigation symbols}{}





\begin{document}
	\title{الرياضيات المتقطعة}
	\author{\textbf{فاطمة كفاء}}
	\date{\textbf{
اشراف \\
د. مضر عباس مجيد}}

\begin{frame}
	\maketitle
\end{frame}
\timesfont
\begin{frame}
\begin{exampleblock}{تعريف}
	البيان $G$ عبارة عن ثنائي مرتب $(V,E)$ حيث $V$ هي مجموعة غير خالية منتهية من العقد \en{Vertcies} و $E$ هي مجموعة الاسهم \en{Edges}.
\end{exampleblock}

\pause
\noindent
\textbf{ملاحظات}
\begin{enumerate}
	
		\pause
	\item كل سهم له عقدة او عقدتين مرتبطتين تسمى أطراف السهم \en{Endpoints}.
	
	\pause
	\item يسمى السهم الذي له عقدة واحدة مرتبطة به بالحلقة \en{Loop}.
	
		\pause
	\item تسمى الاسهم التي تتشارك بنفس النهايات بالاسهم المتوازية \en{Parallel}.
	
		\pause
	\item تسمى العقدتين المرتبطتين بالسهم على انهما متجاورتين \en{Adjacent}.
	
		\pause
	\item  تسمى العقدة التي ليس لها اسهم واردة بالمعزولة \en{Isolated}.
\end{enumerate}
\end{frame}
\begin{frame}

\textbf{مثال}
\begin{figure}[H]
	\centering
	\begin{tikzpicture}[->, thick]
		\node[fill=black, circle, draw, inner sep=2pt, label={left: $v_1$}] (v1) at (0,0) {};
		\node[fill=black, circle, draw, inner sep=2pt, label={left: $v_2$}] (v2) at (0,3) {};
		\node[fill=black, circle, draw, inner sep=2pt, label={right: $v_3$}] (v3) at (3,3) {};
		\node[fill=black, circle, draw, inner sep=2pt, label={right: $v_4$}] (v4) at (3,0) {};
		\node[fill=black, circle, draw, inner sep=2pt, label={right: $v_5$}] (v5) at (5,2) {};
		\node[fill=black, circle, draw, inner sep=2pt, label={right: $v_6$}] (v6) at (7,0) {};
		\node[fill=black, circle, draw, inner sep=2pt, label={right: $v_7$}] (v7) at (10,2) {};
		
		
		\path (v1) edge node[midway, above] {$e_1$} (v4);
		\path (v4) edge [loop below] node[midway, below] {$e_5$} (v4);
		\path (v3) edge node[midway, right] {$e_4$} (v4);
		\draw[->, thick] (v2) .. controls (2,2) .. (v3) node[midway, below]{$e_2$};
		\draw[->, thick] (v3) .. controls (2,4) .. (v2) node[midway, above]{$e_3$};
		\path (v6) edge node[midway,above, sloped] {$e_6$} (v7);
	\end{tikzpicture}
\end{figure}

\pause
في الشكل اعلاه السهم $e_5$ مثال على الحلقة و $e_2,e_3$ أسهم متوازية و العقدة $v_5$ مثال على العقدة المعزولة و العقدتان $v_6,v_7$ مثال على العقد المتجاورة.
\end{frame}

\begin{frame}
	content
\end{frame}
\end{document}