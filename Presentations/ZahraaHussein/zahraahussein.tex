% compile with XeLaTeX
% this template was created by salim bou 
\documentclass[dvipsnames,mathserif]{beamer}
\usepackage{setspace, float}
\setstretch{1.2}
\usepackage{tikz}
\usepackage{subcaption}

\usepackage{amsmath, amssymb, amsthm, nicematrix}

\usepackage{polyglossia}
\setdefaultlanguage[numerals=maghrib,locale=algeria]{arabic} % locale=mashriq, libya, algeria, tunisia, morocco, or mauritania  for names of months in \date 
\setotherlanguage{english}
\newfontfamily\arabicfont[Script=Arabic]{Amiri}
\newfontfamily\arabicfontsf[Script=Arabic]{Amiri}
\newfontfamily{\timesfont}{Times New Roman}

\newcommand{\ar}{\textarabic}
\newcommand{\en}{\textenglish}

\usepackage[T1]{fontenc}
\usepackage{times}

\usepackage[lite, zswash]{mtpro2}

\DeclareMathSymbol{0}{\mathalpha}{operators}{`0}
\DeclareMathSymbol{1}{\mathalpha}{operators}{`1}
\DeclareMathSymbol{2}{\mathalpha}{operators}{`2}
\DeclareMathSymbol{3}{\mathalpha}{operators}{`3}
\DeclareMathSymbol{4}{\mathalpha}{operators}{`4}
\DeclareMathSymbol{5}{\mathalpha}{operators}{`5}
\DeclareMathSymbol{6}{\mathalpha}{operators}{`6}
\DeclareMathSymbol{7}{\mathalpha}{operators}{`7}
\DeclareMathSymbol{8}{\mathalpha}{operators}{`8}
\DeclareMathSymbol{9}{\mathalpha}{operators}{`9}

\usetheme{Warsaw}
%\usecolortheme{crane}

% for RTL liste
\makeatletter
\newcommand{\RTListe}{\raggedleft\rightskip\leftm}
\newcommand{\leftm}{\@totalleftmargin}
\makeatother



% RTL frame title
\setbeamertemplate{frametitle}
{\vspace*{-1mm}
	\nointerlineskip
	\begin{beamercolorbox}[sep=0.3cm,ht=2.2em,wd=\paperwidth]{frametitle}
		\vbox{}\vskip-2ex%
		\strut\hskip1ex\insertframetitle\strut
		\vskip-0.8ex%
	\end{beamercolorbox}
}


% align subsection in toc
\makeatletter
\setbeamertemplate{subsection in toc}
{\leavevmode\rightskip=5ex%
	\llap{\raise0.1ex\beamer@usesphere{subsection number projected}{bigsphere}\kern1ex}%
	\inserttocsubsection\par%
}
\makeatother

% RTL triangle for itemize
\setbeamertemplate{itemize item}{\scriptsize\raise1.25pt\hbox{\donotcoloroutermaths$\blacktriangleleft$}} 

%\setbeamertemplate{itemize item}{\rule{4pt}{4pt}}

\defbeamertemplate{enumerate item}{square2}
{\LR{
		%
		\hbox{%
			\usebeamerfont*{item projected}%
			\usebeamercolor[bg]{item projected}%
			\vrule width2.25ex height1.85ex depth.4ex%
			\hskip-2.25ex%
			\hbox to2.25ex{%
				\hfil%
				{\color{fg}\insertenumlabel}%
				\hfil}%
		}%
}}

\setbeamertemplate{enumerate item}[square2]

\setbeamertemplate{navigation symbols}{}






\title{
\textbf{ بعض مؤثرات لوباس}
}

\author{
\textbf{الطالبة : زهراء حسين سموم}
}
\date{
\textbf{إشراف : م.م. تهاني عبدالمجيد}
}

\begin{document}
	\begin{frame}
		\maketitle
	\end{frame}
	
	\timesfont
	\begin{frame}{مقدمة}
		في الكثير من التطبيقات العلمية في الرياضيات أو الهندسة والتكنولوجيا، يمكن أن نحصل على دوال معقدة وأحيانًا غير مألوفة، وبذلك تصبح عملية دراسة هذه الدوال من حيث اتصالها وقابليتها للتفاضل أو التكامل وغيره من الموضوعات الصعبة جدًا والتي تستغرق الكثير من الوقت.\\
		\noindent  		
		لكل هذه الأسباب وغيرها، يكون من الضروري والمفيد استبدال دالة بسيطة وغير معقدة بدالة ذات شكل رياضي معقد أو دالة بيانات بحيث يمكن اعتبارها بديلاً للدالة المعطاة. في الواقع، أن \textbf{نظرية التقريب} تتعامل مع كل المفاهيم السابقة.\\
		\noindent
		والتقريب يمكن أن يكون بواسطة الدوال كثيرات حدود جبرية أو غير جبرية أو بواسطة دوال أسية أو دوال مثلثية وغيرها.
		
	\end{frame}
	
	\begin{frame}
\begin{center}
			\Huge\textbf{مؤثر لوباس الاعتيادي}
\end{center}
	\end{frame}
	
	\begin{frame}{مؤثر لوباس الاعتيادي}
	\begin{exampleblock}{تعريف}

		
		تعرف متتابعة من المؤثرات الخطية الموجبة من الفضاء $C_h[0, \infty)$ إلى نفسه كما يلي:
		\[
		L_n(f(t); x) = \sum_{k=0}^{\infty} p_{n,k} (x) f\left(\frac{k}{n}\right)
		\]
		حيث:
		\[
		p_{n,k} (x) = \binom{n+k-1}{k} x^k (1+x)^{-(n+k)}, \quad x \in [0, \infty)
		\]
	\end{exampleblock}
	\end{frame}
	
	\begin{frame}
		\begin{center}
			\Huge
			\textbf{بعض النتائج المباشرة للدالة $p_{n,k}(x)$}
		\end{center}
	\end{frame}
	
	\begin{frame}
		\begin{exampleblock}{بعض النتائج المباشرة للدالة $p_{n,k}(x)$}
			\begin{english}
				\begin{enumerate}
					\item $\sum_{k=0}^{\infty} p_{n,k}(x) = 1$
					\item $\sum_{k=0}^{\infty}k p_{n,k}(x) = nx$
					\item $\sum_{k=0}^{\infty}k^2 p_{n,k}(x) = n^2 x^2 + nx^2 + nx$
				\end{enumerate}
			\end{english}
		\end{exampleblock}
	\end{frame}
	\newcommand{\tL}{\tilde{L}}
	\begin{frame}{تعريف المؤثر $\tL_n(f(t);x)$}
		\begin{exampleblock}{تعريف}
			تعرف متتابعة من المؤثرات الخطية الموجبة من الفضاء $C_h[0, \infty)$ الى نفسه كما يلي
			\[
			\tL_n(f(t); x) = \sum_{k=0}^{\infty} p_{n, k}(x) f\left(\frac{k+\alpha}{n+\beta}\right)
			\]
			حيث
			\[
			p_{n,k} (x) = \binom{n+k-1}{k} x^k (1+x)^{-(n+k)}, \quad x \in [0, \infty)
			\]
			و حيث ان $0 \leq \alpha\leq \beta$
			
		\end{exampleblock}
	\end{frame}
	
	\begin{frame}
\begin{center}
	\Huge\textbf{مبرهنة كورفكن للمؤثر $\tL_n(f(t); x)$}
\end{center}
	\end{frame}
	
	\begin{frame}{مبرهنة كورفكن للمؤثر $\tL_n(f(t); x)$}
		\begin{exampleblock}{نظرية}	
		لتكن $\tL_n(f(t);x)$ متتابعة من المؤثرات الخطية الموجبة \en{\textit{L.P.O}} و الشروط التالية متحققة
		\begin{english}
			\begin{enumerate}
				\item $\tL_n(1;x)=1$
				\item $\tL_n(t;x) = \dfrac{nx + \alpha}{n + \beta} \to x$
				\item $\tL_n(t^2;x) = \dfrac{n^2x^2 + nx^2 + 2\alpha n x + \alpha^2}{n^2 + 2n\beta+ \beta^2} \to x^2$
				\[
				\Rightarrow\tL_n(f(t);x) \to f(x) \quad\text{as}\quad n\to \infty
				\]
			\end{enumerate}
		\end{english}
		\end{exampleblock}
	\end{frame}
	
	\begin{frame}
	\begin{center}
		\Huge\textbf{مؤثر لوباس من نوع مجموع - تكامل}
	\end{center}
\end{frame}
	
	\begin{frame}{تعريف المؤثر $B_n(f(t); x)$}
		\begin{exampleblock}{تعريف}
			متتابعة من المؤثرات الخطية الموجبة من الفضاء $C_h[0, \infty)$ الى نفسه كما يلي
			\[
			B_n(f(t); x) = (n-1)\sum_{k=0}^{\infty} p_{n, k}(x) \int_{0}^{\infty} p_{n, k}(t) f\left(\frac{nt + \alpha}{n + \beta}\right) \, dt
			\]
			حيث $0 \leq \alpha\leq \beta$
		\end{exampleblock}
	\end{frame}
	
	\begin{frame}
		\begin{center}
			\Huge\textbf{مبرهنة كورفكن للمؤثر $B_n(f(t); x)$}
		\end{center}
	\end{frame}
	
	\begin{frame}{مبرهنة كورفكن للمؤثر $B_n(f(t); x)$}
		\begin{exampleblock}{نظرية}
			لتكن $B_n(f(t);x)$ متتابعة من المؤثرات الخطية الموجبة \textit{L.P.O} والشروط التالية متحققة 
			\begin{english}
				\begin{enumerate}
					\item $B_n(1; x) =1$
					\item $B_n(t; x) = \frac{n^2x + n}{(n+ \beta)(n-2)} + \frac{\alpha}{n + \beta} \to x$
					\item $B_n(t^2; x) = \frac{n^4x^2 + n^3x^2 + 3n^3x + n^3 x + 2n}{(n+\beta)^2 (n-2)(n-2)} + \frac{2\alpha x n^2 + 2n \alpha}{(n+\beta)^2 (n-2)} + \frac{\alpha^2 }{(n+\beta)^2} \to x^2$
					\[
					\Rightarrow B_n(f(t); x) \to f(x)\,\, \text{ as $n \to \infty$}
					\]
				\end{enumerate}
			\end{english}
		\end{exampleblock}
	\end{frame}
\newcommand{\tB}{\tilde{B}}	
\begin{frame}
	\begin{center}
		\Huge\textbf{تعريف المؤثر $\tB_n(f(t); x)$}
	\end{center}
\end{frame}

\begin{frame}
	\begin{exampleblock}{تعريف}
		تعرف متتابعة من المؤثرات الخطية الموجبة من الفضاء $C_h[0, \infty)$ الى نفسه كما يلي
		\[
		\tB_n(f(t); x) = (n-1)\sum_{k=0}^{\infty} p_{n+2, k}(x) \int_{0}^{\infty} p_{n, k+1}(t) f(t) \, dt
		\]
	\end{exampleblock}
\end{frame}

\begin{frame}
	\begin{center}
		\Huge\textbf{مبرهنة كورفكن للمؤثر $\tB_n(f(t); x)$}
	\end{center}
\end{frame}

\begin{frame}{مبرهنة كورفكن للمؤثر $\tB_n(f(t); x)$}
	\begin{exampleblock}{نظرية}
		لتكن $\tB_n(f(t); x)$ متتابعة من المؤثرات الخطية الموجبة \en{\textit{L.P.O}} تحقق الشروط التالية
		\begin{english}
			\begin{enumerate}
				\item $\tB_n(1; x) = 1$
				\item $\tB_n(t; x) = \dfrac{(n+2)x+2}{n-2}\to x$
				\item $\tB_n(t^2; x) = \dfrac{(n+2)(n+3)x^2 + 6(n+2)x + 6}{(n-3)(n-2)}\to x^2$
				\[
				\Rightarrow \tB(f(t); x) \to f(x)\quad \text{as}\quad n \to \infty
				\]
			\end{enumerate}
		\end{english}
	\end{exampleblock}
\end{frame}

\begin{frame}
	\begin{center}
		\Huge
		\textbf{\LR{Thanks}}\\
		\textbf{شكراً لإصغائكم}
	\end{center}
\end{frame}
\end{document}