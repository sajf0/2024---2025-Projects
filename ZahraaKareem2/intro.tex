\chapter*{مقدمة}
\addcontentsline{toc}{chapter*}{مقدمة}
\noindent
المصفوفات هي إحدى أهم البنى الرياضية التي تُستخدم على نطاق واسع في الرياضيات والهندسة والعلوم التطبيقية. تتكون المصفوفة من مجموعة من الأعداد أو الرموز مرتبة في صفوف وأعمدة داخل جدول مستطيل الشكل، مما يسهل التعامل مع البيانات وتمثيل الأنظمة المعقدة بطريقة منظمة وفعالة.\\
\noindent
تلعب المصفوفات دورًا حيويًا في العديد من المجالات، مثل الجبر الخطي، الإحصاء، الفيزياء، والهندسة. ومن أبرز تطبيقاتها:
\begin{itemize}[leftmargin=*]
	\item \textbf{تمثيل وحل الأنظمة الخطية:} حيث تُستخدم لحل مجموعة من المعادلات الخطية بطريقة مبسطة باستخدام العمليات المصفوفية مثل ضرب المصفوفات وعكسها.
	\item \textbf{تحليل البيانات والذكاء الاصطناعي:} إذ تُستخدم في معالجة الصور، التعلم العميق، وتحليل البيانات الكبيرة.
	\item \textbf{فيزياء الكم والرسومات الحاسوبية:} حيث تُستخدم في تدوير الأجسام وتحويل الإحداثيات في الفضاء ثلاثي الأبعاد.
\end{itemize}