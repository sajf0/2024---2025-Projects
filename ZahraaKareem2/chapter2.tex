\chapter{}
\section{ايجاد معادلة المستقيم المار بنقطة معطاة}
الخط المستقيم تم تعريفه هندسياً على انه خط لا بداية له ولا نهاية له ولا يوجد له طول معين ايضاً وبعد ذلك تم الوصول بأن هذا التعريف خاطئ لان المستقيم في اعتقاد بعض العلماء له بداية وله نهاية لانه يتواجد في كل زاوية حولنا.

\subsection*{المنحني المستوي}
هو منحنى يقع في مستوي ما. قد يكون المنحني المستوي مغلقاً او مفتوحاً. المنحنيات التي تكون مثيرة للاهتمام لسبب ما والتي تم التحقيق في خصائصها تسمى المنحنيات الخاصة.\\
بعض المنحنيات الاكثر شيوعاً هي الخط والقطع الزائد وبعض المنحنيات المغلقة الاكثر شيوعاً هي الدائرة والقطع الناقص.

\subsection*{ميل الخط المستقيم} 
الميل من اهم خصائص الخط المستقيم ويرمز له بالحرف $m$. يصف الميل مدى انحدار
هذا الخط المستقيم عن المحور الافقي (محور السينات او محور $X$) سواء اتجه نحو الاعلى  او انخفض

\section{ايجاد معادلة خط المستقيم بواسطة ميل معطى ونقطة}
\[
y = mx + b
\]
حيث $m$: ميل الخط المستقيم و $b$: العدد ؟\\ [10pt]
\noindent
\textbf{مثال}\\
\noindent
جد معادلة الخط المستقيم الذي ميله 2 ويمر بالنقطة 
$(-1,1)$\\
\noindent
\textbf{الحل}\\
\noindent
$y = mx + b$. الميل $m=2$ اذن $y = 2x + b$. نعوض النقطة $(-1,1)$ في معادلة الخط المستقيم
\begin{gather*}
	-1 = 2(1) + b\\
	-1 = 2+b\\
	b = -1-2 = -3
\end{gather*}
اذن المعادلة $y=2x-3$\\ [10pt]
\textbf{حساب ميل الخط المستقيم}
\[
m = \frac{\Delta y}{\Delta x} = \frac{y_2 - y_1}{x_2 - x_1}
\]
بحيث أن $m$ ميل الخط المستقيم 
\[
y = mx + b
\]
و $(x_1, y_1), (x_2, y_2)$ نقطتان تقع على خط المستقيم\\
\noindent
\textbf{مثال}\\
\noindent
جد ميل المستقيم الذي تقع عليه النقطتين $A(2,4)$ و $B(6,8)$\\
\noindent
\textbf{الحل}
\[
m = \frac{y_2 - y_1}{x_2 - x_1} = \frac{8-4 }{6-2} = \frac{4}{4} = 1
\]

\noindent
\textbf{مثال}\\
\noindent
جد معادلة المستقيم المار بالنقطتين $(2, 5), (4, 7)$

\noindent
\textbf{الحل}\\
\noindent
معادلة المستقيم هي $y = mx + b$. نجد الميل
\begin{align*}
	m &= \frac{y_2-y_1}{x_2-x_1}\\
	&= \frac{7-5}{4-2}\\
	&= \frac{2}{2}=1
\end{align*}
نعوض الميل و احدى النقطتين في معادلة المستقيم
\begin{gather*}
y = mx + b\\
5 = 1(2) + b\\
b = 5-2 \Rightarrow b=3
\end{gather*}
اذن معادلة المستقيم هي $ y = x+3 $ 
\section{ايجاد معادلة المستقيم المار بنقطتين}
اذا كانت لدينا نقطتان $(x_1, y_1)$ و $(x_2, y_2)$ في المستوي ونريد ايجاد معادلة المستقيم المار بهما فإن هذه المعادلة هي على الصورة
\[
a_1 x + a_2 y + a_3 = 0
\]
حيث $a_1,a_2,a_3$ اعداد حقيقية لا تساوي صفر\\
اذا كانت $(x,y)$ اي نقطة على هذا المستقيم فإننا نحصل على نظام المعادلات المتجانس
\begin{align*}
	a_1 x + a_2 y + a_3 &= 0\\
	a_1 x_1 + a_2 y_1 + a_3 &= 0\\
	a_1 x_2 + a_2 y_2 + a_3 &= 0
\end{align*}
حيث $a_1, a_2, a_3$ مجاهيل. لذا فإن محدد مصفوفة المعاملات للنظام يجب ان يساوي صفراً، أي ان
\[
\begin{vmatrix}
	x & y & 1\\
	x_1 & y_1 & 1\\
	x_2 & y_2 & 1\\
\end{vmatrix} =0
\]
\noindent
\textbf{مثال}\\
\noindent
استخدم المحددات لايجاد معادلة المستقيم المار بالنقطتين
$(-1, 1), (2,3)$\\
\noindent
\textbf{الحل}\\
\noindent
\[
\begin{vmatrix}
	x & y & 1\\
	2 &3 & 1\\
	-1 & 1 & 1\\
\end{vmatrix} =0
\Rightarrow
\begin{array}{|ccc|cc}
	\Rnode{a1}{x} &\Rnode{b1}{y} & \Rnode{c1}{1} & \Rnode{A1}{x} & \Rnode{B1}{y} \\
	\Rnode{a2}{2} & \Rnode{b2}{3} & \Rnode{c2}{1} & \Rnode{A2}{2}& 3 \\
	\Rnode{a3}{-1} & \Rnode{b3}{1} & \Rnode{c3}{1} & \Rnode{A3}{-1} & \Rnode{B3}{1}
\end{array} = 0
\psset{linecolor=red, nodesepA=0.5pt, nodesepB = 0.5pt, arrowinset=0.1}
\foreach \s/\t/\u in {a1/b2/c3,b1/c2/A3,c1/A2/B3} {\ncline{-}{\s}{\t}\ncline{->}{\t}{\u}}
\psset{linecolor=blue}
\foreach \s/\t/\u in {a3/b2/c1,b3/c2/A1,c3/A2/B1} {\ncline{-}{\s}{\t}\ncline{->}{\t}{\u}}
\]
\[
\Rightarrow (3x-y +2) - (2y+x-3) = 0 \Rightarrow 2x-3y + 5=0
\]
\newpage
\noindent
\textbf{مثال}\\
\noindent
استخدم المحددات لايجاد معادلة المستقيم المار بالنقطتين
$(2,7), (2,5)$\\
\noindent
\textbf{الحل}\\
\noindent
\[
\begin{vmatrix}
	x & y & 1\\
	2 &5 & 1\\
	2 & 7 & 1\\
\end{vmatrix} =0
\Rightarrow
\begin{array}{|ccc|cc}
	\Rnode{a1}{x} &\Rnode{b1}{y} & \Rnode{c1}{1} & \Rnode{A1}{x} & \Rnode{B1}{y} \\
	\Rnode{a2}{2} & \Rnode{b2}{5} & \Rnode{c2}{1} & \Rnode{A2}{2}& 5 \\
	\Rnode{a3}{2} & \Rnode{b3}{7} & \Rnode{c3}{1} & \Rnode{A3}{2} & \Rnode{B3}{7}
\end{array} = 0
\psset{linecolor=red, nodesepA=0.5pt, nodesepB = 0.5pt, arrowinset=0.1}
\foreach \s/\t/\u in {a1/b2/c3,b1/c2/A3,c1/A2/B3} {\ncline{-}{\s}{\t}\ncline{->}{\t}{\u}}
\psset{linecolor=blue}
\foreach \s/\t/\u in {a3/b2/c1,b3/c2/A1,c3/A2/B1} {\ncline{-}{\s}{\t}\ncline{->}{\t}{\u}}
\]
\[
\Rightarrow (5x+2y +14) - (10+7x+2y) = 0 \Rightarrow -2x+2=0
\]

\noindent
\textbf{مثال}\\
\noindent
استخدم المحددات لايجاد معادلة المستقيم المار بالنقطتين
$(3,4), (1,4)$\\
\noindent
\textbf{الحل}\\
\noindent
\[
\begin{vmatrix}
	x & y & 1\\
	2 &3 & 1\\
	-1 & 1 & 1\\
\end{vmatrix} =0
\Rightarrow
\begin{array}{|ccc|cc}
	\Rnode{a1}{x} &\Rnode{b1}{y} & \Rnode{c1}{1} & \Rnode{A1}{x} & \Rnode{B1}{y} \\
	\Rnode{a2}{3} & \Rnode{b2}{4} & \Rnode{c2}{1} & \Rnode{A2}{3}& 4 \\
	\Rnode{a3}{1} & \Rnode{b3}{4} & \Rnode{c3}{1} & \Rnode{A3}{1} & \Rnode{B3}{4}
\end{array} = 0
\psset{linecolor=red, nodesepA=0.5pt, nodesepB = 0.5pt, arrowinset=0.1}
\foreach \s/\t/\u in {a1/b2/c3,b1/c2/A3,c1/A2/B3} {\ncline{-}{\s}{\t}\ncline{->}{\t}{\u}}
\psset{linecolor=blue}
\foreach \s/\t/\u in {a3/b2/c1,b3/c2/A1,c3/A2/B1} {\ncline{-}{\s}{\t}\ncline{->}{\t}{\u}}
\]
\[
\Rightarrow (4x+y +12) - (4+4x+3y) = 0 \Rightarrow -2y+8=0
\]

\section{معادلة الدائرة}
نعلم من الدروس الهندسة اي ثلاث نقاط
$(x_1, y_1), (x_2, y_2), (x_3, y_3)$
وليست على استقامة واحدة تعين دائرة وحيدة وأن معادلة الدائرة هي على الصورة
\[
a_1(x^2 + y^2) + a_2 x + a_3 y + a_4 =0 
\]
حيث $a_1\neq0$\\
اذا كانت $(x,y)$ اي نقطة على الدائرة فإننا نحصل على النظام المتجانس
\begin{align*}
	a_1(x^2 + y^2) + a_2 x + a_3 y + a_4 =0 \\
	a_1(x_1^2 + y_1^2) + a_2 x_1 + a_3 y_1 + a_4 =0 \\
	a_1(x_2^2 + y_2^2) + a_2 x_2 + a_3 y_2 + a_4 =0\\ 
	a_1(x_3^2 + y_3^2) + a_2 x_3 + a_3 y_3 + a_4 =0 
\end{align*}
حيث $a_1,a_2,a_3,a_4$ مجاهيل وإننا نبحث عن حل غير تافه للنظام.\\
\noindent
\textbf{مثال}\\
\noindent
استخدم المحددات لايجاد معادلة الدائرة المارة بالنقاط
$(-1, 1), (1,-1), (1,0)$\\
\noindent
\textbf{الحل}
\begin{align*}
	\begin{vmatrix}
		x^2+y^2 & x & y & 1\\
		1 & 1 & 0 & 1\\
		2 & 1 & -1 & 1 \\
		2 & -1 & 1 & 1
	\end{vmatrix} & = (x^2 +y^2)
	\begin{vmatrix}
		1&0&1\\
		1&-1&1\\
		-1&1&1
	\end{vmatrix}
	-x
	\begin{vmatrix}
		1&0&1\\
		2&-1&1\\
		2&1&1
	\end{vmatrix}\\
	& \quad +y
	\begin{vmatrix}
		1&1&1\\
		2&1&1\\
		2&-1&1
	\end{vmatrix}
	-
	\begin{vmatrix}
		1&1&0\\
		2&1&-1\\
		2&-1&1
	\end{vmatrix} =0
\end{align*}
اي ان 
\[
(x^2 + y^2)(-2) -x(2) + y(-2) - (-4) =0 
\]
\[
x^2 + y^2 + x + y-2=0
\]

\noindent
\textbf{مثال}\\
\noindent
استخدم المحددات لايجاد معادلة الدائرة المارة بالنقاط
$(2, 1), (1,2), (0,0)$\\
\noindent
\textbf{الحل}
\begin{align*}
	\begin{vmatrix}
		x^2+y^2 & x & y & 1\\
		5 & 2 & 1 & 1\\
		5 & 1 & 2 & 1 \\
		0 & 0 & 0 & 1
	\end{vmatrix} & = (x^2 +y^2)
	\begin{vmatrix}
		2&1&1\\
		1&2&1\\
		0&0&1
	\end{vmatrix}
	-x
	\begin{vmatrix}
		5&1&1\\
		5&2&1\\
		0&0&1
	\end{vmatrix}\\
	& \quad +y
	\begin{vmatrix}
		5&2&1\\
		5&1&1\\
		0&0&1
	\end{vmatrix}
	-
	\begin{vmatrix}
		5&2&1\\
		5&1&2\\
		0&0&0
	\end{vmatrix} =0
\end{align*}
اي ان 
\[
3x^2 + 3y^2 -5x -5y = 0
\]

\noindent
\textbf{مثال}\\
\noindent
استخدم المحددات لايجاد معادلة الدائرة المارة بالنقاط
$(-1, 2), (1,3), (2,0)$\\
\noindent
\textbf{الحل}
\begin{align*}
	\begin{vmatrix}
		x^2+y^2 & x & y & 1\\
		4 & 2 & 0 & 1\\
		1 & 1 & 3 & 1 \\
		1 & -1 & 2 & 1
	\end{vmatrix} & = (x^2 +y^2)
	\begin{vmatrix}
		2&0&1\\
		1&3&1\\
		-1&2&1
	\end{vmatrix}
	-x
	\begin{vmatrix}
		4&0&1\\
		1&3&1\\
		1&2&1
	\end{vmatrix}\\
	& \quad +y
	\begin{vmatrix}
		4&2&1\\
		1&1&1\\
		1&-1&1
	\end{vmatrix}
	-
	\begin{vmatrix}
		4&2&0\\
		1&1&3\\
		1&-1&2
	\end{vmatrix} =0
\end{align*}
اي ان 
\[
7x^2+7y^2 -3x + 6y + 22 =0
\]
