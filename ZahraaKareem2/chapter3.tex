\section*{القطع المكافئ}

القطع المكافئ هو المحل الهندسي لمجموعة نقاط المستوي التي يكون بُعد كل منها عن نقطة ثابتة تسمى البؤرة مساوياً دائماً لبعدها عن مستقيم معلوم يسمى الدليل.

والقطع المكافئ متماثل حول المستقيم العمودي على الدليل والمار بالبؤرة ويسمى هذا المستقيم محور التماثل وتسمى نقطة تقاطع القطع المكافئ مع محور التماثل بالرأس وتسمى القطعة المستقيمة المارة بالبؤرة والعمودية على محور التماثل بالوتر البؤري ويقطع طرفا الوتر البؤري على القطع المكافئ.

\section{خصائص القطع المكافئ}

\subsection*{1. القطع المكافئ المفتوح رأسياً الى اعلى او الى اسفل}
\textbf{معادلة القطع:} \quad $(x-h)^2 = 4p(y-k)$\\[10pt]
\textbf{خصائص القطع:} 
\begin{enumerate}
	\item الاتجاه: \quad المنحني مفتوح رأسياً.
	\item الرأس: \quad $(h, k)$.
	\item البؤرة: \quad $(h, k+p)$
	\item معادلة محور التماثل:  \quad $x=h$
	\item معادلة الدليل:\quad $y=k+p$
	\item طول الوتر البؤري: \quad $|4p|$
\end{enumerate}

\subsection*{2. القطع المكافئ المفتوح افقياً الى اليمين او الى اليسار}
\textbf{معادلة القطع:} \quad $(y-k)^2 = 4p(x-h)$\\[10pt]
\textbf{خصائص القطع:} 
\begin{enumerate}
	\item الاتجاه: \quad المنحني مفتوح رأسياً.
	\item الرأس: \quad $(h, k)$.
	\item البؤرة: \quad $(h+p, k)$
	\item معادلة محور التماثل:  \quad $y=k$
	\item معادلة الدليل:\quad $x=h-p$
	\item طول الوتر البؤري: \quad $|4p|$
\end{enumerate}
\vspace{10pt}
\noindent
\textbf{مثال 1}\\
\noindent
جد معادلة القطع المكافئ المار بالنقاط 
$(2, 0), (1, 3), (-1,2)$.
\noindent
\textbf{الحل}\\
\noindent
معادلة القطع المكافئ العامة 
\begin{equation}
(x-h)^2 = 4p(y-k)
\end{equation}
بما ان النقاط تمر بالقطع فإنها تحقق المعادلة (1)
\begin{align*}
	(2-h)^2 = 4p(0-k)\\
	(1-h)^2 = 4p(3-k)\\
	(-1-h)^2 = 4p(2-k)
\end{align*}
بترتيب المعادلات نحصل على 
\begin{align}
	(h-2)^2 &= -4pk\\
	(h-1)^2 &= 12p - 4pk\\
	(h+1)^2 &= 8p - 4pk
\end{align}
بطرح المعادلتين (3) و (4) نحصل على
\[
(h-1)^2 - (h+1)^2 = 4p
\]
\[
(h^2 -2h+1)-(h^2+2h+1) = 4p
\]
\[
-4h = 4p \Rightarrow \boxed{p = - h}
\]
نعوض في (2) و (4)
\begin{align*}
	(h-2)^2 &= 4hk\\
	(h+1)^2 &= 4hk - 8h 
\end{align*}
بعد التبسيط نحصل على 
\begin{align}
	h^2 -4h + 4 &= 4hk\\
	h^2+2h+1 &= 4hk - 8h  
\end{align}
بطرح المعادلتين (5) و (6) نحصل على
\[
-6h + 3 = 8h \Rightarrow 14h = 3 \Rightarrow \boxed{h = \frac{3}{14}}
\]
بما ان $p = -h$ فإن 
\[
\boxed{p = \frac{-3}{14}}
\]
نعوض عن قيمة $h,p$ في (2) للحصول على قيمة $k$
\[
\left(\frac{3}{14} - 2\right)^2 = - 4 \frac{-3}{14} k \Rightarrow \boxed{k = \frac{625}{168}}
\]
اذن معادلة القطع المكافئ
\[
\left(x- \frac{3}{14}\right)^2 = \frac{-6}{7} \left(y - \frac{625}{168}\right)
\]
بعد التبسيط وفتح الاقواس
\[
7x^2 - 3x + 6y - 22=0
\]
\noindent
\textbf{مثال 2}\\
\noindent
المعادلة $ax^2+bx+cy+d=0$، $a\neq0$،$c\neq0$ تصف قطعاً مكافئاً
\begin{enumerate}	
	\item باستخدام المحددات عين صيغة لمعادلة القطع المكافئ المار بالنقاط $(x_1, y_1) , (x_2, y_2) , (x_3, y_3)$ التي لا تقع على استقامة واحدة.
	\item استخدم (1) لايجاد معادلة القطع المكافئ المار بالنقاط 
	$(2, 0), (1, 3), (-1,2)$.
\end{enumerate} 

\noindent
\textbf{الحل}\\
\noindent
1. المعادلة العامة 
\[
ax^2+bx+cy+d=0
\]
اذا كانت النقاط $(x_1,y_1),(x_2,y_2),(x_3,y_3)$ تمر بالقطع المكافئ نحصل على النظام المتجانس
\begin{align*}
	ax^2+bx+cy+d=0\\
	ax_1^2+bx_1+cy_1+d=0\\
	ax_2^2+bx_2+cy_2+d=0\\
	ax_3^2+bx_3+cy_3+d=0
\end{align*}
باستخدام المحددات
\[
\begin{vmatrix}
	x^2  & x & y & 1\\
	x_1^2  & x_1 & y_1 & 1\\
	x_2^2  & x_2 & y_2 & 1\\
	x_3^2  & x_3 & y_3 & 1
\end{vmatrix} = 0
\]
2. نعوض النقاط 
$(-1,2), (1,3), (2,0)$
في المحدد
\[
\begin{vmatrix}
	x^2 & x& y& 1\\
	4 & 2 & 0 & 1\\
	1 & 1& 3 & 1\\
	1 & -1 & 2&1
\end{vmatrix} = 0
\]
\[
x^2\begin{vmatrix}
	2 & 0&1\\
	1&3&1\\
	-1&2&1
\end{vmatrix}
-x \begin{vmatrix}
	4&0&1\\
	1&3&1\\
	1&2&1
\end{vmatrix}
+ y\begin{vmatrix}
	4&2&1\\
	1&1&1\\
	1&-1&1
\end{vmatrix}
-\begin{vmatrix}
	4&2&0\\
	1&1&3\\
	1&-1&2
\end{vmatrix} = 0
\]
\begin{align*}
	x^2(6+&0+2+3-4-0) - x(12+0+2-3-8-0)\\
	&+ y (4+2-1-1+4-2) - (8+6-0-0+12-4) = 0
\end{align*}
\[
7x^2 - 3x + 6y - 22=0
\]