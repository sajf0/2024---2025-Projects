\chapter{مفاهيم أساسية}

\section{المصفوفات}
يمكن تعريف المصفوفات بأنها ترتيب معين للاعداد على شكل اعمدة وصفوف . تكتب المصفوفات عادة على شكل صندوق مربع او مستطيل الشكل ويسمى الخط العمودي داخل المصفوفة بالعمود اما الخط الافقي فيسمى صفاً ويمكن التعبير عن حجم المصفوفة من خلال عدد الصفوف والاعمدة التي تحتويها كما يلي

\section{حجم المصفوفة}
هو عدد الصفوف وعدد الاعمدة فمثلاً اذا كان عدد الصفوف في مصفوفة ما هو 2 وعدد الاعمدة هو 3 فإنه يتم التعبير عن حجمها بـــ $2\times3$ وهكذا. وتعرف المصفوفات والاعمدة بأبعاد المصفوفة.

\section{خصائص المصفوفة}
يعرف كل ما يوجد داخل المصفوفة بعناصر المصفوفة سواء كانت ارقام او رموز او مقادير جبرية وابرز هذه الخصائص
\begin{enumerate}
	\item اذا كان عدد صفوف واعمدة احدى المصفوفات مساوياً لعدد صفوف واعمدة مصفوفة اخرى فإن هاتين المصفوفتين تعتبران متساويتين بالحجم 
	\item يمكن تسمية المصفوفة بأي حرف من احرف اللغة العربية اما في اللغة الانكليزية فيتم التعبير عنها باستخدام احد الاحرف الكبيرة.
	\item ما داخل المصفوفة اي العناصر فيتم التعبير عنها عن طريق كتابة الحرف الذي يعبر عن اسم المصفوفة وكتابة رقم كل من الصف والعمود لذلك العنصر على الترتيب اسفل ذلك الحرف اي اسم المصفوفة صف وعمود.
\end{enumerate}

\section{العمليات الحسابية على المصفوفات}
\textbf{1. جمع وطرح المصفوفات}\\
يجب عند جمع او طرح المصفوفات ان تكون متساوية في الحجم اي يجب لعدد الصفوف والاعمدة ان يكون متساوياُ في كلا المصفوفتين.
\newpage
\noindent
\textbf{مثال توضيحي:} اذا كان عدد الصفوف في مصفوفة ما 3 صفوف و 5 اعمدة فإنه يمكن جمعها مع مصفوفة اخرى اذا كان عدد صفوفها 3 صفوف وعدد اعمدتها 5. وفي المقابل لا يمكن مثلاً جمعها الى مصفوفة اخرى عدد الصفوف فيها 3 وعدد اعمدتها 4.\\
ويتم الجمع عن طريق جمع كل عنصر متطابقين في الموقع بين المصفوفتين وكذلك الامر في عملية الطرح.\\ [10pt]
\noindent
\textbf{مثال}
\[
\begin{bmatrix}
	3 & 8\\
	4 & 6
\end{bmatrix}_{2\times2}
+
\begin{bmatrix}
	4 & 0\\
	1 & -9
\end{bmatrix}_{2\times2}
=
\begin{bmatrix}
	7 & 8\\
	5 & -3
\end{bmatrix}_{2\times2}
\]
\vspace{10pt}
\[
\begin{bmatrix}
	-1 & 3 & -3\\
	5  & 2 & 0
\end{bmatrix}_{2\times3}
-
\begin{bmatrix}
	7 & -3 & 5\\
	2  & 1 &11
\end{bmatrix}_{2\times3}
=
\begin{bmatrix}
	-8 & 6 & 2\\
	3  & 1 & -11
\end{bmatrix}_{2\times3}
\]
\textbf{2. ضرب المصفوفات}\\
يمكن ضرب مصفوفتين ببعضهما فقط اذا كان عدد الاعمدة في المصفوفة الاولى مساوياً لعدد الصغوف في المصفوفة الثانية ليكون حجم المصفوفة الناتجة هو عدد صفوف المصفوفة الاولى × عدد اعمدة المصفوفة الثانية\\
\noindent
\textbf{مثال}
\[
A = 
\begin{bmatrix}
	2 & 3 & 1\\
	2 & -7 &4
\end{bmatrix}_{2\times 3}
, B =
\begin{bmatrix}
	3 &4& 5\\
	1 &1 &4\\
	2 &1& 4 
\end{bmatrix}_{3\times 3}
\]
حجم المصفوفة الناتجة 
\[
[A]_{2\times \textcolor{red}{3}}\cdot [B]_{\textcolor{red}{3}\times 3} \Rightarrow [AB]_{2\times 3}
\]
\[
AB = 
\begin{bmatrix}
	11& 12& 26\\
	7 &5& -2 
\end{bmatrix}_{2\times 3}
\]
طريقة الحل 
\begin{align*}
	2\cdot3 + 3\cdot1 +1\cdot2&=11\\
	2\cdot4 + 3\cdot1 + 1\cdot1 &= 12\\
	2\cdot5 + 3\cdot4 + 1\cdot 4&= 26 \qquad \text{وهكذا}
\end{align*}

\section{محدد المصفوفة [ ]}
المحدد هو دالة رياضية تعتمد على بعد المصفوفة $n$ ويربط بقيمة قياسية (scalar) هي $\det A$ بكل مصفوفة مربعة $n\times n$ والمعنى الهندسي الاساسي للمحدد هو أنه بمثابة عامل المقياس للحجم عندما تعد المصفوفة $A$ تحويلا خطيا ويرمز عادة للمحدد لمصفوفة ما بالرمز $|A|$ . ولا يمكن حساب المحدد الا للمصفوفة المربعة\\ [10pt]
\noindent
\textbf{تعريف:} لتكن $A$ مصفوفة مربعة من الدرجة $n$ اي ان $A=[a_{ij}]_n$ وتساوي
\[
A =
\begin{bmatrix}
	a_{11} & a_{12} & \cdots & a_{1n}\\
	a_{21} & a_{22} & \cdots & a_{2n}\\
		\vdots & \vdots & \ddots & \vdots\\
	a_{n1} & a_{n2} & \cdots & a_{nn}\\
\end{bmatrix}
\] 
ندعوا الرمز $|A|$ او $\det A$ بمحدد المصفوفة من الرتبة $n$ ونكتب
\[
|A| = |A|_n = |a_{ij}|_{n} = \det A
\]

\section{انواع المحددات}
\textbf{1. المحدد من الرتبة الاولى}\\
\[
A = [a_{11}] \Rightarrow |A|_1 = |a_{11}|
\]
\textbf{طريقة الحساب:} يملك نفس قيمة العنصر الوحيد للمصفوفة المقابلة\\ [10pt]
\textbf{2. المحدد من الرتبة الثانية}
\[
A = 
\begin{bmatrix}
	a_{11} & a_{12} \\
	a_{21} & a_{22}
\end{bmatrix}
\]
حساب المحدد لمصفوفة ابعادها $2\times2$ يكون وفق القانون 
\[
|A| = a_{11}a_{22} - a_{12}a_{21}
\]
\newpage
\noindent
\textbf{3. المحدد من الرتب العليا ($3\times3$) او ($4\times4$)}\\
تحسب وفق الطرائق الآتية \\ [10pt]
\noindent
\textbf{1. طريقة ساروس (الشعاعية)}\\ [10pt]
	تتلخص بأخذ العمودين الاول و الثاني ونضعهما على يمين العمود الثالث ونقوم بمد شعاع على عناصر القطر الرئيسي من اعلى اقصى اليسار الى اسفل اقصى اليمين بعدها نمد شعاع بالاتجاه المعاكس اي من عناصر القطر الثانوي من اسفل اقصى اليسار الى اقصى اليمين وكما موضح
\begin{align*}
			A &= 
		\begin{array}{|ccc|cc}
			\Rnode{a1}{a_{11}} &\Rnode{b1}{a_{12}} & \Rnode{c1}{a_{13}} & \Rnode{A1}{a_{11}} & \Rnode{B1}{a_{12}} \\
			\Rnode{a2}{a_{21}} & \Rnode{b2}{a_{22}} & \Rnode{c2}{a_{23}} & \Rnode{A2}{a_{21}}& a_{22}\\
			\Rnode{a3}{a_{31}} & \Rnode{b3}{a_{32}} & \Rnode{c3}{a_{33}} & \Rnode{A3}{a_{31}} & \Rnode{B3}{a_{32}}
		\end{array}
		\psset{linecolor=red, nodesepA=0.5pt, nodesepB = 0.5pt, arrowinset=0.1}
		\foreach \s/\t/\u in {a1/b2/c3,b1/c2/A3,c1/A2/B3} {\ncline{-}{\s}{\t}\ncline{->}{\t}{\u}}
		\psset{linecolor=blue}
		\foreach \s/\t/\u in {a3/b2/c1,b3/c2/A1,c3/A2/B1} {\ncline{-}{\s}{\t}\ncline{->}{\t}{\u}}\\
		&= [a_{11}a_{22}a_{33} + a_{12}a_{23}a_{31}+a_{13}a_{21}a_{32}]\\
		&\qquad - [a_{31}a_{22}a_{13} + a_{32}a_{23}a_{11} + a_{33}a_{21}a_{12}]
\end{align*}

\noindent
\textbf{مثال}
\[
B= 
\begin{bmatrix}
	2 &1& -3\\
	5 &7 &0 \\
    -1& 3& 4 
\end{bmatrix}_{3\times 3}
\]
\begin{align*}
	|B| &=
	\begin{array}{|ccc|cc}
		\Rnode{a1}{2} &\Rnode{b1}{1} & \Rnode{c1}{-3} & \Rnode{A1}{2} & \Rnode{B1}{1} \\
		\Rnode{a2}{5} & \Rnode{b2}{7} & \Rnode{c2}{0} & \Rnode{A2}{5}& 7 \\
		\Rnode{a3}{-1} & \Rnode{b3}{3} & \Rnode{c3}{4} & \Rnode{A3}{-1} & \Rnode{B3}{3}
	\end{array}
	\psset{linecolor=red, nodesepA=0.5pt, nodesepB = 0.5pt, arrowinset=0.1}
	\foreach \s/\t/\u in {a1/b2/c3,b1/c2/A3,c1/A2/B3} {\ncline{-}{\s}{\t}\ncline{->}{\t}{\u}}
	\psset{linecolor=blue}
	\foreach \s/\t/\u in {a3/b2/c1,b3/c2/A1,c3/A2/B1} {\ncline{-}{\s}{\t}\ncline{->}{\t}{\u}}\\
	&=
	(56 + 0 -45) - (21 + 0 + 20)\\
	&= -30
\end{align*}
\newpage

\subsection*{2. طريقة المحيدد (اختيار صف او عمود)}
تلائم هذه الطريقة المصفوفات من الرتبة $2\times2$ و $3\times3$ و $4\times4$ \\
وهو اختيار احد الصفوف او الاعمدة في مثالنا. قمنا باختيار الصف الاول ثم نثبت الاشارات البدء بالحساب بالاشارات الموجبة وبعدها السالبة وبعدها الموجبة (بالتناوب).\\
\noindent
\textbf{مثال}
\[
B= 
\begin{bmatrix}
	\overbrace{2}^{+} & \overbrace{1}^{-}& \overbrace{-3}^{+}\\
	5 &7 &0 \\
	-1& 3& 4 
\end{bmatrix}_{3\times 3}
\] 
\begin{align*}
	|B| &= 2 
	\begin{vmatrix}
		7&0\\3&4
	\end{vmatrix}
	-1
	\begin{vmatrix}
		5&0 \\ -1&4
	\end{vmatrix}
	+(-3)
	\begin{vmatrix}
		5&7\\
		-1&3
	\end{vmatrix}\\
	&= 2(28) - 20 + (-3)(22)\\
	&= -30
\end{align*}
