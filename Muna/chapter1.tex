\chapter{مفاهيم أساسية} 
\section[المعادلة التفاضلية]{المعادلة التفاضلية \cite{diff_eqs_sols_apps}}
    هي علاقة بين المتغير المعتمد والمتغير المستقل (المتغيرات المستقلة) تدخل فيها المشتقات أو التفاضلات وتسمى المعادلة التفاضلية اعتيادية إذا كان المتغير المعتمد دالة في متغير مستقل واحد وبالتالي لا تحتوي إلا على مشتقات عادية. حيث الشكل العام لها
\[
F(x,y',y'',y''',\dots,y^{(n)})=0
\]
حيث $y^{(n)}=\mfrac{d^ny}{dx^n}$ المشتقة من الرتبة $n$ للمتغير $y$ بالنسبة إلى $x$.

\begin{example}
    ليكن $x$ هو المتغير المستقل و $y$ هو المتغير المعتمد. فالعلاقات التالية تمثل معادلات تفاضلية اعتيادية
    \begin{align}
        &\frac{dy}{dx}+y=3\sin x\label{diff_eq_ex1}\\
        &\frac{d^2y}{dx^2}+\frac{1}{x}\frac{dy}{dx}+y=0\label{diff_eq_ex2}
    \end{align}
\end{example}

\section[رتبة المعادلة التفاضلية]{رتبة المعادلة التفاضلية \cite{diff_eqs_sols_apps}}
    إذا كانت المشتقة النونية $y^{(n)}$ هي أعلى مشتقة تظهر في المعادلة التفاضلية الاعتيادية قيل أن هذه المعادلة من الرتبة $n$.

\begin{example}
    المعادلة \eqref{diff_eq_ex1} هي معادلة اعتيادية من الرتبة الأولى. أما المعادلة \eqref{diff_eq_ex2} هي معادلة تفاضلية اعتيادية من الرتبة الثانية.
\end{example}

\section[درجة المعادلة التفاضلية]{درجة المعادلة التفاضلية \cite{diff_eqs_sols_apps}}
    هي الاس المرفوع إليها أعلى مشتقة تظهر في المعادلة التفاضلية ، وقبل تحديد درجة المعادلة التفاضلية يجب وضعها في أبسط صورة قياسية صحيحة من حيث المشتقات.
\newpage
\begin{example}
    المعادلة
    \[
    \pbracket{\frac{d^2y}{dx^2}}^3+x\pbracket{\frac{dy}{dx}}+x^2y^3=0
    \]
    هذه معادلة تفاضلية من الرتبة الثانية و الدرجة الثالثة. أما المعادلة 
    \[
    \sbracket{1+\pbracket{\frac{dy}{dx}}^3}^{1/2}+3\pbracket{\frac{d^2}{dx^2}}+xy=0
    \]
    قبل تحديد درجة المعادلة يجب وضعها على صورة خالية من الجذور. بإجراء عمليات بسيطة نلاحظ أن
    \begin{gather*}
        1+\pbracket{\frac{dy}{dx}}^3=\pbracket{3\frac{d^2y}{dx^2}+xy}^2\\
        \Rightarrow 9\pbracket{\frac{d^2y}{dx^2}}^2+6xy\frac{d^2y}{dx^2}-\pbracket{\frac{dy}{dx}}^2+x^2y^2-1=0
    \end{gather*}
    وهذه معادلة تفاضلية من الرتبة الثانية ومن الدرجة الثانية
\end{example}

\section[المعادلة التفاضلية الخطية]{المعادلة التفاضلية الخطية \cite{diff_eqs_sols_apps}}
    هي المعادلة الخطية في المتغير المعتمد و مشتقاته جميعاً. الصورة العامة للمعادلة التفاضلية الخطية من الرتبة $n$ 
    \[
    p_n(x)y^{(n)}+p_{n-1}(x)y^{(n-1)}+\dots+p_1(x)y'+p_0(x)y=Q(x)
    \]
    أو
    \[
    \sum_{i=0}^{n}p_i(x)y^{(i)}=Q(x)
    \]
    حيث المتغير المعتمد $y$ و جميع مشتقاته مرفوعة إلى الاس واحد ولا توجد حواصل ضرب مشتركة في ما بينها ، و الدوال $Q(x)$ و $p_i(x)$ هي دوال للمتغير المستقل $x$ خطية أم غير خطية لا تؤثر على خطية المعادلة التفاضلية.

\begin{note}
    من تعريف المعادلة التفاضلية الخطية يمكننا القول أي معادلة تفاضلية فيها المتغير المعتمد $y$ أو أحد مشتقاته مرفوعة إلى اس غير الواحد أو وجدنا حاصل ضرب في ما بينها. سوف نطلق عليها معادلة تفاضلية غير خطية.
\end{note}

\begin{example}
    لدينا المعادلة
    \[
    x^2y^{''}+xy'+y=\cos x
    \]
    معادلة تفاضلية اعتيادية خطية. أما المعادلة
    \[
    yy^{'}+y^{''}=e^{3x}
    \]
    هذه المعادلة التفاضلية غير خطية لوجود حاصل الضرب $yy^{'}$. وأيضا لدنيا
\[
y^{'}+x\sqrt[4]{y}=\cot x
\]
وهذه كذلك ليست خطية فيها المتغير المعتمد $y$ مرفوع إلى الاس $1/4$ (غير الواحد).
\end{example}

\section[حل المعادلة التفاضلية]{حل المعادلة التفاضلية \cite{diff_eqs_pt1}}
تسمى الدالة $y=y(x)$ حلاً للمعادلة التفاضلية $F(x,y,y',y'',\dots,y^{(n)})$ إذا كانت
\begin{enumerate}
    \item قابلة للاشتقاق $n$ من المرات
    \item تحقق المعادلة التفاضلية أي أن $F(x,y(x),y'(x),y''(x),\dots,y^{(n)}(x))$
\end{enumerate}
\begin{example}
    أثبت أن $y(x)=c \sin x$ حلاً للمعادلة التفاضلية $y''+y=0$ حيث $c$ ثابت.
\end{example}
\begin{solution}
    نشتق العلاقة ونقوم بتعويضها في المعادلة التفاضلية
\[
y=c\sin x\Rightarrow y'=c\cos x\Rightarrow y''=-c\sin x
\]
الآن نعوض
\[
y''+y=-c\sin x+c\sin x=0
\]\qed
\end{solution}
\section[الحل العام و الحل الخاص]{الحل العام و الحل الخاص \cite{diff_eqs_pt1}}
الحل العام للمعادلة التفاضلية من الرتبة $n$ هو حل يحتوي على $n$ من الثوابت الاختيارية وبالطبع يحقق المعادلة التفاضلية.\\
أما الحل الخاص هو أي حل يحقق المعادلة التفاضلية لا يشتمل على أي ثوابت اختيارية وقد نحصل عليه احيانا بالتعويض عن الثوابت الاختيارية في الحل العام بقيم محددة.

\begin{example}
    الحل العام للمعادلة التفاضلية $y'''-5y''+6y'=0$ يكون $y(x)=c_1+c_2e^{2x}+c_3e^{3x}$ حيث $c_1,c_2,c_3$ ثوابت اختيارية.\\
    ونجد أن بعض الحلول الخاصة على الصور 
    \[
    y=3+5e^{2x},\quad y=5-2e^{3x},\quad y=e^{2x}+e^{3x}
    \]
\end{example}

\section[المتتابعات و المتسلسلات]{المتتابعات و المتسلسلات \cite{mathanal}}

\begin{definition}
	لتكن $X$ مجموعة ما، نسمي متتابعة من $X$، كل دالة $u$ من $\N$ الى $X$ ، نرمز لذلك بالرمز $(u_n)$ حيث صورة العدد $n$ بواسطة الدالة $u$ ويسمى بالحد العام للمتتابعة $(u_n)_{n\geq 0}$
\end{definition}
\noindent
\textbf{مثال}
\begin{align*}
&u : \N \to \R\\
&n \to u_n = 2n+1
\end{align*}

\begin{definition}
	لتكن $(u_n)$ متتابعة عددية، ان العبارة
	\begin{equation}
		u_1 + u_2 + \cdots + u_n + \cdots
	\end{equation}
	تسمى متسلسلة عددية والاعداد $u_1,u_2,\dots$ تسمى بحدود المتسلسلة. اما $u_n$ يسمى بالحد العام للمتسلسلة.\\
	نعتبر المجاميع الجزئية
	\[
	\left\{
	\begin{array}{l}
		S_1 = u_1\\
		S_2 = u_1+u_2\\
		\vdots\\
		S_n = u_1 + u_2 + \cdots + u_n
	\end{array}
	\right.
	\]
	$(S_n)$ تسمى متتابعة المجاميع الجزئية. اذا كانت الغاية
	\begin{equation}
		\lim\limits_{n\to \infty} S_n
	\end{equation}
	موجودة و منتهية ، أي أن $\lim\limits_{n\to \infty} S_n = S$ فإننا نسميها مجموع المتسلسلة (3). ونكتب
	\begin{equation}
		S = \sum_{n=0}^{\infty} S_n
	\end{equation}
\end{definition}

\begin{definition}
	في الرياضيات، متسلسلة القوى (في متغير واحد) هي متسلسلة لانهائية تأخذ الشكل
	\begin{equation}
		\sum_{n=0}^{\infty} a_n (x-c)^n = a_0 + a_1(x-c) + a_2 (x-c)^2  + \cdots
	\end{equation}
	حيث $c$ يسمى مركز المتسلسلة، في الكثير من الحالات المركز يساوي صفراً $c=0$ وفي هذه الحالة نسمي المتسلسلة بمتسلسلة ماكلورين وتأخذ الشكل
	\begin{equation}
		\sum_{n=0}^{\infty} a_n x^n = a_0 + a_1 x + a_2 x^2 + \cdots
	\end{equation}
\end{definition}

\section{بعض متسلسلات القوى للدوال}
\textbf{1. المتسلسلة الهندسية}
\[
\frac{1}{1-x} = \sum_{n=0}^{\infty} x^n = 1+x+x^2 + \cdots, \quad (|x| < 1)
\]

\noindent
\textbf{2. الدالة الاسية}
\[
e^x = \sum_{n=0}^{\infty} \frac{x^n }{n!} = 1 + x + \frac{x^2}{2!} +  \frac{x^3}{3!}+ \cdots , \quad (|x| < \infty)
\]
\textbf{3. الدوال المثلثية}
\[
\sin x = \sum_{n=0}^{\infty} (-1)^n \frac{x^{2n+1}}{(2n+1)!} = x - \frac{x^3}{3!} + \frac{x^5}{5!} + \cdots, \quad (|x|<\infty)
\]
\[
\cos x = \sum_{n=0}^{\infty} (-1)^n \frac{x^{2n}}{(2n)!} = 1- x^2 + \frac{x^4 }{4!} + \cdots, \quad (|x| < \infty)
\]
\textbf{4. الدوال الزائدية}
\[
\sinh x = \sum_{n=0}^{\infty} \frac{x^{2n+1}}{(2n+1)!} = x + \frac{x^3}{3!} + \frac{x^5}{5!} + \cdots, \quad (|x|<\infty)
\]
\[
\cosh x = \sum_{n=0}^{\infty} \frac{x^{2n}}{(2n)!} = 1+ x^2 + \frac{x^4 }{4!} + \cdots, \quad (|x| < \infty)
\]