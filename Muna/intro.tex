\chapter*{مقدمة}

\addcontentsline{toc}{chapter*}{مقدمة}
تُعَدُّ المعادلات التفاضلية الاعتيادية (ODEs) من الأدوات الأساسية في الرياضيات التطبيقية والعلوم الهندسية، حيث تُستخدم لنمذجة العديد من الظواهر الفيزيائية، والهندسية، والاقتصادية، والبيولوجية. فهي تصف العلاقات بين الدوال ومشتقاتها، مما يساعد في فهم كيفية تغيّر الأنظمة الديناميكية مع الزمن أو عبر متغيرات أخرى. وتبرز أهمية حل هذه المعادلات في مجالات متعددة مثل تحليل الدوائر الكهربائية، ميكانيكا الموائع، علم الفلك، ونظرية التحكم.

\noindent
من بين الطرق المتاحة لحل المعادلات التفاضلية الاعتيادية، تأتي طريقة متسلسلات القوى كأداة فعالة، خاصة عندما لا يمكن إيجاد الحل بصيغة مغلقة باستخدام الطرق التحليلية التقليدية مثل طريقة الفصل أو عامل التكامل. تعتمد هذه الطريقة على تمثيل الحل كمتسلسلة قوى حول نقطة معينة، ثم تحديد معاملات هذه المتسلسلة من خلال التعويض في المعادلة التفاضلية.

\noindent
تتميز طريقة متسلسلات القوى بأنها توفر حلولًا دقيقة في شكل موسّع يمكن استخدامه لإيجاد تقديرات عددية للحل، كما أنها تُعطي تمثيلًا للحل حتى في نقاط يصعب فيها استخدام طرق أخرى، مثل النقاط الشاذة العادية. وتُستخدم هذه الطريقة في حل المعادلات ذات المعاملات المتغيرة التي لا يمكن حلها بالطرق التقليدية، كما تُعد الأساس لطريقة فروبينياس التي تعالج الحالات التي تحتوي على نقاط شاذة مفردة.

\noindent
في هذا السياق، سنتناول في هذا البحث منهجية حل المعادلات التفاضلية باستخدام متسلسلات القوى، بدءًا من التعريف بمفهوم المتسلسلات، مرورًا بالخطوات العملية لتطبيق الطريقة، وانتهاءً ببعض التطبيقات المهمة التي تبرز فاعليتها.


