\chapter{المعادلات التفاضلية}

\section{المعادلات التفاضلية الاعتيادية و الجزئية}
تسمى المعادلات التي تحتوي على مشتقات لمتغير معتمد او اكثر بالنسبة الى متغير مستقل او اكثر بالمعادلة التفاضلية 

\begin{itemize}
\item	يكون الهدف هو ايجاد حل المعادلة التفاضلية الذي يتمثل بايجاد الدالة المجهولة (المتغير المعتمد) الذي يحقق المعادلة التفاضلية

\item  نسمي المعادلة التفاضلية بالاعتيادية اذا كان هناك متغير مستقل واحد وتكون جزئية اذا كان هناك اكثر من متغير مستقل.
\end{itemize}

\begin{example}
	لدينا المعادلات التفاضلية
	\begin{align}
		&k \frac{d^2 y}{dx^2} = \left[1 + \left(\frac{dy}{dx}\right)^2\right]^{3/2}\\
		&\frac{\partial^2 u}{\partial x^2} + \frac{\partial^2 u}{\partial y^2} + \frac{\partial^2 u}{\partial z^2} = 0
	\end{align}
	المعادلة (1) تكون اعتيادية لان فيها متغير مستقل واحد اما المعادلة (2) هي جزئية لان فيها ثلاث متغيرات مستقلة.
\end{example}

في هذا البحث سوف نركز فقط على المعادلات التفاضلية الاعتيادية.

\section{رتبة المعادلة التفاضلية}
رتبة المعادلة التفاضلية هي اعلى رتبة لمشتقة تظهر في المعادلة التفاضلية.

\begin{example}
	لدينا المعادلات التفاضلية
	\begin{align}
		& \frac{dy}{dx} = x^2 -1\\
		& m \frac{d^2 y}{dx^2} = F
	\end{align}
	المعادلة (3) من الرتبة الاولى. اما المعادلة (4) من الرتبة الثانية.
\end{example} 
\noindent
$\bullet$ الشكل العام للمعادلة التفاضلية من الرتبة $n$ يكون
\[
F\left(\frac{d^n y}{dx^n}, \frac{d^{n-1} y}{dx^{n-1}}, \dots, \frac{dy}{dx}, y, t\right) = 0
\]

\section{المعادلات التفاضلية الخطية واللا خطية}
المعادلة التفاضلية من الرتبة $n$ يقال بأنها خطية اذا كانت تأخذ الشكل
\[
a_n(t) \frac{d^n y}{dx^n } + a_{n-1}(t) \frac{d^{n-1}y}{dx^{n-1}} + \cdots + a_1(t) \frac{dy}{dt} + a_0 y = f(t)
\]
اذا كانت $f(t) = 0$ فإن المعادلة تسمى متجانسة وتكون غير متجانسة اذا كانت $f(t) \neq 0$.

\section{المعادلات التفاضلية الخطية المتجانسة من الرتبة الثانية ذات المعاملات الثابتة}
في هذا البند سوف نجد الحل العام للمعادلة التفاضلية التي تكون على الشكل
\begin{equation}
	\label{eq:homlinearconstcoeff}
	a \frac{d^2 x}{dt^2} + b \frac{dx}{dt} + cx = 0
\end{equation}
عند تعويض $x(t) = e^{kt}$ في \eqref{eq:homlinearconstcoeff} نحصل على
\begin{gather*}
	a \frac{d^2}{dt^2} (e^{kt}) + b \frac{d}{dt} (e^{kt}) + c  e^{kt} =0 \\
	ak^2 e^{kt} + bk e^{kt} + cr^{kt} = 0\\
	e^{kt} (ak^2 + bk + c) = 0
 \end{gather*}
 بما ان $e^{kt} \neq 0$ فإن
 \begin{equation}
 	\label{eq:char}
 	ak^2 + bk + c = 0
 \end{equation}
تسمى المعادلة \eqref{eq:char} بالمعادلة المميزة للمعادلة التفاضلية المتجانسة \eqref{eq:homlinearconstcoeff}. وهي معادلة جبرية من الدرجة الثانية في $k$ يكون لها جذران في ثلاث حالات كالآتي:

\subsection*{أ - جذران حقيقيان مختلفان}
لنفرض $k_1, k_2$ جذرا المعادلة \eqref{eq:char} حيث $k_1, k_2\in \R$ و $k_1\neq  k_2$. اذن الحل العام للمعادلة \eqref{eq:homlinearconstcoeff} هو
\[
x(t) = A e^{k_1 t} + B e^{k_2 t}
\]

\begin{example}
	جد الحل العام للمعادلة التفاضلية 
	\begin{equation}
		\label{eq:distnictexample}
		x '' - 2x ' = 0
	\end{equation}
\end{example}
\begin{solution}
	عند تعويض $x(t) = e^{kt}$ في \eqref{eq:distnictexample} نحصل على المعادلة المميزة
	\[
	k^2 - 2k =0 
	\]
	وبحل هذه المعادلة 
	\[
	k(k-2) = 0 \Rightarrow k_1 = 0, k_2= 2
	\]
	اي ان $k_1,k_2$ حقيقيان مختلفان بالتالي الحل العام يكون
	\[
	x(t) = Ae^{0t} + Be^{2t} = A + B e^{2t}
	\]
\end{solution}

\subsection*{ب - جذران حقيقيان متساويان}
نفرض $k_1=k_2\in\R$ جذرا المعادلة \eqref{eq:char} فإن الحل العام للمعادلة التفاضلية \eqref{eq:homlinearconstcoeff} يكون
\[
x(t) = (A + Bt) e^{kt}
\]
حيث $k=k_1=k_2$.

\begin{example}
	جد الحل العام للمعادلة التفاضلية
	\begin{equation}
		\label{eq:repeatedexample}
		x'' + 2x' + x = 0
	\end{equation}
	التي تمتلك الشروط الابتدائية
	\[
	x(0) = 0,\quad x'(0) = 1
	\]
\end{example}
\begin{solution}
	نعوض $x(t) = e^{kt}$ في المعادلة التفاضلية \eqref{eq:repeatedexample} نحصل على المعادلة المميزة
	\[
	k^2 + 2k + 1
	\]
	بحل هذه المعادلة 
	\[
	(k+1)^2 = 0\Rightarrow k_1 = k_2 = -1
	\]
	اذن الجذور حقيقية متكررة لذا الحل العام يكون
	\[
	x(t) = (A + Bt) e^{-t}
	\]
	الآن بما ان $x(0) = 0$ اذن
	\[
	0 = (A + 0) e^0 \Rightarrow A=0
	\]
	بالتالي
	\[
	x(t) = Bt^{-t} \Rightarrow x'(t) = B(e^{-t} - te^{-t})
	\]
	وبما ان $ x'(0) = 1 $ فإن 
	\[
	1 = B(e^0  -0 ) \Rightarrow B = 1
	\]
	وبالتالي فإن الحل النهائي يكون
	\[
	x(t) = t e^{-t}
	\]
\end{solution}

\subsection*{جـ - جذران عقديان}
عندما يكون جذرا المعادلة \eqref{eq:char} غير حقيقيين، أي ان $k_1,k_2\in \C$ فإنهما يكونا على الشكل
\[
k_1, k_2 = p \pm iw
\]
فحل المعادلة التفاضلية \eqref{eq:homlinearconstcoeff} يكون
\[
x(t) = e^{pt} [A \cos(wt) + B \sin(wt)]
\]
حيث $A, B$ ثوابت اختيارية
\newpage
\begin{example}
	جد حل المعادلة التفاضلية
	\begin{equation}
		\label{eq:complexexample}
		x'' + 2x' + 5x = 0, \quad x(0) = 1, x'(0) = 0
	\end{equation}
\end{example}
\begin{solution}
	نفرض $x(t) = e^{kt}$ ونعوض في المعادلة \eqref{eq:complexexample} ، نحصل على
	\[
	k^2 + 2k + 5= 0
	\]
	وبحل هذه المعادلة المميزة نحصل على
	\begin{align*}
		k_{1,2} &= \frac{-2 \pm \sqrt{2^2 - 4(5)}}{2}\\
		&= \frac{-2\pm \sqrt{-16}}{2}\\
		&= \frac{-2 \pm 4i}{2}\\
		&= -1 \pm 2i
	\end{align*}
\end{solution}
اذن $p=-1,w=2$ وبالتالي ان الحل العام للمعادلة \eqref{eq:complexexample} يكون
\[
x(t) = e^{-t} [A \cos(2t) + B \sin(2t)]
\]
\[
x'(t) = e^{-t} [(2B-A)\cos 2t - (2A + B)\sin 2t]
\]
باستخدام الشروط $x(0) = 1, x'(0) =0$ نجد
\[
A =1,\quad 2B - A = 0 
\]
\[
\Rightarrow B = \frac{1}{2}
\]
اذن الحل النهائي
\[
x(t) = e^{-t} [\cos 2t + \frac{1}{2} \sin 2t]
\]

\section{المعادلات التفاضلية الخطية غير المتجانسة ذات المعاملات الثابتة}
في هذا البند نناقش الحصول على الحل العام للمعادلات التفاضلية التي تكون على الشكل
\begin{equation}
	\label{eq:nonhomlinearconst}
	a \frac{d^2 x}{dx^2} + b \frac{dx}{dt} + cx = f(t)
\end{equation}
حيث  $f(t) \neq 0$، يكون الحل العام للمعادلة التفاضلية \eqref{eq:nonhomlinearconst} على الشكل
\[
x(t) = A y_1(t) + B y_2(t) + x_p(t)
\]
حيث $A y_1(t) + By_2(t)$ هو حل المعادلة التفاضلية المتجانسة
\[
	a \frac{d^2 x}{dx^2} + b \frac{dx}{dt} + cx = 0
\]
ونرمز له $x_c(t)$ ويسمى بالحل التام. اما $x_p(t)$ هو الحل الخاص والذي يعتمد على شكل الدالة $f(t)$. الآن نناقش بعض الحالات الخاصة للدالة $f(t)$ وكيفية الحصول على الحل الخاص.

\subsection*{أ - عندما تكون \en{\textit{f}(\textit{t})} تكون كثيرة حدود}

اذا كانت $f(t)$ كثيرة حدود من الدرجة $n$ فإننا نخمن الحل الخاص
\[
x_p(t) = c_n t^n + c_{n-1} t^{n-1} + \cdots + c_1 t + c_0
\]
ونعوض في \eqref{eq:nonhomlinearconst} لايجاد المجاهيل $c_0,\dots,c_n$

\begin{example}
	جد الحل العام للمعادلة التفاضلية
	\begin{equation}
		\label{eq:polyexample1}
		x'' + x' - 6x = 12 
	\end{equation}
\end{example}
\begin{solution}
	نجد اولاً الحل التام $x_c(t)$ الذي يحقق المعادلة التفاضلية المتجانسة
	\[
		x'' + x' - 6x =0
	\]
	التي تمتلك المعادلة المميزة
	\[
	k^2 + k - 6 =0
	\]
	بحل هذه المعادلة نجد $k_1=-3,k_2=2 $ اذن
	\[
	x_c(t) = A e^{-3t} + B e^{2t}
	\]
	الان لأن الطرف الايمن من المعادلة \eqref{eq:polyexample1} هو كثيرة حدود من الدرجة 0 فإننا نفرض $x_p(t) = c$ اذن $x_p'(t) = x_p''(t) = 0$ نعوض في المعادلة \eqref{eq:polyexample1} نجد
	\[
	0 + 0 -6c = 12 \Rightarrow c = -2
	\]
	اذن الحل العام
	\[
	x(t) = x_c(t) + x_p(t) = A e^{-3t} + B e^{2t} -2
	\]
\end{solution}

\begin{example}
	جد الحل العام للمعادلة التفاضلية
	\begin{equation}
		\label{eq:polyexample2}
		x'' + x' - 6x = 216 t^3
	\end{equation}
\end{example}
\begin{solution}
	الحل التام من المثال السابق يكون
	\[
	x_c(t) = A e^{-3t} + B e^{2t}
	\]
	اما الحل الخاص. نفرض 
	\[
	x_p(t) = C t^3 + D t^2 + E t + F
	\]
	لأن الطرف الايمن من المعادلة \eqref{eq:polyexample2} كثيرة حدود من الدرجة الثالثة. الآن
	\begin{gather*}
		x_p'(t) = 3C t^2 + 2Dt + E\\
		x_p''(t) = 6Ct + 2D
	\end{gather*}
	نعوض في المعادلة التفاضلية الاصلية \eqref{eq:polyexample2} نجد
	\[
	(6Ct + 2D) + (3C t^2 + 2Dt + E) - 6(C t^3 + D t^2 + E t + F) = 216t^3
	\]
	ومنه
	\[
	-6C t^3 +(3C - 6D)t^2 + (6C + 2D - 6E) t + (2D + E -6F) = 216t^3
	\]
	اذن
	\[
	\left.
	\begin{array}{r}
		-6C = 216\\
		3C - 6D = 0\\
		6C + 2D - 6E =0\\
		2D + E - 6F = 0 
	\end{array}
	\right\} \Rightarrow 
	\begin{array}{c}
		C= -36\\
		D = -18\\
		E = -42\\
		F = -13
	\end{array}
	\]
	بالتالي الحل الخاص يكون
	\[
	x_p(t) = -36t^3 -18t^2 -42t - 13
	\]
	اذن الحل العام يكون
	\[
	x(t) = A e^{-3t} + B e^{2t} -36t^3 -18t^2 -42t - 13
	\]
\end{solution}


\subsection*{ب - عندما تكون \en{\textit{f}(\textit{t})} تكون دالة أُسية}
لو كانت $f(t) = ae^{kt}$ فإننا نفرض الحل الخاص $x_p(t) = ce^{kt}$ والهدف ايجاد المجهول $c$ من خلال تعويض في المعادلة \eqref{eq:nonhomlinearconst}. ولكن ننوه ان هذه الفرضية تكون في حالة كون الدالة $f(t)$ ليست حلاً للجزء المتجانس
\[
a \frac{d^2 x}{dx^2} + b \frac{dx}{dt} + cx = 0
\]
اما اذا كانت الدالة $e^{kt}$ حلاً للمعادلة التفاضلية المتجانسة، فإنه لا يمكن ان يكون حلاً خاصاًونأخذ الحالات التالية
\begin{enumerate}
	\item اذا كانت المعادلة التفاضلية المتجانسة تمتلك جذور غير متكررة فإننا  نفرض الحل الخاص
	\[
	x_p(t) = c t e^{kt}
	\]
	
	\item اذا كانت المعادلة التفاضلية المتجانسة تمتلك جذور متكررة فإننا  نفرض الحل الخاص
	\[
	x_p(t) = c t^2 e^{kt}
	\]
\end{enumerate}

\begin{example}
	جد الحل العام للمعادلة التفاضلية
	\begin{equation}
		\label{eq:expexample1}
		x'' + x' - 6x = 4 e^{-2t}
	\end{equation}
\end{example}
\newpage
\begin{solution}
	الحل التام من الامثلة السابقة
	\[
	x_c(t) = A e^{-3t} + B e^{2t}
	\]
	نرى ان $e^{-2t}$ ليست حلاً للجزء المتجانس، اذن نفرض الحل الخاص
	\[
	x_p(t) = C e^{-2t}
	\]
	بالتالي
	\[
	x_p'(t) = -2C e^{-2t}
	\]
	\[
	x_p''(t) = 4C e^{-2t}
	\]
	نعوض في المعادلة الاصلية \eqref{eq:expexample1} نحصل على 
	\[
	4 C e^{-2t} -2  C e^{-2t} -6 C e^{-2t} =  4e^{-2t}
	\]
	\[
	(4C -2C - 6C)  e^{-2t} = 4 e^{-2t}
	\]
	اذن
	\[
	-4C = 4 \Rightarrow C =- 1
	\]
	منه
	\[
	x_p(t) =  -e^{-2t}
	\]
	بالتالي
	\[
	x(t) = A e^{-3t} + B e^{2t} -e^{-2t}
	\]
\end{solution}

\begin{example}
	جد الحل العام للمعادلة التفاضلية
	\begin{equation}
		\label{eq:expexample2}
		x'' + x' - 6x = 5 e^{-3t}
	\end{equation}
\end{example}

\begin{solution}
	الحل التام من الامثلة السابقة
	\[
	x_c(t) = A e^{-3t} + B e^{2t}
	\]
	نرى ان $e^{-3t}$ حلاً للجزء من المتجانس الذي يمتلك جذور غير متكررة ، اذن نفرض الحل الخاص
	\[
	x_p(t) = C t e^{-3t}
	\]
	\[
	x_p'(t) = C e^{-3t} - 3C t e^{-3t}
	\]
	\[
	x_p''(t) = -6C e^{-3t} + 9C te^-{3t}
	\]
	نعوض في المعادلة التفاضلية الاصلية \eqref{eq:expexample2} نحصل على
\[
	(-6C e^{-3t} + 9C te^-{3t}) + (C e^{-3t} - 3C t e^{-3t}) 
-6(C t e^{-3t}) = 5 e^{-3t}
\]
اذن
\[
(-6C + C)e^{-3t} + (9C - 3C - 6C) te^{-3t} = 5 e^{-3t}
\]
\[
\Rightarrow -6C + C = 5 \Rightarrow C =-1
\]
اذن الحل الخاص
\[
x_p(t) = -t e^{-3t}
\]
الحل العام يكون
\[
x(t) = A e^{-3t} + B e^{2t} - t e^{-3t}
\]
\end{solution}

\begin{example}
	جد الحل العام للمعادلة التفاضلية
	\begin{equation}
		\label{eq:expexample3}
		x'' + 2x' + x = 6e^{-t}
	\end{equation}
\end{example}
\begin{solution}
	من خلال الامثلة السابقة ، رأينا ان الحل التام
	\[
	x_c(t) = (A + Bt) e^{-t}
	\]
	اذن $e^{-t}$ هو حل المعادلة التفاضلية المتجانسة التي تمتلك جذور متكررة. بالتالي نفرض الحل الخاص
	\[
	x_p(t) = C t^2 e^{-t}
	\]
	اذن
	\begin{gather*}
		x_p'(t) = 2C te^{-t} - Ct^2 e^{-t}\\
		x_p''(t) = 2C e^{-t} - 4C t e^{-t} + C t^2 e^{-t}
	\end{gather*}
	الآن نعوض في \eqref{eq:expexample3} نحصل على 
	\[
	(2C e^{-t} - 4C t e^{-t} + C t^2 e^{-t}) + 2(2C te^{-t} - Ct^2 e^{-t}) + (C t^2 e^{-t}) = 6 e^{-t}
	\]
	بجمع الحدود المتشابهة
	\[
	(C-2C+C)t^2 e^{-t} + (-4C + 4C)te^{-t} + 2C e^{-t}  = 6 e^{-t}
	\]
	ومنه
	\[
	2C = 6 \Rightarrow C= 3
	\]
	اذن الحل الخاص
	\[
	x_p(t) = 3t^2 e^{-t}
	\]
	وبالتالي فإن الحل العام يكون
	\[
	x(t) =  (A + Bt) e^{-t} + 3t^2 e^{-t}
	\]
\end{solution}