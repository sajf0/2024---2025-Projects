\chapter{معادلات الفروق}

\section{التعريف}
المعادلة التي تربط بين قيم $x_n$ لقيم مختلفة من $n$ تسمى معادلة فروق، ورتبة معادلة الفروق هي الفرق الاكبر بين اي دليلين موجودين في المعادلة 

\begin{example}
	لدينا معادلات الفروق التالية
	\begin{align}
		& x_{n+1} = x_n + hf(x_n, nh)\\
		& x_{n+1} = 2^{n+7} + \cos x_n\\
		& x_{n+2} = x_{n+1}^2 - \exp(x_{n-3})
	\end{align}
	نلاحظ ان المعادلتين (1) و (2) من الرتبة الاولى ، ولكن المعادلة (3) من الرتبةالخامسة لأن\\
	 ($(n+2) - (n-3) = 5$)
\end{example}

\section{معادلات الفروق من الرتبة الاولى}
معادلة الفروق من الرتبة الاولى تربط القيمة التالية لـ $x$ مع قيمتها الحالية. وتكون صيغتها العامة بالشكل
\[
F(x_n, x_{n+1}, n) = 0
\]
سوف نقتصر هنا على دراسة معادلات الفروق التي يمكن فيها التعبير عن $x_{n+1}$ بشكل صريح بدلالة $x_n$. 
\[
x_{n+1} = f(x_n, n)
\]

\begin{example}
	لنأخذ معادلة فروق خطية بسيطة من الرتبة الاولى
	\[
	x_{n+1} = k x_n
	\]
	لنفتر	ض اننا نعرف $x_0$ فإنه من السهل ايجاد $x_n$. لدينا
	\begin{align*}
		& x_1 = k x_0\\
		& x_2 = k x_1 = k(kx_0) = k^2 x_0\\
		& x_3 = k x_2 = k(k^2x_0) = k^3 x_0
	\end{align*} 
	وبشكل عام نرى ان $x_n = k^n x_0$.
\end{example}

\section{معادلات الفروق من الرتبة الثانية: الدالة المتممة والحل الخاص}
نركز الآن على معادلات الفروق الخطية من الرتبة الثانية التي تكون على الشكل
\begin{equation}
	\label{eq:secondorderdiffeq}
	x_{n+2} + a x_{n+1} + b x_n = f_n
\end{equation} 
كما فعلنا مع المعادلات التفاضلية سوف نجزء حل المعادلة \eqref{eq:secondorderdiffeq} الى ايجاد حل المعادلة المتجانسة 
\[
y_{n+2} + a y_{n+1} + by_n = 0
\]
ومن ثم ايجاد الحل الخاص للمعادلة \eqref{eq:secondorderdiffeq}.

\section{معادلات الفروق المتجانسة}
اولا نتعامل مع معادلات الفروق المتجانسة
\begin{equation}
	\label{eq:homdiffeq}
	ax_{n+2} + b x_{n+1} + c x_n = 0
\end{equation}
سوف نخمن ان الحل هو $x_n = k^n$ ونعوض في المعادلة \eqref{eq:homdiffeq} نحصل على
\[
a k^{n+2} + b k^{n+1} + ck^n = 0
\]
بإختصار $k^n$ نحصل على المعادلة المساعدة
\[
ak^2 + bk + c = 0
\]
حصلنا على معادلة تربيعية في $k$. و حل المعادلة \eqref{eq:homdiffeq} يعتمد على طبيعة الجذور.

\subsection*{أ - جذور حقيقية مختلفة}
اذا كانت المعادلة المساعدة تمتلك جذران حقيقيان مختلفان مثل $k_1, k_2$ اذن $x_n = k_1^n$ و $x_n = k_2^n$ كلاهما يكون حلاً للمعادلة \eqref{eq:homdiffeq}. بالتالي فإن الحل العام يكون
\[
x_n = A k_1^n + B k_2^n
\]

\begin{example}
	لنجد صيغة لعدد فيبوناتشي النوني 
	(\LR{\textit{nth} Fibonacci number})
	الذي يحقق معادلة الفروق
\begin{equation}
	\label{eq:fibonaccieq}
		x_n = x_{n-1} + x_{n-2}
\end{equation}
	اذا كانت $x_0 =1, x_1=1 $ ، حيث الحدود الاولى تكون
	\[
    1, 1, 2, 3 ,5, 8, 13, 21, 34, 55, \dots
    \]
    لحل معادلة الفروق \eqref{eq:fibonaccieq} نفترض ان $x_n = k^n $ نجد
    \[
    k^2 = k+1 
    \]
    هذه المعادلة تمتلك الجذور
    \[
    k = \frac{1\pm \sqrt{5}}{2}
    \]
    لذا فإن الحل العام للمعادلة \eqref{eq:fibonaccieq} يكون
    \[
    x_n = \alpha \left(\frac{1+\sqrt{5}}{2}\right)^n + \beta\left(\frac{1-\sqrt{5}}{2}\right)^n
    \]
    والشروط الابتدائية تتطلب
    \[
    \alpha+ \beta =1\qquad (1+\sqrt{5})\alpha + (1-\sqrt{5})\beta = 2
    \]
    وبالحل من اجل $\alpha, \beta$ 
    \[
    \alpha = \frac{1+\sqrt{5}}{2\sqrt{5}}, \quad \beta = \frac{\sqrt{5}-1}{2\sqrt{5}}
    \]
    اذن عدد فيبوناتشي النوني 
    \[
    x_n = \frac{1}{\sqrt{5}} \left(\frac{1+\sqrt{5}}{2}\right)^{n+1} - \frac{1}{\sqrt{5}} \left(\frac{1-\sqrt{5}}{2}\right)^{n+1}
    \]
\end{example}


\subsection*{ب - جذور حقيقية متكررة}
اذا كانت المعادلة المساعدة تمتلك جذر حقيقي مكرر $k$ فإن الحل العام لمعادلة الفروق \eqref{eq:homdiffeq} 
\[
x_n = Ak^n + Bn k^n
\]
 \newpage
\begin{example}
 جد الحل العام لمعادلة الفروق 
 \[
 x_n - 4x_{n-1} + 4x_{n-2} =0 
 \]

\end{example}
\begin{solution}
	نفرض $x_n = k^n$ في المعادلة ، نحصل على المعادلة المساعدة 
	\[
	k^2 - 4k + 4=0
	\]
	الذي يكون حلها $k=2 $ (مرتان) ، لذا فإن الحل العام يكون
	\[
	x_n = A 2^n + Bn 2^n
	\]
\end{solution}

\subsection*{جـ - جذور عقدية}
من الممكن الحصول على جذور عقدية للمعادلة المساعدة، اي ان $k = a\pm ib$. فإننا نحتاج ان نكتب $k$ بالصيغة القطبية
\[
k = r e^{\pm i\theta}
\]
حيث
\[
r^2 = a^2 + b^2, \qquad \theta = \tan^{-1}(b/a)
\]
فإن الحل العام لمعادلة الفروق المتجانسة \eqref{eq:homdiffeq} 
\[
x_n = r^n [\cos n\theta + B \sin n\theta]
\]

\begin{example}
	جد الحل العام لمعادلة الفروق 
	\[
	x_{n+2} - 2x_{n+1} + 2x_n = 0
	\]
\end{example}
\begin{solution}
	نعوض $x_n = k^n$ في المعادلة ، ونحصل على المعادلة المساعدة
	\[
	k^2 - 2k + 2 =0
	\]
	التي تمتلك الجذور
	\[
	k = \frac{2\pm\sqrt{4-8}}{2} = 1 \pm i
	\]
	وبما ان 
	\[
	1 \pm i = \sqrt{2} e ^{\pm i \pi/4}
	\]
	الحل يكون
	\[
	x_n = 2^{n/2} [A\cos(n \pi/4) + B \sin(n \pi/4)]
	\]
\end{solution}

\section{الحلول الخاصة}
عندما يكون لدينا معادلة فروق والطرف الايمن فيها غير صفري
\[
a x_{n+2} + bx_{n+1} +cx_n = f_n
\]
هنا نستخدم نفس الاسلوب في المعادلات التفاضلية. حيث نخمن شكل الحل الخاص ونعوضه في المعادلة الاصلية لتحديد قيم المجاهيل.

\subsection*{أ - عندما تكون \en{\textit{f\textsubscript{n}}} كثيرة حدود في \textit{n}}
عندما يكون الطرف الايمن كثيرة حدود بالنسبة الى $n$. فإن التخمين هو كيثرة حدود عامة بنفس درجة كثيرة الحدود في الطرف الايمن. ولكن اذا كان تخميننا يكون حلاً للجزء المتجانس من المعادلة الاصلية ، نضرب التخمين في $n$.

\begin{example}
 جد الحل العام لمعادلة الفروق
 \[
 x_n - x_{n-1} - 6x_{n-2} = -36n
 \]
\end{example}
\begin{solution}
	اولا نحل معادلة الفروق المتجانسة $ y_n - y_{n-1} - 6y_{n-2} = 0 $ حيث نفرض $y_n = k^n$ نحصل على المساعدة
	\[
	k^2 - k - 6=0 \quad \Rightarrow \quad (k-3)(k-2) =0
	\]
	اذن الدالة المتممة $y_n = A3^n + B (-2)^n$. من اجل الحل الخاص نفرض $x_n = \alpha n+\beta$ ونعوض في المعادلة الاصلية ، نحصل على
	\[
	\alpha n+\beta - (\alpha(n-1) + \beta) - 6(\alpha(n-2) + \beta) = -6\alpha n + 13\alpha - 6 \beta
	\]
	بالتالي $\alpha = 6, \beta=13$. اذن الحل الخاص $x_n = 6n + 13 $. ومنه نحصل على الحل العام
	\[
	x_n = A3^n + B (-2)^n + 6n + 13
	\]
\end{solution}

\begin{example}
	جد الحل العام لمعادلة الفروق
	\[
	x_{n+1} - 2x_n + x_{n-1} = 8.
	\]
\end{example}
لحل المعادلة المتجانسة $y_{n+1} - 2y_n + y_{n-1}  = 0 $ نجرب $y_n = k^n$ نحصل على المعادلة المساعدة
\[
k^2 - 2k + 1 = 0
\]
التي تمتلك جذر متكرر $k=1$ لذا فإن الدالة المتممة 
\[
y_n = A + Bn
\]
من اجل الخاص لا يمكننا ان نجرب $x_n = c$ ($A$ جزء من الدالة المتممة). ولا يمكننا ان نجرب $x_n=cn$ (لان $Bn$ جزء من الدالة المتممة)، لذا نجرب $x_n = cn^2$. نعوض في المعادلة الاصلية 
\[
c(n+1)^2 - 2cn^2 + c(n-1)^2 = c[n^2+2n+1-2n^2+n^2-2n+1] =2c=8
\]
اي ان $c=4$. بالتالي الحل الخاص $x_n = 4n^2$. والحل العام
\[
x_n = 4n^2 + A + Bn
\]

\subsection*{ب - عندما تكون \en{\textit{f\textsubscript{n}} = \textit{a\textsuperscript{n}}}}
هذه الحالة مشابهة للحالة في المعادلات التفاضلية عندما يكون في الطرف الايمن دالة اسية. اذا كان $a$ ليس حلاً للمعادلة المساعدة فإننا نجرب $x_n = C a^n$. اذ	ا كان $a$ جذراً غير مكرر للمعادلة المساعدة فإننا نجرب $x_n = Cna^n$. بينما اذا كان $a$ جذراً مكرراً للمعادلة المساعدة نجرب $x_n = Cn^2a^n$
\newpage
\begin{example}
	جد الحل العام لمعادلة الفروق
	\[
	x_{n+2} + x_{n+1} - 6x_n = 12(-2)^n
	\]
\end{example}
\begin{solution}
	نجد الحل اولا للمعادلة المتجانسة $y_{n+2} + y_{n+1} - 6y_n  = 0 $ نفرض $y_n = k^n$ نحصل على المعادلة المساعدة
	\[
	k^2 + k - 6 = 0\qquad \Rightarrow \qquad (k+3)(k-2)=0
	\]
	اذن $k=-3, k=2 $ والدالة المتممة تكون
	\[
	y_n = A 2^n + B (-3)^n
	\]
	بما ان $(-2)^n$ ليس حلاً للمعادلة المتجانسة. نفرض $x_n = C (-2)^n$ من اجل الحل الخاص. ونعوض في المعادلة الاصلية
	\[
	C(-2)^{n+2} + C(-2)^{n+1} - 6C(-2)^{n} = 12(-2)^n
	\]
	بالقسمة على $(-2)^n$ نحصل على 
	\[
	(-2)^2 C + (-2)C -6C = 12
	\]
	اي ان $C=-3$ بالتالي الحل الخاص يكون $x_n = -3(-2)^n$ والحل العام 
	\[
	x_n = A2^n + B(-3)^n -3(-2)^n
	\]
\end{solution}

\begin{example}
	جد الحل العام لمعادلة الفروق 
	\[
	x_{n+2} + x_{n+1} - 6x_n = 30 \times 2^n
	\]
\end{example}
\begin{solution}
	وجدنا اعلاه ان الدالة المتممة تكون  $y_n = A 2^n + B (-3)^n$. بما ان الطرف الايمن موجود في الدالة المتممة وبما ان 2 حل غير مكرر للمعادلة المساعدة، نفرض الحل الخاص $x_n = Cn2^n$ ونعوض في المعادلة الاصلية
	\[
	C(n+2)2^{n+2} + C(n+1)2^{n+1} - 6Cn2^{n} = 30 \times2^n
	\] 
	بالقسمة على $2^n$ 
	\[
	C[4(n+2) + 2(n+1) -6n] = 10C = 30
	\]
	اذن $C = 3$. بالتالي الحل الخاص يكون $x_n = 3n2^n$. والحل العام 
	\[
	x_n = A2^n + B(-3)^n + 3n2^n
	\]
\end{solution}