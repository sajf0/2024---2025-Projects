\chapter*{مقدمة}
\addcontentsline{toc}{chapter*}{مقدمة}
	
	تُعد معادلات الفروق (\LR{Difference Equations}) من الأدوات الرياضية الأساسية في تحليل الأنظمة المتقطعة والزمنية، حيث تمثل العلاقة بين قيم متتابعة لكمية معينة عبر فترات زمنية منفصلة. ويمكن اعتبارها النظير التفاضلي للمعادلات التفاضلية، ولكن بدلاً من التعامل مع التغير المستمر، فإنها تتعامل مع تغير القيم عند نقاط زمنية محددة.\\
	\noindent
	تُستخدم معادلات الفروق على نطاق واسع في مجالات متعددة مثل الاقتصاد، علم الحاسوب، الهندسة، ونظرية التحكم، حيث تُستخدم لنمذجة الأنظمة الرقمية، ودراسة النمو السكاني، وتحليل الأسواق المالية، وتصميم الأنظمة التفاعلية. ومن أبرز مزاياها أنها تُستخدم بكفاءة في دراسة الأنظمة التي لا يمكن نمذجتها بسهولة باستخدام المعادلات التفاضلية بسبب الطبيعة المتقطعة للبيانات أو الزمن.\\
	\noindent
	وتكمن أهمية معادلات الفروق في قدرتها على التنبؤ بالقيم المستقبلية بناءً على القيم السابقة، مما يجعلها أداة قوية في مجالات التنبؤ، التحكم، والتحليل العددي. كما أنها تُعد الأساس الرياضي للعديد من الخوارزميات العددية والحسابات الرقمية المستخدمة في الحواسيب والبرمجيات الهندسية.\\
	\noindent
	في الختام، تمثل معادلات الفروق جسراً بين الرياضيات النظرية والتطبيقات العملية، وهي تُمكّن الباحثين والمهندسين من دراسة وتحليل الأنظمة الديناميكية المتقطعة بطريقة منظمة وفعالة.
	
