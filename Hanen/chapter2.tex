\chapter{التشاكل الزمري والتشاكل الحلقي}

\section{التشاكل الزمري}

\begin{definition}
	لتكن كل من $(G_1, *)$ و $(G_2, \circ)$ زمرة ، تسمى الدالة $f : G_1\to G_2 $ انها تشاكل زمري اذا حققت الشرط  $f(a * b) = f(a) \circ f(b)$ لكل $a, b \in G_1 $ 
\end{definition}

\begin{example}
	لتكن كل من $(G_1, *)$ و $(G_2, \circ)$ زمرة بعنصر محايد $e_1 $ و $e_2$ على التوالي ولتكن الدالة $f : G_1\to G_2 $ معرفة بالشكل 
	\[
	f(a) = e_2, \forall a\in G_1
	\]
	الدالة $f$ تحقق شرط التشاكل ويسمى هذا التشاكل بالتشاكل التافه.
\end{example}
\noindent
\textbf{الحل}\\
$
\forall a, b\in G_1 , \quad f(a * b) = e_2 = e_2 \circ e_2 = f(a) \circ f(b)
$\\
بالتالي $f$ دالة تشاكل.

\begin{example}
	لتكن 
	$f : (\Z, +) \to (\Z_n, +_n)$ دالة معرفة بالشكل التالي $f(a) = [a], \forall a\in \Z$. بين هل ان $f$ تمثل تشاكل
\end{example}
\noindent
\textbf{الحل}\\
\noindent
لكل $a, b\in \Z$
\[
f(a + b) = [a + b] = [a] +_n [b] = f(a) +_n f(b)
\]
بالتالي $f$ دالة تشاكل.

\begin{definition}
	ليكن 
	$f : (G_1, *) \to (G_2, \circ)$ تشاكل زمري ، فإن
	\begin{enumerate}
		\item اذا كانت $f$ دالة شاملة فإن التشاكل يسمى تشاكل شامل.
		\item اذا كانت $f$ دالة متباينة فإن التشاكل يسمى تشاكل متباين.
		\item اذا كانت $f$ دالة تقابل فإن التشاكل يسمى تشاكل تقابلي.
	\end{enumerate}
\end{definition}
\newpage
\begin{theorem}
	ليكن 	$f : (G_1, *) \to (G_2, \circ)$ تشاكل زمري ، فإن
	\begin{enumerate}[leftmargin=*]
		\item $f(e_1) = e_2$.
		\item  $f(a^{-1}) = [f(a)]^{-1}$.
	\end{enumerate}
\end{theorem}
\noindent
\textbf{البرهان}\\
\noindent
1. $f(e_1)  \circ e_2= f(e_1 * e_1) = f(e_1) \circ f(e_1)$ وبقانون الاختصار نحصل على $f(e_1) = e_2$
2. $f(e_1) = f(a * a^{-1}) = f(a) \circ f(a^{-1})$\\
\hspace*{9pt} $f(e_1) = f(a^{-1} * a) = f(a^{-1}) \circ f(a)$\\
\hspace*{9pt} بالتالي $f(a^{-1}) = [f(a)]^{-1}$

\begin{definition}
		ليكن 	$f : (G_1, *) \to (G_2, \circ)$ تشاكل زمري ، فإن مجموعة كل عناصر المجموعة $G_1$ التي تكون صورتها عنصر المحايد للزمرة $G_2$ تسمى نواة التشاكل ويرمز لها بالرمز $\ker f$ اي
		\[
		\ker f = \{a\in G : f(a) = e_2\}
		\]
\end{definition}

\begin{example}
	لتكن
	$f : (\R, +) \to (\R-\{0\}, \cdot)$
	دالة معرفة بالشكل 
	$f(a) = 2^a, \forall a\in \R$
	جد نواة التشاكل
\end{example}
\noindent
\textbf{الحل}
\begin{align*}
	\ker f &= \{a \in \R  : f(a) = 1\}\\
&= \{a\in \R : 2^a = 1\}\\
&= \{a\in \R : a =0 \}\\
&= \{0\}
\end{align*}
	\begin{example}
		لتكن 
		$f : (\Z, +) \to (\Z_n, +_n)$ دالة معرفة بالشكل التالي $f(a) = [a], \forall a\in \Z$. جد نواة التشاكل؟
	\end{example}
	\noindent
	\textbf{الحل}
	\begin{align*}
		 \ker f &= \{a \in \Z: f(a) = [0]\}\\
		 &= \{a\in \Z: [a] = [0]\}\\
&= (n)
	\end{align*}
	
	\begin{theorem}
			ليكن 	$f : (G_1, *) \to (G_2, \circ)$ تشاكل زمري ، فإن $\ker f\leq G_1$.
	\end{theorem}
	\noindent
	\textbf{البرهان}\\
	\noindent
	بما ان $f(e_1) = e_2$ اذن $e_1\in \ker f$ وبالتالي $\ker f\neq \varnothing$ وبواسطة التعريف $\ker f \subseteq G_1$ \\
	الآن نفرض $a, b\in \ker f$ اذن $f(a) = f(b) = e_2$ اذن
	\begin{align*}
		f(a * b^{-1}) &= f(a) \circ f(b^{-1})\\
		&= f(a) \circ  [f(b)]^{-1}\\
		&= e_2 \circ  e_2^{-1}\\
		&= e_2
	\end{align*}
	بالتالي $a * b^{-1} \in \ker f$. اذن $\ker f\leq G_1$.\qed
	
	\begin{theorem}
					ليكن 	$f : (G_1, *) \to (G_2, \circ)$ تشاكل زمري ، فإن $\ker f=\{e_1\}$ اذا وفقط اذا كانت $f$ دالة متباينة.
	\end{theorem}
	\noindent
	\textbf{البرهان}\\
	\noindent
	($\Leftarrow$) لنفرض ان $a, b\in G_1$ بحيث $f(a) = f(b)$ اذن $f(a) \circ [f(b)]^{-1} = e_2$ وبالتالــــي $f(a * b^{-1}) = f(a) \circ f(b^{-1}) = e_2$ ولكن $\ker f = \{e_1\}$ اذن $a*b^{-1} = e_1$ اي ان $a = b$. اذن $f$ دالة متباينة.\\
	($\Rightarrow$) لنفرض ان $a \in \ker f$ بحيث $a\neq e_1$ وبما ان $f(e_1) = e_2$ اذن $f(a) = f(e_1) = e_2$ ولكن $f$ دالة متباينة اذن $a = e_1$ وهذا تناقض اذن $\ker f = \{e_1\}$.\qed
	\newpage
	\begin{theorem}
	ليكن 	$f : (G_1, *) \to (G_2, \circ)$ تشاكل زمري ، ولتكن $(H, *)$ زمرة جزئية من 
	$(G_1, *)$ فأن $(f(H), \circ )$ زمرة جزئية من $(G_2, \circ)$.
	\end{theorem}
	\noindent
	\textbf{البرهان}\\
	\noindent
	بما ان
	$f(H) = \{ f(h) : h\in H\}$ اذن $f(H) \subseteq G_2$ و $f(H) \neq \varnothing$ لأن $f(e_1) = e_2$ حيث $e_1 \in H$. الآن نفرض $f(h_1) , f(h_2) \in f(H)$ فإن 
	\[
	f(h_1) \circ f(h_2)^{-1} = f(h_1) \circ f(h_2^{-1}) = f(h_1 * h_2^{-1} )  = f(e_1)\in f(H). 
 	\]
 	اذن $(f(H), \circ )$ زمرة جزئية من $(G_2, \circ)$.\qed
 	
 	\begin{theorem}
 ليكن $f : (G_1, *) \to (G_2, \circ)$ تشـــــــــــاكل زمـــــــري شـــامل وأن 
 $(H, *) \trianglelefteq (G_1, *)$ ، فأن\\
 $(f(H), \circ) \trianglelefteq (G_2, \circ)$
 ، هذا يعني الصورة التشاكلية الشاملة لأي زمرة جزئية سوية تكون أيضاً زمرة جزئية سوية.
 	\end{theorem}
\noindent
\textbf{البرهان}\\
\noindent
 $(f(H), \circ) \leq (G_2, \circ)$ من المبرهنة السابقة ، الآن نفرض ان $f(h) \in f(H)$ و $a\in G_2$. وليكن $a\circ f(h) \circ a^{-1} \in a \circ f(H) \circ a^{-1}$، وبما أن $a\in G_2$ و $f$ دالة شاملة اذن يوجد $b\in G_1$ بحيث ان $a = f(b)$.  اذن
 $
 a\circ f(h) \circ a^{-1}= f(b) \circ f(h) \circ f(b)^{-1} = f(b*h*b^{-1})
 $
 وبما أن
 $b * h * b^{-1} \in H$ بالتالي $(f(H), \circ) \trianglelefteq (G_2, \circ)$.\qed
 
 \begin{theorem}
 ليكن $f : (G_1, *) \to (G_2, \circ)$ تشاكل زمري ولتكن 
 $(H, \circ) \trianglelefteq (G_2, \circ)$
 فأن 
 $(f^{-1}(H), *) \trianglelefteq (G_1, *)$
 \end{theorem}
\noindent
\textbf{البرهان}\\
\noindent
لنفرض ان $x \in f^{-1}(H)$ و $a\in G_1$ وليكن 
$a * x * a^{-1} \in a * f^{-1}(H) *a^{-1}$ بالتالــي
$f(a * x * a^{-1}) = f(a) * f(x) * [f(a)]^{-1} \in H$ اذن $(f^{-1}(H), *) \trianglelefteq (G_1, *)$.\qed
\newpage
\begin{corollary}
ليكن $f : (G_1, *) \to (G_2, \circ)$ تشاكل زمري فأن 
$(\ker f, *) \leq (G_1, *)$ 
هذا يعني ان نواة اي تشاكل زمري يكون زمرة جزئية.
\end{corollary}
\noindent
\textbf{البرهان}\\
\noindent
بما ان
$\ker f = f^{-1}(\{e\})$
اذن بواسطة المبرهنة السابقة نستنتج ان 
$(\ker f, *) \leq (G_1, *)$. \qed 

\begin{definition}
	ليكن 
	$(G_1, *) , (G_2, \circ)$ زمرتان ، يقال انهما متشاكلتان اذا وجدت دالة بينهما تشاكل تقابلي و نكتب $(G_1, *) \cong (G_2, \circ)$.
\end{definition}

\begin{example}
	ببين أن 
	$(\Z_2, +_2) \cong (\{1,-1\}, \cdot)$.
\end{example}
\begin{solution}
	لتكن 
	$f  : (\Z_2, +_2) \to (\{1,-1\}, \cdot)$  دالة معرفة بالشكل
	\[
	f(0) = 1, \quad f(1) = -1
	\]
	اذن نلاحظ
	$
	\begin{gathered}[t]
		f(0 +_2 0) = f(0) = 1 = f(0)\cdot f(0)\\
		f(1+_2 0) = f(1) = -1 = f(1) \cdot f(0)\\
		f(1 +_2 1) = f(0) = 1 = f(1) \cdot f(1)
	\end{gathered}
	$\\
	اذن $f$ دالة تشاكل ، ايضاً $f(\Z_2) = \{1,-1\}$ اذن الدالة شاملة وواضح انها متباينة لذلك هي تقابل. اذن 	$(\Z_2, +_2) \cong (\{1,-1\}, \cdot)$.\qed
\end{solution}
