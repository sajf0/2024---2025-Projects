\chapter{التشاكل الزمري والتشاكل الحلقي}

\begin{definition}[\cite{groups1}]
	لتكن $A$ و $B$ مجموعتين. الدالة هي علاقة تربط كل عنصر $a \in A$ بعنصر وحيد $f(a) \in B$، ويُرمز لها بـ:
	\[
	f : A \to B \quad \text{حيث} \quad a \mapsto f(a)
	\]
	وتُسمى $A$ مجال الدالة و$B$ المجال المقابل أو المدى.
\end{definition}

\subsection*{انواع الدوال}
	\begin{itemize}
		\item \textbf{دالة شاملة (Surjective):} تُسمى $f : A \to B$ شاملة إذا:
		\[
		\forall b \in B, \exists a \in A : f(a) = b
		\]
		أي أن صورة $f$ تغطي كامل المجموعة $B$.
		
		\item \textbf{دالة متباينة (Injective):} تُسمى $f : A \to B$ متباينة إذا:
		\[
		f(a_1) = f(a_2) \Rightarrow a_1 = a_2
		\]
		أو مكافئًا:
		\[
		a_1 \neq a_2 \Rightarrow f(a_1) \neq f(a_2)
		\]
		أي أن كل عنصر في $B$ له أصل واحد على الأكثر.
		
		\item \textbf{دالة تقابلية (Bijective):} تُسمى $f : A \to B$ تقابلية إذا كانت شاملة ومتباينة معًا، أي أن:
		\[
		\forall b \in B, \exists! a \in A : f(a) = b
		\]
		وفي هذه الحالة تكون $f$ قابلة للعكس ويكون لها معكوس $f^{-1} : B \to A$.
	\end{itemize}



\begin{definition}[\cite{groups2}]
	\addcontentsline{toc}{section}{تعريف التشاكل الزمري}
	لتكن كل من $(G_1, *)$ و $(G_2, \circ)$ زمرة ، تسمى الدالة $f : G_1\to G_2 $ انها تشاكل زمري اذا حققت الشرط  $f(a * b) = f(a) \circ f(b)$ لكل $a, b \in G_1 $ 
\end{definition}

\begin{example}
	لتكن كل من $(G_1, *)$ و $(G_2, \circ)$ زمرة بعنصر محايد $e_1 $ و $e_2$ على التوالي ولتكن الدالة $f : G_1\to G_2 $ معرفة بالشكل 
	\[
	f(a) = e_2, \forall a\in G_1
	\]
	الدالة $f$ تحقق شرط التشاكل ويسمى هذا التشاكل بالتشاكل التافه.
\end{example}
\noindent
\textbf{الحل}\\
$
\forall a, b\in G_1 , \quad f(a * b) = e_2 = e_2 \circ e_2 = f(a) \circ f(b)
$\\
بالتالي $f$ دالة تشاكل.

\begin{example}
	لتكن 
	$f : (\Z, +) \to (\Z_n, +_n)$ دالة معرفة بالشكل التالي $f(a) = [a], \forall a\in \Z$. بين هل ان $f$ تمثل تشاكل
\end{example}
\noindent
\textbf{الحل}\\
\noindent
لكل $a, b\in \Z$
\[
f(a + b) = [a + b] = [a] +_n [b] = f(a) +_n f(b)
\]
بالتالي $f$ دالة تشاكل.

\begin{definition}[\cite{groups2}]
	ليكن 
	$f : (G_1, *) \to (G_2, \circ)$ تشاكل زمري ، فإن
	\begin{enumerate}
		\item اذا كانت $f$ دالة شاملة فإن التشاكل يسمى تشاكل شامل.
		\item اذا كانت $f$ دالة متباينة فإن التشاكل يسمى تشاكل متباين.
		\item اذا كانت $f$ دالة تقابل فإن التشاكل يسمى تشاكل تقابلي.
	\end{enumerate}
\end{definition}
\newpage
\begin{theorem}[\cite{groups2}]
	ليكن 	$f : (G_1, *) \to (G_2, \circ)$ تشاكل زمري ، فإن
	\begin{enumerate}[leftmargin=*]
		\item $f(e_1) = e_2$.
		\item  $f(a^{-1}) = [f(a)]^{-1}$.
	\end{enumerate}
\end{theorem}
\noindent
\textbf{البرهان}\\
\noindent
1. $f(e_1)  \circ e_2= f(e_1 * e_1) = f(e_1) \circ f(e_1)$ وبقانون الاختصار نحصل على $f(e_1) = e_2$
2. $f(e_1) = f(a * a^{-1}) = f(a) \circ f(a^{-1})$\\
\hspace*{9pt} $f(e_1) = f(a^{-1} * a) = f(a^{-1}) \circ f(a)$\\
\hspace*{9pt} بالتالي $f(a^{-1}) = [f(a)]^{-1}$

\begin{definition}[\cite{groups2}]
	\addcontentsline{toc}{section}{نواة التشاكل الزمري}
		ليكن 	$f : (G_1, *) \to (G_2, \circ)$ تشاكل زمري ، فإن مجموعة كل عناصر المجموعة $G_1$ التي تكون صورتها عنصر المحايد للزمرة $G_2$ تسمى نواة التشاكل ويرمز لها بالرمز $\ker f$ اي
		\[
		\ker f = \{a\in G : f(a) = e_2\}
		\]
\end{definition}

\begin{example}
	لتكن
	$f : (\R, +) \to (\R-\{0\}, \cdot)$
	دالة معرفة بالشكل 
	$f(a) = 2^a, \forall a\in \R$
	جد نواة التشاكل
\end{example}
\noindent
\textbf{الحل}
\begin{align*}
	\ker f &= \{a \in \R  : f(a) = 1\}\\
&= \{a\in \R : 2^a = 1\}\\
&= \{a\in \R : a =0 \}\\
&= \{0\}
\end{align*}
	\begin{example}
		لتكن 
		$f : (\Z, +) \to (\Z_n, +_n)$ دالة معرفة بالشكل التالي $f(a) = [a], \forall a\in \Z$. جد نواة التشاكل؟
	\end{example}
	\noindent
	\textbf{الحل}
	\begin{align*}
		 \ker f &= \{a \in \Z: f(a) = [0]\}\\
		 &= \{a\in \Z: [a] = [0]\}\\
&= (n)
	\end{align*}
	
	\begin{theorem}[\cite{groups1}]
			ليكن 	$f : (G_1, *) \to (G_2, \circ)$ تشاكل زمري ، فإن $\ker f\leq G_1$.
	\end{theorem}
	\noindent
	\textbf{البرهان}\\
	\noindent
	بما ان $f(e_1) = e_2$ اذن $e_1\in \ker f$ وبالتالي $\ker f\neq \varnothing$ وبواسطة التعريف $\ker f \subseteq G_1$ \\
	الآن نفرض $a, b\in \ker f$ اذن $f(a) = f(b) = e_2$ اذن
	\begin{align*}
		f(a * b^{-1}) &= f(a) \circ f(b^{-1})\\
		&= f(a) \circ  [f(b)]^{-1}\\
		&= e_2 \circ  e_2^{-1}\\
		&= e_2
	\end{align*}
	بالتالي $a * b^{-1} \in \ker f$. اذن $\ker f\leq G_1$.\qed
	
	\begin{theorem}[\cite{groups2}]
					ليكن 	$f : (G_1, *) \to (G_2, \circ)$ تشاكل زمري ، فإن $\ker f=\{e_1\}$ اذا وفقط اذا كانت $f$ دالة متباينة.
	\end{theorem}
	\noindent
	\textbf{البرهان}\\
	\noindent
	($\Leftarrow$) لنفرض ان $a, b\in G_1$ بحيث $f(a) = f(b)$ اذن $f(a) \circ [f(b)]^{-1} = e_2$ وبالتالــــي $f(a * b^{-1}) = f(a) \circ f(b^{-1}) = e_2$ ولكن $\ker f = \{e_1\}$ اذن $a*b^{-1} = e_1$ اي ان $a = b$. اذن $f$ دالة متباينة.\\
	($\Rightarrow$) لنفرض ان $a \in \ker f$ بحيث $a\neq e_1$ وبما ان $f(e_1) = e_2$ اذن $f(a) = f(e_1) = e_2$ ولكن $f$ دالة متباينة اذن $a = e_1$ وهذا تناقض اذن $\ker f = \{e_1\}$.\qed
	\newpage
	\begin{theorem}[\cite{groups1}]
	ليكن 	$f : (G_1, *) \to (G_2, \circ)$ تشاكل زمري ، ولتكن $(H, *)$ زمرة جزئية من 
	$(G_1, *)$ فأن $(f(H), \circ )$ زمرة جزئية من $(G_2, \circ)$.
	\end{theorem}
	\noindent
	\textbf{البرهان}\\
	\noindent
	بما ان
	$f(H) = \{ f(h) : h\in H\}$ اذن $f(H) \subseteq G_2$ و $f(H) \neq \varnothing$ لأن $f(e_1) = e_2$ حيث $e_1 \in H$. الآن نفرض $f(h_1) , f(h_2) \in f(H)$ فإن 
	\[
	f(h_1) \circ f(h_2)^{-1} = f(h_1) \circ f(h_2^{-1}) = f(h_1 * h_2^{-1} )  = f(e_1)\in f(H). 
 	\]
 	اذن $(f(H), \circ )$ زمرة جزئية من $(G_2, \circ)$.\qed
 	
 	\begin{theorem}[\cite{groups2}]
 ليكن $f : (G_1, *) \to (G_2, \circ)$ تشـــــــــــاكل زمـــــــري شـــامل وأن 
 $(H, *) \trianglelefteq (G_1, *)$ ، فأن\\
 $(f(H), \circ) \trianglelefteq (G_2, \circ)$
 ، هذا يعني الصورة التشاكلية الشاملة لأي زمرة جزئية سوية تكون أيضاً زمرة جزئية سوية.
 	\end{theorem}
\noindent
\textbf{البرهان}\\
\noindent
 $(f(H), \circ) \leq (G_2, \circ)$ من المبرهنة السابقة ، الآن نفرض ان $f(h) \in f(H)$ و $a\in G_2$. وليكن $a\circ f(h) \circ a^{-1} \in a \circ f(H) \circ a^{-1}$، وبما أن $a\in G_2$ و $f$ دالة شاملة اذن يوجد $b\in G_1$ بحيث ان $a = f(b)$.  اذن
 $
 a\circ f(h) \circ a^{-1}= f(b) \circ f(h) \circ f(b)^{-1} = f(b*h*b^{-1})
 $
 وبما أن
 $b * h * b^{-1} \in H$ بالتالي $(f(H), \circ) \trianglelefteq (G_2, \circ)$.\qed
 
 \begin{theorem}[\cite{groups1}]
 ليكن $f : (G_1, *) \to (G_2, \circ)$ تشاكل زمري ولتكن 
 $(H, \circ) \trianglelefteq (G_2, \circ)$
 فأن 
 $(f^{-1}(H), *) \trianglelefteq (G_1, *)$
 \end{theorem}
\noindent
\textbf{البرهان}\\
\noindent
لنفرض ان $x \in f^{-1}(H)$ و $a\in G_1$ وليكن 
$a * x * a^{-1} \in a * f^{-1}(H) *a^{-1}$ بالتالــي
$f(a * x * a^{-1}) = f(a) * f(x) * [f(a)]^{-1} \in H$ اذن $(f^{-1}(H), *) \trianglelefteq (G_1, *)$.\qed
\newpage
\begin{corollary}[\cite{groups2}]
ليكن $f : (G_1, *) \to (G_2, \circ)$ تشاكل زمري فأن 
$(\ker f, *) \leq (G_1, *)$ 
هذا يعني ان نواة اي تشاكل زمري يكون زمرة جزئية.
\end{corollary}
\noindent
\textbf{البرهان}\\
\noindent
بما ان
$\ker f = f^{-1}(\{e\})$
اذن بواسطة المبرهنة السابقة نستنتج ان 
$(\ker f, *) \leq (G_1, *)$. \qed 

\begin{definition}[\cite{groups1}]
	ليكن 
	$(G_1, *) , (G_2, \circ)$ زمرتان ، يقال انهما متشاكلتان اذا وجدت دالة بينهما تشاكل تقابلي و نكتب $(G_1, *) \cong (G_2, \circ)$.
\end{definition}

\begin{example}
	ببين أن 
	$(\Z_2, +_2) \cong (\{1,-1\}, \cdot)$.
\end{example}
\begin{solution}
	لتكن 
	$f  : (\Z_2, +_2) \to (\{1,-1\}, \cdot)$  دالة معرفة بالشكل
	\[
	f(0) = 1, \quad f(1) = -1
	\]
	اذن نلاحظ
	$
	\begin{gathered}[t]
		f(0 +_2 0) = f(0) = 1 = f(0)\cdot f(0)\\
		f(1+_2 0) = f(1) = -1 = f(1) \cdot f(0)\\
		f(1 +_2 1) = f(0) = 1 = f(1) \cdot f(1)
	\end{gathered}
	$\\
	اذن $f$ دالة تشاكل ، ايضاً $f(\Z_2) = \{1,-1\}$ اذن الدالة شاملة وواضح انها متباينة لذلك هي تقابل. اذن 	$(\Z_2, +_2) \cong (\{1,-1\}, \cdot)$.\qed
\end{solution}

\newpage

\begin{theorem}[\cite{groups1}]
	\addcontentsline{toc}{section}{مبرهنة التشاكل الاساسية في الزمر}
	لتكن 
	$f : (G_1, *) \to (G_2, \circ)$ تشاكل شامل فإن 
	$(G/\ker f, \otimes) \cong (G_2, \circ)$
\end{theorem}
\noindent
\textbf{البرهان}\\
\noindent
ليكن $H = \ker f$ بما ان
 $(\ker f, *) \leq (G_1, *)$ و 
 $(G_1/H, \otimes)$ 
زمرة\\
 لنفرض ان $\phi : G_1/H\to G_2$ دالة معرفة بالشكل الآتي $\phi(a*H) = f(a)$ لبرهان انها معرفة تعريفاً حسناً ليكن $a * H = b * H$ حيث 
 $a*H, b*H \in G_1/H$ اذن $a^{-1} * b\in H$ اذن $f(a^{-1}*b) = e_2$ لذلك $[f(a)]^{-1} \circ f(b) = e_2$ اذن $f(a) = f(b)$  وهذا يعني ان الدالة $\phi $ دالة حسنة التعريف.
 لبرهان ان $\phi$ تشاكل
 \begin{align*}
 	 \phi(a * H \otimes b*H) &= \phi[(a*b)*H]\\
 	  &= f(a*b)\\
 	   &= f(a)\circ f(b)\\
 	    &= \phi(a*H)\circ \phi(b*H) 
 \end{align*}
 لبرهان ان $\phi$ متباينة
 \begin{align*}
 	\ker f &= \{a*H \in G_1/H : \phi(a*H) = e_2\}\\
 	&= \{a*H \in G_1/H : a\in H\}\\
 	&= H
 \end{align*}
 وليكن $f(a) \in G_2$ حيث ان $a\in G_1$ يؤدي الى $a*H\in G_1/H$ اذن $\phi(a*H) = f(a)$
اذن $\phi$ تشاكل متقابل لذلك $(G/\ker f, \otimes) \cong (G_2, \circ)$
\newpage
\begin{definition}[\cite{rings}]
	\addcontentsline{toc}{section}{تعريف التشاكل الحلقي}
	لنفترض ان $R$ و $S$ حلقتان ، تسمى الدالة $f:R\to S$ تشاكل حلقي اذا وفقط اذا كان
	\setLR
	\begin{enumerate}[label=(\arabic*)]
		\item $f(a + b) = f(a) + f(b)$.
		\item $f(a\cdot b) = f(a) \cdot f(b)$.
	\end{enumerate}
\end{definition}
\setRL
\noindent
\textbf{ملاحظة}
	\begin{enumerate}
	\item اذا كانت $f$ دالة شاملة فإن التشاكل يسمى تشاكل شامل.
	\item اذا كانت $f$ دالة متباينة فإن التشاكل يسمى تشاكل متباين.
	\item اذا كانت $f$ دالة تقابل فإن التشاكل يسمى تشاكل تقابلي.
\end{enumerate}

\begin{example}
		لنفترض ان $R$ و $S$ حلقتان ، نعرف الدالة $f:R\to S$ على انها  $f(a) = 0, \forall a\in R$ هي تشاكل وتسمى بالتشاكل التافه.
\end{example}
\noindent
\textbf{الحل}\\
\noindent
1. $f(a + b) = 0 = 0 + 0 = f(a) + f(b)$.\\
2. $f(a\cdot b) = 0 = 0\cdot 0 = f(a ) \cdot f(b)$.\\
اذن $f$ تشاكل.

\begin{example}
	افترض ان $R$ حلقة ، نعرف الدالة $f : R\to R$ بالشكل 
	$f(a) = a, \forall a\in R$
	تكون تشاكل تقابلي.
\end{example}
\noindent
\textbf{الحل}\\
\noindent
1. $f(a + b) = a + b = f(a) + f(b)$.\\
2. $f(a\cdot b) = a\cdot b = f(a) \cdot f(b)$. \\
اذا كان 
$f(x) = f(y)$ لبعض $x,y \in R$ فأن $x=y$. اذن $f$ تباين.\\
الآن لكل $y\in R$ فأن $ f(y) = y$ اذن $f$ شاملة. بالتالي $f$ تشاكل تقابلي.

\begin{definition}[\cite{rings}]
	\addcontentsline{toc}{section}{نواة التشاكل الحلقي}
	لتكن $f$ تشاكل من الحلقة $R$ الى الحلقة $S$. فإن المجموعة 
	\[
	\ker f = \{a\in R : f(a) = 0\}
	\]
	تسمى بنواة التشاكل $f$.
\end{definition} 

\begin{theorem}[\cite{rings}]
		لتكن $f$ تشاكل من الحلقة $R$ الى الحلقة $S$. فإن $\ker f$ هي مثالية في $R$.
\end{theorem}
\noindent
\textbf{البرهان}

\begin{english}
	\begin{enumerate}
		\item  $f(0) = 0 \Rightarrow 0 \in \ker f \Rightarrow \ker f \neq \varnothing$.
	\item  $\forall x, y\in \ker f \Rightarrow f(x) = f(y) = 0\Rightarrow f(x-y) = f(x) - f(y) = 0-0 =0 \Rightarrow x-y \in \ker f$.
	\item  $\forall x\in \ker f, \forall r\in R \Rightarrow f(x) = 0 \Rightarrow f(rx) = f(r)\cdot f(x) = f(r) \cdot 0 = 0 \Rightarrow rx \in \ker f.$
	\end{enumerate}
	\end{english}
	
	\noindent اذن $\ker f$ مثالية في $R$.
	
\begin{definition}[\cite{rings}]
	نفرض ان $R, S$ حلقتان بحيث $f:R\to S$ تشاكل تقابلي ، نقول ان $R$ تماثل $S$ ونكتب $R\simeq S$.
\end{definition}

\begin{example}
لنعرف الدالة 
$f : \Z_6\to \Z_2\times \Z_3$ بالشكل 
$f([a]_6) = ([a]_2, [a]_3)$ ، $f$ تكون دالة تشاكل تقابلي وبالتالي $\Z_6 \simeq\Z_2\times \Z_3$.
\end{example}

\begin{definition}
	ليكن $I$ مثاليًا (ideal) في الحلقة $(R, +, \cdot)$، نُعرِّف التشاكل الطبيعي 
	$\pi : R \to R/I$ بالشكل التالي:
	\[
	\pi(r) = r + I
	\]
	ويسمى هذا التشاكل {بالتشاكل الطبيعي}.
\end{definition}
\begin{theorem}[\cite{rings}]
	ليكن $I$ مثاليًا (ideal) في الحلقة $(R, +, \cdot)$، فإن التطبيق الطبيعي 
$\pi : R \to R/I$ 
يكون تشاكل و 
$\ker(\pi) = I$
\end{theorem}
\noindent
\textbf{البرهان}\\
\noindent
نلاحظ أن:
\begin{align*}
	\pi(r + s) &= (r + s) + I = (r + I) + (s + I) = \pi(r) + \pi(s)\\
	\pi(r \cdot s) &= (rs) + I = (r + I)(s + I) = \pi(r) \cdot \pi(s)
\end{align*}

إذن $\pi$ يحفظ الجمع والضرب، وبالتالي هو تشاكل حلقات.

\bigskip

\noindent
\textbf{برهان أن نواة $\pi$ هي $I$}\\
\noindent
نحسب:
\begin{align*}
	\ker \pi &= \{r \in R : \pi(r) = 0 + I\}\\
	&= \{r \in R : r + I = I\}\\
	&= \{r \in R : r \in I\}\\
	&= I
\end{align*}

وبالتالي:
\[
\ker \pi = I
\]


\begin{theorem}[\cite{rings}]
	\addcontentsline{toc}{section}{مبرهنة التشاكل الأساسية في الحلقات}
	لتكن 
	$f : (R, +, \cdot) \to (S, +, \cdot)$ تشاكل شامل بين حلقتين، فإن 
	$(R/\ker f, +, \cdot) \cong (S, +, \cdot)$.
\end{theorem}
\noindent
\textbf{البرهان}\\
\noindent
ليكن $K = \ker f$ بما أن 
$K$ مثالي من الحلقة $(R, +, \cdot)$ فإن 
$(R/K, +, \cdot)$ حلقة.\\
لنفرض أن $\phi : R/K \to S$ دالة معرفة بالشكل الآتي: 
$\phi(r + K) = f(r)$\\
لبرهان أنها معرفة تعريفاً حسناً، لنأخذ $r + K = s + K$ حيث 
$r + K, s + K \in R/K$، إذن $r - s \in K$، إذن 
$f(r - s) = 0$، وبالتالي 
$f(r) - f(s) = 0$ أي أن $f(r) = f(s)$، وهذا يعني أن الدالة $\phi$ معرفة تعريفاً حسناً.

لبرهان أن $\phi$ تشاكل:
\begin{align*}
	\phi((r + K) + (s + K)) &= \phi((r + s) + K)\\
	&= f(r + s)\\
	&= f(r) + f(s)\\
	&= \phi(r + K) + \phi(s + K)\\
   &= \phi(rs + K)\\
	&= f(rs)\\
	&= f(r) \cdot f(s)\\
	&= \phi(r + K) \cdot \phi(s + K)
\end{align*}

لبرهان أن $\phi$ متباينة:
\begin{align*}
	\ker \phi &= \{r + K \in R/K : \phi(r + K) = 0_S\}\\
	&= \{r + K \in R/K : f(r) = 0_S\}\\
	&= \{r + K : r \in K\}\\
	&= K
\end{align*}

وليكن $f(r) \in S$ حيث أن $r \in R$ يؤدي إلى $r + K \in R/K$ إذن 
$\phi(r + K) = f(r)$،\\
إذن $\phi$ تشاكل متقابل، وبالتالي:
\[
(R/\ker f, +, \cdot) \cong (S, +, \cdot)
\]


