\chapter{نظرية الزمر}

\section{مفاهيم أولية}

\begin{definition}[(العملية الثنائية)]
	لتكن $G$ مجموعة غير خالية ، نطلق على التطبيق $*:G\to G\times G$ بأنه عملية ثنائية على $G$.
\end{definition}

\begin{note}
	اذا كانت $*$ عملية ثنائية على مجموعة $G$ سنكتب العلاقة بين عناصرها بالشكل $a*b$ بدل من $*(a, b)$ لغرض السهولة.
\end{note}

\begin{example}
عملية الجمع الاعتيادية على مجموعة الاعداد الصحيحة والطبيعية والنسبية والحقيقية تمثل عملية ثنائية وكذلك عملية الضرب الاعتيادي. 
\end{example}

\begin{example}
	لتكن $X = \{1,2,3\}$ ، العملية $*$ معرفة على المجموعة $X$ بالشكل
	\begin{table}[H]
		\renewcommand{\arraystretch}{1.4}
		\centering
		\setLR
		\begin{tabular}{|c|c|c|c|}
			\hline
			$*$ & 1&2&3\\
			\hline
			1&1&2&3\\
			\hline
			2&2&2&1\\
			\hline
			3&3&1&3\\
			\hline
		\end{tabular}
	\end{table}
	\setRL\noindent
	نلاحظ ان $*$ تمثل عملية ثنائية.
\end{example}

\begin{definition}[( الانغلاق )]
	لتكن $*$ عملية ثنائية على المجموعة $X$ ، المجموعة الجزئية $A$ من $G$ تسمى مغلقة تحت العملية $*$ اذا كان $a*b \in A$ لكل عنصرين $a, b\in A$.
\end{definition}

\begin{example}
	نحن نعلم ان $+$ عملية الجمع الاعتيادي على مجموعة الاعداد الحقيقية ، نلاحظ ان $+$ عملية ثنائية مغلقة على مجموعة الاعداد الصحيحة لأن
	\[
	a+b \in \Z ,\quad \forall a, b\in \Z
	\]
\end{example}

\begin{definition}[( النظام الرياضي )]
	هو مجموعة غير خالية $G$ مع عملية ثنائية واحدة او اكثر معرفة عليه. ويرمز له بالرمز المرتب $(G, *, \#)$ او $(G, *)$.
\end{definition}

\begin{definition}[( العملية التجميعية )]
	ليكن $(G, *)$ نظاماً رياضياً مع $*$ عملية ثنائية معرفة عليه ، يقال ان العملية $*$ تجميعية اذا حققت الشرط
	\[
	a*(b*c) = (a*b)*c,\quad \forall a,b,c\in G
	\]
\end{definition}

\begin{example}
	لتكن $X = \{1,2,3\}$ ، العملية $*$ معرفة على المجموعة $X$ بالشكل
	\begin{table}[H]
		\renewcommand{\arraystretch}{1.4}
		\centering
		\setLR
		\begin{tabular}{|c|c|c|c|}
			\hline
			$*$ & 1&2&3\\
			\hline
			1&1&2&3\\
			\hline
			2&2&1&2\\
			\hline
			3&3&3&3\\
			\hline
		\end{tabular}
	\end{table}
	\setRL\noindent
	نلاحظ ان $*$ تمثل عملية تجميعية.
\end{example}

\begin{example}
	لتكن $*$ عملية معرفة على $\Z$ كما يأتي : $a*b =a+b-1$ لكل عنصرين $a, b\in \Z$ ، فإن $*$ عملية تجميعية.
\end{example}

\begin{definition}[( العنصر المحايد )]
	ليكن $(G, *)$ نظاماً رياضياً ، يقال ان النظام الرياضي $(G, *)$ يمتلك عنصراً محايداً بالنسبة للعملية الثنائية $*$ اذا وجد عنصر $e \in G$ بحيث ان
	\[
	a * e = e*a = a,\quad \forall a\in G
	\]
\end{definition}

\begin{theorem}
	لتكن $(G, *)$ نظاماً رياضياً بعنصر محايد فإن المحايد وحيد.
\end{theorem}
\noindent
\textbf{البرهان}\\
\noindent
لتكن $e, e'$ عنصران محايدان بالنسبة للعملية $*$ اذن\\
$e*e' =e'$ لان $e$ عنصر محايد.\\
$e*e' =e'$ لان $e'$ عنصر محايد.\\
اذن $e=e'$. \qed

\begin{definition}[(monoid)]
	لتكن $(G, *)$ شبه زمرة ، اذا كانت تمتلك عنصر محايد فإنها تسمى (monoid).
\end{definition}

\begin{definition}[(  المعكوس )]
	لتكن $(G, *)$ شبه زمرة بمحايد اذا كان $a\in G$ يحقق الخاصية : $a'*a=a*a'=e$ حيث ان $a'\in G$ ، فإن العنصر $a'$ يسمى معكوس العنصر $a$ بالنسبة للعملية $*$ ويرمز له بالرمز $a^{-1}$.
\end{definition}

\begin{note}
	لتكن $(G, *)$ شبه زمرة بعنصر محايد $e$ فأن $e^{-1}=e$
\end{note}

\begin{theorem}
	لتكن $(G, *)$ شبه زمرة بعنصر محايد وليكن $a\in G$ وله معكوس في $G$ فأن المعكوس وحيد.
\end{theorem}

\begin{definition}[( العملية الابدالية )]
	ليكن $(G, *)$ نظاماً رياضياً مع $*$ عملية ثنائية معرفة عليه ، يقال ان العملية $*$ ابدالية اذا حققت الشرط
	\[
	a*b = b*a,\quad \forall a,b\in G
	\]
\end{definition}

\begin{example}
	عمليتي الجمع والضرب الاعتياديتين على مجموعة الاعداد الحقيقية والصحيحة والنسبية $\Q, \Z, \R$ تمثل عمليات ثنائية ابدالية. 
\end{example}
\newpage
\section{تعريف الزمرة وبعض خصائصها}
\begin{definition}[( الزمرة )]
	لتكن $(G, *)$ شبه زمرة بعنصر محايد فأن $G$ تسمى زمرة Group اذا كان كل عنصر فيها له معكوس بالنسبة للعملية الثنائية $*$. او نقول ان $(G, *)$ زمرة اذا تحققت الشروط التالية
	\begin{enumerate}[label=$\boxed{\arabic*}$]
		\item مغلقة بالنسبة للعملية $*$ اي : $a*b \in G , \forall a, b\in G$.
		\item العملية $*$ تجميعية : $a*(b*c) = (a*b)*c , \forall a, b,c\in G$.
		\item $G$ تمتلك عنصر محايد مثل $e$ : $a*e =e*a=a, \forall a\in G$.
		\item كل عنصر $a\in G$ يمتلك معكوس : $\forall a\in G, \exists a^{-1}\in G : a*a^{-1}=a^{-1}*a=e$.
	\end{enumerate}    
\end{definition}

\begin{example}
	كل الانظمة الرياضية التالية تمثل زمرة
	\[
	(\Q-\{0\}, \cdot), \quad(\R-\{0\}, \cdot),\quad (\Q, +) ,\quad (\R, +) ,\quad (\Z, +)
	\]
\end{example}

\begin{definition}[( الزمرة الابدالية )]
	الزمرة $(G, *)$ تسمى زمرة ابدالية اذا كانت العملية $*$ عملية ثنائية ابدالية.
\end{definition}

\begin{example}
	كل الانظمة الرياضية التالية تمثل زمرة ابدالية
	\[
	(\R-\{0\}, \cdot),\quad (\Q, +) ,\quad (\R, +) ,\quad (\Z, +)
	\]
\end{example}

\begin{definition}[( الزمرة المنتهية وغير المنتهية )]
	الزمرة $(G, * )$ تسمى زمرة منتهية اذا كانت المجموعة $G$ منتهية. عدا ذلك تسمى الزمرة $(G, *)$ زمرة غير منتهية. 
\end{definition}

\begin{definition}[( رتبة الزمرة )]
	لتكن $(G, *)$ زمرة منتهية ، نطلق على عدد عناصر المجموعة $G$ اسم رتبة الزمرة ويرمز له بالرمز $O(G)$ اما اذا كانت المجموعة غير منتهية فتكون رتبتها غير منتهية ايضاً.
\end{definition}

\begin{definition}[( قوى العنصر )]
	لتكن $(G, *)$ زمرة وليكن $n$ عدد موجب فأن 
	$a^n =\underbrace{a*a*\cdots*a}_{\text{$n$ من المرات}}$ 
\end{definition}

\begin{theorem}
	لتكن $(G, *)$ زمرة وليكن $n, m\in \Z$ فإن
	\setLR
	\begin{enumerate}[label=$\boxed{\arabic*}$]
		\item $e^n =e$
		\item $a^0 =e, \forall a\in G$
		\item $(a^m)^n = a^{mn}$
		\item $a^m * a^n = a^{m+n}$
		\item $a^{-m} = (a^{-1})^m$
	\end{enumerate}
\end{theorem}

\begin{definition}[( الزمرة الدوارة )]
	$(G, *)$ تسمى زمرة دوارة اذا وجد عنصر فيها $a\in G$ بحيث ان اي عنصر اخر في الزمرة $b\in G$ يمكن كتابته بالصيغة $b =a^k , k \in \Z$ ، يطلق على العنصر $a$ بالعنصر المولد للزمرة. او يقال ان الزمرة $G$ مولدة بواسطة العنصر $a$ ونكتب $G= \langle a\rangle$ او $G=(a)$
 \end{definition}

\begin{example}
	لتكن $G=\{1,-1,i,-i\}$ حيث ان $i = \sqrt{-1}$ ، فإن $(G, \cdot)$ تمثل زمرة دوارة.
\end{example}

\begin{theorem}
	الزمرة $(G, *)$ تكون ابدالية اذا وفقط اذا كان $(a*b)^{-1} = a^{-1} * b^{-1}$.
\end{theorem}

\section{زمرة التناظر (التباديل)}

\begin{definition}[( زمرة التناظر )]
	لتكن $X$ مجموعة غير خالية ، الدالة $f:X\to X$ تسمى تباديل على $X$ اذا كانت $f$ تقابل على $X$ ، مجموعة كل التباديل على $X$ يرمز لها بالرمز $\sym{X}$ حيث ان 
	\[
	\sym{X} = \{ f\mid f:X\to X \,\,\text{\LR{bijective}}\}
	\]
\end{definition}

\begin{example}
	لتكن $X=\{1,2,3\}$ و $f \in S_3$ معرفة كالآتي
	\[
	f(1)=1, f(2)=3, f(3)=2
	\]
	نستطيع كتابة عناصر الدالة $f$ بالشكل التالي
	\[
	f = \begin{pmatrix}
		1&2&3\\
		1&3&2
	\end{pmatrix}
	\]
	ترتيب الاعمدة هنا ليس مهماً فنستطيع كتابة الدالة كما يأتي
	\[
	\begin{pmatrix}
		2&1&3\\
		3&1&2
	\end{pmatrix} \quad \text{او} \quad 
	\begin{pmatrix}
		2&3&1\\
		3&2&1
	\end{pmatrix}
	\]
\end{example}

\begin{note}
	طريقة تعريف دالة التركيب على عناصر $f, g\in S_n$ 
	\[
	f = \begin{pmatrix}
		1&2&\cdots&n\\
		f(1)&f(2)&\cdots&f(n)
	\end{pmatrix}, \quad 
	g = \begin{pmatrix}
	1&2&\cdots&n\\
	g(1)&g(2)&\cdots&g(n)
\end{pmatrix}, \quad 
	\]
	فأن دالة التركيب تكتب بالشكل
	\[
		f\circ g = \begin{pmatrix}
		1&2&\cdots&n\\
		f(g(1))&f(g(2))&\cdots&f(g(2))
	\end{pmatrix}, \quad 
	\]
	وتكتب الدالة المحايدة بالشكل
	\[
		(1)= \begin{pmatrix}
		1&2&\cdots&n\\
		1&2&\cdots&n
	\end{pmatrix}, \quad 
	\]
	وتكتب دالة المعكوس بالشكل
	\[
	f^{-1}=
		 \begin{pmatrix}
		f(1)&f(2)&\cdots&f(n)\\
				1&2&\cdots&n
	\end{pmatrix}, \quad 
	\]
\end{note}

\begin{definition}
	لتكن $f\in S_n$ بحيث ان $x_1, x_2, \dots, x_n$ فإذا كان $f(x_i) = x_{i+1}$ لكل $1\leq i\leq n-1$ وأن $f(x_n)=x_1$ اذن نستطيع كتابة $f$ بشكل دورة $\cycle{x_1, x_2, \cdots, x_n}$ وتسمى دورة ذات طول  $n$ 
\end{definition}

\begin{example}
	لنفرض ان $f, g\in S_5$ حيث ان
	\[
	f = \begin{pmatrix}
		1&2&3&4&5\\
		2&3&4&1&5
	\end{pmatrix}, \quad
		g = \begin{pmatrix}
		1&2&3&4&5\\
		1&3&2&4&5
	\end{pmatrix}, 
	\]
	اذن $f=\cycle{1,2,3,4}$ وهذا يعني انها دورة ذات طول 4 ، اما $g=\cycle{2,3}$ وهذا يعني انها دورة ذات طول 2.
\end{example}

\begin{definition}
	اي دورة طولها 2 فقط تسمى مناقلة.
\end{definition}

\begin{lemma}
معكوس اي مناقلة هي المناقلة نفسها.
\end{lemma}
\noindent
\textbf{مثال: }$\cycle{2,3}^{-1} = \cycle{2,3}$

\begin{lemma}
	لتكن $\cycle{x_1, x_2, \cdots, x_n}$ دورة ذات طول $n$ فإن
	\[
	\cycle{x_1, x_2, \cdots, x_n}^{-1} = = \cycle{x_1, x_{n-1}, x_{n-2}, \cdots, x_2}
	\]
\end{lemma}

\begin{example}
	معكوس الدورة $\cycle{4,5,6,7}$ الدورة $\cycle{4, 7,6,5}$.
\end{example}

\begin{note}
	نكتب العنصر المحايد في الزمرة $S_n$ بشكل دورة ذات طول 1، ويرمز له  بالرمز $(1)$
\end{note}

\begin{theorem}
	كل دورة $\cycle{x_1, x_2, \cdots, x_n}$ ممكن كتابتها على صورة حاصل ضرب مناقلات وهذا التعبير غير وحيد
	\[
	\cycle{x_1, x_2, \cdots, x_n} = \cycle{x_1, x_n}\cycle{x_2, x_n}\cdots\cycle{x_{n-1}, x_n}
	\]
\end{theorem}
\begin{example}
	في الزمرة $S_8$ 
	\[
	\cycle{3,4,5,6,7,8} = \cycle{3, 8} \cycle{3, 7} \cycle{3, 6} \cycle{3, 5} \cycle{3, 4}
	\]
	\[
	\cycle{1,3,2}\cycle{5,8,6,7} =  \cycle{1,2 } \cycle{1,3} \cycle{5,7 }  \cycle{5,6 }\cycle{5,8}
	\]
\end{example}

\begin{corollary}
	كل تبديل نستطيع كتابته على شكل حاصل ضرب مناقلات
\end{corollary}

\begin{definition}
	التبديل $f$ يسمى تبديل زوجي (فردي) اذا كان يكتب على شكل حاصل ضرب عدد زوجي (فردي) من المناقلات
\end{definition}

\begin{example}
	$\cycle{1,2}$ تبديل فردي.\\
	$\cycle{1,2,3} = \cycle{1,3}\cycle{1,2}$ تبديل زوجي.
\end{example}

\begin{lemma}
	الدورة (التباديل) ذات الطول $n$ تكون تبديل فردي اذا كان الطول زوجي والعكس بالعكس.
\end{lemma}

\begin{example}
	$\cycle{1,2}$ دورة ذات طول 2 اذن تبديل فردي\\
	$\cycle{1,2,3}$ دورة ذات طول 3 اذن تبديل زوجي.
\end{example}

\begin{theorem}
	عند ضرب تبديلين زوجيين او فرديين فالناتج يكون تبديل زوجي ، اما عند ضرب تبديل فردي بتديل زوجي او العكس فالناتج تبديل فردي.
\end{theorem}

\begin{example}
	حاصل الضرب $\cycle{1,2,3} \cycle{5,4}\cycle{7,8,9}$ ، التبديل الاول والثالث زوجيان اما التبديل الثاني فردي ، اذن الناتج يكون تبديل زوجي.
\end{example}

\begin{lemma}
	$(S_n, \circ)$ ليست زمرة ابدالية لكل $n\geq 3$.
\end{lemma}
\noindent
\textbf{البرهان}\\
\noindent
لنأخذ $\cycle{1,2}, \cycle{2,3} \in S_n$ نلاحظ ان ، 
$\cycle{1,2}\cycle{2,3} = \cycle{1,2,3}$ بينما
$\cycle{2,3}\cycle{1,2} = \cycle{1,3,2}$
لذلك فإن 
$\cycle{1,2}\cycle{2,3} \neq\cycle{2,3}\cycle{1,2}$
بالتالي فإن $(S_n , \circ)$ ليست زمرة ابدالية.\qed

\begin{definition}
	مجموعة كل التبديلات الزوجية في الزمرة $S_n$ مع عملية التركيب $\circ$ تسمى الزمرة المتناوبة ويرمز لها بالرمز $(A_n, \circ)$ ورتبتها $O(A_n) = \dfrac{n!}{2}$.
\end{definition}

\section{زمرة الاعداد الصحيحة مقياس \textit{n}}
\begin{definition}
	ليكن $n\in \Z_+$ نعرف العلاقة $\equiv_n$ (او قياس $n$) على $\Z$ كما يلي: $a\equiv_n b$ اذا وفقط اذا $a-b$ يقبل القسمة على $n$ او $a-b = kn$ او $a=b+kn$
\end{definition}
\noindent\textbf{مثال:} $3\equiv1\Mod 2$ او $3\equiv_2 1$

\begin{theorem}
	علاقة القياس $n$ ($\equiv_n$) لمجموعة الاعداد الصحيحة هي علاقة تكافؤ.
\end{theorem}

\begin{note}
	بما ان العلاقة $\equiv_n$ هي علاقة تكافؤ على الاعداد الصحيحة (عكسية ، متناظرة ، متعدية) اذن هي تشكل تجزئة على $\Z$ وتوجد صفوف تكافؤ او صفوف التكافؤ، حيث نستخدم اقل عدد موجب لتمثيل صف التكافؤ بشكل عام، اذا كان $a\in \Z$ فإن
	\begin{align*}
			[a] &= \{x\in \Z \mid a=x\Mod n\}\\
			&= \{x\mid a-x = kn, k\in \Z\}\\
			&= \{x\mid a = x + kn, k\in \Z \}
	\end{align*}
\end{note}

\begin{theorem}
	الزوج المرتب  $(\Z_n, +_n)$ يشكل زمرة ابدالية التي يطلق عليها زمرة الاعداد الصحيحة مقياس $n$
\end{theorem}
\noindent
\textbf{البرهان}\\
\noindent
$\boxed{1}$ لنبرهن ان $\Z_n$ مغلقة تحت العملية $+_n$ ، لنفرض ان $[a], [b] \in \Z_n$ نجد ان
\[
[a] +_n [b] = [a + b] \in \Z_n
\]
$\boxed{2}$ لبرهان ان $+_n$ تجميعية على $\Z_n$ لنفرض ان $[a], [b], [c] \in \Z_n$ نجد
\begin{align*}
	[a] +_n \Big([b] +_n [c]\Big)&= [a] +_n [b+c]\\
	&= [a + b+ c] \\
	&= [a + b] +_n [c\\
	&= \Big( [a] +_n [b]\Big) +_n [c]
\end{align*}
$\boxed{3}$ $[0] \in \Z_n$ هو  العنصر المحايد لــ $\Z_n $ لأن لكل $[a] \in \Z_n $
\[
[a] +_n [0] = [a+0] = [a]
\]
$\boxed{4}$ ليكن $[a]\in \Z$ فأن $[a]^{-1} = [n-a] $ لأن 
\[
[a] +_n [n-a] = [a+n-a] = [n] = [0]
\]
$\boxed{5}$ لبرهان خاصية الابدال لنفرض ان $[a], [b] \in \Z_n$ فان 
\[
[a] +_n [b] = [a+b] = [b+a] = [b] +_n [a]
\]

\begin{note}
	تكتب عناصر $\Z_n$ بالشكل $a$ بدل من $[a]$ و $-a$ بدل من $[n-a]$
\end{note}

\begin{theorem}
	الزمرة $(\Z_n, +_n)$ تشكل زمرة دوارة مولدة بواسطة العنصر 1.
\end{theorem}

\begin{note}
	اي عنصر في $\Z_n$ ممكن ان يكون مولد للزمرة اذا وفقط اذا $\gcd(a,n)=1$
\end{note}
\begin{note}
 تعرف عملية الضرب على المجموعة $\Z_n$ كالآتي 
\[
[a] \cdot_n [b] = [a\cdot b] , \forall [a],[b] \in \Z_n
\]
\end{note}

\begin{example}
	$5\cdot_6 4 =2,\quad 7\cdot_9 2=5,\quad 3\cdot_4 2=2$
\end{example}

\begin{note}
	الزوج المرتب $(\Z_n - \{0\}, \cdot_n)$ ربما يكون زمرة اذا كان $n$ عدد اولي. للتوضيح اكثر $(\Z_4-\{0\}, \cdot_4)$ لا تمثل زمرة لان 2 لا يملك معكوس ضربي بينما $(\Z_5-\{0\},\cdot_5)$ تمثل زمرة.
\end{note}

\begin{note}
	العنصر $a$ في $\Z_n$ يمتلك معكوس ضربي اذا وفقط اذا كان $\gcd(a,n)=1$
\end{note}

\section{الزمرة الجزئية}

\begin{definition}
	لتكن $(G, *)$ زمرة و $H$ مجموعة غير خالية جزئية من $G$ فإن $(H, *)$ تسمى زمرة جزئية من الزمرة $(G, *)$ اذا كانت $H$ هي زمرة كذلك ونكتب $(H, *)\leq (G, *)$
\end{definition}
\begin{example}
	كل زمرة على الاقل لها زمرتان جزئيتان هما $(G, *) $ و $(\{e\}, *)$.
\end{example}

\begin{definition}
	الزمرة الجزئية $(H, *)$ تسمى زمرة جزئية فعلية من $(G, *)$ اذا كانت $H$ هي مجموعة جزئية فعلية $H \subset G$.
\end{definition}

\begin{definition}
		الزمرة الجزئية $(H, *)$ تسمى زمرة جزئية غير تافهة اذا كانت $\varnothing\neq H\neq G$.  
\end{definition}

\begin{example}
	$(\R, +). (\Z, +). (\Q, +), (\R-\{0\}, \cdot), (\Q-\{0\}, \cdot)$ جميع هذه الانظمة هي زمر جزئية غير تافهة من زمرة الاعداد المركبة $(\C, +), (\C-\{0\}, \cdot)$ على التوالي
\end{example}

\begin{theorem}
	لتكن $(G, *)$ زمرة و $\varnothing\neq H\subseteq G$ اذن $(H, *)$ تكون زمرة جزئية من $(G, *)$ اذا وفقط اذا حققت الشرط 
	\[
	a * b^{-1} \in H, \forall a, b\in H
	\]
\end{theorem}

\begin{example}
	$(\Z_e, +) \leq (\Z, +)$
\end{example}
\noindent
\textbf{الحل}\\
\noindent
لفترض ان $a, b\in \Z_e$ اذن يوجد $n,m\in Z$ بحيث ان $a=2n, b=2m$ اذن
\[
a * b^{-1} = a- b=2n-2m=2\underbrace{(n-m)}_{\in \Z} \in \Z_e
\]
اذن 	$(\Z_e, +) \leq (\Z, +)$

\begin{example}
	وضح ان $(A_3, \circ) \leq (S_3, \circ)$
\end{example}
\noindent
\textbf{الحل}
\begin{table}[H]
	\centering
	\renewcommand{\arraystretch}{1.3}
	\setLR
	\begin{tabular}{c|c|c|c}
		$\circ$ & $\cycle{1}$ & $\cycle{1,2,3}$ & $\cycle{1,3,2}$\\
		\hline
		$\cycle{1}$ & $\cycle{1}$ & $\cycle{1,2,3}$ & $\cycle{1,3,2}$\\
		$\cycle{1,2,3}$& $\cycle{1,2,3}$ & $\cycle{1,3,2}$&$\cycle{1}$\\
		$\cycle{1,3,2}$& $\cycle{1,3,2}$&$\cycle{1}$ & $\cycle{1,2,3}$ \\
		\hline
	\end{tabular}
\end{table}
\setRL
من الجدول اعلاه نلاحظ ان
 $a\circ b^{-1} \in A_3$
 لأي  $a, b\in A_3$ اذن  $(A_3, \circ) \leq (S_3, \circ)$
 
 \begin{example}
 	وضح ان  $(A_n, \circ) \leq (S_n, \circ)$
 \end{example}
 \noindent
 \textbf{الحل}\\
 \noindent
  نلاحظ ان $A_n \subseteq S_n $ و $(1) \in A_n$ اذن $A_n \neq \varnothing$ ، لنفترض ان $f, g \in A_n$ وهذا يعني كل من $f,g$ تبديلات زوجية ، نبين ان $f^{-1}$ تبديل زوجي ، نلاحظ ان $f\circ f^{-1}=(1)$ وهذا يعني $f^{-1}$ تبديل زوجي وبالتالي 
  $g\circ f^{-1}$ تبديل زوجي كذلك اي ان $g\circ f^{-1} \in A_n$ اذن $A_n \leq S_n$.\qed
  
   \begin{note}
   	كل زمرة جزئية من زمرة ابدالية تكون ابدالية.
   \end{note}
   
   \begin{definition}
   	لتكن $(G, *)$ زمرة ، مجموعة كل العناصر في $G$ التي تتبادل مع جميع عناصر $G$ تسمى مركز الزمرة $G$ ويرمز لها بالرمز $\cent{G}$.
   	\[
   	\cent G = \{c \in G : x*c=c*x , \forall x\in G\}
   	\]
   \end{definition}
   
   \begin{example}
   	$\cent\Z_n=\Z_n, \quad \cent S_n = (1)$.
   \end{example}
   
   \begin{note}
   	$(\cent G, *) \leq (G, *)$.
   \end{note}
   
   \begin{lemma}
   	الزمرة $(G, *)$ تكون ابدالية اذا وفقط اذا $\cent G = G$.
   \end{lemma}
   
   \begin{theorem}
   	لتكن كل من $(K, *)$ و $(H, *)$ زمرة جزئية من $(G, *)$ فإن $(K \cap H, *) \leq (G, *)$ بمعنى اخر تقاطع اي زمرتين جزئيتين هو زمرة جزئية.
   \end{theorem}
\noindent
\textbf{البرهان}\\
\noindent
لتكن $a, b\in K\cap H$ اذن $a, b\in H$ و $a, b\in K$ ولأن كل منهما زمرة جزئية ، فإن 
$a*b^{-1} \in H$ و $a*b^{-1} \in G$
لذلك فإن $a*b^{-1} \in K\cap H$ وبالتالي  $(K \cap H, *) \leq (G, *)$.\qed
   \newpage
   \begin{note}
   	اذا كان كل من $(K, *)$ و $(H, *)$ زمرة جزئية من $(G, *)$  فليس من الضروري ان يكــــــــــــــون \\
   	$(K \cup H, *) \leq (G, *)$  ، بمعنى آخر اتحاد اي زمرتين جزئيتين لا يعطي بالضرورة زمرة جزئية.
   \end{note}
   
   \begin{theorem}
   	 	لتكن كل من $(K, *)$ و $(H, *)$ زمرة جزئية من $(G, *)$ فإن $(K \cup H, *) \leq (G, *)$ اذا وفقط اذا كان اما $K\subseteq H$ أو $H\subseteq K$.
   \end{theorem}
   
   \begin{definition}
   	لتكن $(G, *)$ زمرة و $\varnothing\neq S\subseteq G$ ولتكن
   	$(S) = \{H : S\subseteq H , (H, *) \leq (G, *)\}$
   	تسمى $((S), *)$ زمرة جزئية مولدة بواسطة المجموعة $S$.
   \end{definition}
   
   \begin{note}
   	الزمرة الجزئية $((S), *)$ هي اصغر زمرة تحتوي على المجموعة $S$.
   \end{note}
   
\begin{definition}
	لتكن $(G, *)$ زمرة وليكن $a\in G$ فإن الزمرة الجزئية 	$\big((\{a\}) , *\big)$ وتكتب بالصيغة $((a), *) $ وهي الزمرة الجزئية المولدة بواسطة العنصر $a$.
\end{definition}

\noindent
\textbf{ملاحظة}
\begin{enumerate}
	\item لتكن $(G, *)$ زمرة ، اذا كان $a\in G$ يمتلك رتبة منتهية فإن $O(a) = O\Big((a)\Big)$
	\item  لتكن $(G, *)$ زمرة ، اذا كان $a\in G$ فإن $(a) = \{a^n : n\in \Z\}$.
\end{enumerate}

\begin{example}
	جد $(3)$ في $(\Z, +)$
\end{example}
   \noindent
   \textbf{الحل}\\
   \noindent
   $(3) = \{3^n : n\in \Z\} = \{3n : n\in \Z\}$.
   
   \begin{example}
   	اوجد $(2)$ فــــي $(\Z_8, +_8)$.
   \end{example}
   \noindent
   \textbf{الحل}\\
   \noindent
   $(2) = \{2^n : n\in \Z\} = \{2n : n\in \Z\} = \{0,2,4,6\}$.
   
\begin{definition}
	لتكن كل من $(K, *)$ و $(H, *)$ زمرة جزئية من $(G, *)$ فإن حاصل ضرب الزمرتان الجزئيتيان يعرف بالشكل
	$
	H * K = \{h * k : h \in H, k \in K\}
	$.
\end{definition}

\begin{example}
	ليكن $H=(2)$ و $K=(3)$ زمرتان جزئيتيان من الزمرة $(\Z_{12}, +_{12})$ اوجد $H_{12}K$
\end{example}
\noindent
\textbf{الحل}\\
\noindent
$H=\{0,2,4,6,8,10\}$ ، $K=\{0,3,6,9\}$ اذن\\
 $H+_{12} K=\{0,1,2,3,4,5,6,7,8,9,10,11\} = \Z_{12}$
 
 \begin{note}
 		لتكن كل من $(K, *)$ و $(H, *)$ زمرة جزئية من $(G, *)$ فإن حاصل ضرب الزمرتان الجزئيتيان $H*K$ ربما لا يكون زمرة جزئية من $(G,*)$.
 \end{note}
 
 \begin{theorem}
 		لتكن كل من $(K, *)$ و $(H, *)$ زمرة جزئية من $(G, *)$ فإن حاصل ضرب الزمرتان الجزئيتيان $(H*K, *)$ يكون زمرة اذا كان $H*K=K*H$.
 \end{theorem}
 
 \begin{note}
 		لتكن كل من $(K, *)$ و $(H, *)$ زمرة جزئية من $(G, *)$ فإن $(H*K, *) = (H\cup K, *)$.
 \end{note}
 
 \begin{example}
 	لتكن $H=\{3\}$  و $K=\{4\}$ زمرتان جزئيتيان من الزمرة $(\Z_{12}, +_{12})$ اوجد 
 	$(H\cup K, +_{12})$
 \end{example}
 \noindent
 \textbf{الحل}\\
 \noindent
 $H = \{0,3,6,9\}$ ، $K=\{0,4,8\} $ اذن \\
 $H \cup K = H+_{12} K = \{0,4,8,3,7,11,6,10,2,9,1,5\} = \Z_{12}$

 \begin{theorem}
 	لتكن $(G, *)$ زمرة ابدالية و	لتكن كل من $(K, *)$ و $(H, *)$ زمرة جزئية من $(G, *)$ فإن\\
 	 $(H*K, *) \leq (G, *)$.
 \end{theorem}
 
 \begin{theorem}
 	لتكن $((a), *)$ زمرة دائرية تمتلك رتبة منتهية $n$ فإن 
 	$(a) = \{a^0 = e, a, a^2, \dots, a^{n-1}\}$.
 \end{theorem}
 
 
 \begin{example}
 	اوجد $\big(\cycle{1,2,3}\big)$ فـــــي $(S_3, \circ)$.
 \end{example}
 \noindent
 \textbf{الحل}\\
 \noindent
 $O\big(\cycle{1,2,3}\big) = 3$ $\Leftarrow$ $\big(\cycle{1,2,3}\big) = \{(1), \cycle{1,2,3} , \cycle{1,3,2}\}$
 
 \begin{definition}
 	لتكن $(H, *)$ زمرة جزئية من الزمرة $(G, *)$ وأن $a\in G$ ،تسمى  المجموعة المعرفة بالشكل التالي 
 	$a* H = \{a * h : h\in G\}$
 	بالمجموعة المشاركة (المصاحبة)  اليسارية لــ $H$ في $G$  ، وتسمى $H*a = \{h*a:h\in G\}$ بالمجموعة المشاركة (المصاحبة) اليمينية.  
 \end{definition}
 
 \begin{example}
 	ليكن $H=\{0,2,4\}$ اوجد
 	 $1+_{6} H, 2+_{6} H, 3 +_{6} H, 4+_6 H, 5+_6 H$
 \end{example}
 \noindent
 \textbf{الحل}
 \begin{gather*}
 	1 +_{6} H =\{1 +_6 0, 1 +_6 2, 1 +_6 4\} = \{1,3,5\}\\
 	2 +_{6} H = \{2 +_6 0, 2 +_6 2, 2 +_6 4\} = \{2,4,0\}\\
 	3 +_{6} H = \{3 +_6 0, 3 +_6 2, 3 +_6 4\} =\{3,5,1\}\\
  	4 +_{6} H =\{4+_6 0, 4+_6 2, 4 +_6 4\} = \{4,0,2\}\\
   	5 +_{6} H = \{5 +_6 0, 5 +_6 2, 5 +_6 4\} =\{5,1,3\}
 \end{gather*}
 
 \begin{theorem}
 	لتكن $(H, *)$ زمرة جزئية من الزمرة $(G, *)$ فإن
 	\begin{enumerate}
 		\item $a*H=H \iff a\in H$.
 		\item $H*a = H \iff a\in H$
 	\end{enumerate}
 \end{theorem}
 \noindent
 \textbf{البرهان}\\
 \noindent
 ليكن $a*H=H$ اذن $a\in G$ و $e\in H$ نحصل على $a = a*e \in a*H =H$ اذن $a\in H$.\\
 لنفترض $a\in H$ وليكن $x \in a*H$ يوجد $h\in H$ بحيث $x=a*h$ اذن $a*H\subseteq H$ ، الآن نفترض $y\in H$ فإن 
 $y =a*(a^{-1}*y)  \in a*H $ 
 اذن $H\subseteq a*H$ وبالتالي $a*H = H$ \qed
 
 \begin{definition}
 	لتكن $(H, *)$ زمرة جزئية من الزمرة $(G, *)$ ، يسمى عدد كل من المجموعات المشتركة اليمنى او  اليسرى بدليل الزمرة الجزئية $H$ في $G$ ويرمز له بالرمز $[G:H]$.
 \end{definition}
 \begin{example}
 	$[A_n : S_n] = 2$.
 \end{example}
 
 \begin{theorem}[( مبرهنة لاكرانج)]
 	لتكن $(G, *)$ زمرة منتهية ، فإن كل من رتبة ودليل اي زمرة جزئية $H$ منها تقسم رتبتها $O(G)$. 
 \end{theorem}
 \noindent
 \textbf{البرهان}\\
 \noindent
 بما ان $(G, *)$ زمرة منتهية و $(H, *)$ زمرة جزئية منها ، اذن مجموعة كل المجموعات المشاركة تشكل تجزئة للزمرة $G$ وكذلك يوجد عدد منتهي من المجموعات المشاركة المختلفة كالآتي\\
 $H, a_1*H, a_2*H, \dots, a_k*H$ $\Leftarrow$
 $
 G= H\cup a_1 * H \cup a_2*H \cup \cdots \cup a_k*H
 $
 اذن 
 $O(G) = O(H) + O(a_1*H) + O(a_2*H) + \cdots + O(a_k*H)$
ولكن يوجد تقابل بين اي اثنين من المجموعات المشاركة المختلفة لذلك 
$O(G) = \underbrace{O(H)+O(H)+\cdots+O(h)}_{\text{$k$ من المرات}} $ 
 ومنه نحصل على $O(G) = k\cdot O(H) $ بالتالي $O(G) = [G:H] O(H)$.\qed
 
 \begin{note}
 	عكس مبرهنة لاكرانج ربما لا يكون صحيح دائماً لأي زمرة منتهية (لأي قاسم من قواسم رتبة الزمرة المنتهية توجد هنالك زمرة جزئية رتبتها ذلك القاسم).
 \end{note}
 
 \begin{example}
 	الزمرة $A_4$ رتبتها 12 لكن لا توجد زمرة جزئية منها رتبتها 6.
 \end{example}
 
 \begin{corollary}
 	لتكن $(G, *) $ زمرة منتهية وليكن $a\in G$ فإن رتبة العنصر $O(a)$ عامل من عوامل رتبة الزمرة ، هذا يعني $a^{O(G)} =e$.
 \end{corollary}
 \noindent
 \textbf{البرهان}\\
 \noindent
 الزمرة الجزئية $\big((a), *\big)$ رتبتها تساوي رتبة العنصر $a$ اي
 $O\big((a)\big) = O(a)$
 هو عامل من عوامل $O(G)$.
 
 \begin{corollary}
 	لتكن $(G, *)$ زمرة منتهية رتبتها مركبة (قابلة للتحليل الى عوامل) فإن $(G, *)$ تمتلك زمرة جزئية غير تافهة.
 \end{corollary}
 \noindent
 \textbf{البرهان}\\
 \noindent
 اذا كانت $(G, *)$ زمرة ليست دائرية فإنه يوجد عنصر $a\in G$ بحيث يولد زمرة جزئية $\big((a),*\big)$ غير تافهة.\\
 اما اذا كانت $(G, *)$ زمرة دائرية مولدة بواسطة العنصر $a$ ، بما ان رتبة الزمرة قابلة للتحليل فأن $O(G)=mn$ حيث $m,n\in \Z$ ان كل من $m,n\neq 1 $ اذن $a^{mn} = (a^n)^m=e$ ولكن 
 $(a^n)^{m_1} = e$ حيث ان $ 0 < m_1 <m$
 ، اذن 
 $\big((a), *\big)$ رتبتها تساوي $m_1$ وبالتالي فإنها زمرة جزئية غير تافهة.\qed
 
 
 \begin{corollary}
 	كل زمرة منتهية ذات رتبة اولية تكون دائرية.
 \end{corollary}
 
 \section{الزمر الجزئية الناظمية}
 
 \begin{definition}
 	لتكن $(H, *)$ زمرة جزئية من $(G, *)$ فإن $(H, *)$ تسمى زمرة جزئية ناظمية (سوية) اذا وفقط اذا كان $a*H = H*a$ لكل $a\in G$ ونكتب $ H \trianglelefteq G$.
 \end{definition}
 
 \begin{example}
 	كل زمرة جزئية دليلها 2 تكون زمرة ناظمية.
 \end{example}
 \noindent
 \textbf{البرهان}\\
 \noindent
 لتكن $(H, *)\leq (G, *)$ بما ان $[G:H]=2$ اذن $G$ تمتلك مجموعتين مشاركتين يسرى هما $H, a*H$ وكذلك مجموعتين مشاركتين يمنى $H, H*a$ \\
 ليكن $b\in G$. اذا كان $b\in H$ نحصل على $H=H*b$ و $H=b*H$ $\Leftarrow$ $H*a=a*H$\\
 اما اذا كان $b\notin H$ فإن $b*H\neq H$ اذن $H*b = a*H$ كذلك $H*b \neq H$ ومنه $H*b = H*a=a*H$ اذن $H\trianglelefteq G$.\qed
 
 \begin{example}
 	اثبت ان 
 	$(A_n, \circ) \trianglelefteq (S_n, \circ)$ 
 \end{example}
 \noindent
 \textbf{البرهان}\\
 \noindent
 بما ان 
 $[S_n : A_n] = \dfrac{O(S_n)}{O(A_n) } = \dfrac{n!}{\mfrac{n!}{2}} = 2$ ، 
 اذن  	$(A_n, \circ) \trianglelefteq (S_n, \circ)$ \qed
 
 \begin{theorem}
 	كل زمرة جزئية من زمرة ابدالية تكون زمرة سوية.
 \end{theorem}
 \noindent
 \textbf{البرهان}\\
 \noindent
 نفترض ان $(H, *)\leq (G, *)$ ،
 لبرهان $a*H = H*a$ لكل $a\in G$ ،
 ليكن $a*h \in a*H$ ، وبما ان $(G, *)$ زمرة ابدالية ، اذن $a*h = h*a \in H*a$ ،
  اذن $a*H\subseteq H*a$ ، وبنفس الطريقة نبرهن $H*a\subseteq a*H$ وبالتالي $a*H=H*a$ ومنه 
  $(H, *) \trianglelefteq (G, *)$.\qed
  
  \begin{definition}
  	الزمرة $(G, *)$ تسمى زمرة بسيطة اذا كانت تحوي فقط زمرتين سويتين هما 
  	$(G, *) , \big(\{e\}, *\big)$.
  \end{definition}
  
  \newpage
  
  \section{زمرة القسمة}
  \begin{definition}
  	لتكن $(H, *)$ زمرة جزئية من $(G, *)$ نعرف المجموعة
  $G/H = \{a*H : a\in G\}$  او $ G/H = \{H*a : a\in G\}$ ، بأنها مجموعة القسمة لــ $G$ على $H$ وتمثل مجموعة كل المجموعات المصاحبة اليمنى او اليسرى للزمرة الجزئية $H$ في الزمرة $G$.
  \end{definition}
  
  \begin{definition}
  	لتكن $(H, *)$ زمرة جزئية من $(G , *)$ ، نعرف العملية الثنائية $\circledast$ على $G/H$ بالشكل التالي\\
  	$
  	(a*H)\circledast (b*H) = (a*b) *H, \forall a, b\in G
  	$
  	ويسمى الزوج المرتب $(G/H, \circledast)$ بزمرة القسمة.
  \end{definition}
  
  \begin{theorem}
  	لتكن $(G, *)$ زمرة و $(H, *)$ زمرة جزئية سوية منها فإن الثنائي $(G/H, \circledast)$ يشكل زمرة.
  \end{theorem}
  
  \begin{theorem}
  	لتكن $(G, *)$ زمرة ابدالية ، فإن $(G/H, \circledast)$ زمرة ابدالية لأي زمرة جزئية سوية $(H, *)$.
  \end{theorem}
  
  \begin{theorem}
  	لتكن $(G, *)$ زمرة دائرية ، فإن $(G/H, \circledast)$ زمرة دائرية لأي زمرة جزئية سوية $(H, *)$.
  \end{theorem}
  
  
\newpage
 
 \section{تعريف الحلقة}
 
 \begin{definition}
 	الحلقة هي ثلاثي مرتب $(R, +, \cdot)$ مكون من مجموعة غير خالية $R$ وعمليتي الجمع والضرب بحيث 
 	\begin{enumerate}[label=$\boxed{\arabic*}$]
 		\item $(R, +)$ زمرة ابدالية.
 		\item  $(R, \cdot) $ شبه زمرة.
 		\item العملية $\cdot$ تتوزع على العملية + ، أي أن:
 		\begin{gather*}
 			a\cdot (b+c) = a\cdot b + a\cdot c \tag{التوزيع من اليسار}\\
 			(b + c) \cdot a = b\cdot a + c\cdot a \tag{التوزيع من اليمين}
 		\end{gather*}
 		لكل $a ,b,c\in R$
 	\end{enumerate}
 \end{definition}
 
 \begin{example}
 	الانظمة التالية تمثل حلقات
 	$
 	(\Z, + ,\cdot), (\Q, + ,\cdot),(\R, + ,\cdot),(\C, + ,\cdot)
 	$\\
 	بينما الانظمة 
 	$
 	(\N, +, \cdot), (\R, \cdot, +)
 	$
 	لا تمثل حلقات
 \end{example}
 
 \begin{definition}
 	يقال ان حلقة $(R, +, \cdot)$ تحتوي على قواسم الصفر ، اذا كان هناك عنصرين $a, b\in R$ بحيث $a\neq 0$ ، $b\neq 0$ مع ذلك فأن $a\cdot b=0$ ، يطلق على العناصر $a,b$ قواسم الصفر
 \end{definition}
 
 \begin{example}
 	الحلقة
 	$(\Z_6, +_6, \cdot_6)$ تمتلك قواسم للصفر ، لأن $2\cdot_6 3=0 $ بينما الحلقة 
 	$(\Z_7, +_7, \cdot_7)$ لا تمتلك قواسم للصفر.
 \end{example}
 
 \begin{theorem}
 	لتكن $(R, +, \cdot)$ حلقة بحيث $R\neq \{0\}$ ، عندئذٍ تكون العناصر 0 و 1 مختلفة ($0\neq 1$).
 \end{theorem}
 
 \begin{theorem}
 	لتكن $(R, +, \cdot)$ حلقة و $a,b\in R$ فإن 
 	$-(a\cdot b) = a\cdot (-b) = (-a)\cdot b$.
 \end{theorem}
 
 \begin{corollary}
 	لكل $a,b\in R$ فإن
 	$a\cdot (b-c) = a\cdot b - a\cdot c$\,\,  و \,\,$(b-c) \cdot a = b\cdot a - c\cdot a$\\
 	اي ان عملية الضرب تتوزع على الطرح.
 \end{corollary}
 
 \begin{theorem}
 	الحلقة $(R, +, \cdot)$ لا تحتوي على قواسم صفرية اذا وفقط اذا كان قانون الاختصار ينطبق على عملية الضرب.
 \end{theorem}
 
 \begin{corollary}
 	لتكن $(R, +, \cdot)$ حلقة لا تحتوي على قواسم صفرية فإن الحلول الوحيدة للمعادلة $a^2=a$ هي  $a=0, a=1$ 
 \end{corollary}
 \noindent
 \textbf{البرهان}\\
 \noindent
 $a^2 =a\Rightarrow a^2-a=0\Rightarrow a\cdot(a-1)=0 $ ، وبما ان $R$ ليس لها قواسم صفرية فإن $a=0$ او $a-1=0$ وبالتالي $a=0 $ أو $a=1$.\qed
 
 \begin{definition}[( الساحة التامة )]
 	لتكن $(R, +, \cdot)$ حلقة مع عنصر محايد لعملية الضرب ، نقول ان $(R, +, \cdot)$ ساحة تامة اذا لم تحوي على قواسم الصفر وكانت ابدالية بالنسبة لعملية الضرب.
 \end{definition}
 
 \begin{example}
 	الحلقة $(\Z, +, \cdot)$ هي ساحة تامة لتحقيقها الشروط ، ولكن الحلقة 
 	$\Z_6, +_6, \cdot_6$ ليست ساحة تامة لأحتوائها على قواسم الصفر.
 \end{example}
 \newpage
 \section{الحلقة الجزئية}
 
 \begin{definition}
 	لتكن $(R, +, \cdot)$ حلقة و أن $\varnothing\neq S\subseteq R$ ، اذا كانت $(S, +, \cdot)$ حلقة بحد ذاتها نقول بأنها حلقة جزئية من $(R, +, \cdot)$ و اختصاراً نقول $S$ حلقة جزئية من $R$.
 \end{definition}
 
 \begin{note}
 	نقول ان $S$ حلقة جزئية من $R$ اذا تحقق الآتي
 	\begin{enumerate}[label=$\boxed{\arabic*}$]
 		\item $S\neq \varnothing$.
 		\item $\forall a,b\in S \Rightarrow a-b\in S$.
 		\item $\forall a,b\in S\Rightarrow a\cdot b\in S$.
 	\end{enumerate}
 \end{note}
 
 \begin{example}
 	لتكن $S = \{a+b\sqrt{3} : a, b\in \R \}$ فإن $S$ حلقة جزئية من $\R$.
 \end{example}
 
 \begin{definition}
 	لنفرض ان $R$ حلقة ، اذا كان هناك عدد صحيح موجب $n$ بحيث يكون $na=0$ لكل $a\in R$ ، فإن اقل عدد صحيح موجب يحقق هذه الخاصية يسمى المميز للحلقة $R$ و نكتب $\Char R = n$ ، اذا لم يوجد عدد صحيح موجب يحقق هذه الخاصية لجميع عناصر $R$ ، فإننا نقول $R$ ليست لها مميز او يكون المميز للحلقة يساوي 0 ($\Char R=0$)
 \end{definition}
 
 \begin{example}
 	الحلقة 
 	$(\Z_4, +_4, \cdot_4) $ تمتلك المميز 4. اي ان $\Char \Z_4=4$
 \end{example}
 
 \begin{theorem}
 	لتكن $R$ حلقة ذات محايد ، فأن $\Char R=n >0$ اذا وفقط اذا كان $n$ هو اقل عدد صحيح موجب بحيث $n1=0$.
 \end{theorem}
 \newpage
 
 \section{المثاليات}
 \begin{definition}
 	لتكن $R$ حلقة و $I$ مجموعة جزئية من $R$ ، نقول ان $I$ هي مثالية في $R$ اذا تحققت الشروط
 	\begin{enumerate}[label=$\boxed{\arabic*}$]
 		\item $a-b\in I, \forall a, b\in I$.
 		\item $r\cdot a \in I$ و $a\cdot r\in I$ لكل $r\in R, a\in I$.
 	\end{enumerate}
 \end{definition}
 
 \begin{example}
 	المجموعة 
 	$I = \{3k : k\in \Z\}$ تمثل مثالية في الحلقة $\Z$.
 \end{example}
 \noindent
 \textbf{الحل}
 
 \setLR
 \begin{enumerate}[leftmargin=*, label=$\boxed{\arabic*}$]
 	\item $\forall n,m\in I \Rightarrow n=3r, m=3s, \exists r,s\in \Z \Rightarrow n-m =3\underbrace{(r-s)}_{\in \Z} \in I$.
 	\item $\forall n\in I, \forall r\in \Z \Rightarrow rn = r(3m) = 3\underbrace{(rm)}_{\in \Z} \in I$. 
 \end{enumerate}
 
 \setRL
 \section{بعض الانواع الخاصة للمثاليات}
 
 \begin{definition}[( المثالية الاعظمية )]
 	لتكن $R$ حلقة و $I$ مثالية فيها ، تسمى $I$ مثالية اعظمية في $R$ ، اذا كانت $I\neq R$ و عندما توجد مثالية $J$ بحيث 
 	$I \subseteq J \subseteq R$ فإن $J=R$.
 \end{definition}
 
 \begin{note}
 	لتكن $R$ حلقة و ان $I$ مثالية في $R$ بحيث  $I\neq R$ و $a\in R - I$ فإن
 	\begin{enumerate}
 		\item $\subset (I, a) \subseteq R $.
 		\item  اذا كانت $I$ مثالية اعظمية فإن $(I, a) = R$.
 	\end{enumerate}
 \end{note}
 
 \begin{theorem}
 	في الحلقة $\Z$ و حيث $n>1$ ، فأن $(n)$ مثالية اعظمية اذا وفقط اذا كان $n$ عدد أولي.
 \end{theorem}
 
 \noindent
 \textbf{البرهان}\\
 \noindent
 ($\Leftarrow$) نفرض $(n)$ مثالية اعظمية في $\Z$ ونفرض ان $n$ ليس عدد اولي ، اي يمكن كتابته بالشكل $n=a\cdot b$ لبعض $a, b\in \Z$ ، من الواضح ان $(n) \subset (a)$ لأن $n=a\cdot b$ و لكن $(a)\neq \Z$ وبالتالي حصلنا على $(n) \subset (a) \subset \Z$ وهذا تناقض مع كون $(n)$ مثالية اعظمية.\\
 ($\Rightarrow$) اذا كان $n$ عدداً اولياً ، نفترض وجود مثالية $I$ في $\Z$ بحيث $(n )\subset I \subset \Z$ ، لنأخذ $a\in I$ حيث $a\notin (n)$ ، بما أن $n$  عدد أولي فأن $\gcd(a, n) = 1$ ، وبالتالي يوجد $x, y\in \Z$ بحيث $ax + ny=1$ وبما ان $I$ مثالية فأن $ax, ny\in I$ وبالتالي $1\in I$ اذن $I = \Z$ وبالتالي $(n)$ مثالية اعظمية.\qed  
 
 