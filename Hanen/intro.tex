\chapter*{مقدمة}
\addcontentsline{toc}{chapter*}{مقدمة}
تُعد نظرية الزمر والحلقات من المواضيع الأساسية في علم الجبر المجرد، وتهدف إلى دراسة البنى الجبرية التي تقوم على مجموعات من العناصر مرتبطة بعمليات رياضية محددة. ظهرت هذه النظرية لتعميم مفاهيم العمليات الحسابية المعروفة، مثل الجمع والضرب، وتطبيقها على مجموعات أكثر تجريداً.

\noindent
تُعنى نظرية الزمر بدراسة الخواص التي تنشأ عن وجود عملية واحدة تُطبق على مجموعة من العناصر، بينما تتعامل نظرية الحلقات مع بنيات تحتوي على عمليتين (غالبًا الجمع والضرب) وتحاول فهم التفاعل بينهما.

\noindent
تُستخدم هذه النظريات في العديد من فروع الرياضيات والعلوم التطبيقية، بما في ذلك الفيزياء، علوم الحاسوب، التشفير، ونظرية الأعداد، وهي تمثّل خطوة أساسية لفهم العديد من المفاهيم الرياضية المتقدمة.