\chapter*{مقدمة}
\addcontentsline{toc}{chapter*}{مقدمة}
الرياضيات المتقطعة هي فرع من فروع الرياضيات التي تهتم بدراسة الكيانات المنفصلة والمحدودة، مثل الأعداد الصحيحة والمجموعات والمنطق. تختلف الرياضيات المتقطعة عن الرياضيات التحليلية التي تركز على الكيانات المستمرة كالزمن أو المسافة. وتعتبر الرياضيات المتقطعة حجر الزاوية في العديد من التطبيقات العملية في علوم الكمبيوتر، نظرية المعلومات، والذكاء الصناعي، حيث تُستخدم في معالجة البيانات واتخاذ القرارات وتنظيم المعلومات.\\
\noindent
من بين المواضيع الأساسية في الرياضيات المتقطعة، نجد نظرية البيان و الجبر البولياني، اللتين تلعبان دورًا محوريًا في بناء أساسيات العديد من التطبيقات الحديثة.\\
\noindent
نظرية البيان هي فرع من فروع الرياضيات التي تدرس مجموعات البيانات وكيفية تمثيل هذه البيانات بطريقة منظمة وفعالة. تهتم بتحديد العناصر المميزة للمجموعات، علاقات التداخل بينها، وكيفية دمجها وتقسيمها. هذه النظرية تعد حجر الزاوية في تطوير أنظمة إدارة البيانات، والخوارزميات المستخدمة في تصنيف البيانات وتحليلها.\\
\noindent
أما الجبر البولياني فهو فرع آخر يعنى بدراسة العمليات المنطقية التي تتعامل مع القيم الثنائية (صواب وخطأ، 1 و0). يعتمد الجبر البولياني على العمليات الأساسية مثل الاتحاد، التقاطع، والفرق، وهي مفاهيم تُستخدم بكثرة في تصميم الدوائر الكهربائية الرقمية وكتابة الخوارزميات في البرمجة.