\chapter{الجبر البولياني}

\section{التعاريف الاساسية}
نفرض $B$ مجموعة غير خالية معرف عليها عمليتان ثنائيتان (+) و (.) و عملية أحادية يرمز لها بالرمز ( $\bar{}$ ) و معها عنصران مختلفان هما 0 و 1.\\
عندئذ نسمي $B$ جبراً بوليانياًاذا تحققت المسلمات التالية حيث $x,y,z$ عناصر من $B$
\begin{enumerate}
	\item قوانين التبديل
	\[
	\forall x, y \in B \Rightarrow \begin{cases}
		x + y = y + x\\
		x\cdot y = y \cdot x
	\end{cases}
	\]
	\item قوانين التوزيع
	\[
	\forall x,y,z \in B \Rightarrow \begin{cases}
		x + (y\cdot z) = (x+y)\cdot(x+z) \\
		x \cdot(y+z) = (x\cdot y) + (x\cdot z)
	\end{cases}
	\]
	\item قوانين التطابق (العنصر المحايد)
	\[
	\forall x \in B \Rightarrow x + 0 = 0 + x = x
	\]
	نقول أن العنصر (0) هو عنصر محايد بالنسبة للعملية (+).
	\[
	\forall x \in B \Rightarrow x \cdot 1= 1\cdot  x = x
	\]
	نقول أن العنصر (1) هو عنصر محايد بالنسبة للعملية ($\cdot$).
	\item قوانين الاتمام 
	\[
	\forall x \in B \exists \bar{x}: \begin{cases}
		x + \bar{x} = \bar{x} + x = 1\\
		x \cdot\bar{x} = \bar{x} \cdot x = 0\\
	\end{cases}
	\]
\end{enumerate}
\noindent
\textbf{ملاحظة}\\
\noindent
في بعض الاحيان نرمز للجبر البولياني بالرمز
\[
\langle B, +, \cdot, \bar{}, 0, 1\rangle
\]
عندما نريد التأكيد على أجزائه الستة
\section*{قاعدة الأسبقية}
اذا لم تكتب اقواس فإن العملية ( $\bar{}$ ) يكون لها الاسبقية على العملية ($\cdot$) و العملية ($\cdot$) يكون لها الاسبقية على العملية (+)
\begin{example}
	$x+y\cdot z$ تعني $x+(y\cdot z)$ و ليس $(x+y)\cdot z$\\
	$x \cdot \bar{y}$ تعني $x\cdot (\bar{y})$ و ليس  $\overline{(x\cdot y)}$
\end{example}
\section{الجبر البولياني بقيمتين}
يعرف الجبر البولياني بقيمتين على مجموعة من عنصرين $B = \{0, 1\}$ حيث العمليتان الثنائيتان (+) و ($\cdot$) و عملية الاتمام معطاة كما يلي:
\begin{english}
	\begin{table}[H]
		\centering
		\begin{minipage}{0.3\textwidth}
			\centering
			\begin{tabular}{|c|c|c|}
				\hline
				$x$ & $y$ & $x\cdot y$ \\
				\hline
				0 & 0 & 0 \\
				0 & 1 & 0 \\
				1 & 0 & 0 \\
				1 & 1 & 1 \\
				\hline
			\end{tabular}
		\end{minipage}
		\begin{minipage}{0.3\textwidth}
			\centering
				\begin{tabular}{|c|c|c|}
				\hline
				$x$ & $y$ & $x + y$ \\
				\hline
				0 & 0 & 0 \\
				0 & 1 & 1 \\
				1 & 0 & 1 \\
				1 & 1 & 1 \\
				\hline
			\end{tabular}
		\end{minipage}
		\begin{minipage}{0.3\textwidth}
			\centering
			\begin{tabular}{|c|c|}
				\hline
				$x$ & $\bar{x}$\\
				\hline
				0 & 1\\
				1 & 0\\
				\hline
			\end{tabular}
		\end{minipage}
	\end{table}
\end{english}

\noindent
\textbf{ملاحظة}\\
\noindent
إن العمليات السابقة هي نفسها العمليات المنطقية:\\
	(+) تقابل ($\vee$) or  و ($\cdot$) تقابل ($\wedge$) and و ( $\bar{}$ ) تقابل $(\sim)$ not .
	\begin{example}
			لنثبت صحة قانون التوزيع التالي
		\[
		x\cdot (y+z) = x\cdot y + x \cdot z
		\]
		\begin{english}
			\begin{table}[H]
				\renewcommand{\arraystretch}{1.3}
				\centering
				\begin{tabular}{|c|c|c|c|c|c|c|c|c|}
					\hline
					$x$ & $y$ & $z$ & $y+z$ & $x\cdot(y+z)$ & $x\cdot y$ & $x\cdot z$ & $(x\cdot y)+(x\cdot z)$\\
					\hline
					0 & 0 & 0 & 0 & 0 & 0 & 0 & 0\\
					\hline
					0 & 0 & 1 & 1 & 0 & 0 & 0 &0\\
					\hline
					0 &1 &0&1& 0& 0& 0& 0\\
					\hline
					0 &1 &1 &1 &0 &0 &0& 0\\
					\hline
					1 &0 &0 &0 &0 &0 &0 &0\\
					\hline
					1 &0& 1& 1& 1& 0& 1& 1\\
					\hline
					1 &1 &0 &1 &1 &1 &1 &1\\
					\hline
					1& 1 &1& 1& 1& 1& 1& 1\\
					\hline 
				\end{tabular}
			\end{table}
		\end{english}
		بمطابقة العمود الخامس و الاخير ينتج صحة القانون
	\end{example}

\section{الدوال البوليانية}
من أجل تعريف الدالة البوليانية نعرف أولاً المتغير البولياني
\section*{المتغير البولياني}
نقول أن المتغير $x$ انه متغير بولياني اذا كان يأخذ قيمة من المجموعة $B=\{0,1\}$ فقط. أي أن اذا كانت قيمته 0 أو 1.\\
من  التعريف السابق يكون قد تحدد لدينا المجال المقابل. الآن نحدد المجال\\
نأخذ الضرب الديكارتي للمجموعة $B$ بنفسها $n$ من المرات أي 
$\underbrace{B\times B \times \cdots \times B}_{\text{$n$ من المرات}}$\\
نحصل على $B^n$ حيث 
\[
B^n = \{(x_1, x_2, \dots, x_n) \mid x_i \in B ; 1 \leq i\leq n\}
\]
\section*{الدالة البوليانية} 
هي تعبير جبري يتألف من المتغيرات الثنائية و الثوابت 0 و 1 و العمليات المنطقية، مجاله المجموعة $B^n$ و مجاله المقابل $B$. و نحدد درجة الدالة حسب قيمة $n$.

\begin{example}
	الدالة
	$F_1 = x + \bar{y} + \bar{x}\cdot y$
	هي دالة بوليانية من الدرجة الثانية لأنها يقرن كل زوج $(x,y)$ من $B^2$ بـــ $x + \bar{y} + \bar{x}\cdot y$\\
	أما الدالة
	$F_2 = x + \bar{y}\cdot z$
	هي دالة من الدرجة الثالثة لأنها تقرن كل ثلاثي $(x,y,z)$ من $B^3$ بــ $x + \bar{y}\cdot z$
\end{example}


\section{كيفية تحديد قيم الدالة البوليانية}
من أجل تحديد قيم الدالة البوليانية ننشئ جدول الصواب للتعبير الذي يمثل الدالة.
\begin{example}
	لنوجد قيم الدالة البوليانية المعطاة كما يلي: $F(x,y,z) = x\cdot y + \bar{z}$
	\begin{english}
		\begin{table}[H]
			\centering
			\renewcommand{\arraystretch}{1.3}
			\begin{tabular}{|c|c|c|c|c|c|}
				\hline
				$x$ & $y$ & $z$ & $x\cdot y$ & $\bar{z}$ & $F(x,y,z) = x\cdot y + \bar{z}$\\
				\hline
				0 &0& 0& 0& 1& 1\\
				\hline
				0 &0& 1& 0& 0& 0\\
				\hline
				0&1&0&0&1&1\\
				\hline
				0&1&1&0&0&0\\
				\hline
				1&0&0&0&1&1\\
				\hline
				1&0&1&0&0&0\\
				\hline
				1&1&0&1&1&1\\
				\hline
				1&1&1&1&0&1\\
				\hline  
			\end{tabular}
		\end{table}
	\end{english}
\end{example}

\section*{تساوي الدوال البوليانية}
تتساوى الدالتان البوليتيان $F,G$ بــ $n$ من المتغيرات اذا و فقط اذا
\[
F(x_1, x_2, \dots, x_n) = G(x_1,x_2, \dots, x_n), \quad \forall (x_1, x_2,\dots. x_n) \in B^n
\]

\noindent
\textbf{ملاحظة}\\
\noindent
نقول عن تعبيرين بوليانيي مختلفين انهما متكافئان اذا كان كل منهما يمثل الدالة البوليانية نفسها
\begin{example}
	التعابير: $xy, xy+0, xy\cdot 1$ هي تعابير بوليانية متكافئة
\end{example}
\noindent
\textbf{ملاحظة}\\
\noindent
ان عدد الدوال البوليانبة المختلفة من الدرجة $n$ الممكنة هو $2^{2^n}$ لان عدد عناصر الضرب الديكارتي $B^n$ هو $2^n$ و بما اننا نسند للدالة القيمة 0 أو 1 الى كل من المتعددات ذات الطول $n$ و المختلفة.

\section{العمليات على الدوال البوليانية}
\subsection*{1- الجمع البولياني للدوال}
نعرف عملية الجمع للدالتين البوليانيتين $F, G$ كمايلي:
\[
(F+G)(x_1, x_2, \dots, x_n) = F(x_1, x_2, \dots, x_n) + G(x_1, x_2, \dots, x_n)
\]

\subsection*{2- الضرب البولياني للدوال}
نعرف عملية الضرب للدالتين البوليانيتين $F, G$ كمايلي:
\[
(F\cdot G)(x_1, x_2, \dots, x_n) = F(x_1, x_2, \dots, x_n) \cdot G(x_1, x_2, \dots, x_n)
\]

\subsection*{3- متمم الدالة البوليانية}
اذا كانت $F$ دالة بوليانية من $B^n$ الى $B$. فإننا نعرف متمم هذه الدالة بالعلاقة
\[
\overline{F}(x_1, x_2, \dots, x_n) = \overline{F(x_1, x_2, \dots, x_n)}
\]

\subsection*{كيفية الحصول على متمم الدالة البوليانية F}
نرمز له بــ $\overline{F}$ نحصل عليه بإستبدال الاصفار بواحدات و الواحدات بأصفار في قيم الدالة $F$ وهذا يمكن اشتقاق متمم الدالة جبرياً من خلال قانوني ديمورغان حيث نستطيع تحديد هذه القوانين لأكثر من متغيرين.\\ [5pt]
\noindent
\textbf{مثال}\\
\noindent
لتكن لدينا الدالة البوليانية
\[
F(x, y, z) = x + y + z
\]
اوجد متمم هذه الدالة.
\newpage
\noindent
\textbf{الحل}\\
\noindent
من اجل ايجاد 
$\overline{F(x, y, z)} = \overline{(x+y + z)}$
نفرض $y + z = A$ ونعوض
\[
\overline{(x + y + z)} = \overline{(x + A)}
\]
حسب قانون ديمورغان:
\[
\overline{(x+A)} = \overline{x}\cdot \overline{A} = \overline{x}\cdot\overline{(y+z)} = \overline{x}\cdot(\overline{y}\cdot\overline{z}) = \overline{x}\cdot\overline{y}\cdot\overline{z}
\]
اذن
\[
\overline{F(x, y, z)}  = \overline{x}\cdot\overline{y}\cdot\overline{z}
\]


\section{قوانين الجبر البولياني}

\begin{table}[H]
	\centering
	\renewcommand{\arraystretch}{1.7}
	\begin{tabular}{|c|c|}
		\hline
		\textbf{الاسم} & \textbf{القانون}\\
		\hline
		الاتمام المضاعف & $\overline{\overline{x}} = x$\\
				\hline
		قوانين تساوي القوى & $x+x =x, \quad x\cdot x=x$\\
				\hline
		قوانين المحايد & $x+0=x, \quad x\cdot1=x$\\
				\hline
		قوانين الهيمنة & $x\cdot 0 =0,\quad x+1 =1 $\\
				\hline
		قوانين الابدال & $x+y = y + x, \quad x\cdot y = y\cdot x$\\
				\hline
		قوانين التجميع & $x+(y+z) = (x+y)+z, \quad x(yz)=(xy)z$\\
				\hline
		قوانين التوزيع & $x+yz = (x+y)\cdot(x+z), \quad x(y+z)=xy+xz$\\
				\hline
		قوانين ديمورغان & $\overline{xy} = \overline{x}+\overline{y}, \quad \overline{x+y}= \overline{x}+\overline{y}$\\
				\hline
		قوانين الامتصاص & $x+xy=x, \quad x(x+y)=x$\\
				\hline
		الخاصية الواحدية & $x+\overline{x}=1$ \\
				\hline
		الخاصية الصفرية & $x\cdot \overline{x} = 0$\\
		\hline
	\end{tabular}
\end{table}
