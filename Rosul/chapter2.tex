\chapter{بعض طرق حل الانظمة الخطية}

سنتعرف في هذا الفصل على بعض طرق حل الانظمة الخطية وهنالك طرق عديدة. سنقتصر على بعض انواع الطرق وهي طريقة كرامر وطريقة معكوس المصفوفة ثم طريقة كاوس جوردان للحذف.

\section[طريقة كرامر]{طريقة كرامر}
لحل النظام $AX=b$ فإن 
\[
x_i = \frac{|A_i|}{|A|}, \quad i = 1,2,\dots,n
\]
حيث $A_i$ هي المصفوفة $A$ مع استبدال العمود $i$ مع المتجه $b$.

\noindent
\textbf{مثال:}\\ \noindent
 اوجد حل النظام الخطي التالي بإستخدام طريقة كرامر
\begin{align*}
	2X_1 - 3X_2 =8\\
	3X_1 + X_2 =1
\end{align*}
\noindent
\textbf{الحل:}\\ \noindent
نكتب النظام بالشكل
\[
\underbrace{
\begin{pmatrix}
	2&-3\\
	3&1
\end{pmatrix}}_{A}
\underbrace{
\begin{pmatrix}
	X_1\\X_2
\end{pmatrix}}_{X}
=
\underbrace{
\begin{pmatrix}
	8\\1
\end{pmatrix}}_{b}
\]
\[
\Rightarrow |A| = 2+9=11
\]
\[
A_1 =
\begin{pmatrix}
	8&-3\\
	1&1
\end{pmatrix}
\Rightarrow |A_1| = 8+3=11
\]
\[
\Rightarrow X_1 = \frac{|A_1|}{|A|} = \frac{11}{11}=1
\]
\[
A_1 =
\begin{pmatrix}
	2&8\\
	3&1
\end{pmatrix}
\Rightarrow |A_2| = 2-24=-22
\]
\[
\Rightarrow X_2 = \frac{|A_2|}{|A|} = \frac{-22}{11} = -2
\]

\section[طريقة معكوس المصفوفة]{طريقة معكوس المصفوفة}
ليكن $AX=b$ منظومة المعادلات الخطية مكونة من $n$ من المعادلات والمتغيرات. في حال عدد المعادلات يساوي عدد المجاهيل نستخدم القانون التالي لحل النظام
\[
X = A^{-1}b
\]
حيث $X$ قيم المتغيرات و $A$ مصفوفة المعاملات وهي قابلة للانعكاس و $b$ متجه القيم المطلقة.\\
\noindent\textbf{مثال:}\\ \noindent
بإستخدام معكوس المصفوفة جد حل النظام الخطي التالي
\begin{align*}
	4X_1 - 2X_2 = 10\\
	3X_1 - 5X_2 = 11
\end{align*}
\noindent\textbf{الحل:}\\ \noindent
\[
A = 
\begin{pmatrix}
	4&-2\\
	3&-5
\end{pmatrix}
\Rightarrow |A| = -20+6=-14
\]
\[
A^{-1} = \frac{1}{|A|}\cdot \text{adj}(A)
\]
\begin{align*}
	C_{11} = (-1)^{1+1} |M_{11}| = -5\\
	C_{12} = (-1)^{1+2} |M_{12}| = -3\\
	C_{21} = (-1)^{2+1} |M_{21}| = 2\\
	C_{22} = (-1)^{2+2} |M_{22}| = 4
\end{align*}
\[
\Rightarrow C =
\begin{pmatrix}
	-5& -3\\
	2&4
\end{pmatrix}
\Rightarrow C^T
=\begin{pmatrix}
	-5&2\\
	-3&4
\end{pmatrix}
\]
\begin{align*}
	A^{-1} &= \frac{1}{|A|}C^T\\
	&=\frac{1}{-14} 
	\begin{pmatrix}
		-5&2\\
		-3&4
	\end{pmatrix}\\
	&=
	\begin{pmatrix}
		5/14 & -1/7\\
		3/14 & -2/7
	\end{pmatrix}
\end{align*}
نحسب قيم المتغيرات $X_1, X_2 $ باستخدام القانون $X=A^{-1}b$
\[
X = 
\begin{pmatrix}
	5/14 & -1/7\\
	3/14 & -2/7
\end{pmatrix} \cdot
\begin{pmatrix}
	10\\11
\end{pmatrix}=
\begin{pmatrix}
	50/14 - 11/7\\
	30/14 - 22/7
\end{pmatrix}
\]
\[
X_1 = \frac{50}{14} - \frac{11}{7} = 2
\]
\[
X_2 = \frac{30}{14} - \frac{22}{7} = -1
\]

\section[طريقة كاوس - جوردان للحذف]{طريقة كاوس - جوردان للحذف}

لحل النظام $AX=B$ بطريقة كاوس - جوردان للحذف نتبع الخطوات التالية:
\begin{enumerate}[leftmargin=*]
	\item نحول النظام الخطي الى المصفوفة الممتدة.
	\item نحول المصفوفة الممتدة الى المصفوفة المحايدة.
	\item عند تحويل المصفوفة الى مصفوفة محايدة نستخدم العمليات الصفية الاولية.
	\item نصفر العناصر الواقعة تحت القطر الرئيسي.
	\item نصفر العناصر الواقعة فوق القطر الرئيسي.
	\item نجعل عناصر القطر الرئيسي تساوي 1.
\end{enumerate}

\noindent
\textbf{مثال}\\ \noindent
جد حل النظام الخطي التالي
\begin{align*}
	X-6Y = -11\\
	5X - Y = 3
\end{align*}
\textbf{الحل}
\[
(A\mid b) =
\left(
\begin{array}{cc|c}
	1&-6&-11\\
	5&-1&3
\end{array}
\right)
\]
نصفر العناصر الواقعة تحت القطر الرئيسي. حيث نضرب الصف الاول بــ $-5$ ونضيفه الى الصف الثاني
\[
R_2\to-5R_1 + R_2  \quad 
\left(
\begin{array}{cc|c}
	1&-6&-11\\
	0&29&58
\end{array}
\right)
\]
الآن نضرب الصف الثاني بــ $6$ ونضرب الصف الاول بــ $29$ ونجمع
\[
R_1 \to 6R_2 + 29R_1 \quad
\left(
\begin{array}{cc|c}
	29&0&29\\
	0&29&58
\end{array}
\right) 
\]
نجعل عناصر القطر الرئيسي يسواي 1.
\[
\begin{array}{c}
	R_1 \to \dfrac{1}{29}R_1\\[10pt]
	R_2 \to \dfrac{1}{29}R_2
\end{array}\quad
\left(
\begin{array}{cc|c}
	1&0&1\\
	0&1&2
\end{array}
\right) 
\]
نستخرج قيم $X, Y$ 
\[
X=1, \quad Y=2 
\]
