\section{Boolean Rings}
\phantomsection
%\subsection*{Definition and Examples}
%\addcontentsline{toc}{subsection}{Definition and Examples}

\begin{mydef}[\cite{modern_algebra}]
    A Boolean ring \((R, +, \cdot)\) is a ring with identity in which every element is idempotent, that is, \(a^2=a\) for every \(a \in R\).
\end{mydef}

\begin{example}
  The ring  of the integers modulo 2, \((\Z_2,+_2,\cdot_2)\) forms a Boolean ring, since, \(0^2=0\cdot_2 0=0\) and \(1^2=1\cdot 1=1\).
\end{example}

\begin{example}
    The ring \((\p(X),\Delta,\cap)\) of subsets of a nonempty set X is easily verified to be a Boolean ring, since 
    \(A^2=A\ \cap A=A\) for every \(A\subseteq X\).
\end{example}

%\subsection*{Some Theorems and Results}
%\addcontentsline{toc}{subsection}{Some Theorems and Results}

\begin{theo}[\cite{modern_algebra}]
    Every Boolean ring \((R,+,\cdot)\) is a commutative ring of characteristic \(2\).
\end{theo}
\begin{myproof}
For every \(a,b\in R\) we have
\[
a+b=(a+b)^2=a^2+a\cdot b+b\cdot a+b^2=a+a\cdot b+b\cdot a+b
\]
then \(a\cdot b+b\cdot a=0\). In particular setting \(a=b\) we get
\(2a=a+a=a^2+a^2=0\), this shows that \(\mathrm{char}(R)=2\). Now for every \(a,b\in R\)
\begin{align*}
    a\cdot b&=a\cdot b+(a\cdot b+b\cdot a)\\
    &=(a\cdot b+a\cdot b)+b\cdot a\\
    &=2(a\cdot b)+b\cdot a\\
    &=b\cdot a
\vspace{-5pt}
\end{align*}
Hence \(R\) is commutative. \qed
\end{myproof}

\vspace{.5cm}

\begin{theo}[\cite{modern_algebra}]
    Let $(R,+,\cdot)$ be a Boolean ring , A proper ideal $(I,+,\cdot)$ of $(R,+,\cdot)$ is prime ideal if and only if it's maximal ideal.
\end{theo}
\begin{myproof}
   Assume \(I\) is maximal ideal then by Theorem \ref{max_prime_theorem} \(I\) is prime ideal. \noindent
   Conversely, let \(I\) be prime ideal and \(J\) ideal of \(R\) such that \(I\subset J\subseteq R\) , we want to prove \(J=R\)\\  
   Now, consider any \(x \in J - I\). Since \(x = x^2\), it implies \(x(1-x) = 0 \in I\) . Using the fact \(I\) is a prime ideal with \(x\notin I\), we conclude 
   \[
   1-x \in I \subset J
   \] 
   As both elements \(x\) and \(1-x\) lie in \(J\) , it follows that,
   \[
   1=x+(1-x) \in J
   \]
   The ideal \(J\) thus contains the identity , and consequently \(J=R\).\qed
\end{myproof}

\vspace{.5cm}

\begin{theo}[\cite{modern_algebra}]
\label{thm3}
  A Boolean ring $(R,+,\cdot)$ is a field if and only if $(R,+,\cdot) \simeq (\Z_2,+_2,\cdot_2)$
\end{theo}
\begin{myproof}
   ($\Leftarrow$)
   Let $(R,+,\cdot) \simeq (\Z_2,+_2,\cdot_2)$ , then $(R,+,\cdot)$ is a field since $(\Z_2,+_2,\cdot_2)$ is itself a field.\\
   ($\Rightarrow$)
   Let $(R,+,\cdot)$ be a Boolean field, then for every nonzero element $x\in R$ we have :$$\displaystyle x=x\cdot 1=x\cdot (x\cdot x^{-1})=x^2\cdot x^{-1}=x\cdot x^{-1}=1$$Then the only nonzero element in $R$ is $1$ $\Longrightarrow$ $R=\{0,1\}\simeq \Z_2$ by the homomorphism $\phi:R\longrightarrow\Z_2$ where $\phi(0)=0$ and $\phi(1)=1$.\qed
\end{myproof}

\begin{cor}[\cite{modern_algebra}]
\label{cor1}
    A proper ideal $(I,+,\cdot)$ of the Boolean ring $(R,+,\cdot)$ is a maximal ideal if and only if $(R/I,+,\cdot)\simeq(\Z_2,+_2,\cdot_2)$
\end{cor}
\begin{myproof}
    First we notice that $R/I$ is itself Boolean ring since, $$(x+I)^2=x^2+I=x+I\ ,\forall\ x\in R$$ By Theorem \ref{max_field_theorem} we know $I$ is a maximal ideal if and only if $R/I$ is a field and using Theorem \ref{thm3} we conclude that $I$ is a maximal ideal if and only if $R/I\simeq\Z_2$.\qed
\end{myproof}

\begin{lemma}[\cite{modern_algebra}]
\label{lemma1}
    Let $(R,+,\cdot)$ be a Boolean ring. For each nonzero element $x\in R$ there exists a homomorphism $\phi$ from $(R,+,\cdot)$ onto the field $(\Z_2,+_2,\cdot_2)$ such that $\phi(x)=1$
\end{lemma}
\begin{myproof}
    Let $I$ be the principle ideal generated by $1+x$ that is $$I=(1+x)=\{r\cdot (1+x):r\in R\}$$
    $I$ is proper ideal , since $1\notin I$ because if so , then for some $r\in R$ we have
    $$1=r\cdot (1+x)=r\cdot (1+x)^2=[r\cdot (1+x)]\cdot(1+x)=1\cdot(1+x)$$$$\Longrightarrow 1=1+x\Longrightarrow x=0$$
   which is contradiction (since $x\neq 0$)\\
    Now since $I$ is proper ideal of $R$ by Theorem \ref{krull_zorn} there is a maximal ideal $M$ of $R$ such that $I\subseteq M$ , so by the corollary \ref{cor1} we have $R/M\simeq\Z_2$ by some homomorphism $f:R/M\to\Z_2$\\
    Now define the function $\phi:R\to\Z_2$ by taking $\phi=f\circ \nat_M$ where $\nat_M$ is the natural homomorphism from $R$ onto $R/M$.
\begin{figure}[ht]
\centering
\begin{tikzpicture}
   \node (R) at (0,0) {$R$};
   \node (Z) at (6,0) {$\Z_2$};
   \node (RM) at (3,-3) {$R/M$};
   \node (C) at (3,-4/3) 
   {\Huge$\circlearrowleft$};
   
   \draw [->] (R) to node[above] {$\phi$} (Z);
   \draw [->] (R) to node[left] {$\nat_M$} (RM);
   \draw [->] (RM) to node[right] {$f$} (Z);
\end{tikzpicture}
\end{figure}


We now show $\phi$ is onto homomorphism , so for every $a,b\in R$ : 

with respect to addition : $$\phi(a+b)=f(\nat_M(a+b))=f(\nat_M(a)+\nat_M(b))=f(\nat_M(a))+_2f(\nat_M(b))$$
$$\Longrightarrow \phi(a+b)=\phi(a)+_2\phi(b)$$
with respect to multiplication : 
$$\phi(a\cdot b)=f(\nat_M(a\cdot b))=f(\nat_M(a)\cdot \nat_M(b))=f(\nat_M(a))\cdot_2f(\nat_M(b))$$
$$\Longrightarrow \phi(a\cdot b)=\phi(a)\cdot_2\phi(b)$$
Now let $k\in\Z_2$ be arbitrary , since $f$ is onto , then there exists $x\in R$ such that $f(x+M)=k$ so we have : 
$$\forall\ k\in\Z_2,\exists\ a\in R: \phi(a)=f(\nat_M(a))=f(a+M)=k.$$
So we showed that $\phi$ is onto homomorphism.
Now since $1+x\in I\subseteq M\Longrightarrow$ the coset $(1+x)+M=M$ so that
$$1+_2\phi(x)=\phi(1+x)=f(\nat_M(1+x))=f((1+x)+M)=f(M)=0$$
$$\Longrightarrow \phi(x)=1$$\qed
\end{myproof}

\begin{theo}[(Stone Representation Theorem)\cite{modern_algebra}]
 Every Boolean ring $(R,+,\cdot)$ is isomorphic to a ring of subsets of some fixed set.   
\end{theo}
\begin{myproof}
    Let \[
    H=\left\{f: R \longrightarrow \Z_2 \mid f\right.\text{ is homomorphism }\}
    \]    
Now, define the function \(h\) from the ring \((R,+,\cdot)\) into the ring \((\p(H),\Delta,\cap)\) such that 
\[
h(x) = \{f\in H \mid f(x)=1\} \text{ for every } x\in R
\]

\noindent Now for every $x, y \in R$ if we assume \(f:R\longrightarrow\Z_2\) is homomorphism then,
\begin{align*}
f\in h(x\cdot y) &\Longleftrightarrow f(x\cdot y)=1
\\&\Longleftrightarrow f(x)\cdot_2 f(y)=1\\&\Longleftrightarrow f(x)=1 \text{ and } f(y)=1\\&\Longleftrightarrow f\in h(x) \text{ and } f\in h(y)\\&\Longleftrightarrow f\in h(x)\cap h(y)
\end{align*}
Therefore,  $$h(x \cdot y)=h(x) \cap h(y)$$
Similarly for the addition , but first we observe that
\[
f \notin h(x) \Longleftrightarrow f(x)\neq 1 \Longleftrightarrow f(x)=0
\]
So we have,
\begin{align*}
  f \in h(x+y) &\Longleftrightarrow f(x+y)=1\\&\Longleftrightarrow f(x)+_2f(y)=1\\&\Longleftrightarrow (f(x)=1 \text{ and }f(y)=0)\text{ or }(f(x)=0\text{ and }f(y)=1)\\&\Longleftrightarrow (f\in h(x)\text{ and }f\notin h(y))\text{ or }(f\notin h(x)\text{ and }f\in h(y))\\&\Longleftrightarrow f\in h(x)\ \Delta\ h(y)
\end{align*}
Therefore,
\[h(x+y)=h(x)\ \Delta\ h(y)\]
Hence $h$ is homomorphism. Now to prove $h$ is one-to-one we find the kernel
\[\text{ker}(h)=\{x\in R\mid h(x)=\varnothing\}\]
but \(h(x)=\varnothing\) if and only if \(x=0\) by Lemma \ref{lemma1} \(\Longrightarrow \text{ker}(h)=\{0\}\)
Hence $R$ is isomorphic to a subring of $(\p(H),\Delta,\cap)$ namely the image of $R$ under the mapping $h$
\[(R,+,\cdot)\simeq (h(R),\Delta,\cap)\]\qed
\end{myproof}