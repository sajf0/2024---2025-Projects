\chapter*{Introduction}
\addcontentsline{toc}{chapter}{Introduction}
{\setstretch{1.3}
In mathematics and mathematical logic, Boolean algebra is a branch of algebra distinguished by the values of its variables, typically represented as true and false, denoted as 1 and 0, respectively. Unlike elementary algebra, which operates with numerical values, Boolean algebra employs logical operators such as conjunction ($\wedge$), disjunction ($\vee$), and negation ($\sim$), whereas elementary algebra employs arithmetic operators like addition, multiplication, subtraction, and division. Boolean algebra provides a formal framework for describing logical operations, similar to how elementary algebra deals with numerical operations.

George Boole introduced Boolean algebra in his first book, "The Mathematical Analysis of Logic" (1847), further expounded in "An Investigation of the Laws of Thought" (1854). Henry M. Sheffer first proposed the term "Boolean algebra" in 1913, though Charles Sanders Peirce had earlier referred to it as "A Boolian Algebra with One Constant" in the first chapter of his work "The Simplest Mathematics" (1880). Boolean algebra has played a pivotal role in the advancement of digital electronics and is incorporated into all modern programming languages. It also finds applications in set theory and statistics.}