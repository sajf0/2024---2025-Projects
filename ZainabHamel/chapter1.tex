\chapter{مفاهيم أساسية}

\section{العلاقات والدوال}

\begin{definition}
	العلاقة بين مجموعتين $A, B$ هي مجموعة جزئية $R$ من $A\times B$. ونقرأ $(a, b)\in R$:\\
	"$a$ مرتبط بالعنصر $b$"
	ونكتب $aRb$
\end{definition}

\begin{definition}
	الدالة $\phi$ من $X$ الى $Y$ هي علاقة بين $X$ و $Y$ مع الخاصية لكل $x\in X$ يظهر كعنصر اول في زو ج مرتب واحد $(x, y)$ في $\phi$ ونكتب $\phi:X\to Y$
\end{definition}

\section{العملية الثنائية وخصائصها}

\begin{definition}
	العملية الثنائية $*$ على مجموعة $S$ هي دالة من $S\times S$ الى $S$ لكل $(a,b) \in S\times S$. نرمز الى العنصر $*\big((a, b)\big)$ بالرمز $a*b$
\end{definition}

\begin{definition}
	العملية الثنائية * على $S$ تكون ابدالية اذا وفقط اذا $a*b = b*a$ لكل $a, b\in S$.
\end{definition}

\begin{definition}
	العملية الثنائية * على $S$ تكون تجميعية اذا كان $(a*b)*c = a*(b*c)$ لكل $a, b, c\in S$.
\end{definition}

\begin{example}
	العمليتان $+$ (الجمع) و $\cdot$ (الضرب) ابداليتيين و تجميعتين على مجموعة الاعداد الحقيقية $\R$.
\end{example}

\section{الزمرة}

\begin{definition}
	الزمرة $(G, *)$ هي مجموعة $G$ غير خالية تكون مغلقة تحت العملية * مع تحقيق البديهيات التالية
	\begin{enumerate}
		\item (التجميعية) لكل $a, b, c\in G$ لدينا
		\[
		(a*b) * c = a*(b*c)
		\]
		\item (العنصر المحايد) يوجد عنصر $e\in G$ بحيث ان لكل $x\in G$ 
		\[
		e * a = a * e = a
		\]
		\item (العنصر النظير)  لكل $a\in G$ يوجد عنصر مثل $a' \in G$ بحيث 
		\[
		a * a' = a' * a = e
		\]
	\end{enumerate}
\end{definition}

\begin{definition}
	الزمرة $G$ تكون تبديلية (Abilian) اذا كانت العملية الثنائية تبديلية.
\end{definition}

\begin{example}
	المجموعة $\Z^+$ تحت عملية الجمع + لا تشكل زمرة. لعدم وجود عنصر ممحايد.
\end{example}

\begin{example}
	المجموعة $M_{m\times n}(\R)$ مجموعة كل المصفوفات مع عملية جمع المصفوفات مع العنصر المحايد (المصفوفة الصفرية) تشكل زمرة ابدالية.
\end{example}

\section{الزمرة الجزئية}

\begin{definition}
	اذا كانت $H$ مجموعة جزئية من الزمرة $(G, *)$ ومغلقة تحت العملية الثنائية للزمرة فإذا كانت $(H, *)$ زمرة فإن $H$ زمرة جزئية من $G$ ونكتب $H\leq G$.
\end{definition}

\begin{definition}
	لتكن $G$ زمرة و $H$ زمرة جزئية منها. وليكن $x\in G$. المجموعة المصاحبة اليسارية $xH$ تعرف بالشكل
	\[
	x*H = \{x*h : h \in H\}
	\]
	اما المصاحبة اليمينية $Hx$ تعرف بالشكل
	\[
	H*x = \{ h*x : h\in H\} 
	\]
\end{definition}

\subsection*{العمليات على المجموعات المصاحبة \cite{abstract_algebra3}}
لتكن $G$ زمرة. و $H$ زمرة جزئية منها وليكن $x\in G$ سوف نرمز الى مجموعة كل المجموعات المصاحبة اليسارية بالرمز $G/H$ اي ان 
\[
G/H = \{x*H : x\in G\}
\]
نعرف العملية $\otimes$ على $G/H$ بالشكل
\[
(a*H)\otimes(b*H) = (a*b)*H
\]

\section{الزمر السوية و زمرة القسمة}

\begin{definition}
	لتكن $G$ زمرة. الزمرة الجزئية $H$ تسمى زمرة جزئية سوية اذا تحقق الشرط $x * H = H * x$ لكل $x\in G$. و نكتب $H \triangleleft G$
\end{definition}

\begin{definition}
	لتكن $G$ زمرة و $H \triangleleft G$ فإن $G/H$ تكون زمرة مع العملية $\otimes$ المعرفة بالشكل
	\[
	(a*H)\otimes(b*H) = (a*b)*H
	\]
	نسمي الزوج $(G/H, \otimes)$ بزمرة القسمة.
\end{definition}

\section{الحلقة وبعض خصائصها}

\begin{definition}
	الحلقة $(R, +, \cdot)$ هي مجموعة $R$ مع عمليتان ثنائيتيان. الجمع (+) و الضرب ($\cdot$) مع البديهيات التالية
	\begin{enumerate}
		\item $(R, +)$ زمرة ابدالية.
		\item $a\cdot(b\cdot c) = (a\cdot b)\cdot c$ لكل $a, b, c\in R$.
		\item $a\cdot(b+c) = (a\cdot b) + (a\cdot c)$ و $(a+b)\cdot c = (a\cdot c) + (a\cdot b)$ لكل $a, b, c\in R$. 
	\end{enumerate}
\end{definition}

\begin{definition}
	لتكن $R$ حلقة. فإن $R$ تكون حلقة ابدالية اذا كان $a\cdot b = b\cdot a$ لكل $a, b\in R$.
\end{definition} 

\begin{definition}
	لتكن $R$ حلقة. المحايد هو العنصر $1\in R$ بحيث ان $1\cdot x= x\cdot1=x$ لكل $x\in R$.
\end{definition}

\begin{example}
	$(\Z, +, \cdot), (\Q, +, \cdot), (\R, +, \cdot), (\Z_n, +_n, \cdot_n)$ حلقات ابدالية ذات محايد.
\end{example}
\newpage
\begin{example}
	$(M_{n\times n}(\R), +, \cdot)$ حلقة ذات محايد ولكن غير ابدالية.
\end{example}
\noindent
\textbf{الحل}\\
\noindent
نثبت اولا ان 
$(M_{n\times n}, +)$ زمرة ابدالية:\\
1. المجموعة $M_{n\times n}$ مغلقة بالنسبة للعملية +\\
2. لكل $A, B, C\in M_{n\times n}$ 
\begin{align*}
	A + (B + C) &= [a_{ij}] + \Big([b_{ij}] + [c_{ij}]\Big)\\
	&= [a_{ij}] + [b_{ij} + c_{ij}]\\
	&= [a_{ij} + b_{ij} + c_{ij}]\\
	&= \Big([a_{ij}] + [b_{ij}]\Big) + [c_{ij}]\\
	&= (A + B) + C
\end{align*}
3. المصفوفة الصفرية تكون العنصر المحايد لأن لكل $A\in M_{n\times n}$ لدينا $A + 0 = 0 + A = A$.\\
4. لكل $A\in M_{n\times n}$ النظير الجمعي يكون $-A = [-a_{ij}]$.\\
5. العملية + تكون ابدالية 
\[
A + B = [a_{ij}] + [b_{ij}] = [a_{ij} + b_{ij}] = [b_{ij} + a_{ij}] = B + A
\] 
اذن $(M_{n\times n}, +)$ زمرة ابدالية.\\
نلاحظ أن العملية $\cdot$ تكون تجميعية. اي ان
$
A\cdot (B\cdot C) = (A\cdot B)\cdot C
$\\
ايضا نلاحظ ان العملية $\cdot$ تتوزع على العملية + من اليمين و من اليسار اي ان
\[
A\cdot (B+C) = A\cdot B + A\cdot C
\]
\[
(A + B) \cdot C = A\cdot C + B\cdot C
\]
والمصفوفة الواحدية (المحايدة) $I$ تمثل العنصر المحايد لعملية الضرب اي ان 
$A\cdot I = I \cdot A = A$
اذن $(M_{n\times n}(\R), +, \cdot)$ حلقة ذات محايدة ولكن غير تبديلية لان في 
$M_{2\times 2}(\R)$
\[
\begin{bmatrix}
	1 & 2\\
	3 & 4
\end{bmatrix}\cdot
\begin{bmatrix}
	2 & -1\\
	3 & -5
\end{bmatrix}
=
\begin{bmatrix}
	8 & -11\\
	18 & -23
\end{bmatrix}
\]
ولكن
\[
\begin{bmatrix}
	2 & -1\\
	3 & -5
\end{bmatrix}\cdot
\begin{bmatrix}
	1 & 2\\
	3 & 4
\end{bmatrix}=
\begin{bmatrix}
	-1 & 0\\
	-12 & -14
\end{bmatrix}
\]


