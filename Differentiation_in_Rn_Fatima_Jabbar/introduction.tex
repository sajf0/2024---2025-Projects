\subchapter*{مقدمة}
\addcontentsline{toc}{chapter*}{مقدمة}
الإشتقاق يعتبر من أهم المواضيع في الرياضيات التطبيقية أن لم يكن أهمها ، حيث يعد من الركائز الأساسية التي يستند عليها معظم مجالات الرياضيات المختلفة فضلاً عن العلوم الأخرى ، و بهذا أي نتيجة نحصل عليها تخص الاشتقاق نتوقع لها تطبيقاً في الرياضيات أو في العلوم الأخرى. في بادئ الأمر تمت دراسة الاشتقاق على الدوال في الفضاء أحادي البعد $\R$ و تم تعريف المشتقة بأنها ميل المماس لدالة عند نقطة معينة وبعد ذلك تم توسعة مفهوم المشتقة لدالة على مجموعة من النقاط و بعد ذلك تم اثبات العديد من النتائج المهمة التي تخص المشتقة في $\R$. و لكن في معظم الحالات في حياتنا العملية نتعامل مع مسائل في أكثر من بعد ، من هذا المنطلق كان يجب أن نوسع مفهوم المشتقة لكي نستطيع تطبيقه بشكل موسع في حياتنا\\
في بحثنا هذا سوف نحاول توسعة مفهوم الاشتقاق من الفضاء أحادي البعد إلى الفضاء متعدد الأبعاد $\R^n$ ، حيث نحاول أن نفهم ماذا يعني أن تكون دالة إتجاهية قابلة للإشتقاق عند نقطة من نقاط مجالها التي سوف تمثل متجه بطبيعة الحال ، وبعد أن نفهم قابلية الاشتقاق سوف ننتقل إلى مفهوم التفاضل التام الذي سوف يساعدنا في تقريب بعض المسائل العددية التي يصعب ايجاد الحل الحقيقي لها.
