%\newgeometry{a4paper, width=160mm,top=25mm, bottom=25mm}
\begin{titlepage}
\begin{minipage}{0.2\textwidth}
\raggedleft\includegraphics[scale=0.2]{Title Page/20231126094730!جامعة_البصرة.png}
\end{minipage}
\hfill
\begin{minipage}{0.5\textwidth}
    \centering
 %  \large
   \textbf{وزارة التعليم العالي والبحث العلمي}\\
    \textbf{جامعة البصرة}\\
    \textbf{كلية التربية للعلوم الصرفة}\\
    \textbf{قسم الرياضيات}
\end{minipage}
\hfill
\begin{minipage}{0.2\textwidth}
\raggedright\includegraphics[scale=0.16]{Title Page/img_1_1700235785069.png}   
\end{minipage}

\vspace{1cm}

\begin{center}
\rule{100mm}{0.5mm}\\
    \vspace{1cm}
    \large \textbf{الإشتقاق في $\R^n$}\\
    \large \textbf{\en{Differentiation in $\R^n$}}\\
    \vspace{12pt}
    \rule{100mm}{0.5mm}
\end{center}
\vfill
\begin{center}
    %\large
    \textbf{مقدم إلى قسم الرياضيات كلية التربية للعلوم الصرفة جامعة البصرة\\
    \vspace{6pt}
    وهو جزء من متطلبات نيل شهادة بكالوريوس علوم الرياضيات}
\end{center}
\vfill
\begin{center}
    %\large
    \textbf{من قبل الطالبة}\\
    \vspace{8pt}
   % \large
    \textbf{فاطمة جبار حسن}
\end{center}
\vspace{10pt}
\begin{center}
    %\large
    \textbf{إشراف}\\
    \vspace{8pt}
   % \large
   \textbf{د. خالد عبدالاله}
\end{center}
\vspace{10pt}
\begin{center}
  %\large 
  \textbf{1445 ه‍ــ - 2024 م}
\end{center}
\end{titlepage}
%\restoregeometry