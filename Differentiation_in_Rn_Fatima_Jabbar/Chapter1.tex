\chapter{مفاهيم أساسية}
\thispagestyle{empty}
\newpage
\section{المشتقة}

\begin{definition}
    $f$ دالة حقيقية يقال بأنها قابلة للإشتقاق عند نقطة $a\in\R$ إذا وفقط إذا كانت الدالة $f$ معرفة عند فترة تحوي $a$ والغاية
    \begin{equation}
    \label{diff_def}
        f^{\prime}(a):=\lim\limits_{h\rightarrow 0}\frac{f(a+h)-f(a)}{h}
    \end{equation}
    موجودة. في هذه الحالة $f^{\prime}(a)$ تسمى مشتقة الدالة $f$ عند $a$. إذا فرضنا $x=a+h$ فإن \eqref{diff_def} تصبح بالشكل
    \begin{equation*}
        f'(a)=\lim\limits_{x\to a}\frac{f(x)-f(a)}{x-a}
    \end{equation*}
\end{definition}
\noindent
إذن عندما $x\rightarrow a$ فإن ميل الوتر الذي يمر بالنقاط $(x,f(x))$ و $(a,f(a))$ يُقرب ميل المماس الدالة $y=f(x)$ عند $x=a$ كما موضح بالشكل \ref{fig:tangent}
\begin{figure}[ht]
\centering
\begin{tikzpicture}
\begin{axis}[
axis equal,
    axis lines = middle,
    xlabel = \(x\),
    ylabel = {\(y\)},
    xtick={1,2,3},
    xticklabels={$a$,$x_1$,$x_2$},
    ytick=\empty,
    ymin=0, ymax=5,
    xmin=0, xmax=6.5,
]
% Function plot
\addplot [
    domain=0.5:4.5,
    samples=100,
    thick,
]
{exp(x)/10};
% Tangent line at x = 1
\addplot [
    domain=0:5,
    samples=100
]
{exp(1)/10*x};
% Line segment approaching tangent from left
\addplot [
    domain=0:4,
    samples=100,
    dashed,
]
{(exp(3)-exp(1))/20 *(x-1)+exp(1)/10};
% Line segment approaching tangent from right
\addplot [
    domain=0:4,
    samples=100,
    dashed,
]
{(exp(2)-exp(1))/10 *(x-1)+exp(1)/10};
\addplot [mark=*] coordinates {(1,e/10)};
\addplot [mark=*] coordinates {(2,e^2/10)};
\addplot [mark=*] coordinates {(3,e^3/10)};
\end{axis}
\end{tikzpicture}
\caption{تقارب مستقيمات الوتر للمماس}
\label{fig:tangent}
\end{figure}
\begin{note}
    إذا كانت $f$ قابلة للإشتقاق عند كل نقطة في مجموعة $E$ ، فإن $f^{\prime}$ هي دالة على $E$. هذه الدالة لها عدة ترميزات
\[
D_x f= \frac{df}{dx} =f^{(1)}=f^{\prime}
\]
عندما $y=f(x)$ ، سوف نستخدم الترميز $dy/dx$ أو $y^{\prime}$
\end{note}

\begin{note}
    المشتقات من الرتب العليا تعرف بشكل متكرر، ذلك بأن، إذا كان $n\in\N$ ، إذن \\$f^{(n+1)}(a)=(f^{(n)})^{\prime}(a)$
    ونستخدم للمشتقات العليا الرموز $d^nf/dx^n$ أو $f^{(n)}$ أو بحالة \\$y=f(x)$ فإننا نكتب $d^ny/dx^n$ أو $y^{(n)}$.
\end{note}
\begin{theorem}
    \label{linear_function_diff'able_Theorem_R}
    $f$ دالة حقيقية تكون قابلة للإشتقاق عند $a$ إذا وفقط إذا وجدت دالة $T$ من الشكل $T(x):=mx$ بحيث
\begin{equation}
    \label{linear_diff_limit}
    \lim\limits_{h\to0}\frac{f(a+h)-f(a)-T(h)}{h}=0
\end{equation}
\end{theorem}
\begin{myproof}
    أفرض أن $f$ قابلة للإشتقاق ، و أجعل $m=f'(a)$. إذن بواسطة \eqref{diff_def} نحصل على
    \[
    \frac{f(a+h)-f(a)-T(h)}{h}=\frac{f(a+h)-f(a)}{h}-f'(a)\to0
    \]
    عندما $h\to0$

    بالعكس إذا كانت \eqref{linear_diff_limit} متحققة للدالة $T(x):=mx$ و $h\neq0$ ، إذن
    \begin{align*}
        \frac{f(a+h)-f(a)}{h}&=m+\frac{f(a+h)-f(a)-mh}{h}\\
       &= m+\frac{f(a+h)-f(a)-T(h)}{h}
    \end{align*}
    بواسطة \eqref{linear_diff_limit}.الغاية للصيغة الأخيرة عليها وتساوي $m$. وبالتالي $\mfrac{f(a+h)-f(a)}{h}\to m$ عندما $h\to0$؛ وهذا يعني $f'(a)$ موجودة وتساوي $m$.
\end{myproof}

\begin{theorem}
    إذا كانت $f$ قابلة للإشتقاق عند $a$ ، إذن $f$ مستمرة عند $a$.
\end{theorem}
\begin{myproof}
\vspace{-30pt}
    \begin{align*}
        \lim\limits_{x\to a}f(x)-f(a)&=\lim\limits_{x\to a}f(x)-f(a)\cdot\frac{x-a}{x-a}\\
       &= \lim\limits_{x\to a}\frac{f(x)-f(a)}{x-a}\cdot(x-a)\\
       &= \lim\limits_{x\to a}\frac{f(x)-f(a)}{x-a}\cdot\lim\limits_{x\to a}(x-a)\\
        &=f'(a)\cdot0\\
        &=0
    \end{align*}
    إذن $f(x)\to f(a)$ عندما $x\to a$ ، وبالتالي $f$ مستمرة عند $a$.
\end{myproof}

\begin{example}
    برهن أن الدالة $f(x)=\abs{x}$ مستمرة عند $0$ ولكن غير قابلة للإشتقاق عند $0$
\end{example}
\begin{myproof}
    عندما $x\to0$ هذا يؤدي إلى أن $\abs{x}\to0$ و بالتالي فإن $f$ مستمرة عند $a$ ، من الناحية الأخرى ، بما أن
   \[
   \abs{h}=\begin{cases}
       h & h>0\\
      -h & h<0
  \end{cases}
   \]
    لدينا
    \[
    \lim\limits_{h\rightarrow0^+}\frac{f(h)-f(0)}{h}=1\quad\text{و}\quad\lim\limits_{h\rightarrow0^-}\frac{f(h)-f(0)}{h}=-1
    \]
\end{myproof}
\begin{definition}
    لتكن $I$ فترة
    \begin{enumerate}
        \item الدالة $f:I\to\R$ قابلة للإشتقاق على $I$ إذا وفقط إذا كانت الغاية
        \[
        f'_I(a):=\lim_{\substack{x\to a \\ x\in I}}\frac{f(x)-f(a)}{x-a}
        \]
        موجودة ومنتهية لكل $a\in I$.

        \item يقال بأن $f$ قابلة للإشتقاق بشكل مستمر على $I$ إذا وفقط إذا كانت $f'_I$ موجودة و مستمرة على $I$.
    \end{enumerate}
\end{definition}

\begin{note}
  إذا كانت $a$ ليست نقطة نهاية للفترة $I$ فإنه $f'_I(a)=f'(a)$ ، لذا بالعادة نستخدم الرمز $f'$ بدلاً من $f'_I$. بالخصوص إذا كانت $f$ قابلة للإشتقاق على $[a,b]$ فإن
  \[
  f'(a)=\lim\limits_{h\to0^+}\frac{f(a+h)-f(a)}{h},\quad f'(b)=\lim\limits_{h\to0^-}\frac{f(b+h)-f(b)}{h}
  \]
\end{note}

\begin{note}
    اشتقاق بعض الدوال المهمة 
\[
\begin{array}{c|c}
f(x) & f'(x) \\
\hline
a & 0\\
x^n & nx^{n-1} \\
e^{\alpha x} & \alpha e^{\alpha x} \\
\ln(x) & \mfrac{1}{x} \\
\sin(x) & \cos(x) \\
\cos(x) & -\sin(x) \\
\tan(x) & \sec^2(x) \\
\cot(x) & -\csc^2(x) \\
\sec(x) & \sec(x) \tan(x) \\
\csc(x) & -\csc(x) \cos(x) \\
\end{array}
\]
\end{note}

\begin{example}
    الدالة $f(x)=x^{3/2}$ قابلة للإشتقاق على الفترة $[0,\infty)$ و $f'(x)=3\sqrt{x}/2$ لكل $x\in[0,\infty)$.
\end{example}
\begin{myproof}
    باستخدام قانون المشتقة للقوى فإن $f'(x)=3\sqrt{x}/2$ لكل $x\in(0,\infty)$ وبإستخدام التعريف نجد أن
    \[
    f'(0)=\lim\limits_{h\to0^+}\frac{h^{3/2}-0}{h}=\lim\limits_{h\to0^+}\sqrt{h}=0
    \]
\end{myproof}

\begin{definition}
    نرمز إلى مجموعة الدوال التي تمتلك $n$ من المشتقات الموجودة و المستمرة على الفترة $I$ بالرمز $C^{(n)}(I)$ أي أن
    \[
    C^{(n)}(I):=\cbracket{f\mid f:I\to\R ; f^{(n)}\: \text{موجودة و مستمرة}}
    \]
وسوف نرمز إلى مجموعة الدوال التي تنتمي إلى $C^{(n)}(I)$ لكل $n\in\N$ بالرمز $C^{\infty}(I)$.
\end{definition}

\begin{theorem}
\label{derivatives_rules}
    لتكن $f,g$ دوال حقيقية و $\alpha\in\R$. إذا كانت $f,g$ دوال قابلة للإشتقاق عند $a$ فإن
    \begin{gather}
        (f\pm g)'(a)=f'(a)\pm g'(a)\\[7pt]
        (\alpha f)'(a)=\alpha f'(a)\\[7pt]
        (f\cdot g)'(a)=g(a)f'(a)+f(a)g'(a)\\[7pt]
        \pbracket{\frac{f}{g}}'(a)=\frac{g(a)f'(a)-f(a)g'(a)}{[g(a)]^2},\quad g(a)\neq 0
    \end{gather}
\end{theorem}

\begin{theorem}[{[قاعدة السلسلة]}]
    لتكن $f,g$ دوال حقيقية. اذا كانت $f$ قابلة للإشتقاق عند $a$ و $g$ قابلة للإشتقاق عند $f(a)$ فإن $g\circ f$ قابلة للإشتقاق عند $a$ مع 
    \[
    (g\circ f)'(a)=g'[f(a)]f'(a)
    \]
\end{theorem}

\section{مبرهنة القيمة المتوسطة}
\begin{theorem}
\label{mean_value_theorem}
    افرض أن $a,b\in\R$ مع $a<b$.
    \begin{enumerate}
        \item[.i][مبرهنة القيمة المتوسطة المعممة] إذا كانت $f,g$ دوال مستمرة على $[a,b]$ و قابلة للإشتقاق على $(a,b)$ إذن يوجد عدد $c\in(a,b)$ بحيث
        \[
        g'(c)(f(b)-f(a))=f'(c)(g(b)-g(a))
        \]

        \item[.ii] [مبرهنة القيمة المتوسطة]  إذا كانت $f$ دوال مستمرة على $[a,b]$ و قابلة للإشتقاق على $(a,b)$ إذن يوجد عدد $c\in(a,b)$ بحيث
        \[
        f(b)-f(a)=f'(c)(b-a)
        \]
       
    \end{enumerate}
\end{theorem}

\section{الضرب الديكارتي}
\begin{definition}
    لتكن $E_1,E_2,\dots,E_n$ تجمع منتهِ من المجموعات ، فإن الضرب الديكارتي يعرف بالشكل
    \[
    E_1\times E_2\times\dots\times E_n:=\cbracket{(x_1,x_2,\dots,x_n):x_j\in E_j\text{ for } j=1,2,\dots,n}
    \]
    إذن الضرب الديكارتي إلى $n$ من المجموعات الجزئية من $\R$ هو مجموعة جزئية من $\R^n$
\end{definition}
