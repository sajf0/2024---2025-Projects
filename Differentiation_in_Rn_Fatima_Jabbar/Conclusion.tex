\subchapter*{الخلاصة}
\addcontentsline{toc}{chapter*}{الخلاصة}
بعد الإنتهاء من البحث صرنا عارفين بكيفية التعامل مع الإشتقاق داخل الفضاء $\R^n$. حيث تم توسعة تعريف المشتقة أحادية البعد وتطبيق نظرياتها في الفضاء $\R^n$ ، وأيضاً تعرفنا على مفهوم قابلية الإشتقاق الذي كان يشابه إلى حد كبير (لكن ليس تماماً) مفهوم قابلية الإشتقاق في $\R$ ، و بعد ذلك تعرفنا على الشرط الكافي و الضروري لتكون الدالة الإتجاهية قابلة للإشتقاق و أثناء ذلك عرّفنا الإشتقاق التام (الكلي) الذي هو من نقاط الشبه مع مشتقة الدالة في الفضاء $\R$. و أخيراً تعرفنا على مفهوم التفاضل الذي مكّننا من تقريب بعض المسائل التي يصعب إيجاد قيمة حقيقية لها.