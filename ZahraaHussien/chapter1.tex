\chapter{مفاهيم اساسية}
	
	\section*{المقدمة}
	
	\section{نظرية التقريب \LR{(Approximation theory)}}
	
	في الكثير من التطبيقات العلمية في الرياضيات أو الهندسة والتكنولوجيا، يمكن أن نحصل على دوال معقدة وأحيانًا غير مألوفة، وبذلك تصبح عملية دراسة هذه الدوال من حيث اتصالها وقابليتها للتفاضل أو التكامل وغيره من الموضوعات الصعبة جدًا والتي تستغرق الكثير من الوقت.  
	كذلك يمكن أن نتقابل مع ما يسمى \textbf{الدوال المجدولة أو دوال البيانات} وهي تلك الدوال التي تعطى على شكل الثنائيات المرتبة:
	\[
	\{ (x_i, y_i) \}_{i=0}^{n}
	\]
	حيث يكون المطلوب هو معرفة شكل العلاقة أو الدالة التي تربط بين فئة العقد \(\{x_i\}\) وفئة قيم الدالة \(\{y_i\}\) عند هذه العقد.
	
	لكل هذه الأسباب وغيرها، يكون من الضروري والمفيد استبدال دالة بسيطة وغير معقدة بدالة ذات شكل رياضي معقد أو دالة بيانات بحيث يمكن اعتبارها بديلاً للدالة المعطاة. في الواقع، أن \textbf{نظرية التقريب} تتعامل مع كل المفاهيم السابقة.  
	
	والتقريب يمكن أن يكون بواسطة الدوال كثيرات حدود جبرية أو غير جبرية أو بواسطة دوال أسية أو دوال مثلثية وغيرها.
	
	\section{الدالة (Function)}
	
	\textbf{تعريف:}  
	هي علاقة من المجموعة \( A \) إلى المجموعة \( B \) بحيث يقترن كل عنصر في \( A \) بعنصر واحد فقط في \( B \). وتسمى المجموعة \( A \) \textbf{مجال الدالة} والمجموعة \( B \) \textbf{النطاق المرافق}. ويرمز لمجموعة القيم بالصورة:
	\[
	f: A \to B
	\]
	والمجموعة الجزئية من \( B \) التي تتألف من جميع صور عناصر \( A \) في الدالة \( f \) تسمى \textbf{مدى الدالة} ويرمز له بالرمز:
	\[
	f(A) = \{ f(a) \mid a \in A \}
	\]
	
	\section{الدالة الخطية \LR{(Linear function)}}
	
	\textbf{تعريف:}  
	هي علاقة تربط عددًا حقيقيًا \( x \) بعدد حقيقي آخر \( ax \)، وتسمى \textbf{دالة خطية} معاملها هو \( a \). الدالة الخطية تكتب على الشكل:
	\[
	f(x) = ax
	\]
	
	\section{المؤثر (Operator)}
	
	\textbf{تعريف:}  
	المجال \( D \) للمؤثر \( S \) عبارة عن مجموعة غير خالية من الدوال الحقيقية جميعها لها نفس المجال \( X \)، ولكل \( f \in D \) تكون \( S(f) \) دالة حقيقية في المجال \( X \).
	
	\section{مؤثر لوباس \LR{(Lupas operator)}}
	
	هو نوع من المؤثرات الخطية المستخدمة في التقريب وتحليل الدوال، وهو مرتبط بأساليب التقريب من خلال المؤثرات الإيجابية التي تحافظ على بعض الخصائص المهمة للدوال.  
	يتم استخدام \textbf{مؤثر لوباس} في التقريب التوافقي وتحليل الوظائف الرياضية، حيث يستعمل بشكل خاص في تقريب الدوال المستمرة بواسطة متعددات الحدود.
	
	\section{المؤثر الخطي \LR{(Linear operator)}}
	
	\textbf{تعريف:}  
	المؤثر \( F \) من \( X \) إلى \( Y \) يقال إنه خطي إذا تحقق:
	\[
	F(\alpha x_1 + \beta x_2) = \alpha F(x_1) + \beta F(x_2), \quad \forall x_1, x_2 \in D(F), \forall \alpha, \beta \in \mathbb{R}
	\]
	ويرمز لمجموعة المؤثرات الخطية من \( X \) إلى \( Y \) بالرمز \( L(X,Y) \).
	
	\section{فضاء المتجهات \LR{(Vector space)}}
	
	\textbf{تعريف:}  
	لتكن \( V \) مجموعة غير خالية معرف عليها عمليتا الجمع والضرب بعدد ثابت. يقال إن \( V \) فضاء متجهات إذا تحققت البديهيات التالية لكل متجه \( u, v, w \) ولكل عدد حقيقي \( c, d \):
\begin{english}
		\begin{enumerate}[label=\arabic*)]
		\item \( u + v \in V \)
		\item \( u + v = v + u \)
		\item \( u + (v + w) = (u + v) + w \)
	\end{enumerate}
	\end{english}
	يوجد متجه صفري في $V$ بحيث ان لكل $u\in V$
	\begin{english}
		\begin{enumerate}[label=\arabic*), start=4]
			\item \(0 + u = u + 0 = u\)
		\end{enumerate}
	\end{english}
	لكل $u \in V$ يوجد متجه $(-u \in V)$ بحيث
	\begin{english}
		\begin{enumerate}[label=\arabic*), start=5]
\item $u + (-u) = (-u) + u - 0$
		\item \( c u \in V \)
		\item \( c (u + v) = cu + cv \)
		\item \( (c + d) u = cu + du \)
		\item \( c (d u) = (cd) u \)
		\item \( 1 \cdot u = u \)
	\end{enumerate}
\end{english}
	
	\section{الفضاء الجزئي \LR{Subspace}}
		\textbf{تعريف:} لتكن $M$ مجموعة جزئية من فضاء المتجهات $V$ يقال ان $M$ فضاء جزئي من $V$ اذا كان $M$  هو فضاء متجهات بالنسبة لعمليتي الجمع والضرب بعدد معرف على $V$.
	\section{فضاء $C_h[0, \infty)$}
	\begin{center}
		$C_h[0, \infty) = \{f\in C[0, \infty) : |f(t)| \leq m(1+t)^h \text{\LR{for some $m > 0, h > 0$}}\}$
	\end{center}
	
	\newpage
	\section{مبرهنة كوروفكن \LR{(Korovkin's Theorem)}}
	لتكن \( S_n(f(x); x) \) متتابعة من المؤثرات الخطية الموجبة \( L.P.O \) والشروط التالية متحققة:
\begin{english}
		\begin{enumerate}
		\item \( S_n(1; x) = 1 + \alpha_n(x) \)
		\item \( S_n(t; x) = x + \beta_n(x) \)
		\item \( S_n(t^2; x) = x^2 + \delta_n(x) \)
	\end{enumerate}
\end{english}
	حيث \( \alpha_n(x), \beta_n(x), \delta_n(x) \) متتابعات متقاربة بانتظام إلى 0 في الفترة \([a, b]\)، فإن:
	\[
	S_n(f(t); x) \to f(x) \quad \text{as} \quad n \to \infty
	\]
	