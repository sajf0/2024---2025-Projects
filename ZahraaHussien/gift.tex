\chapter*{الإهداء}
\begin{center}
	
إلى سيدي ومولاي، صاحب العصر والزمان (عج)، وإلى أهل بيت النبوة (عليهم السلام)، منبع العلم والحكمة، الذين علمونا معنى الصبر والتضحية، أهدي ثمرة جهدي هذا، سائلًا الله القبول والتوفيق.\\ [10pt]
\textcolor{red}{إلى أمي الحبيبة}\\
التي كانت لي وطنًا وسكنًا،  وصبرها كان قوتي، وحنانها كان بلسمًا يخفف عني عناء الأيام. يا نبع الحب والحنان، لو كان للعرفان شكلٌ، لكنتِ أنتِ صورته الأجمل\\[10pt]
\textcolor{red}{إلى أبي العزيز}\\
 السند والقوة، الذي لم يبخل يومًا بتقديم العون والنصيحة،يا من كنت دائمًا سندي لك كل الشكر والتقدير.\\[10pt]
\textcolor{red}{إلى أخواتي الغاليات}\\
اللواتي كنَّ لي العائلة التي لا تُعوَّض، والصديقات اللاتي لم يفارقنني في دربي، دعمكنَّ منحني القوة للاستمرار، وشراكتكنَّ في هذه الرحلة زادتها جمال\\[10pt]
\textcolor{red}{إلى أصدقائي وصديقاتي}\\
 الذين كانوا سندًا في كل مرحلة من مراحل دراستي لاتقتصر الرحلة الدراسية على الكتب فقط، بل على الأشخاص الذين يحيطون بنا ويساعدوننا في تخطي الصعاب أنتم جزء من أجمل ذكرياتي، وستبقون دائمًا في قلبي."\\[10pt]
\textcolor{red}{أخيرًا، إلى نفسي...}\\
إلى كل لحظة شعرتُ فيها بالتعب، إلى كل ليلة سهرتُ فيها لأصل إلى هذه اللحظة، لكل الجهد الذي بذلته، والساعات التي قضيتها بين الكتب ،إلى كل دمعة نزلت يأسًا، ولكل ابتسامة رسمتها رغم الصعوبات، إلى كل مرة شعرت فيها بالرغبة في الاستسلام لكنني قاومت، إلى كل جهد بذلته، وكل حلم تمسكت به... اليوم أهديكِ هذا الإنجاز، فهو ثمرة صبركِ وكفاحكِ. قد لا يكون الطريق كان سهلًا، لكنه كان يستحق كل خطوة فيه. شكرًا لكِ لأنكِ لم تستسلمي، ولأنكِ وقفتِ من جديد كلما تعثرتِ. هذا الإهداء لكِ، لأنكِ كنتِ تستحقينه دائمًا.
\end{center}