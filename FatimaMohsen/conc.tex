\chapter*{الاستنتاجات}
\addcontentsline{toc}{chapter*}{الاستنتاجات}
من خلال دراسة موضوع الدوال المستمرة وغير المستمرة، يمكن الوصول إلى مجموعة من الاستنتاجات العامة التي تبرز أهمية هذا الموضوع في الرياضيات ومجالاتها التطبيقية:

\begin{enumerate}
	\item الدوال تُعد من الأسس الجوهرية في الرياضيات، وفهم خصائصها يُسهم في بناء قاعدة علمية متينة لتحليل المسائل الرياضية بمختلف أنواعها.
	
	\item التمييز بين الدوال المستمرة وغير المستمرة أمر ضروري لفهم سلوك النماذج الرياضية، حيث أن هذا التمييز يساعد في اختيار الطرق المناسبة للتحليل أو المعالجة.
	
	\item الاستمرارية تُعتبر خاصية مركزية تضمن سلاسة وانتظام تغير القيم، وهي ضرورية في الكثير من التطبيقات مثل الفيزياء والهندسة والاقتصاد.
	
	\item الدوال غير المستمرة، رغم تعقيداتها، تحمل أهمية كبيرة، فهي تمثل أنماطًا من التغيرات الفجائية أو الظواهر غير المنتظمة التي تظهر في الواقع.
	
	\item دراسة خصائص الدوال، كقابلية الاشتقاق والتكامل، تفتح المجال لفهم أعمق لكيفية تعامل هذه الدوال مع التغيرات وتأثيرها في الحلول الرياضية.
	
	\item الربط بين الجوانب النظرية والتطبيقية للدوال يساهم في تطوير مهارات التحليل الرياضي، ويجعل من الرياضيات أداة أكثر فاعلية لفهم العالم من حولنا.
\end{enumerate}
