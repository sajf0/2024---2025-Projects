\chapter*{مقدمة}
\addcontentsline{toc}{chapter*}{مقدمة}
	\noindent
	تعد الدوال من أهم المفاهيم الرياضية التي تعبر عن العلاقة بين المتغيرات، حيث تربط كل عنصر في مجموعة معينة بعنصر وحيد في مجموعة أخرى. تلعب الدوال دورًا أساسيًا في مختلف فروع الرياضيات، مثل التحليل والجبر والإحصاء، كما تمتد تطبيقاتها إلى العلوم الطبيعية والهندسية والاقتصادية. فهي تساعد في فهم التغيرات، التنبؤ بالاتجاهات، وحل المشكلات المعقدة في مجالات متعددة.\cite{mathanalysis}\\[10pt]
	\noindent
	من بين الخصائص المهمة للدوال خاصية الاستمرارية، التي تحدد مدى سلاسة تغير القيم دون انقطاعات. ومع ذلك، هناك العديد من الظواهر التي لا يمكن تمثيلها بدوال مستمرة، مما يجعل دراسة \textbf{الدوال غير المستمرة} ضرورية. هذه الدوال هي التي تحتوي على نقاط يحدث فيها تغير مفاجئ في القيم، مما يعني أنها لا تأخذ مسارًا سلسًا كما هو الحال في الدوال المستمرة.\cite{realanal}\\[10pt]
	\noindent
	هذا النوع من الدوال له أهمية كبيرة في العديد من المجالات، حيث يمثل الظواهر التي تتغير بشكل مفاجئ أو غير منتظم. في الرياضيات، تعتبر دراسة الدوال غير المستمرة ضرورية في التحليل الرياضي ونظرية الدوال، حيث تساعد في فهم السلوكيات غير المتوقعة وحل المعادلات التي تتضمن تغيرات فجائية. وفي الفيزياء، تظهر الدوال غير المستمرة في النماذج التي تصف الانتقالات الطورية، والنبضات الكهربائية، والموجات غير المنتظمة. أما في الهندسة، فهي تُستخدم في تحليل الإشارات، ونظرية التحكم، والنظم الديناميكية. كما أن للاقتصاد دورًا في استخدام الدوال غير المستمرة في دراسة الأسواق المالية والتغيرات المفاجئة في الأسعار والطلب.\cite{introrealanal}\\[10pt]
	\noindent
	لذلك، فإن دراسة الدوال غير المستمرة لا تقل أهمية عن دراسة الدوال المستمرة، حيث تساهم في تحليل وفهم الظواهر الطبيعية والاصطناعية التي لا يمكن نمذجتها باستخدام الدوال المستمرة فقط.\cite{elemrealanal}
	
