\chapter{الدوال المستمرة وخصائصها}

\section{تركيب الدوال المستمرة}

\begin{definition}[( التركيب ) \cite{introrealanal}]
	لتكن $f$ و $g$ دوال معرفة على المجالات $D_f$ و $D_g$ على التوالي. اذا كانت $D_g$ تمتلك مجموعة جزئية غير خالية $T$ بحيث $g(x) \in D_f$ لكل $x\in T$ ، فأن تركيب الدالتين $f\circ g$ يعرف بالشكل
	\[
	(f \circ g)(x) = f(g(x)), \quad \forall x\in T.
	\]
\end{definition}

\begin{theorem}
	افرض ان $g(x)$ دالة مستمرة عند $x_0 $ و $f(x)$ مستمرة عند $g(x_0)$  ، فأن $f\circ g$ مستمرة عند $x_0$.
\end{theorem}
\noindent
\textbf{البرهان}\\
\noindent
نفرض ان $\epsilon > 0 $ ، بما ان $f$ مستمرة عند $g(x_0)$ ، اذن يوجد $\delta_1> 0 $ بحيث
\begin{equation}
	| f(t)  - f(g(x_0))| < \epsilon \quad\text{if}\quad |t - g(x_0)| < \delta_1
\end{equation}
وبما ان $g$ مستمرة عند $x_0 $ اذن يوجد $\delta > 0 $ بحيث
\begin{equation}
	|g(x) - g(x_0) | < \delta_1 \quad\text{if}\quad |x-x_0| < \delta
\end{equation}
اذن من المعادلتين (1) و (2) نحصل على 
\[
|f(g(x)) - f(g(x_0)) | < \epsilon \quad\text{if}\quad |x-x_0| < \delta.
\] 
بالتالي $f\circ g$ دالة مستمرة عند $x_0$.\qed

\begin{example}
	الدالة $f(x) = \sqrt{x}$ مستمرة لكل $x> 0 $ ، والدالة
	\[
	g(x) = \frac{9 - x^2}{x + 1} 
	\]
	مستمرة لكل $x \neq -1 $ وبالتالي فأن دالة التركيب 
	\[
	(f\circ g)(x) = \SQRT{\frac{9-x^2}{x+1}}
	\]
	مستمرة لكل نقاط مجالها  
	$(-\infty, -3) \cup (1,3)$
\end{example}

\section{الاستمرارية المنتظمة \cite{introrealanal}}

\begin{definition}
	يقال ان $f$ دالة مستمرة بإنتظام على المجموعة $S$ ، اذا كان لكل  $\epsilon > 0 $ يوجد $\delta$ بحيث
	\[
	|f(x) - f(y) | < \epsilon \quad\text{whenever}\quad |x-y| < \delta \,\, \text{and}\,\, x,y\in S
	\] 
\end{definition}

\begin{example}
	الدالة $ f(x) = 2x$ مستمرة بإنتظام على 
	$(-\infty, \infty)$
	لأن
	\[
	|f(x) - f(y)| < \epsilon \quad\text{if}\quad |x-y| < \frac{\epsilon}{2} 
	\]
\end{example}

\begin{example}
	اذا كان $r> 0 $ فإن الدالة $g(x) = x^2 $ مستمرة بإنتظام على الفترة $[-r,r]$ ، لرؤية ذلك ، نلاحظ
	\[
	|g(x) - g(y) | = |x^2 - y^2| = |x-y| |x+y| \leq 2r |x-y|
	\]
	لذا
	\[
    |f(x) - f(y)| < \epsilon \quad\text{if}\quad |x-y| < \delta = \frac{\epsilon}{2r} \quad\text{and}\quad -r < x,y < r
 	\]
\end{example}

\begin{example}
	الدالة
	\[
	f(x) = \cos \frac{1}{x}
	\]
	مستمرة على الفترة
	$(0,1]$
	ولكن غير مستمرة بأنتظام على نفس الفترة ، لأن
	\[
	\left|f\left(\frac{1}{n\pi} \right) - f\left(\frac{1}{(n+1)\pi}\right)\right| =2 , \quad n=1,2,\dots
	\]
\end{example}

\newpage

\begin{theorem}
	اذا كانت $f$ دالة مستمرة على الفترة المغلقة  $[a, b]$ ، فإن $f$ دالة مستمرة بأنتظام على الفترة $[a, b]$.
\end{theorem}
\noindent
\textbf{البرهان}\\
\noindent
لنفرض ان $\epsilon > 0 $ ، بما أن $f$ دالة مستمرة على $[a, b]$ ، لذا فأن لكل $t\in [a, b]$ يوجد عدد موجب $\delta_t > 0 $ بحيث
\begin{equation}
	|f(x) - f(t) | < \frac{\epsilon}{2}  \quad\text{if}\quad |x-t| < 2\delta_t \quad\text{and}\quad x \in [a, b]
 \end{equation}
 اذا كانت $I_t = (t-\delta_t, t+\delta_t)$ ، فأن التجميعة
 \[
 H = \{I_t : t\in [a, b]\}
 \]
 تمثل غطاء مفتوح للفترة المغلقة $[a, b]$ ، وبما أن $[a, b]$ متراصة ، بأستخدام مبرهنة هاين - بورل يوجد عدد منته من النقاط $t_1, t_2,\dots, t_n$ في $[a, b]$ بحيث $I_{t_1}, I_{t_2}, \dots, I_{t_n} $ تغطي $[a, b]$ ، الآن نعرف 
\begin{equation}
	 \delta = \min \{\delta_1, \delta_2, \dots, \delta_n\}
\end{equation}
 الآن سوف نثبت اذا كان 
 \begin{equation}
 	| x - y| < \delta \quad\text{and}\quad  x, y\in [a, b]
 \end{equation}
 فأن $|f(x) - f(y)| < \epsilon$. من المتراجحة المثلثية
 \begin{equation}
 	\begin{aligned}
 		|f(x) - f(y)| &= |f(x) - f(t_r) + f(t_r) - f(y)|\\
 		&\leq |f(x) - f(t_r) | + |f(t_r) - f(y).|
 	\end{aligned}
 \end{equation}
 بما ان  $I_{t_1}, I_{t_2}, \dots, I_{t_n} $ تغطي $[a, b]$ ، العنصر $x$ يجب ان ينتمي الى واحدة من هذه الفترات ، نفرض ان $x\in I_r$ ، أي أن
 \begin{equation}
 	|x-t_r| < \delta_{t_r}
 \end{equation}
من (3) مع $t=t_r$ 
\begin{equation}
	|f(x) - f(t_r)| < \frac{\epsilon}{2}
\end{equation}
من (5) و (7) و المتراجحة المثلثية
\[
|y-t_r| = |y-x + x -t_r| \leq |y-x| + |x-t_r| < \delta + \delta_{t_r} \leq 2\delta_{t_r}
\]
وبالتالي من (3) مع $t=t_r$ و استبدال $x$ بـــ $y$ نحصل على 
\[
|f(y) - f(t_r)| < \frac{\epsilon}{2}
\]
(6) و (8) تؤدي الى 
$$|f(x) - f(y)| < \epsilon$$
وبالتالي الدالة $f$ مستمرة بأنتظام على الفترة $[a, b]$.\qed

\begin{corollary}
	لتكن $f$ دالة مستمرة على مجموعة $T$ ، فأن $f$ مستمرة بأنتظام على اي فترة مغلقة محتواة في $T$.
\end{corollary}


\section{الاستمرارية وقابلية الاشتقاق \cite{introrealanal}}

في هذا البند سوف نناقش علاقة قابلية الاشتقاق لدالة معينة مع استمراريتها عند نقطة معينة
\begin{theorem}
	اذا كانت $f$ دالة قابلة للاشتقاق عند النقطة $x_0 $ ، فأنها مستمرة عند نفس النقطة.
\end{theorem} 
\noindent
\textbf{البرهان}\\
\noindent
من تعريف المشتقة عند النقطة $x_0 $ فأن 
\[
f'(x_0) = \lim\limits_{x\to x_0} \frac{f(x) - f(x_0 )}{x-x_0}
\]
هذه الغاية موجودة و منتهية ، نريد اثبات ان $f$ دالة مستمرة عند $x_0 $ اي بمعنى 
\[
\lim\limits_{x\to x_0} f(x) = f(x)
\]
الآن 
\[
f(x) = f(x_0) + \left[\frac{f(x) - f(x_0)}{x-x_0}\right] (x-x_0)
\]
الآن، نأخذ الغاية للطرفين عندما $x\to x_0 $.
\begin{align*}
\lim\limits_{x\to  x_0}f(x) &= f(x_0) + \lim\limits_{x\to x_0}
\left\{
\left[\frac{f(x) - f(x_0)}{x-x_0}\right] (x-x_0)
\right\}\\
&= f(x_0 ) + f'(x_0) \cdot 0\\
&= f(x_0)
\end{align*}
بالتالي $f$ دالة مستمرة عند $x_0$.\qed

\begin{note}
	عكس المبرهنة السابقة ليس دائماً صحيح. ويمكن رؤية ذلك من خلال المثال القادم
\end{note}

\begin{example}
	الدالة $f(x) = |x|$ مستمرة لكل $x\in\R$ ولكنها غير قابلة للاشتقاق عند 0. لأن 
	\[
	\lim\limits_{x\to 0^+} \frac{f(x) - f(0)}{x-0} = \lim\limits_{x\to 0^+} \frac{|x|}{x} = 1
	\]  
	\[
		\lim\limits_{x\to 0^-} \frac{f(x) - f(0)}{x-0} = \lim\limits_{x\to 0^-} \frac{|x|}{x} = -1
	\]
	بالتالي فأن الغاية للمشتقة غير موجودة ، اذن الدالة غير قابلة للاشتقاق عند 0.
	
	\begin{figure}[H]
		\centering
		\begin{tikzpicture}
			\begin{axis}[
				axis lines = middle,
				xlabel = $x$,
				ylabel = {$f(x)$},
				xmin = -5, xmax = 5,
				ymin = -2, ymax = 10,
				samples = 100
				]
				\addplot[domain=-5:0,blue,thick] {-x};
				\addplot[domain=0:5,red,thick] {x};
			\end{axis}
		\end{tikzpicture}
		\caption{\en{Plot of \(f(x) = |x|\)}}
	\end{figure}
\end{example}

\section{الاستمرارية وقابلية التكامل \cite{introrealanal}}

\begin{theorem}
	اذا كانت $f$ دالة مستمرة على الفترة $[a, b]$ فأنها دالة قابلة للتكامل على الفترة $[a, b]$.
\end{theorem}
\noindent
\textbf{البرهان}\\
\noindent
لتكن $P = \{ x_0 , x_1, \dots, x_n\}$ تجزئة للفترة $[a, b]$ ، بما ان الدالة $f$ مستمرة على $[a, b]$ ، توجد نقاط $c_j, c_j'$ بحيث
\[
f(c_j) = M_j = \sup_{x_{j-1}\leq x\leq x_j} f(x)
\]
و 
\[
f(c_j') = m_j = \inf_{x_{j-1}\leq x\leq x_j} f(x)
\]
وبالتالي

\begin{equation}
	S(P) - s(P) = \sum_{j=1}^{n} \left[f(c_j) - f(c_j')\right](x_j - x_{j-1})
\end{equation}

بما ان الدالة مستمرة بأنتظام على الفترة $[a, b]$ (مبرهنة 2 - 2 -1) ، اذن لكل $\epsilon > 0 $ يوجد $\delta > 0 $ بحيث
\[
|f(x') - f(x)| < \frac{\epsilon}{b-a}
\]
اذا كانت $x, x'\in [a, b]$ و $|x-x'| > \delta$ ، اذا كان $||P|| > \delta$ فأن $|c_j - c_j'| > \delta$ ، ومن (9) 
\[
S(P) - s(P) < \frac{\epsilon}{b-a} \sum_{j=1}^{n} (x_j - x_{j-1} ) =\epsilon.
\]
وبالتالي $f$ دالة قابلة للتكامل على الفترة $[a, b]$.

\begin{note}
	عكس المبرهنة السابقة ليس صحيح دائماً. سنوضح ذلك في المثال القادم
\end{note}

\begin{example}
	لنأخذ الدالة 
	\[
	f(x) = \begin{cases}
		1 & x=1/2 \\
		0 & x\in [0,1] ,x \neq 1/2
	\end{cases}
	\]
	هذه الدالة قابلة للتكامل ، لاثبات ذلك ، نفرض $\epsilon > 0 $ ونكون التجزئة $P$ حيث
	\[
	P = \left[0, \frac{1}{2} - \frac{\delta}{2} \right] \cup \left[\frac{1}{2} - \frac{\delta}{2}, \frac{1}{2} + \frac{\delta}{2}\right]  \cup \left[\frac{1}{2} + \frac{\delta}{2}, 1\right]
	\]
	الآن
	\[
	S(P ) = 1\cdot \delta
	\]
	\[
	s(P) = 0
	\]
	وبالتالي
	\[
	|S(P) - s(P)| = \delta
	\]
	بأخذ $\delta < \epsilon$ نثبت قابلية التكامل. ولكن الدالة غير مستمرة عند $1/2$ لأن	
\[
\lim\limits_{x\to 1/2} f(x) = 0 \neq 1 = f(1/2). 
\] 
\end{example}