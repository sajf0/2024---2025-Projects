\chapter{الدوال المستمرة وخصائصها}

\section{تركيب الدوال المستمرة}

\begin{definition}[( التركيب ) \cite{introrealanal}]
	لتكن $f$ و $g$ دوال معرفة على المجالات $D_f$ و $D_g$ على التوالي. اذا كانت $D_g$ تمتلك مجموعة جزئية غير خالية $T$ بحيث $g(x) \in D_f$ لكل $x\in T$ ، فأن تركيب الدالتين $f\circ g$ يعرف بالشكل
	\[
	(f \circ g)(x) = f(g(x)), \quad \forall x\in T.
	\]
\end{definition}

\begin{theorem}
	افرض ان $g(x)$ دالة مستمرة عند $x_0 $ و $f(x)$ مستمرة عند $g(x_0)$  ، فأن $f\circ g$ مستمرة عند $x_0$.
\end{theorem}
\noindent
\textbf{البرهان}\\
\noindent
نفرض ان $\epsilon > 0 $ ، بما ان $f$ مستمرة عند $g(x_0)$ ، اذن يوجد $\delta_1> 0 $ بحيث
\begin{equation}
	| f(t)  - f(g(x_0))| < \epsilon \quad\text{if}\quad |t - g(x_0)| < \delta_1
\end{equation}
وبما ان $g$ مستمرة عند $x_0 $ اذن يوجد $\delta > 0 $ بحيث
\begin{equation}
	|g(x) - g(x_0) | < \delta_1 \quad\text{if}\quad |x-x_0| < \delta
\end{equation}
اذن من المعادلتين (1) و (2) نحصل على 
\[
|f(g(x)) - f(g(x_0)) | < \epsilon \quad\text{if}\quad |x-x_0| < \delta.
\] 
بالتالي $f\circ g$ دالة مستمرة عند $x_0$.\qed

\begin{example}
	الدالة $f(x) = \sqrt{x}$ مستمرة لكل $x> 0 $ ، والدالة
	\[
	g(x) = \frac{9 - x^2}{x + 1} 
	\]
	مستمرة لكل $x \neq -1 $ وبالتالي فأن دالة التركيب 
	\[
	(f\circ g)(x) = \SQRT{\frac{9-x^2}{x+1}}
	\]
	مستمرة لكل نقاط مجالها  
	$(-\infty, -3) \cup (1,3)$
\end{example}