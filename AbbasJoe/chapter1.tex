\chapter{طريقة التفاضل التربيعي}
\section[مقدمة]{مقدمة \LR{Introduction}}

في هذا الفصل سنوضح كيفية الحصول على الصيغ الأساسية لتقريب المشتقة الأولى و المشتقات من الرتب العليا بإستعمال طريقة التفاضيل التربيعي \en{(DQM)}, بعد ذلك سوف نذكر الصيغ التي يمكن حساب معاملات الوزن من خلالها و دورها في تحديد دقة الحلول العددية.

\section[صيغ التفاضل التربيعي]{صيغ التفاضل التربيعي \LR{Differential Quadrature Formulas}}

تتلخص فكرة عمل طريقة التفاضل التربيعي في تقريب المشتقات الجزئية (أو الاعتيادية) بواسطة مجموع الاوزان لمتغيرات الدالة المعطاة في كل نقاط الشبكة وضمن المجال المحدد لحساب قيم الدالة. لتكن الدالة $y=f(x)$ معرفة على الفترة $[a,b]$ حيث $a,b$ ثوابت ولنفرض أن المنطقة قسمت إلى $N$ من النقاط كما موضح في الشكل \ref{fig:numberline}\cite{chang_shu}

\begin{figure}[H]
	\centering
	\begin{tikzpicture}
		\draw[latex-latex] (-3.5,0) -- (3.5,0) ; %edit here for the axis
		\foreach \x in  {-3,-2,-1,0,1,2,3} % edit here for the vertical lines
		\draw[shift={(\x,0)},color=black] (0pt,3pt) -- (0pt,-3pt);
		\draw[shift={(-3,0)}] (0pt,0pt) -- (0pt,-3pt) node [below] {$x_1$};
		\draw[shift={(0,0)}] (0pt,0pt) -- (0pt,-3pt) node [below] {$x_i$};
		\draw[shift={(3,0)}] (0pt,0pt) -- (0pt,-3pt) node [below] {$x_N$};
	\end{tikzpicture}
	
	\caption{تقسيم الفترة $[a,b]$}
	\label{fig:numberline}
\end{figure}

لذلك يكون تقريب المشتقة الأولى و الثانية للدالة $f(x)$ عند النقطة $x_i$ بالشكل
\begin{align}
	\subs{\frac{dy}{dx}}_{x=x_i}=\sum_{j=1}^{N}a_{ij}^{(1)}f(x_j)\\
	\subs{\frac{d^2y}{dx^2}}_{x=x_i}=\sum_{j=1}^{N}a_{ij}^{(2)}f(x_j)
\end{align}
وبشكل عام
\begin{equation}
	\label{eq:DQM_formula}
	\subs{\frac{d^ry}{dx^r}}_{x=x_i}=\sum_{j=1}^{N}a_{ij}^{(r)}f(x_j)
\end{equation}
حيث $a_{ij}^{(r)}$ تمثل معاملات الوزن من الرتبة $r^{th}$ وسنوضح في البند القادم كيف يتم حساب معاملات الوزن وبيان دورها في تحديد دقة الحلول الناتجة من استعمال طريقة التفاضل التربيعي.

\section[معاملات الوزن من الرتبة الأولى]{معاملات الوزن من الرتبة الأولى \LR{Weight Coefficients of First Order}}

أن تحديد نقاط الشبكة و معاملات الوزن هما عاملان مهمان في تطبيق صيغ التفاضل التربيعي \eqref{eq:DQM_formula} وأن لمعاملات الوزن دوراً رئيسياً و مهماً في طريقة التفاضل التربيعي و تعد أحد مفاتيح هذه الطريقة لما لها من أهمية في التأثير على دقة الحلول العددية. الآن سوف نتطرق إلى طرق حساب معاملات الوزن من الرتبة الأولى


\subsection[طريقة بيلمان الأولى]{طريقة بيلمان الأولى \cite{Bellman} \LR{(Billman's First Approach)}}

في هذه الطريقة استخدم بيلمان دوال الاختبار التالية للحصول على معاملات الوزن
\begin{equation}
	\label{test_function_1}
	g_k(x)=x^k,\quad k =0,1,2,\dots,N-1
\end{equation}
من الواضح أن معادلة \eqref{test_function_1} تعطي $N$ من دوال الإختبار. معاملات الوزن $a_{ij}^{(1)}$ في \eqref{eq:DQM_formula} ، $i$ و $j$ تأخذ قيم من $1$ إلى $N$ وبالتالي مجموع معاملات الوزن هو $N\times N$. بتطبيق دوال الإختبار على نقاط الشبكة $x_1,x_2,\dots,x_N$ ، نتيجة لذلك نحصل على $N\times N$ من المعادلات

\begin{align}
	\label{equations_system}
	&\sum_{j=1}^{N}a_{ij}^{(1)}=0\notag\\
	&\sum_{j=1}^{N}a_{ij}^{(1)}\,x_j=1\\
	&\sum_{j=1}^{N}a_{ij}^{(1)}\,x_j^k=k\,x_i^{k-1},\quad k=2,3,\dots,N-1\notag
\end{align}
لكل $i=1,2,\dots,N$. نظام المعادلات في \eqref{equations_system} يملك حل وحيد لأن مصفوفة النظام تأخذ شكل \\\textbf{Vandermonde}. لسوء الحظ عندما تكون $N$ كبيرة يصعب إيجاد حل لهذا النظام لهذا يتم إختيار قيم صغيرة إلى $N$ (أقل من $13$).

\begin{note}
	لا توجد أي قيود على اختيار نقاط الشبكة $x_i$ في طريقة بيلمان لحساب معاملات الوزن من الرتبة الأولى.
\end{note}

\subsection[طريقة بيلمان الثانية]{طريقة بيلمان الثانية \cite{Bellman} \LR{(Billman's Second Approach)}}

في هذه الطريقة أستخدم بيلمان دوال الإختبار التالية للحصول على معاملات الوزن
\begin{equation}
	\label{test_function_2}
	g_k(x)=\frac{L_N(x)}{(x-x_k)L_N^{(1)}(x)},\quad k=1,2,\dots,N
\end{equation}
حيث $L_N(x)$ هي متعددة حدود ليجندر من الدرجة $N$ و $L_N^{(1)}(x)$ هي المشتقة الأولى إلى $L_N(x)$. يتم في هذه الطريقة إختيار نقاط الشبكة $x_1,x_2,\dots,x_N$ لتكون جذور متعددة حدود ليجندر المُزاحة إلى الفترة $[0,1]$. وبتطبيق دوال الإختبار في \eqref{test_function_2} على نقاط الشبكة. بيلمان توصل إلى أن صياغة جبرية بسيطة لحساب معاملات الوزن $a_{ij}$.
\begin{equation}
	\label{bilman2_equations}
	\begin{aligned}
		&a_{ij}=\frac{L_N^{(1)}(x_i)}{(x_i-x_j)L_N^{(1)}(x_j)},\quad i\neq j\\[10pt]
		& a_{ii}=\frac{1-2x_i}{2x_i(x_i-1)}   
	\end{aligned}
\end{equation}
رغم هذه البساطة إلا أن هذه الطريقة لبست بمرونة الطريقة الأولى والسبب يعود إلى إختيار نقاط الشبكة $x_1,x_2,\dots,x_N$ حيث لا نستطيع تحديدها بشكل إختياري ، بدلاً من ذلك يتم اختيارها كجذور متعددة حدود ليجندر من الدرجة $N$. لهذا السبب فإن الطريقة الأولى تُفَضّل في التطبيقات العملية.

\begin{note}
	أن متعددات حدود ليجندر المُزاحة إلى الفترة $[0,1]$ تعطى بالصيغة 
	\begin{equation*}
		L_N^{*}(x)=\sum_{k=0}^{N}(-1)^{N+k}\frac{(N+k)!}{(N-k)!(k!)^2}\,x^k
	\end{equation*}
	أول خمس متعددات حدود ليجندر هي
	\begin{align*}
		&L_0^{*}(x)=1\\
		&L_1^{*}(x)=2x-1\\
		&L_2^{*}(x)=6x^2-6x+1\\
		&L_3^{*}(x)=20x^3-30x^2+12x-1\\
		&L_4^{*}(x)=70x^4-140x^3+90x^2-20x+1
	\end{align*}
\end{note}

\subsection[طريقة كوان و جانك]{طريقة كوان و جانك \cite{Quan} \LR{(Quan \& Chang's Approach)}}

لتطوير طُرق بيلمان في حساب معاملات الوزن ، العديد من المحاولات تمت بواسطة العديد من الباحثين. واحدة من أكثر الطرق فائدة هي الطريقة المقدمة من الباحثين كوان و جانك. حيث استعملا متعددات حدود لاكرانج كدوال إختيار
\begin{equation}
	\label{test_function_3}
	g_k(x)=\frac{M(x)}{(x-x_k)M^{(1)}(x_k)},\quad k=1,2,\dots,N
\end{equation}
حيث
\begin{align*}
	&M(x)=(x-x_1)(x-x_2)\cdots(x-x_N)\\
	&M^{(1)}(x_i)=\prod_{k=1,k\neq i}^{N}(x_i-x_k)
\end{align*}
وبتطبيق هذه الدوال على $N$ من نقاط الشبكة ، نحصل على الصيغ الجبرية لحساب معاملات الوزن $a_{ij}^{(1)}$.
\begin{equation}
	\label{quan_chang_equations}
	\begin{aligned}
		& a_{ij}^{(1)}=\frac{1}{x_j-x_i}\prod_{k=1,k\neq i,j}^{N}\frac{x_i-x_k}{x_j-x_k},\quad i\neq j \\
		& a_{ii}^{(1)}=\sum_{k=1,k\neq i}^{N}\frac{1}{x_i-x_k}
	\end{aligned}
\end{equation}
ومن المهم معرفة أنه لا توجد أية قيود على اختيار نقاط الشبكة في هذه الطريقة

\begin{note}
	طريقة كوان و جانك مكافئة لطريقة بيلمان الأولى لهذا هنا أيضاً لا توجد قيود في اختيار نقاط الشبكة $x_i$.
\end{note}

\section[معاملات الوزن من الرتبة الثانية]{معاملات الوزن من الرتبة الثانية \LR{Weight Coefficients of Second Order}}

في هذا البند سوف نتعرف على طرق حساب معاملات الوزن من الرتبة الثانية حيث
\begin{equation}
	\label{eq:DQM_formula_order2}
	\subs{\frac{d^2f}{dx^2}}_{x=x_i}=\sum_{j=1}^{N}a_{ij}^{(2)}f(x_j)
\end{equation}
حيث $a_{ij}^{(2)}$ هي معاملات الوزن من الرتبة الثانية.

\subsection[طريقة شو العامة]{طريقة شو العامة \cite{chang_shu} \en{(Shu's General Approach)}}
بإستخدام تقريب متعددات الحدود و فضاء المتجهات توصل شو إلى صياغة لمعاملات الوزن من الرتبة الثانية ، كما يلي
\begin{equation}
	\label{eq:shus_equations}
	a_{ij}^{(2)}=2a_{ij}^{(1)}\pbracket{a_{ii}^{(1)}-\frac{1}{x_i-x_j}},\quad i\not=j
\end{equation}
حيث نلاحظ من \eqref{eq:shus_equations} إذا كانت $i\neq j$ فإن $a_{ij}^{(2)}$ يمكن أن تحسب بسهولة. يمكن تطبيق نظام المعادلات في \eqref{equations_system} لـــ $k=1$ ، نحصل على
\[
\sum_{j=1}^{N}a_{ij}^{(2)}=0\Longrightarrow a_{ii}^{(2)}=-\sum_{j=1,i\neq j}^{N}a_{ij}^{(2)}
\]


\subsection[طريقة ضرب المصفوفات]{طريقة ضرب المصفوفات \cite{chang_shu} \LR{(Matrix Multiplication Method)}}
من تعريف المؤثر التفاضلي لدينا
\[
\der{^2f}{x^2}=\der{}{x}\pbracket{\der{f}{x}}
\]
بتطبيق طريقة التفاضل التربيعي على الطرفين ، نحصل على
\begin{align*}
	\sum_{j=1}^{N}a_{ij}^{(2)}\cdot f(x_j)&=\sum_{k=1}^{N}a_{ik}^{(1)}\cdot \subs{\der{f}{x}}_{x_k}\\
	&=\sum_{k=1}^{N}a_{ik}^{(1)}\sum_{j=1}^{N}a_{kj}^{(1)}\cdot f(x_j)\\
	&=\sum_{j=1}^{N}\sbracket{\sum_{k=1}^{N}a_{ik}^{(1)}\cdot a_{kj}^{(1)}}\cdot f(x_j)
\end{align*}
وبمقارنة الطرفين نحصل على
\begin{equation}
	a_{ij}^{(2)}=\sum_{k=1}^{N}a_{ik}^{(1)}\cdot a_{kj}^{(1)}
\end{equation}
وبلغة المصفوفات هذا يعني
\begin{equation}
	\label{second_order_equations}
	[a_{ij}^{(2)}]=[a_{ij}^{(1)}]\times[a_{ij}^{(1)}]
\end{equation}

\begin{note}
	يُفضل استخدام طريقة شو العامة لأنها تتطلب عمليات حسابية أقل من طريقة ضرب المصفوفات
\end{note}

\section[اختيار نقاط الشبكة]{اختيار نقاط الشبكة \LR{Choice of Grid Points}}

أن اختيار نقاط الشبكة واحد من العوامل المهمة التي تؤثر على دقة التقريبات الناتجة من استعمال طريقة التفاضل التربيعي. لذلك ركز العديد من الباحثين منهم \textbf{شو \en{(Shu)}} على كيفية دراسة تأثير نقاط الشبكة على دقة الحلول في هذه الطريقة. ولكن حين يمكننا التحكم بنقاط الشبكة ففي طريقة بيلمان الثانية لا يمكننا ذلك.

\subsection[النقاط متساوية الأبعاد]{النقاط متساوية الأبعاد \LR{(Equally Spaced Grid Points)}}

\begin{equation}
	\label{eq:equally_spaced_points}
	x_i=a+\frac{b-a}{N-1}(i-1),\quad i=1,2,\dots,N
\end{equation}
هذا النوع من النقاط كان قيد الإستعمال من قبل الكثير من الباحثين لبساطتها وملائمتها في حل الكثير من المسائل

\subsection[نقاط شيبيشيف-كاوس-لوباتو]{نقاط شيبيشيف-كاوس-لوباتو \LR{(Chebyshev-Gauss-Lobatto Points)}}
ويطلق عليها اختصاراً \textbf{نقاط لوباتو \en{(Lobatto Points)}}
\begin{equation}
	\label{eq:lobatto-points}
	x_i=a+\frac{1}{2}\sbracket{1-\cos\pbracket{\frac{i-1}{N-1}}\pi}(b-a),\quad i=1,2,\dots,N
\end{equation}
وتسمى في بعض الأحيان النقاط غير متساوية الأبعاد \en{(Unequally Spaced Points)}. وقد أثبت العديد من الباحثين بأن هذا النوع من النقاط يعطي نتائج أكثر دقة من النقاط متساوية الأبعاد.




