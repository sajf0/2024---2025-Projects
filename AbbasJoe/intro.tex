\chapter*{مقدمة \en{Introduction}}
\addcontentsline{toc}{chapter*}{مقدمة}
أغلب المسائل الهندسية محكومة بمجموعة من المعادلات التفاضلية الجزئية \textbf{\en{(PDEs)}} مع شروط حدودية ، على سبيل المثال تدفقات السوائل النيوتونية محكومة بمعادلات نيفر-ستوكس\\ \textbf{\en{(Navier-Stockes equations)}  \cite{navier_stokes}}. بشكل عام أنه من الصعب جداً علينا الحصول على الحل الدقيق (الحل الحقيقي) لهكذا معادلات. لذا من المهم أن نطور بعض الحلول العددية التقريبية للمعادلات التفاضلية الجزئية.\\
\noindent
في معظم الحالات الحل التقريبي يُقدم على شكل قيم دالية عند نقاط متقطعة \textbf{\en{(discrete points)}} أو نقاط الشبكة \textbf{\en{(grid points)}}. في هذه المرحلة قد يسأل أحدهم عن العلاقة التي تربط بين المشتقات في المعادلة التفاضلية الجزئية و القيم الدالية عند نقاط الشبكة يبدو أن هناك جسر يربط بينهم.\\
\noindent
تقنية التقدير العددية \textbf{\en{(numerical discretization technique)}} هي الجسر المنشود، كثير من الحلول و الطرائق طُوِرت من الباحثين أبرزها الفروقات المحددة \textbf{\en{(finite differences)} \cite{finite_difference}} و الحجومات المحددة \textbf{\en{(finite volumes)} \cite{finite_volume}}.\\
\noindent
أغلب المسائل العددية في الهندسة قد تُحل بواسطة هذه الطرائق و لكنها تتطلب عدد كبير من نقاط الشبكة بينما الحلول المطلوبة تكون عند عدد محدد من نقاط الشبكة.\\
\noindent
في الجهة الأخرى ، في السعي نحو الحصول على طريقة تحقق حلول عددية دقيقة إلى حد كبير مع عدد قليل من نقاط الشبكة. قدم بلمان \textbf{\en{(Bellman)} \cite{Bellman} 
} سنة (1971 , 1972) طريقة التفاضل التربيعي \textbf{\en{(differential quadrature method)}}، حيث المشتقات الجزئية تُمثل على شكل مجموع خطي موزون لكل القيم الدالية عند كل نقاط الشبكة على طول ذلك الاتجاه. أن فكرة التفاضل التربيعي مستوحاة من فكرة التكامل التربيعي \textbf{\en{(integral quadrature)}}.\\
\noindent
مفتاح طريقة \textbf{\en{DQM}} هي تحديد معاملان الوزن التي سوف تحدد دقة الطريقة، ان هذه الطريقة تستخدم عدد قليل من نقاط الشبكة للحصول على دقة عالية مقارنة مع الطرائق الاخرى مثل \textbf{\en{(finite differences)} \cite{finite_difference}}.\\
\noindent
هنا تم تطبيق طريقة التفاضل التربيعي لحل مسائل قيم حدودية متنوعة . و تم حساب مقدار الخطأ للحلول التقريبية التي حصلنا عليها . فكان مقدار الخطأ صغير مما يدل على دقة الطريقة . و ايضاً من خلال مقارنة حلول الطريقة مع الحل المضبوط وجدنا أن تقارب الطريقة جيد و الجداول و الرسومات للنتائج التي حصلنا عليها تبين ذلك.
\restoregeometry