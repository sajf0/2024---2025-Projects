\chapter*{مقدمة \en{Introduction}}
\pagestyle{plain}
\addcontentsline{toc}{chapter*}{مقدمة}
\noindent
تتحكم معظم المشكلات الهندسية بمجموعة من المعادلات التفاضلية الجزئية (PDEs) بشروط حدودية مناسبة. على سبيل المثال، يتم نمذجة تدفقات الموائع النيوتونية بمعادلات نافييه-ستوكس \cite{navier_stokes}؛ بينما يحكم اهتزاز الألواح الرقيقة معادلة تفاضلية جزئية من الرتبة الرابعة \cite{chang_shu}؛ في حين يمكن محاكاة الموجات الصوتية والموجات الدقيقة بمعادلة هلمهولتز \cite{chang_shu}. بشكل عام، يصعب جدًا الحصول على الحل المغلق لهذه المعادلات. من ناحية أخرى، هناك حاجة دائمة لحلول هذه المعادلات بسبب الأهمية العملية. على سبيل المثال، عند تصميم طائرة، نحتاج إلى معرفة منحنى \( c_l \) (معامل الرفع) مقابل \( c_d \) (معامل السحب) لشكل جناح معين. يمكن الحصول على قيم \( c_l \) و \( c_d \) من حل معادلات نافييه-ستوكس. لذلك، من المهم تطوير حلول تقريبية للمعادلات التفاضلية الجزئية المعطاة.\\
\noindent
في معظم الحالات، يتم تمثيل الحل التقريبي بقيم الدالة عند نقاط منفصلة معينة (نقاط الشبكة أو نقاط الشبكة). عند هذه المرحلة، قد يتساءل المرء عن العلاقة بين المشتقات في المعادلة التفاضلية الجزئية وقيم الدالة عند نقاط الشبكة. يبدو أن هناك جسرًا يربط بينهما. تقنية التقطيع العددي هي هذا الجسر، ويسمى الحل التقريبي المقابل بالحل العددي.\\
\noindent
حاليًا، هناك العديد من تقنيات التقطيع العددي المتاحة. من بينها، تقع طرق الفروق المحدودة (FD) \cite{navier_stokes}، والعناصر المحدودة (FE)، والحجوم المحدودة (FV) \cite{finite_difference} ضمن فئة الطرق منخفضة الرتبة، بينما تعتبر الطرق الطيفية والشبيهة بالطيفية طرقًا شاملة. تعتمد طريقة FD على متسلسلة تايلور أو التقريب متعدد الحدود، بينما تعتمد طريقة FE على المبدأ التبايني أو مبدأ البواقي الموزونة. تطبق طريقة FV قانون الحفظ الفيزيائي مباشرة على خلية محدودة. يمكن النظر إلى الطريقة الطيفية على أنها تطور متطرف لفئة مخططات التقطيع المعروفة باسم طرق البواقي الموزونة. العناصر الأساسية للطرق الطيفية هي دوال الأساس ودوال الترجيح. هناك علاقة وثيقة بين طرق FE والطرق الطيفية بمعنى أن كلا الطريقتين تستخدمان مجموعة من دوال الأساس لتقريب الحل. اختيار دوال الأساس هو أحد الميزات التي تميز الطريقة الطيفية عن طريقة FE. دوال الأساس للطرق الطيفية هي دوال قابلة للاشتقاق بشكل لانهائي ولها خصائص شاملة. في حالة طرق FE، يتم تقسيم المجال إلى عناصر صغيرة،\\
\noindent
ويتم تحديد دالة أساس في كل عنصر. وبالتالي، تكون دوال الأساس محلية في طبيعتها، ومناسبة جيدًا للتعامل مع الأشكال الهندسية المعقدة. يمكن اعتبار الطرق الطيفية امتدادًا لطرق FE، ويمكن النظر إليها على أنها تقنية تقريب للفضاء الكامل.\\
\noindent
يمكن إجراء معظم المحاكاة العددية للمشكلات الهندسية باستخدام الطرق منخفضة الرتبة FD وFE وFV باستخدام عدد كبير من نقاط الشبكة. ومع ذلك، في بعض التطبيقات العملية، تكون الحلول العددية للمعادلات التفاضلية الجزئية مطلوبة فقط في عدد قليل من النقاط المحددة في المجال الفيزيائي. لتحقيق درجة مقبولة من الدقة، لا تزال الطرق منخفضة الرتبة تتطلب استخدام عدد كبير من نقاط الشبكة للحصول على حلول دقيقة في هذه النقاط المحددة. يمكن العثور على مثال في تحليل الاهتزازات. عند التقطيع العددي للمعادلات التفاضلية الجزئية الحاكمة، توفر القيم الذاتية لنظام المعادلات الجبرية الناتجة الترددات الاهتزازية للمشكلة. عادةً، يكون عدد نقاط الشبكة الداخلية مساويًا لبعد نظام المعادلات الجبرية الناتج، مما يعطي نفس عدد الترددات الذاتية. من بين جميع الترددات الذاتية المحسوبة، فقط الترددات المنخفضة هي ذات أهمية عملية. ومع ذلك، نظرًا لأن الترددات الذاتية المحسوبة لها نفس درجة الدقة، فلا يزال هناك حاجة إلى عدد كبير من نقاط الشبكة للحصول على هذه الترددات المنخفضة بدقة. نتيجة لذلك، يتطلب ذلك الكثير من التخزين الافتراضي والجهد الحسابي. يبدو أن عيوب الطرق منخفضة الرتبة المذكورة أعلاه يمكن تحسينها باستخدام الطرق عالية الرتبة والطرق الشاملة. بشكل عام، تتمتع الطرق عالية الرتبة بخطأ قطع عالي الرتبة. وبالتالي، لتحقيق نفس درجة الدقة، يمكن أن يكون تباعد الشبكة المستخدم في الطرق عالية الرتبة أكبر بكثير من ذلك المستخدم في الطرق منخفضة الرتبة. نتيجة لذلك، يمكن للطرق عالية الرتبة أن تنتج حلولًا عددية دقيقة باستخدام عدد قليل جدًا من نقاط الشبكة. الطريقة الطيفية هي الخيار الطبيعي لهذا الغرض. حاليًا، الطريقة الطيفية ناجحة للغاية في عدة مجالات: نمذجة الاضطراب، التنبؤ بالطقس، الموجات غير الخطية، نمذجة الزلازل، إلخ. يتطلب تطبيق الطريقة الطيفية معرفة رياضية كبيرة بالنظرية.\\
\noindent
من ناحية أخرى، في البحث عن تقنية تقطيع فعالة للحصول على حلول عددية دقيقة باستخدام عدد صغير نسبيًا من نقاط الشبكة، قدم بلمان وآخرون (1971، 1972) \cite{bellman_roth} \cite{Bellman} \cite{bert} طريقة التربيع التفاضلي (DQ)، حيث يتم التعبير عن المشتق الجزئي للدالة بالنسبة لاتجاه إحداثي كمجموع خطي موزون لجميع قيم الدالة عند جميع نقاط الشبكة في ذلك الاتجاه. تم بدء طريقة DQ من فكرة التربيع التكاملي. مفتاح DQ هو تحديد معاملات الترجيح لتقطيع المشتقات من أي رتبة. اقترح بلمان وآخرون (1972) طريقتين لتحديد معاملات الترجيح للمشتق من الرتبة الأولى. الطريقة الأولى تحل نظام معادلات جبرية. الطريقة الثانية تستخدم صيغة جبرية بسيطة، ولكن مع اختيار إحداثيات نقاط الشبكة كجذور متعدد الحدود ليجاندر المزاح. معظم التطبيقات المبكرة لـ DQ في الهندسة (بلمان وآخرون 1971، 1972، 1974، 1975، 1986، كاشيف وبلمان 1974، هو وهو 1974، مينجل 1977، وانغ 1982، سيفان وسليبتشوفيتش 1983، 1984، نوديموثو وآخرون 1984، بيرت وآخرون 1988، 1989، جانغ وآخرون 1989) استخدمت طرق بلمان لحساب معاملات الترجيح. من بين طريقي بلمان، عادةً ما يتم تطبيق الطريقة الأولى لأنها تسمح باختيار إحداثيات نقاط الشبكة بشكل تعسفي. لسوء الحظ، عندما يكون ترتيب نظام المعادلات الجبرية كبيرًا،\\
\noindent
تصبح مصفوفته سيئة التكييف. وبالتالي، من الصعب جدًا الحصول على معاملات الترجيح لعدد كبير من نقاط الشبكة باستخدام هذه الطريقة. هذا هو السبب على الأرجح وراء استخدام التطبيقات المبكرة لهذا المخطط عدد نقاط الشبكة أقل من أو يساوي 13 فقط. 