\chapter*{استنتاجات و توصيات}
\addcontentsline{toc}{chapter*}{استنتاجات و توصيات}
\section*{استنتاجات}
\noindent
في نهاية البحث يمكن استنتاج ان طريقة التفاضل التربيعي هي طريقة جيدة في ايجاد نتائج قريبة من الحل الدقيق وبنقاط شبكة قليلة نسبياً مقارنةً بطرائق اخرى ، وكذلك يمكن استنتاج ان طريقة التفاضل التربيعي مستقرة عند قيم مختلفة من عدد نقاط الشبكة ، ولكن لا يمنع ان نعطي بعض التوصيات عند تطبيق طريقة التفاضل التربيعي في مختلف المجالات:
\section*{توصيات}

\begin{enumerate}
	\item يُوصى باختيار نوع نقاط الشبكة (منتظمة أو غير منتظمة مثل نقاط تشيبيشيف) بما يتناسب مع طبيعة المسألة المدروسة، إذ تؤدي النقاط غير المنتظمة إلى دقة أعلى في المسائل ذات التغيرات الحادة.
	
	\item يُنصح بإجراء تحليل دقيق للاستقرارية والدقة العددية للطريقة، خاصة عند تطبيقها على المعادلات غير الخطية أو المعادلات الزمنية، لضمان موثوقية النتائج.
	
	\item يُفضل مقارنة نتائج DQM مع طرق عددية أخرى كطريقة الفروق المنتهية والطريقة الطيفية، للتحقق من دقة الحل وتقييم فعالية الطريقة.
	
	\item يُستحسن استخدام خوارزميات فعالة ودقيقة لحساب أوزان التفاضل التربيعي، مثل الطريقة التحليلية المشتقة من متعددات حدود لاغرانج، لتقليل الخطأ العددي وزيادة الكفاءة.

	
	\item يُنصح بالاستفادة من البرمجيات المتقدمة مثل \textbf{MATLAB} أو \textbf{Python} والتي تتيح أدوات جاهزة لحساب الأوزان وتطبيق الطريقة بكفاءة عالية.
	
	\item يُستحسن توسيع استخدام الطريقة في تطبيقات هندسية وعلمية متنوعة مثل ميكانيكا الموائع، تحليل الإجهادات، وانتقال الحرارة، نظراً لما أثبتته من دقة وكفاءة مقارنة بطرق عددية تقليدية.
\end{enumerate}