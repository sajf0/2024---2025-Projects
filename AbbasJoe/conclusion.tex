\chapter*{توصيات \en{Recommendations}}
\addcontentsline{toc}{chapter*}{توصيات}
بشكل عام طريقة التفاضل التربيعي جيدة في التطبيقات كما شاهدنا في الفصل الثاني ولكن نقدم التوصيات الآتية لمساعدة الباحثين على اتخاذ القرارات الصحيحة و الملائمة
\begin{enumerate}
	\item إذا كان مجال التطبيق يتطلب بيانات كثيرة و توجد طريقة أخرى مختلفة عن التفاضل التربيعي يجب مقارنة الطريقتين لنرى أي منهما أكثر كفاءة.
	
	\item كما شاهدنا في معادلة الحرارة فإن النتائج أقل دقة من نظيرها في المعادلات من الرتبة الأولى ، أي لا ينصح استخدام التفاضل التربيعي في المعادلات التفاضلية من الرتبة الثانية و أعلى.
	
	\item كذلك لا ينصح بأخذ كميات كبيرة لقيمة $x$ حيث نتوقع أن الخطأ سوف يزداد و بالتالي تقل دقة الطريقة.
	
	\item لا ينصح بأخذ مقادير ضخمة لـــ $N$ حيث أن الطريقة لا تتحمل هذا القدر من البيانات ومن الممكن عدم الحصول على نتيجة أو أن تكون النتائج غير دقيقة تماماً و بعيدة كل البعد عن الواقع.
\end{enumerate}