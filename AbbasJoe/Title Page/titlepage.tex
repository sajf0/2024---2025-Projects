\newgeometry{
margin=0.5in
}
\begin{titlepage}
	\begin{minipage}{0.2\textwidth}
		\raggedleft\includegraphics[scale=0.2]{Title Page/university.png}
	\end{minipage}
	\hfill
	\begin{minipage}{0.5\textwidth}
		\centering
		% \large
		\textbf{وزارة التعليم العالي والبحث العلمي}\\
		\textbf{جامعة البصرة}\\
		\textbf{كلية التربية للعلوم الصرفة}\\
		\textbf{قسم الرياضيات}
	\end{minipage}
	\hfill
	\begin{minipage}{0.2\textwidth}
		\raggedright\includegraphics[scale=0.16]{Title Page/college.png}   
	\end{minipage}
	
	\vspace{1cm}
	
	\begin{center}
		\Large \textbf{طريقة التفاضل التربيعي لحل مسائل القيم الحدودية}\\
	\end{center}
	\vfill
	
	\begin{center}
		\large
		\textbf{بحث تخرج تقدم به الى}
	\end{center}
	
	\begin{center}
		\large
		\textbf{قسم الرياضيات كلية التربية للعلوم الصرفة جامعة البصرة\\
			\vspace{6pt}
			وهو جزء من متطلبات نيل شهادة بكالوريوس علوم الرياضيات}
	\end{center}
	\vfill
	\begin{center}
		\large
		\textbf{من قبل الطالب}\\
		\vspace{8pt}
		 \large
		\textbf{عباس حمود ضيدان}
	\end{center}
	\vspace{10pt}
	\begin{center}
		\large
		\textbf{إشراف}\\
		\vspace{8pt}
		 \large
		\textbf{ا.د. عبدالستار جابر علي}
	\end{center}
	\vspace{80pt}
	\begingroup
\large{\raggedleft \textbf{2025 م}} {\hfill \textbf{1446 هــ}}
\endgroup
\end{titlepage}
\restoregeometry