\chapter*{الخلاصة \en{Abstract}}
\addcontentsline{toc}{chapter*}{الخلاصة}
قدمنا في هذا البحث كيفية عمل طريقة التفاضل التربيعي و تطبيقات على معادلات تفاضلية مختلفة و لاحظنا ان النتائج متقاربة للحل المضبوط مما يعني ان طريقة التفاضل التربيعي طريقة جيدة في التطبيقات العملية و لكن يؤخذ عليها بعض الامور منها ، عدم استقرارية الحلول عند زيادة عدد نقاط ( $N$ ) ، أيضاً لاحظنا من الامثلة بأن مقدار الخطأ يزداد مع زيادة قيمة $N$ ( اكثر من $N=17$ ) بشكل خاص في عملها.