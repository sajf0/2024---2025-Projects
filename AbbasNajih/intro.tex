\chapter*{مقدمة}
\addcontentsline{toc}{chapter*}{مقدمة}

​تحويل لابلاس هو أداة رياضية قوية تُستخدم لتحويل المعادلات التفاضلية، خاصة الخطية ذات المعاملات الثابتة، إلى معادلات جبرية أبسط في مجال التردد المركب. يُسهّل هذا التحويل عملية الحل، خصوصًا عندما تكون المعادلة مصحوبة بشروط ابتدائية، حيث يتم دمج هذه الشروط مباشرة في المعادلة المحوّلة، مما يلغي الحاجة إلى حساب الثوابت بشكل منفصل كما في الطرق التقليدية.​[5]

\noindent
يُستخدم تحويل لابلاس على نطاق واسع في مجالات متعددة مثل الهندسة الكهربائية لتحليل الدوائر الكهربائية، والهندسة الميكانيكية لدراسة الاهتزازات، وفي أنظمة التحكم لتحليل استجابات الأنظمة. بعد حل المعادلة الجبرية في مجال التردد، يتم استخدام تحويل لابلاس العكسي للعودة إلى الحل في المجال الزمني، مما يوفر فهمًا دقيقًا لسلوك النظام الأصلي.​[3]

\noindent
بفضل قدرته على تبسيط المعادلات المعقدة والتعامل مع الشروط الابتدائية بفعالية، يُعد تحويل لابلاس أداة لا غنى عنها في تحليل الأنظمة الديناميكية وحل المعادلات التفاضلية في مختلف التخصصات العلمية والهندسية.[2]